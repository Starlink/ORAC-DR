\documentclass[twoside,11pt]{article}

% +
%  Name:
%     sun236.tex

%  Purpose:
%     SUN documentation for ORAC-DR spectroscopy (SUN/236)

%  Authors:
%     Paul Hirst (JAC)
%     Tim Jenness (JAC)
%     Brad Cavanagh (JAC)

%  Copyright:
%     Copyright (C) 2003 Particle Physics and Astronomy
%     Research Council. All Rights Reserved.

%  History:
%     $Log$
%     Revision 1.13  2004/05/27 21:40:20  bradc
%     updates for v4.1
%
%     Revision 1.12  2003/06/12 23:19:54  bradc
%     updates for v4.0
%
%     Revision 1.17  2003/06/12 22:21:28  bradc
%     June, not May
%
%     Revision 1.16  2003/04/25 01:29:20  bradc
%     updates for 4.0 release
%
%     Revision 1.15  2002/09/16 04:30:21  timj
%     Add stardocversion
%
%     Revision 1.14  2002/09/16 03:21:54  timj
%     Fix minor tweaks
%
%     Revision 1.13  2002/09/16 03:13:33  timj
%     Fix spelling mistake
%
%     Revision 1.12  2002/09/16 03:09:48  timj
%     Brad was correct :-) \ORACDR\ is meant to be a latex command.
%
%     Revision 1.11  2002/09/14 01:49:19  phirst
%     fix Brads LaTeX
%
%     Revision 1.10  2002/05/28 21:12:22  bradc
%     Clarified use of calibration options
%
%     Revision 1.9  2001/12/14 02:58:17  timj
%     Wrong copyright
%
%     Revision 1.8  2001/12/14 02:53:06  timj
%     Minor updates for V3.0-3
%
%     Revision 1.7  2001/12/14 02:20:54  timj
%     even more missing typos
%
%     Revision 1.6  2001/12/14 02:18:17  timj
%     stardoccopyright as 2001
%
%     Revision 1.5  2001/12/14 02:16:53  timj
%     couple of typos plus increment doc number
%
%     Revision 1.4  2001/12/13 01:32:21  phirst
%     Added FAINT_POINT_SOURCE
%
%     Revision 1.3  2001/11/28 03:18:28  timj
%     Add xref to other oracdr docs
%
%     Revision 1.2  2001/11/28 01:58:23  timj
%     - Fix spelling mistakes
%     - Add logo
%     - Remove DESCRIPTION paragraph headings
%     - Tweak some sections to be real lists
%     - Some verbatim tables were not formatted verbatim
%

%  Revision:
%     $Id$

% -


% ? Specify used packages
\usepackage{graphicx}        %  Use this one for final production.
% \usepackage[draft]{graphicx} %  Use this one for drafting.
% ? End of specify used packages

\pagestyle{myheadings}

% -----------------------------------------------------------------------------
% ? Document identification
% Fixed part
\newcommand{\stardoccategory}  {Starlink User Note}
\newcommand{\stardocinitials}  {SUN}
\newcommand{\stardocsource}    {sun\stardocnumber}
\newcommand{\stardoccopyright} {Copyright \copyright\ 2004 Particle Physics and Astronomy Research Council}


% Variable part - replace [xxx] as appropriate.
\newcommand{\stardocnumber}    {236.4}
\newcommand{\stardocauthors}   {Paul Hirst \\ Brad Cavanagh \\
                                Joint Astronomy Centre, Hilo, Hawaii}
\newcommand{\stardocdate}      {June 2004}
\newcommand{\stardoctitle}     {ORAC-DR -- spectroscopy data reduction}
\newcommand{\stardocversion}   {4.1}
\newcommand{\stardocmanual}    {User Guide}
\newcommand{\stardocabstract}  {ORAC-DR is a
general-purpose automatic data-reduction pipeline environment.  This
document describes its use to reduce spectroscopy data collected at the
United Kingdom Infrared Telescope (UKIRT) with the CGS4, UIST and MICHELLE
instruments. }
% ? End of document identification
% -----------------------------------------------------------------------------

% +
%  Name:
%     sun.tex
%
%  Purpose:
%     Template for Starlink User Note (SUN) documents.
%     Refer to SUN/199
%
%  Authors:
%     AJC: A.J.Chipperfield (Starlink, RAL)
%     BLY: M.J.Bly (Starlink, RAL)
%     PWD: Peter W. Draper (Starlink, Durham University)
%
%  History:
%     17-JAN-1996 (AJC):
%        Original with hypertext macros, based on MDL plain originals.
%     16-JUN-1997 (BLY):
%        Adapted for LaTeX2e.
%        Added picture commands.
%     13-AUG-1998 (PWD):
%        Converted for use with LaTeX2HTML version 98.2 and
%        Star2HTML version 1.3.
%      1-FEB-2000 (AJC):
%        Add Copyright statement in LaTeX
%     {Add further history here}
%
% -

\newcommand{\stardocname}{\stardocinitials /\stardocnumber}
\markboth{\stardocname}{\stardocname}
\setlength{\textwidth}{160mm}
\setlength{\textheight}{230mm}
\setlength{\topmargin}{-2mm}
\setlength{\oddsidemargin}{0mm}
\setlength{\evensidemargin}{0mm}
\setlength{\parindent}{0mm}
\setlength{\parskip}{\medskipamount}
\setlength{\unitlength}{1mm}

% -----------------------------------------------------------------------------
%  Hypertext definitions.
%  ======================
%  These are used by the LaTeX2HTML translator in conjunction with star2html.

%  Comment.sty: version 2.0, 19 June 1992
%  Selectively in/exclude pieces of text.
%
%  Author
%    Victor Eijkhout                                      <eijkhout@cs.utk.edu>
%    Department of Computer Science
%    University Tennessee at Knoxville
%    104 Ayres Hall
%    Knoxville, TN 37996
%    USA

%  Do not remove the %begin{latexonly} and %end{latexonly} lines (used by 
%  LaTeX2HTML to signify text it shouldn't process).
%begin{latexonly}
\makeatletter
\def\makeinnocent#1{\catcode`#1=12 }
\def\csarg#1#2{\expandafter#1\csname#2\endcsname}

\def\ThrowAwayComment#1{\begingroup
    \def\CurrentComment{#1}%
    \let\do\makeinnocent \dospecials
    \makeinnocent\^^L% and whatever other special cases
    \endlinechar`\^^M \catcode`\^^M=12 \xComment}
{\catcode`\^^M=12 \endlinechar=-1 %
 \gdef\xComment#1^^M{\def\test{#1}
      \csarg\ifx{PlainEnd\CurrentComment Test}\test
          \let\html@next\endgroup
      \else \csarg\ifx{LaLaEnd\CurrentComment Test}\test
            \edef\html@next{\endgroup\noexpand\end{\CurrentComment}}
      \else \let\html@next\xComment
      \fi \fi \html@next}
}
\makeatother

\def\includecomment
 #1{\expandafter\def\csname#1\endcsname{}%
    \expandafter\def\csname end#1\endcsname{}}
\def\excludecomment
 #1{\expandafter\def\csname#1\endcsname{\ThrowAwayComment{#1}}%
    {\escapechar=-1\relax
     \csarg\xdef{PlainEnd#1Test}{\string\\end#1}%
     \csarg\xdef{LaLaEnd#1Test}{\string\\end\string\{#1\string\}}%
    }}

%  Define environments that ignore their contents.
\excludecomment{comment}
\excludecomment{rawhtml}
\excludecomment{htmlonly}

%  Hypertext commands etc. This is a condensed version of the html.sty
%  file supplied with LaTeX2HTML by: Nikos Drakos <nikos@cbl.leeds.ac.uk> &
%  Jelle van Zeijl <jvzeijl@isou17.estec.esa.nl>. The LaTeX2HTML documentation
%  should be consulted about all commands (and the environments defined above)
%  except \xref and \xlabel which are Starlink specific.

\newcommand{\htmladdnormallinkfoot}[2]{#1\footnote{#2}}
\newcommand{\htmladdnormallink}[2]{#1}
\newcommand{\htmladdimg}[1]{}
\newcommand{\hyperref}[4]{#2\ref{#4}#3}
\newcommand{\htmlref}[2]{#1}
\newcommand{\htmlimage}[1]{}
\newcommand{\htmladdtonavigation}[1]{}

\newenvironment{latexonly}{}{}
\newcommand{\latex}[1]{#1}
\newcommand{\html}[1]{}
\newcommand{\latexhtml}[2]{#1}
\newcommand{\HTMLcode}[2][]{}

%  Starlink cross-references and labels.
\newcommand{\xref}[3]{#1}
\newcommand{\xlabel}[1]{}

%  LaTeX2HTML symbol.
\newcommand{\latextohtml}{\LaTeX2\texttt{HTML}}

%  Define command to re-centre underscore for Latex and leave as normal
%  for HTML (severe problems with \_ in tabbing environments and \_\_
%  generally otherwise).
\renewcommand{\_}{\texttt{\symbol{95}}}

% -----------------------------------------------------------------------------
%  Debugging.
%  =========
%  Remove % on the following to debug links in the HTML version using Latex.

% \newcommand{\hotlink}[2]{\fbox{\begin{tabular}[t]{@{}c@{}}#1\\\hline{\footnotesize #2}\end{tabular}}}
% \renewcommand{\htmladdnormallinkfoot}[2]{\hotlink{#1}{#2}}
% \renewcommand{\htmladdnormallink}[2]{\hotlink{#1}{#2}}
% \renewcommand{\hyperref}[4]{\hotlink{#1}{\S\ref{#4}}}
% \renewcommand{\htmlref}[2]{\hotlink{#1}{\S\ref{#2}}}
% \renewcommand{\xref}[3]{\hotlink{#1}{#2 -- #3}}
%end{latexonly}
% -----------------------------------------------------------------------------
% ? Document specific \newcommand or \newenvironment commands.

\newcommand{\ARD}{{\footnotesize ARD}}
\newcommand{\CCDPACK}{{\footnotesize CCDPACK}}
\newcommand{\CONVERT}{{\footnotesize CONVERT}}
\newcommand{\CURSA}{{\footnotesize CURSA}}
\newcommand{\GAIA}{{\footnotesize GAIA}}
\newcommand{\EXTRACTOR}{\mbox{\footnotesize EXTRACTOR}}
\newcommand{\FIGARO}{\mbox{\footnotesize FIGARO}}
\newcommand{\KAPPA}{{\footnotesize KAPPA}}
\newcommand{\ORACDR}{{\footnotesize ORAC-DR}}
\newcommand{\PHOTOM}{{\footnotesize PHOTOM}}
\newcommand{\PISA}{{\footnotesize PISA}}
\newcommand{\POLPACK}{{\footnotesize POLPACK}}


% ? End of document specific commands
% -----------------------------------------------------------------------------
%  Title Page.
%  ===========
\renewcommand{\thepage}{\roman{page}}
\begin{document}
\setcounter{secnumdepth}{5}
\thispagestyle{empty}

%  Latex document header.
%  ======================
\begin{latexonly}
   CCLRC / \textsc{Rutherford Appleton Laboratory} \hfill \textbf{\stardocname}\\
   {\large Particle Physics \& Astronomy Research Council}\\
   {\large Starlink Project\\}
   {\large \stardoccategory\ \stardocnumber}
   \begin{flushright}
   \stardocauthors\\
   \stardocdate
   \end{flushright}
   \vspace{-4mm}
   \rule{\textwidth}{0.5mm}
   \vspace{5mm}
   \begin{center}
   {\Huge\textbf{\stardoctitle \\ [2.5ex]}}
   {\LARGE\textbf{\stardocversion \\ [4ex]}}
   {\Huge\textbf{\stardocmanual}}
   \end{center}
   \vspace{5mm}

% ? Add picture here if required for the LaTeX version.
%   e.g. \includegraphics[scale=0.3]{filename.ps}
\begin{center}
\includegraphics[width=1.0in]{sun236_logo.eps}
\end{center}
% ? End of picture

% ? Heading for abstract if used.
   \vspace{10mm}
   \begin{center}
      {\Large\textbf{Abstract}}
   \end{center}
% ? End of heading for abstract.
\end{latexonly}

%  HTML documentation header.
%  ==========================
\begin{htmlonly}
   \xlabel{}
   \begin{rawhtml} <H1> \end{rawhtml}
      \stardoctitle\\
      \stardocversion\\
      \stardocmanual
   \begin{rawhtml} </H1> <HR> \end{rawhtml}

% ? Add picture here if required for the hypertext version.
%   e.g. \includegraphics[scale=0.7]{filename.ps}
\includegraphics[width=1.0in]{sun236_logo.eps}
% ? End of picture

   \begin{rawhtml} <P> <I> \end{rawhtml}
   \stardoccategory\ \stardocnumber \\
   \stardocauthors \\
   \stardocdate
   \begin{rawhtml} </I> </P> <H3> \end{rawhtml}
      \htmladdnormallink{CCLRC / Rutherford Appleton Laboratory}
                        {http://www.cclrc.ac.uk} \\
      \htmladdnormallink{Particle Physics \& Astronomy Research Council}
                        {http://www.pparc.ac.uk} \\
   \begin{rawhtml} </H3> <H2> \end{rawhtml}
      \htmladdnormallink{Starlink Project}{http://www.starlink.rl.ac.uk/}
   \begin{rawhtml} </H2> \end{rawhtml}
   \htmladdnormallink{\htmladdimg{source.gif} Retrieve hardcopy}
      {http://www.starlink.rl.ac.uk/cgi-bin/hcserver?\stardocsource}\\

%  HTML document table of contents. 
%  ================================
%  Add table of contents header and a navigation button to return to this 
%  point in the document (this should always go before the abstract \section). 
  \label{stardoccontents}
  \begin{rawhtml} 
    <HR>
    <H2>Contents</H2>
  \end{rawhtml}
  \htmladdtonavigation{\htmlref{\htmladdimg{contents_motif.gif}}
        {stardoccontents}}

% ? New section for abstract if used.
  \section{\xlabel{abstract}Abstract}
% ? End of new section for abstract
\end{htmlonly}

% -----------------------------------------------------------------------------
% ? Document Abstract. (if used)
%  ==================
\stardocabstract
% ? End of document abstract

% -----------------------------------------------------------------------------
% ? Latex Copyright Statement
%  =========================
\begin{latexonly}
\newpage
\vspace*{\fill}
\stardoccopyright
\end{latexonly}
% ? End of Latex copyright statement

% -----------------------------------------------------------------------------
% ? Latex document Table of Contents (if used).
%  ===========================================
  \newpage
  \begin{latexonly}
    \setlength{\parskip}{0mm}
    \tableofcontents
    \setlength{\parskip}{\medskipamount}
    \markboth{\stardocname}{\stardocname}
  \end{latexonly}
% ? End of Latex document table of contents
% -----------------------------------------------------------------------------

\cleardoublepage
\renewcommand{\thepage}{\arabic{page}}
\setcounter{page}{1}

% ? Main text

\section{Introduction}

This document describes the spectroscopy recipes and primitives included with
\ORACDR. See \xref{SUN/230}{sun230}{} for general \ORACDR\ documentation,
\xref{SUN/232}{sun232}{} for information on \ORACDR\ imaging data reduction,
and \xref{SUN/246}{sun246}{} for information on \ORACDR\ integral field
spectroscopy data reduction.

At this time (June 2004), \ORACDR's spectroscopy suite is designed
to reduce data from the UKIRT spectrometer suite: CGS4, UIST and
MICHELLE.; data from the AAT imaging spectrometer IRIS2, and data
from the ESO imaging spectrometer ISAAC. The recipes and primitives
are written in such a way as to make the addition of new instruments
simple.

\section{Recipe and Primitive Architecture}

In general, an \ORACDR\ recipe is simply a list of primitives, which
are executed in turn on the data from each observation that is passed
through the pipeline. The primitives form building blocks, each
carrying out an atomic data reduction operation. Thus, the recipe is
readable, understandable and to some extent modifiable by someone not
familiar with the primitive internals -- for example an observer or
scientist reducing data and wishing to add or remove steps from the
data reduction process. The primitives themselves are implemented in
Perl5 and are not designed to be usefully modifiable by a
non-programmer.

In actual fact, it has been found to be beneficial in many places to
take this abstraction one level deeper -- in the spectroscopy pipeline
many of the steps which we wish to denote in the recipe as a single
primitive are themselves too complex to be considered an atomic data
reduction operation, perhaps a good analogy is that be considered a
molecular operation, themselves implemented as a simple list of
primitives which carry out the true atomic operations. In some cases
there is simple logic, generally simple flow control switches, for
example ``if'' clauses at the molecular level, though this code is
simple and these molecular level primitives can be usefully modified
to add or remove steps by a typical astronomer.

\section{The Recipes}

Each recipe has an in-code documentation section, which documents the
recipe concerned. Here, we simply divide the recipes into groups
corresponding to the type of observations they reduce, and give the
recipe documentation.

Some of the recipes have dedicated primitives, this is simply to
avoid having the perl code in the recipe. In these cases, we include
the primitive documentation with that of the recipe. For the main
science recipes, we will describe the recipes at molecular level
here, with the details of the primitives in a subsequent section.

\subsection{System Verification Recipes and Primitives}

\subsubsection{ARRAY\_TESTS}

Strictly speaking, the ARRAY\_TESTS recipe is not part of the \ORACDR\
spectroscopy suite; each instrument provides its own array tests
recipe if it requires one. However, if one is provided, it can
generate bad pixel masks and read-noise measurements that will be used
by the rest of the spectroscopy pipeline. In fact, a read-noise number
is mandatory for the spectroscopy pipeline, and thus if not determined
by an array tests procedure, has to be supplied manually, usually by
the instrument scientist. The CGS4 specific array tests recipe is
described here, as it is currently a good example.

%% ORACDRDOC_RECIPE:ARRAY_TESTS
\paragraph{ARRAY\_TESTS\label{ARRAY_TESTS}\index{ARRAY\ TESTS}}


Tests the array, determining the readnoise and dark current.


\mbox{}


Performs array tests for an array. This recipe uses the raw data,
reduces it, and calculates read noise, median and modal dark currents
for the array. It assumes that data is taken in the standard CGS4
array tests configuration .



See the \_ARRAY\_TESTS\_ CGS4 primitive for details.


%% ORACDRDOC_PRIMITIVE:_ARRAY_TESTS_
\paragraph{\_ARRAY\_TESTS\_\label{_ARRAY_TESTS_}\index{\ ARRAY\ TESTS\ }}


CGS4 array tests.


\mbox{}


Calculates the readnoise and dark current parameters for the array,
and forms a bad pixel mask.



This primitive ensures that the BIAS and DARK frames taken as part of
the array tests are also filed with the calibration system so that
they can be used to reduce subsequent data.



The \_FIND\_BAD\_PIXELS\_ primitive is called to detect bad pixels at each
stage of the array tests.



The \_ARRAY\_TESTS\_STATISTICS\_ primitive is called to determine the
readnoise and dark currents.


%% ORACDRDOC_PRIMITIVE:_ARRAY_TESTS_STATISTICS_
\paragraph{\_ARRAY\_TESTS\_STATISTICS\_\label{_ARRAY_TESTS_STATISTICS_}\index{\ ARRAY\ TESTS\ STATISTICS\ }}


Determines array tests statistics.


\mbox{}


Does the array tests statistics for CGS4.



The DARK frames used must have been reduced with the \_REDUCE\_DARK\_
primitive; this is normally called from the \_ARRAY\_TESTS\_ primitive
during array test reduction.



The basic principle used to determine the readnoise is to subtract 2
dark frames of the same exposure time, on the assumption that the
readnoise is the only difference between them.


%% ORACDRDOC_PRIMITIVE:_FIND_BAD_PIXELS_
\paragraph{\_FIND\_BAD\_PIXELS\_\label{_FIND_BAD_PIXELS_}\index{\ FIND\ BAD\ PIXELS\ }}


Finds bad pixels in a frame.


\mbox{}


Used in array tests to add the bad pixels detected in this frame into the 
bad pixel mask.



Applies thresholds to the data frames and flags pixels outside
the limits as bad.



For BIAS frames, we do a 3-sigma clipped stats and flag pixels more
than 6 sigma from the mean. The actual threshold limits applied are
reported. In addition, we also flag pixels whose BIAS variance is more
than 6-sigma above the mean in a 3-sigma clipped stats.



For DARK frames, we set the thresholds to 1700 and 15 if the exposure
is greater than 80 seconds, and 1500 and -100 if it is not. This will
need updating if MICHELLE elects to use an automated bad pixel
detection scheme in the future and when UIST support is added - these
values are CGS4 specific.



The bad pixels detected are added into the current bad pixel mask and
then this is filed with the calibration system as a new and current
bad pixel mask.


\subsubsection{EMISSIVITY}

It is not strictly necessary for each instrument to be able to take
emissivity data, though currently both CGS4 and MICHELLE do, and it is
strongly anticipated that UIST also will in the near future.

%% ORACDRDOC_RECIPE:EMISSIVITY
\paragraph{EMISSIVITY\label{EMISSIVITY}\index{EMISSIVITY}}


A recipe to measure telescope emissivity.


\mbox{}


Used to measure the telescope emissivity. See the \_EMISSIVITY\_
primitive for details of the calculation, this recipe simply calls
\_REDUCE\_SINGLE\_FRAME\_ followed by \_EMISSIVITY\_.


%% ORACDRDOC_PRIMITIVE:_EMISSIVITY_
\paragraph{\_EMISSIVITY\_\label{_EMISSIVITY_}\index{\ EMISSIVITY\ }}


Calculates telescope emissivity.


\mbox{}


Expects to be run on a SKY frame (taken pointing at the sky) and an
OBJECT frame (taken pointing at the mirror covers ie dome).



File extensions created:



Frames:

\begin{description}

\item[\textbf{\_ess}] \mbox{}

Emissivity Sky Spectrum


\item[\textbf{\_eds}] \mbox{}

Emissivity Dome Spectrum

\end{description}


Group:



Group file contains ratio of sky and dome spectra

\begin{description}

\item[\textbf{\_sm}] \mbox{}

Smoothed by 5 pixels - used to select region

\end{description}


Spectra are extracted from both files and divided by the exposure time
for that frame.



The "emissivity" spectrum is the ratio of these two spectra.



The emissivity spectrum is smoothed with a box of 5 pixels.  The
location of the minimum of the smoothed spectrum is found, and a
5-pixel window centred at this point is extracted from the sky and
dome spectra.  The mean values of these two windows are calculated and
printed.  The ratio of these two mean values is printed as the
emissivity.


\subsection{Calibration Frame Recipes}

\subsubsection{Flat Fields}

The REDUCE\_FLAT recipe will reduce a FLAT field observation. FLAT
observations are usually taken by observing the black-body source in
the CGS4 calibration unit, though the Tungsten-Halogen lamp is used at
the shorter CGS4 wavelengths. Standard operating procedures for
MICHELLE flats have yet to be ascertained, though REDUCE\_FLAT will
work fine if the (warm) shutter of the instrument is used as a black
body source. Sky flats will probably be reduced by an (as yet
unwritten) separate recipe.

%% ORACDRDOC_RECIPE:REDUCE_FLAT
\paragraph{REDUCE\_FLAT\label{REDUCE_FLAT}\index{REDUCE\ FLAT}}


Reduces a spectroscopy flat field.


\mbox{}


Reduces a flat field in the conventional manner, including
normalisation by a model of the blackbody source. Files the normalised
flat field frame for use by subsequent flat fielding operations.


%% ORACDRDOC_PRIMITIVE:_NORMALISE_FLAT_BY_BB_
\paragraph{\_NORMALISE\_FLAT\_BY\_BB\_\label{_NORMALISE_FLAT_BY_BB_}\index{\ NORMALISE\ FLAT\ BY\ BB\ }}


Normalise a flat field frame with a black body curve.


\mbox{}


Normalises a frame (usually a CGS4 flat field frame).



This primitive first creates a black body spectrum using the temperature
from the BBTEMP header, grows this to the size of the image, and divides by
it. It then divides the image by the image's mean pixel value so as to
normalise its absolute level to 1.

\subparagraph*{NOTE\label{_NORMALISE_FLAT_BY_BB__NOTE}\index{ NORMALISE FLAT BY BB !NOTE}}


Uses the BBTEMP FITS header. Perhaps this should have a system internal
translated value to cope with future instrumentation.


\subsubsection{Arc Lamp observations}

The REDUCE\_ARC recipe reduces a CGS4 ARC lamp observation. Currently
the arc frame is not actually used for wavelength calibration, though
this will probably change shortly.

%% ORACDRDOC_RECIPE:REDUCE_ARC
\paragraph{REDUCE\_ARC\label{REDUCE_ARC}\index{REDUCE\ ARC}}


Reduces an arc lamp observation.


\mbox{}


Reduces an arc lamp observation in the conventional manner, and
applies an estimated wavelength scale (based on the CGS4 motor
positions) so that you can easily refer to an arc line list or map to
check that you are hitting your wavelength region of interest.

\subparagraph*{NOTE\label{REDUCE_ARC_NOTE}\index{REDUCE ARC!NOTE}}


Currently, this recipe does not attempt to use the arc lamp data to
carry out proper wavelength calibration. This will change at some
point in the future.


\subsubsection{Dark Frames}

The REDUCE\_DARK recipe will reduce and file a DARK observation. In
practice this is very rarely used as the dark current of the array is
intrinsically subtracted when subtracting a sky or offset beam frame.

%% ORACDRDOC_RECIPE:REDUCE_DARK
\paragraph{REDUCE\_DARK\label{REDUCE_DARK}\index{REDUCE\ DARK}}


Reduce a dark frame.


\mbox{}


Reduces a Dark frame. This is generally only used by the array test
recipe and for engineering. It is unusual to require a dark frame with CGS4 as
the dark frame would be the same in the offset beam or sky image and
thus cancels when sky subtraction is carried out.


\subsubsection{Bias Frames}

The REDUCE\_BIAS recipe reduces a bias observation. The default BIAS
observation takes 3 integrations, each containing many exposures. The
REDUCE\_BIAS recipe forms a variance array based on the variance of the
3 integrations in the observation.


%% ORACDRDOC_RECIPE:REDUCE_BIAS
\paragraph{REDUCE\_BIAS\label{REDUCE_BIAS}\index{REDUCE\ BIAS}}


Reduce a spectrocopy BIAS observation.


\mbox{}


Reduces a spectroscopy BIAS observation, including coadding multiple
integrations. Files the reduced bias frame for use by subsequent
reduction of STARE and CHOP mode data.


%% ORACDRDOC_PRIMITIVE:_REDUCE_BIAS_
\paragraph{\_REDUCE\_BIAS\_\label{_REDUCE_BIAS_}\index{\ REDUCE\ BIAS\ }}


Reduces a spectroscopy BIAS frame.


\mbox{}


Averages together multiple integrations to make the output file data array.



If there are more than 3 integrations, a variance array is created as the
statistical variance of the individual input integrations.



Otherwise, the variance is simply from the readnoise added to the integrations
before averaging.



\subsection{Calibration Star Recipes}

\subsubsection{STANDARD\_STAR}
%% ORACDRDOC_RECIPE:STANDARD_STAR
\paragraph{STANDARD\_STAR\label{STANDARD_STAR}\index{STANDARD\ STAR}}


Reduce a standard star observation.


\mbox{}


This recipe reduces a standard star observation, assuming that it is
observed in a conventional nod-along-slit manner. It extracts the
spectrum of the standard, acquires details of the star (either from a
locally held list or from SIMBAD), and files it such that it can be
later used to create divided by standard and flux calibrated spectra
of targets that are observed.

%% ORACDRDOC_RECIPE:STANDARD_STAR_NOFLAT
\paragraph{STANDARD\_STAR\_NOFLAT\label{STANDARD_STAR_NOFLAT}\index{STANDARD\ STAR\ NOFLAT}}


Reduce a standard star without flat-fielding.


\mbox{}


This recipe reduces a standard star observation, assuming that it is
observed in a conventional nod-along-slit manner. It extracts the
spectrum of the standard, acquires details of the star (either from a
locally held list, or from SIMBAD), and files it such that it can be
later used to create divided by standard and flux calibrated spectra
of targets that are observed.



This recipe does not attempt to flat-field the frames.

%% ORACDRDOC_PRIMITIVE:_STANDARD_STAR_
\paragraph{\_STANDARD\_STAR\_\label{_STANDARD_STAR_}\index{\ STANDARD\ STAR\ }}


Wrapper primitive for standard star filing.


\mbox{}


Looks up standard star parameters, uses these parameters to blackbody
correct the spectrum, then files the spectrum with the calibration system.


\subsection{Science target Recipes}

\subsubsection{POINT\_SOURCE}

The POINT\_SOURCE recipe reduces observations of point sources, with
nodding along the slit or off to sky (the reason you'd go off to sky
with a point source being that it's in a crowded field).


Variants: \\
\_NOFLAT - does not do flat fielding \\
\_NOSTD - does not ratio by standard star \\
\_NOFLAT\_NOSTD - does not flat field or ratio by standard star \\

%% ORACDRDOC_RECIPE:POINT_SOURCE
\paragraph{POINT\_SOURCE\label{POINT_SOURCE}\index{POINT\ SOURCE}}


For reducing point source observations.


\mbox{}


A spectroscopy recipe for observations of Point Sources
Assumes that the data are taken in a pair-wise manner.



This recipe is suitable for all point source data taken in a pair-wise
manner, including nodding along the slit, nodding to sky, and
chop-mode observations.



The final product of this recipe is an spectrum of the target, with an
approximate wavelength and flux scale applied. Processing includes
optimal extraction and division by a standard star.



For details of the data reduction process used, see the documentation
for the individual primitives.

%% ORACDRDOC_RECIPE:POINT_SOURCE_NOFLAT
\paragraph{POINT\_SOURCE\_NOFLAT\label{POINT_SOURCE_NOFLAT}\index{POINT\ SOURCE\ NOFLAT}}


POINT\_SOURCE but without flat-fielding.


\mbox{}


See the POINT\_SOURCE recipe documentation. This version is identical except
that it does not attempt to flat field the data.



If you acquire suitable flat fields later in the night, you should
reduce them first, then re-process your target data with the
POINT\_SOURCE recipe.



Note that to be suitable, such data must be taken before driving
of the spectrometer optics motors. You cannot change configurations
in between.

%% ORACDRDOC_RECIPE:POINT_SOURCE_NOSTD
\paragraph{POINT\_SOURCE\_NOSTD\label{POINT_SOURCE_NOSTD}\index{POINT\ SOURCE\ NOSTD}}


POINT\_SOURCE but without division by a standard star.


\mbox{}


See the POINT\_SOURCE recipe documentation. This version is identical
except that it does not use a standard star, and thus processing stops
after extraction of the spectrum from the group frame.



If you acquire suitable standard star observations later in the night,
you should reduce them first, then re-process your target data with
the POINT\_SOURCE recipe.



Note that to be suitable, such data must be taken before driving
of the spectrometer optics motors. You cannot change configurations
in between.

%% ORACDRDOC_RECIPE:POINT_SOURCE_NOFLAT_NOSTD
\paragraph{POINT\_SOURCE\_NOFLAT\_NOSTD\label{POINT_SOURCE_NOFLAT_NOSTD}\index{POINT\ SOURCE\ NOFLAT\ NOSTD}}


POINT\_SOURCE without flat-fielding or division
by a standard star.


\mbox{}


See the POINT\_SOURCE recipe documentation. This version is identical
except that it does not flat field the data or use a standard
star. Thus processing stops after extraction of the spectrum from the
group frame.



If you acquire suitable flat fields and standard star observations
later in the night, you should reduce them first, then re-process your
target data with the POINT\_SOURCE recipe.



Note that to be suitable, such data must be taken before driving
of the spectrometer optics motors. You cannot change configurations
in between.


\subsubsection{FAINT\_POINT\_SOURCE}

The FAINT\_POINT\_SOURCE recipe is similar to POINT\_SOURCE, except in
determining where to centre the opt-extract windows. With
POINT\_SOURCE, the windows are centred on peaks and troughs detected
in the y-profile of the sky subtracted group image. This is also the
case for STANDARD\_STAR. However, STANDARD\_STAR writes the locations
of the beams to the rows calibration system. FAINT\_POINT\_SOURCE gets
these values from the calibration system rather than the y-profile of
the image. 

Thus, FAINT\_POINT\_SOURCE is useful both for crowded fields or
faint targets, where it's not obvious from the group image where to
extract. For this to be useful, the object of interest must be centred
on the same rows as the standard star was. This will be the case unless
you deliberately change the telescope-instrument aperture, peak-up row
or offset distance between observing the standard and the target.

Variants: \\
\_NOFLAT - does not do flat fielding \\
\_NOSTD - does not ratio by standard star \\
\_NOFLAT\_NOSTD - does not flat field or ratio by standard star \\

%% ORACDRDOC_RECIPE:FAINT_POINT_SOURCE
\paragraph{FAINT\_POINT\_SOURCE\label{FAINT_POINT_SOURCE}\index{FAINT\ POINT\ SOURCE}}


For reducing faint point source observations.


\mbox{}


A spectroscopy recipe for observations of point sources.
Assumes that the data are taken in a pair-wise manner.



This recipe is suitable for all point source data taken in a pair-wise
manner, including nodding along the slit, nodding to sky, and
chop-mode observations.



The final product of this recipe is an spectrum of the target, with an
approximate wavelength and flux scale applied. Processing includes
optimal extraction and division by a standard star.



For details of the data reduction process used, see the documentation
for the individual primitives.



The optimal extraction windows used to extract spectra from the group
image are centred on the rows determined for the standard star.

%% ORACDRDOC_RECIPE:FAINT_POINT_SOURCE_NOFLAT
\paragraph{FAINT\_POINT\_SOURCE\_NOFLAT\label{FAINT_POINT_SOURCE_NOFLAT}\index{FAINT\ POINT\ SOURCE\ NOFLAT}}


FAINT\_POINT\_SOURCE but without flat-fielding.


\mbox{}


See the FAINT\_POINT\_SOURCE recipe documentation. This version is
identical except that it does not attempt to flat field the data.



If you acquire suitable flat fields later in the night, you should
reduce them first, then re-process your target data with the
FAINT\_POINT\_SOURCE recipe.



Note that to be suitable, such data must be taken before driving
of the spectrometer optics motors. You cannot change configurations
in between.

%% ORACDRDOC_RECIPE:FAINT_POINT_SOURCE_NOSTD
\paragraph{FAINT\_POINT\_SOURCE\_NOSTD\label{FAINT_POINT_SOURCE_NOSTD}\index{FAINT\ POINT\ SOURCE\ NOSTD}}


FAINT\_POINT\_SOURCE but without division by 
a standard star.


\mbox{}


See the FAINT\_POINT\_SOURCE recipe documentation. This version is identical
except that it does not use a standard star, and thus processing stops
after extraction of the spectrum from the group frame.



If you acquire suitable standard star observations later in the night,
you should reduce them first, then re-process your target data with
the FAINT\_POINT\_SOURCE recipe.



Note that to be suitable, such data must be taken before driving
of the spectrometer optics motors. You cannot change configurations
in between.

%% ORACDRDOC_RECIPE:FAINT_POINT_SOURCE_NOFLAT_NOSTD
\paragraph{FAINT\_POINT\_SOURCE\_NOFLAT\_NOSTD\label{FAINT_POINT_SOURCE_NOFLAT_NOSTD}\index{FAINT\ POINT\ SOURCE\ NOFLAT\ NOSTD}}


FAINT\_POINT\_SOURCE but without flat-fielding
or division by a standard star.


\mbox{}


See the FAINT\_POINT\_SOURCE recipe documentation. This version is
identical except that it does not flat field the data or use a
standard star, and thus processing stops after extraction of the
spectrum from the group frame.



If you acquire suitable flat field and standard star observations
later in the night, you should reduce them first, then re-process your
target data with the FAINT\_POINT\_SOURCE recipe.



Note that to be suitable, such data must be taken before driving
of the spectrometer optics motors. You cannot change configurations
in between.


\subsubsection{EXTENDED\_SOURCE}

The EXTENDED\_SOURCE recipe reduces observations of extended sources.

Variants: \\
\_NOFLAT - does not do flat fielding \\
\_NOSTD - does not ratio by standard star \\
\_NOFLAT\_NOSTD - does not flat field or ratio by standard star \\

%% ORACDRDOC_RECIPE:EXTENDED_SOURCE
\paragraph{EXTENDED\_SOURCE\label{EXTENDED_SOURCE}\index{EXTENDED\ SOURCE}}


For reducing extended source observations.


\mbox{}


A spectroscopy recipe for observations of extended sources.
Assumes that the data are taken in a pair-wise manner.



This recipe is suitable for all extended source data taken in a pair-wise
manner, including nodding along the slit, nodding to sky, and
chop-mode observations.



The final product of this recipe is a sky subtracted image spectrum of
the target, with an approximate wavelength and flux scale
applied. Processing includes division by a standard star.



For details of the data reduction process used, see the documentation
for the individual primitives.

%% ORACDRDOC_RECIPE:EXTENDED_SOURCE_NOFLAT
\paragraph{EXTENDED\_SOURCE\_NOFLAT\label{EXTENDED_SOURCE_NOFLAT}\index{EXTENDED\ SOURCE\ NOFLAT}}


EXTENDED\_SOURCE without flat-fielding.


\mbox{}


See the documentation for the EXTENDED\_SOURCE recipe. This version is
identical except that it does not attempt to flat-field the data.



If you acquire suitable flat fields later in the night, you should
reduce them first, then re-process your target data with the
EXTENDED\_SOURCE recipe.



Note that to be suitable, such data must be taken before driving
of the spectrometer optics motors. You cannot change configurations
in between.

%% ORACDRDOC_RECIPE:EXTENDED_SOURCE_NOSTD
\paragraph{EXTENDED\_SOURCE\_NOSTD\label{EXTENDED_SOURCE_NOSTD}\index{EXTENDED\ SOURCE\ NOSTD}}


EXTENDED\_SOURCE without division by a standard star.


\mbox{}


See the documentation for the EXTENDED\_SOURCE recipe. This version is
identical except that it does not attempt to use a standard star and
thus processing stops after forming the group image.



If you acquire suitable standard star observations later in the night,
you should reduce them first, then re-process your target data with
the EXTENDED\_SOURCE recipe.



Note that to be suitable, such data must be taken before driving
of the spectrometer optics motors. You cannot change configurations
in between.

%% ORACDRDOC_RECIPE:EXTENDED_SOURCE_NOFLAT_NOSTD
\paragraph{EXTENDED\_SOURCE\_NOFLAT\_NOSTD\label{EXTENDED_SOURCE_NOFLAT_NOSTD}\index{EXTENDED\ SOURCE\ NOFLAT\ NOSTD}}


EXTENDED\_SOURCE without flat fielding or
division by a standard star.


\mbox{}


See the documentation for the EXTENDED\_SOURCE recipe. This version is
identical except that it does not attempt to flat-field the data or
use a standard star and thus processing stops after forming the group
image.



If you acquire suitable flat-field and standard star observations later
in the night, you should reduce them first, then re-process your
target data with the EXTENDED\_SOURCE recipe.



Note that to be suitable, such data must be taken before driving
of the spectrometer optics motors. You cannot change configurations
in between.


\subsubsection{Extended source with a stable sky background}

The EXTENDED\_SOURCE\_SEPARATE\_SKY recipe is for use when the sky is
sufficiently stable, that you do not need to spend half your time
observing it. This is only expected to be of use with CGS4 echelle
observations. For point sources, you would nod along the slit anyway.

Variants: \\
\_NOFLAT - does not do flat fielding \\
\_NOSTD - does not ratio by standard star \\
\_NOFLAT\_NOSTD - does not flat field or ratio by standard star \\

%% ORACDRDOC_RECIPE:EXTENDED_SOURCE_WITH_SEPARATE_SKY
\paragraph{EXTENDED\_SOURCE\_WITH\_SEPARATE\_SKY\label{EXTENDED_SOURCE_WITH_SEPARATE_SKY}\index{EXTENDED\ SOURCE\ WITH\ SEPARATE\ SKY}}


For extended source on stable sky background.


\mbox{}


For use when you want to take several on-target OBJECT frames for each
off-target SKY frame. It is only sensible to use this recipe when
working with low, stable sky counts, and with extended targets where
you cannot nod along the slit. Currently, this can only really be
considered to apply to certain observations using the CGS4 echelle
grating.



Before the pipeline sees an OBJECT frame with this recipe, you must
have had it reduce a suitable SKY frame with the REDUCE\_SKY recipe,
otherwise the pipeline cannot continue. Thus you should use a sequence
like "SKY, OBJECT, OBJECT" rather than "OBJECT, SKY, OBJECT". In each
set of observations you can follow the SKY frame with as many OBJECT
frames as you like. There is obviously a trade-off in that the more
OBJECT frames you do, the more time you spend observing your target
rather than the sky, but also the longer the interval between SKY
frames, and thus the more liable you are to be affected by sky
variation.

%% ORACDRDOC_RECIPE:EXTENDED_SOURCE_WITH_SEPARATE_SKY_NOFLAT
\paragraph{EXTENDED\_SOURCE\_WITH\_SEPARATE\_SKY\_NOFLAT\label{EXTENDED_SOURCE_WITH_SEPARATE_SKY_NOFLAT}\index{EXTENDED\ SOURCE\ WITH\ SEPARATE\ SKY\ NOFLAT}}


EXTENDED\_SOURCE\_WITH\_SEPARATE\_SKY without flat-fielding.


\mbox{}


See the documentation for the EXTENDED\_SOURCE\_WITH\_SEPARATE\_SKY
recipe. This version is identical except that it does not attempt to
flat-field the data.



If you acquire suitable flat fields later in the night, you should
reduce them first, then re-process your target data with the
EXTENDED\_SOURCE\_WITH\_SEPARATE\_SKY recipe.



Note that to be suitable, such data must be taken before driving
of the spectrometer optics motors. You cannot change configurations
in between.

%% ORACDRDOC_RECIPE:EXTENDED_SOURCE_WITH_SEPARATE_SKY_NOSTD
\paragraph{EXTENDED\_SOURCE\_WITH\_SEPARATE\_SKY\_NOSTD\label{EXTENDED_SOURCE_WITH_SEPARATE_SKY_NOSTD}\index{EXTENDED\ SOURCE\ WITH\ SEPARATE\ SKY\ NOSTD}}


EXTENDED\_SOURCE\_WITH\_SEPARATE\_SKY without
division by a standard star.


\mbox{}


See the documentation for the EXTENDED\_SOURCE\_WITH\_SEPARATE\_SKY
recipe. This version is identical except that it does not attempt to
use a standard star, and thus processing stops after forming the group
image.



If you acquire suitable standard star observations later in the night,
you should reduce them first, then re-process your target data with
the EXTENDED\_SOURCE\_WITH\_SEPARATE\_SKY recipe.



Note that to be suitable, such data must be taken before driving
of the spectrometer optics motors. You cannot change configurations
in between.

%% ORACDRDOC_RECIPE:EXTENDED_SOURCE_WITH_SEPARATE_SKY_NOFLAT_NOSTD
\paragraph{EXTENDED\_SOURCE\_WITH\_SEPARATE\_SKY\_NOFLAT\_NOSTD\label{EXTENDED_SOURCE_WITH_SEPARATE_SKY_NOFLAT_NOSTD}\index{EXTENDED\ SOURCE\ WITH\ SEPARATE\ SKY\ NOFLAT\ NOSTD}}


EXTENDED\_SOURCE\_WITH\_SEPARATE\_SKY without
flat-fielding or without division by a standard star.


\mbox{}


See the documentation for the EXTENDED\_SOURCE\_WITH\_SEPARATE\_SKY
recipe. This version is identical except that it does not attempt to
flat field the data or use a standard star, and thus processing stops
after forming the group image.



If you acquire suitable flat fields and standard star observations
later in the night, you should reduce them first, then re-process your
target data with the EXTENDED\_SOURCE\_WITH\_SEPARATE\_SKY recipe.



Note that to be suitable, such data must be taken before driving
of the spectrometer optics motors. You cannot change configurations
in between.


\subsubsection{Blank Sky Observations}

%% ORACDRDOC_RECIPE:REDUCE_SKY
\paragraph{REDUCE\_SKY\label{REDUCE_SKY}\index{REDUCE\ SKY}}


Reduces a sky frame.


\mbox{}


Reduces a blank sky observation, and files it with the calibration
system for use in subsequent data reduction. Note that in pair-wise
observing procedures, you should generally not be taking SKY frames -
both beam positions are classified as OBJECT frames, and should be
handled by whatever pair-wise recipe is being used for the main-beam
frames.


\subsection{Utility Recipes}

\subsubsection{Night Logs}

The NIGHT\_LOG recipe is the default recipe for generating summary
logs of a list of observations. The NIGHT\_LOG\_LONG variant adds more
details to the log file.

%% ORACDRDOC_RECIPE:NIGHT_LOG
\paragraph{NIGHT\_LOG\label{NIGHT_LOG}\index{NIGHT\ LOG}}


Creates a text log summarising file headers.


\mbox{}


This recipe is used to create a text log summarising the file headers
of a group of observations. It is often used to create a log file
describing a whole night's worth of observations.



For full details, see the documentation for the spectroscopy
\_NIGHT\_LOG\_ primitive.



This recipe calls the primitive in such a way that the log file
appears in \$ORAC\_DATA\_IN.



An "on-the-fly" night log is created in \$ORAC\_DATA\_OUT as 
spectroscopy data is reduced by the pipeline. This is done by a
call to \_NIGHT\_LOG\_ from \_SPECTROSCOPY\_HELLO\_.

%% ORACDRDOC_RECIPE:NIGHT_LOG_LONG
\paragraph{NIGHT\_LOG\_LONG\label{NIGHT_LOG_LONG}\index{NIGHT\ LOG\ LONG}}


Creates a text log detailing file headers.


\mbox{}


This recipe does the same as the NIGHT\_LOG recipe, except that it
produces a more detailed log.

%% ORACDRDOC_PRIMITIVE:_NIGHT_LOG_
\paragraph{\_NIGHT\_LOG\_\label{_NIGHT_LOG_}\index{\ NIGHT\ LOG\ }}


Produces a text listing a summary of a frame's headers.


\mbox{}


Produces a line of text in a log file summarising the header values of
the frame. Is used both as part of the generall data reduction, so as
to produce an on-the-fly listing of what has been reduced so far, and
as part of the NIGHT\_LOG recipe to provide a summary of a set (usually
the whole night's worth) of observations.


\subsection{Molecular Primitives}

\subsubsection{To Reduce a Single Observation to a \_wce frame}

%% ORACDRDOC_PRIMITIVE:_REDUCE_SINGLE_FRAME_
\paragraph{\_REDUCE\_SINGLE\_FRAME\_\label{_REDUCE_SINGLE_FRAME_}\index{\ REDUCE\ SINGLE\ FRAME\ }}


Reduces a spectroscopy frame.


\mbox{}


Intended to be run on all OBJECT and SKY science data, and also things
like ARC frames aswell.



Contains all the steps necessary to get from raw data to a \_wce file.
Variance should be propogated throughout.



This should be the first major primitive in any recipe handling on-sky
data.


\subsubsection{Pairwise Grouping}

%% ORACDRDOC_PRIMITIVE:_PAIRWISE_GROUP_
\paragraph{\_PAIRWISE\_GROUP\_\label{_PAIRWISE_GROUP_}\index{\ PAIRWISE\ GROUP\ }}


Create a group file from reduced single frames taken
in a pairwise sequence.


\mbox{}


Takes reduced single frames taken in a pairwise sequence, and groups them
to make a group file. Extracts spectra.


\subsubsection{Extracting and Coadding Spectra}

%% ORACDRDOC_PRIMITIVE:_EXTRACT_SPECTRA_
\paragraph{\_EXTRACT\_SPECTRA\_\label{_EXTRACT_SPECTRA_}\index{\ EXTRACT\ SPECTRA\ }}


Extracts sepctra from an image.


\mbox{}


Extracts spectra from an image. This primitive only runs if a pair
has been completed.

%% ORACDRDOC_PRIMITIVE:_EXTRACT_DETERMINE_NBEAMS_
\paragraph{\_EXTRACT\_DETERMINE\_NBEAMS\_\label{_EXTRACT_DETERMINE_NBEAMS_}\index{\ EXTRACT\ DETERMINE\ NBEAMS\ }}


Determine the number of beams to extract.


\mbox{}


Looks at the chop and offset headers to determine the number of beams
there should be in the group image. Leaves the result in the NBEAMS
group user header.

%% ORACDRDOC_PRIMITIVE:_EXTRACT_FIND_ROWS_
\paragraph{\_EXTRACT\_FIND\_ROWS\_\label{_EXTRACT_FIND_ROWS_}\index{\ EXTRACT\ FIND\ ROWS\ }}


Find spectra rows.


\mbox{}


Finds the rows in a group image at which to centre the spectra 
extraction windows.



These are stored in a user-header called BEAMS, which is a reference
to an array of references to hashes, each hash having keys POS and
MULT - the beam position and multiplier



When determining the location of the rows on which the spectra fall,
a y-profile spectrum is created in a file ending with \_ypr.

%% ORACDRDOC_PRIMITIVE:_EXTRACT_ALL_BEAMS_
\paragraph{\_EXTRACT\_ALL\_BEAMS\_\label{_EXTRACT_ALL_BEAMS_}\index{\ EXTRACT\ ALL\ BEAMS\ }}


Optimally extracts all beams in a group file.


\mbox{}


Optimally extracts all the beams in a group file. The primitive defines
the optimal extraction profile window to be as wide as half the separation
between two beams (if two beams exist), or 50 pixels (if one beam exists).



In obtaining the optimal extraction profile, this primitive temporarily
fills bad pixels in the input file. It then uses this profile to optimally
extract the spectra for the image. After it has done so, it optionally 
files the profile with the calibration system for future use.



This primitive may also optionally use a predetermined optimal extraction
profile obtained from a standard star observation.



As output this primitive creates a file ending in \_oep for the optimal
extraction profile, \_oer for the residuals from the profile fitting, and
\_oes for the optimally extracted spectrum.

%% ORACDRDOC_PRIMITIVE:_DERIPPLE_ALL_BEAMS_
\paragraph{\_DERIPPLE\_ALL\_BEAMS\_\label{_DERIPPLE_ALL_BEAMS_}\index{\ DERIPPLE\ ALL\ BEAMS\ }}


Deripple interleaved observations.


\mbox{}


This primitive deripples interleaved observations by creating a ripple
flat, then dividing this ripple flat into the observation. If the amplitude
of the ripple is greater than 70\%, then no derippling is done.



The resulting spectrum is created in a file with a \_dri suffix. This
file is created even if no derippling is performed because the amplitude
is greater than 70\% -- the spectrum is copied directly from source.



The generated flat-field is created in a file with a \_rif suffix.



If there is only one observation used in interleaving (i.e. no interleaving
is done) then no derippling is performed, and the original spectrum
is propagated through. If this is the case, then no \_dri file is created.

%% ORACDRDOC_PRIMITIVE:_CROSS_CORR_ALL_BEAMS_
\paragraph{\_CROSS\_CORR\_ALL\_BEAMS\label{_CROSS_CORR_ALL_BEAMS}\index{\ CROSS\ CORR\ ALL\ BEAMS}}


Cross correlates and shifts the extracted beams.


\mbox{}


Takes the extracted beams from \_EXTRACT\_ALL\_BEAMS\_ and cross
correlates each beam with the first one, then shifts each beam, so
that they're all shift-aligned with the first beam.



The resulting spectra are created in an HDS container with a filename
ending in \_ccs, and the cross correlation functions are stored in an
HDS container with a filename ending in \_ccf.



If the maximum value of the cross correlation function is less than
0.6, or if the shift is greater than 2 pixels, then the spectra are
not aligned and shifted.

%% ORACDRDOC_PRIMITIVE:_COADD_EXTRACTED_BEAMS_
\paragraph{\_COADD\_EXTRACTED\_BEAMS\_\label{_COADD_EXTRACTED_BEAMS_}\index{\ COADD\ EXTRACTED\ BEAMS\ }}


Coadds the beams which were previously extracted.


\mbox{}


Adds together the beams in the group file. Normally, these will have been aligned
using \_CROSS\_CORR\_ALL\_BEAMS\_ first.


\subsubsection{Using Standard Stars}

%% ORACDRDOC_PRIMITIVE:_DIVIDE_BY_STANDARD_
\paragraph{\_DIVIDE\_BY\_STANDARD\_\label{_DIVIDE_BY_STANDARD_}\index{\ DIVIDE\ BY\ STANDARD\ }}


Divides a spectrum or an array by a suitable standard.


\mbox{}


Asks the calibration system for a suitable standard star, and divides by it.
This primitive works for either 1D or 2D data.



This primitive outputs a file with a \_dbs suffix for 1D data, or a \_dbsi
suffix for 2D data.

%% ORACDRDOC_PRIMITIVE:_ALIGN_SPECTRUM_TO_STD_
\paragraph{\_ALIGN\_SPECTRUM\_TO\_STD\_\label{_ALIGN_SPECTRUM_TO_STD_}\index{\ ALIGN\ SPECTRUM\ TO\ STD\ }}


Cross correlate and shift before divide by
standard star.


\mbox{}


Cross correlates the spectrum with the standard star spectrum and shifts
it, so as to get better atmospheric cancellation if the instrument has
flexed between the standard and the target observations.



Takes a STANDARD parameter telling it the name of the standard to use.



Doesn't do anything unless the group NDIMS is 1

%% ORACDRDOC_PRIMITIVE:_FLUX_CALIBRATE_
\paragraph{\_FLUX\_CALIBRATE\_\label{_FLUX_CALIBRATE_}\index{\ FLUX\ CALIBRATE\ }}


Flux calibrate a spectrum.


\mbox{}


Flux calibrate a spectrum that has been divided by a standard star by
multiplying by an appropriate scaling factor. This scaling factor depends
on the magnitude and spectral type of the standard star.



This primitive works on either 1D or 2D observations. If a 1D observation
is flux calibrated, the resulting file ends in \_fc. If a 2D observation is
flux calibrated, the resulting file ends in \_fci.


\subsection{Historical Recipe Names}

Historical data taken with the main historical recipes, SOURCE\-\_PAIRS\-\_ON\-\_SLIT
and SOURCE\-\_PAIRS\-\_TO\-\_SKY should now be reduced with POINT\-\_SOURCE or
EXTENDED\_SOURCE as appropriate. Use POINT\_SOURCE if in doubt.

Historical data taken with SOURCE\_WITH\_NOD\_TO\_BLANK\_SKY should now
be reduced with EXTENDED\_SOURCE\_WITH\_SEPARATE\_SKY.

\section{Calibration Information}

\ORACDR\ records calibration information, such as dark frames, flat
fields, and the read noise, within index files, one for each type of
calibration information.  When the pipeline needs a calibration frame
it searches the index file for the best matching entry subject to a
set of rules. Each recipe reports the calibrations it has used.  If no
suitable calibration exists, the pipeline exits with an error message
stating this fact.  For further details see
\xref{SUN/230}{sun230}{calibration_selection}.

You can also select a specific calibration using the {\tt -calib}
command-line option, provided the chosen calibration has an entry
in the appropriate index file.  See
\begin{latexonly}
the section on
\end{latexonly}
\xref{calibration options}{sun230}{calibration_options}
\begin{latexonly}
in SUN/230
\end{latexonly}
for details and examples.

\subsection{Available Calibration Methods}

The following calibration methods are available for CGS4:

\begin{itemize}

\item baseshift - Use the given comma separated doublet (i.e. ``0,0'') as the
frame's base position.

\item bias - Use the given bias frame.

\item dark - Use the given dark frame.

\item flat - Use the given flat frame.

\item mask - Use the given mask. Usually used for bad pixel masks.

\item profile - Use the given frame as an extraction profile.

\item readnoise - Use the given value for the detector readnoise.

\item rotation - Use the given frame as a rotation matrix.

\item rowname - Use the given frame to calculate the positions of the positive
and negative rows.

\item sky - Use the given sky frame.

\item standard - Use the given standard star frame.

\end{itemize}

% ? End of main text
\end{document}
