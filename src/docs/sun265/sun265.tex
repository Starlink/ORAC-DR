\documentclass[twoside,11pt]{article}

% ? Specify used packages
\usepackage{graphicx}        %  Use this one for final production.
% \usepackage[draft]{graphicx} %  Use this one for drafting.
% ? End of specify used packages

\pagestyle{myheadings}

% -----------------------------------------------------------------------------
% ? Document identification
% Fixed part
\newcommand{\stardoccategory}  {Starlink User Note}
\newcommand{\stardocinitials}  {SUN}
\newcommand{\stardocsource}    {sun\stardocnumber}
\newcommand{\stardoccopyright}
{Copyright \copyright\ 2013 University of British Columbia and the Science \& Technology Facilities Council}

% Variable part - replace [xxx] as appropriate.
\newcommand{\stardocnumber}    {265.0}
\newcommand{\stardocauthors}   {A.G. Gibb, T. Jenness, F. Economou}
\newcommand{\stardocdate}      {1 Mar 2013}
\newcommand{\stardoctitle}     {PICARD --- a PIpeline for Combining and Analyzing Reduced Data}
\newcommand{\stardocversion}   {Version 0.3.0}
\newcommand{\stardocmanual}    {User's Guide}
\newcommand{\stardocabstract}  {

  \picard\ is a facility for combining and analyzing reduced data,
  normally the output from the \oracdr\ data reduction pipeline. This
  document describes an introduction to using \picard\ for processing
  instrument-independent data.

}
% ? End of document identification
% -----------------------------------------------------------------------------

% +
%  Name:
%     sun265.tex
%
%  Purpose:
%     Documentation for PICARD
%
%  Authors:
%     AGG: Andy Gibb (UBC)
%
%  History:
%     2011-01-20 (AGG):
%        Initial version
%     2012-07-30 (AGG):
%        Updates for Kapuahi
%     2013-03-01 (AGG):
%        Updates for Hikianalia
%     {Add further history here}
%
% -

\newcommand{\stardocname}{\stardocinitials /\stardocnumber}
\markboth{\stardocname}{\stardocname}
\setlength{\textwidth}{160mm}
\setlength{\textheight}{230mm}
\setlength{\topmargin}{-2mm}
\setlength{\oddsidemargin}{0mm}
\setlength{\evensidemargin}{0mm}
\setlength{\parindent}{0mm}
\setlength{\parskip}{\medskipamount}
\setlength{\unitlength}{1mm}

% -----------------------------------------------------------------------------
%  Hypertext definitions.
%  ======================
%  These are used by the LaTeX2HTML translator in conjunction with star2html.

%  Comment.sty: version 2.0, 19 June 1992
%  Selectively in/exclude pieces of text.
%
%  Author
%    Victor Eijkhout                                      <eijkhout@cs.utk.edu>
%    Department of Computer Science
%    University Tennessee at Knoxville
%    104 Ayres Hall
%    Knoxville, TN 37996
%    USA

%  Do not remove the %begin{latexonly} and %end{latexonly} lines (used by
%  LaTeX2HTML to signify text it shouldn't process).
%begin{latexonly}
\makeatletter
\def\makeinnocent#1{\catcode`#1=12 }
\def\csarg#1#2{\expandafter#1\csname#2\endcsname}

\def\ThrowAwayComment#1{\begingroup
    \def\CurrentComment{#1}%
    \let\do\makeinnocent \dospecials
    \makeinnocent\^^L% and whatever other special cases
    \endlinechar`\^^M \catcode`\^^M=12 \xComment}
{\catcode`\^^M=12 \endlinechar=-1 %
 \gdef\xComment#1^^M{\def\test{#1}
      \csarg\ifx{PlainEnd\CurrentComment Test}\test
          \let\html@next\endgroup
      \else \csarg\ifx{LaLaEnd\CurrentComment Test}\test
            \edef\html@next{\endgroup\noexpand\end{\CurrentComment}}
      \else \let\html@next\xComment
      \fi \fi \html@next}
}
\makeatother

\def\includecomment
 #1{\expandafter\def\csname#1\endcsname{}%
    \expandafter\def\csname end#1\endcsname{}}
\def\excludecomment
 #1{\expandafter\def\csname#1\endcsname{\ThrowAwayComment{#1}}%
    {\escapechar=-1\relax
     \csarg\xdef{PlainEnd#1Test}{\string\\end#1}%
     \csarg\xdef{LaLaEnd#1Test}{\string\\end\string\{#1\string\}}%
    }}

%  Define environments that ignore their contents.
\excludecomment{comment}
\excludecomment{rawhtml}
\excludecomment{htmlonly}

%  Hypertext commands etc. This is a condensed version of the html.sty
%  file supplied with LaTeX2HTML by: Nikos Drakos <nikos@cbl.leeds.ac.uk> &
%  Jelle van Zeijl <jvzeijl@isou17.estec.esa.nl>. The LaTeX2HTML documentation
%  should be consulted about all commands (and the environments defined above)
%  except \xref and \xlabel which are Starlink specific.

\newcommand{\htmladdnormallinkfoot}[2]{#1\footnote{#2}}
\newcommand{\htmladdnormallink}[2]{#1}
\newcommand{\htmladdimg}[1]{}
\newcommand{\hyperref}[4]{#2\ref{#4}#3}
\newcommand{\htmlref}[2]{#1}
\newcommand{\htmlimage}[1]{}
\newcommand{\htmladdtonavigation}[1]{}

\newenvironment{latexonly}{}{}
\newcommand{\latex}[1]{#1}
\newcommand{\html}[1]{}
\newcommand{\latexhtml}[2]{#1}
\newcommand{\HTMLcode}[2][]{}

%  Starlink cross-references and labels.
\newcommand{\xref}[3]{#1}
\newcommand{\xlabel}[1]{}

%  LaTeX2HTML symbol.
\newcommand{\latextohtml}{\LaTeX2\texttt{HTML}}

%  Define command to re-centre underscore for Latex and leave as normal
%  for HTML (severe problems with \_ in tabbing environments and \_\_
%  generally otherwise).
\renewcommand{\_}{\texttt{\symbol{95}}}

% -----------------------------------------------------------------------------
%  Debugging.
%  =========
%  Remove % on the following to debug links in the HTML version using Latex.

% \newcommand{\hotlink}[2]{\fbox{\begin{tabular}[t]{@{}c@{}}#1\\\hline{\footnotesize #2}\end{tabular}}}
% \renewcommand{\htmladdnormallinkfoot}[2]{\hotlink{#1}{#2}}
% \renewcommand{\htmladdnormallink}[2]{\hotlink{#1}{#2}}
% \renewcommand{\hyperref}[4]{\hotlink{#1}{\S\ref{#4}}}
% \renewcommand{\htmlref}[2]{\hotlink{#1}{\S\ref{#2}}}
% \renewcommand{\xref}[3]{\hotlink{#1}{#2 -- #3}}
%end{latexonly}
% -----------------------------------------------------------------------------
% ? Document specific \newcommand or \newenvironment commands.

% A new environment for quoting verbatim
% Environment for indenting and using a small font.
\newenvironment{myquote}{\begin{quote}\begin{small}}{\end{small}\end{quote}}

\newcommand{\starlink}{\htmladdnormallink{Starlink}{http://starlink.jach.hawaii.edu}}

% Shorthand and HTML references for other Starlink tasks
\newcommand{\CCDPACK}{\textsc{ccdpack}}
\newcommand{\CCDPACKref}{\xref{\CCDPACK}{sun139}{}}
\newcommand{\CUPID}{\textsc{cupid}}
\newcommand{\CUPIDref}{\xref{\CUPID}{sun255}{}}
\newcommand{\FLUXES}{\textsc{fluxes}}
\newcommand{\FLUXESref}{\xref{\FLUXES}{sun213}{}}
\newcommand{\GAIA}{\textsc{gaia}}
\newcommand{\GAIAref}{\xref{\GAIA}{sun214}{}}
\newcommand{\HDSTRACE}{\textsc{hdstrace}}
\newcommand{\HDSTRACEref}{\xref{\HDSTRACE}{sun102}{}}
\newcommand{\KAPPA}{\textsc{kappa}}
\newcommand{\CURSA}{\xref{\textsc{cursa}}{sun190}{}}
\newcommand{\KAPPAref}{\xref{(SUN/95)}{sun95}{}}
\newcommand{\SMURF}{\textsc{smurf}}
\newcommand{\SMURFcook}{\xref{SC/19}{sc19}{}}
\newcommand{\SMURFsun}{\xref{SUN/258}{sun258}{}}
\newcommand{\ADAMsgref}{\xref{SG/4}{sg4}{}}
\newcommand{\ADAMsunref}{\xref{SUN/101}{sun101}{}}
\newcommand{\astref}{\xref{SUN/211}{sun211}{}}
\newcommand{\ndfref}{\xref{SUN/33}{sun33}{}}

\newcommand{\oracdr}{\textsc{orac-dr}}
\newcommand{\oracsun}{\xref{SUN/230}{sun230}{}}
\newcommand{\oracprogsun}{\xref{SUN/233}{sun233}{}}
\newcommand{\scubasun}{\xref{SUN/231}{sun231}{}}
\newcommand{\scubaiisun}{\xref{SUN/264}{sun264}{}}
\newcommand{\picard}{\textsc{picard}}

% Application tasks
\newcommand{\task}[1]{\textsf{#1}}

% SMURF tasks
\newcommand{\badbolos}{\xref{\task{badbolos}}{sun258}{BADBOLOS}}
\newcommand{\calcdark}{\xref{\task{calcdark}}{sun258}{CALCDARK}}
\newcommand{\calcflat}{\xref{\task{calcflat}}{sun258}{CALCFLAT}}
\newcommand{\calcnoise}{\xref{\task{calcnoise}}{sun258}{CALCNOISE}}
\newcommand{\calcresp}{\xref{\task{calcresp}}{sun258}{CALCRESP}}
\newcommand{\copyflat}{\xref{\task{copyflat}}{sun258}{COPYFLAT}}
\newcommand{\dreamsolve}{\xref{\task{dreamsolve}}{sun258}{DREAMSOLVE}}
\newcommand{\dreamweights}{\xref{\task{dreamweights}}{sun258}{DREAMWEIGHTS}}
\newcommand{\gsdtoacsis}{\xref{\task{gsd2acsis}}{sun258}{GSD2ACSIS}}
\newcommand{\gsdshow}{\xref{\task{gsdshow}}{sun258}{GSDSHOW}}
\newcommand{\smurfhelp}{\xref{\task{smurfhelp}}{sun258}{SMURFHELP}}
\newcommand{\impaztec}{\xref{\task{impaztec}}{sun258}{IMPAZTEC}}
\newcommand{\makecube}{\xref{\task{makecube}}{sun258}{MAKECUBE}}
\newcommand{\qlmakemap}{\xref{\task{qlmakemap}}{sun258}{QLMAKEMAP}}
\newcommand{\rawunpress}{\xref{\task{rawunpress}}{sun258}{RAWUNPRESS}}
\newcommand{\rawfixmeta}{\xref{\task{rawfixmeta}}{sun258}{RAWFIXMETA}}
\newcommand{\sctwosim}{\xref{\task{sc2sim}}{sun258}{SC2SIM}}
\newcommand{\sctwothreadtest}{\xref{\task{sc2threadtest}}{sun258}{SC2THREADTEST}}
\newcommand{\scanfit}{\xref{\task{scanfit}}{sun258}{SCANFIT}}
\newcommand{\skynoise}{\xref{\task{skynoise}}{sun258}{SKYNOISE}}
\newcommand{\smurfcopy}{\xref{\task{smurfcopy}}{sun258}{SMURFCOPY}}
\newcommand{\stackframes}{\xref{\task{stackframes}}{sun258}{STACKFRAMES}}
\newcommand{\starecalc}{\xref{\task{starecalc}}{sun258}{STARECALC}}
\newcommand{\timesort}{\xref{\task{timesort}}{sun258}{TIMESORT}}
\newcommand{\unmakecube}{\xref{\task{unmakecube}}{sun258}{UNMAKECUBE}}

\newcommand{\extinction}{\xref{\task{extinction}}{sun258}{EXTINCTION}}
\newcommand{\flatfield}{\xref{\task{flatfield}}{sun258}{FLATFIELD}}
\newcommand{\jcmtstate}{\xref{\task{jcmtstate2cat}}{sun258}{JCMTSTATE2CAT}}
\newcommand{\dumpocscfg}{\xref{\task{dumpocscfg}}{sun258}{DUMPOCSCFG}}
\newcommand{\makemap}{\xref{\task{makemap}}{sun258}{MAKEMAP}}
\newcommand{\gettsys}{\xref{\task{gettsys}}{sun258}{GETTSYS}}

\newcommand{\remsky}{\xref{\task{remsky}}{sun258}{REMSKY}}
\newcommand{\clean}{\xref{\task{sc2clean}}{sun258}{SC2CLEAN}}
\newcommand{\concat}{\xref{\task{sc2concat}}{sun258}{SC2CONCAT}}
\newcommand{\fft}{\xref{\task{sc2fft}}{sun258}{SC2FFT}}
\newcommand{\fts}{\xref{\task{sc2fts}}{sun258}{SC2FTS}}

\newcommand{\rebin}{\texttt{rebin}}
\newcommand{\iterate}{\texttt{iterate}}

% Other tasks
\newcommand{\makemos}{\xref{\task{makemos}}{sun139}{MAKEMOS}}
\newcommand{\csub}{\xref{\task{csub}}{sun95}{CSUB}}
\newcommand{\clinplot}{\xref{\task{clinplot}}{sun95}{CLINPLOT}}
\newcommand{\mlinplot}{\xref{\task{mlinplot}}{sun95}{MLINPLOT}}
\newcommand{\collapse}{\xref{\task{collapse}}{sun95}{COLLAPSE}}
\newcommand{\fitsedit}{\xref{\task{fitsedit}}{sun95}{FITSEDIT}}
\newcommand{\kapdiv}{\xref{\task{div}}{sun95}{DIV}}
\newcommand{\ndfcopy}{\xref{\task{ndfcopy}}{sun95}{NDFCOPY}}
\newcommand{\provshow}{\xref{\task{provshow}}{sun95}{PROVSHOW}}
\newcommand{\thresh}{\xref{\task{thresh}}{sun95}{THRESH}}
\newcommand{\wcsmosaic}{\xref{\task{wcsmosaic}}{sun95}{WCSMOSAIC}}
\newcommand{\wcsalign}{\xref{\task{wcsalign}}{sun95}{WCSALIGN}}
\newcommand{\wcsattrib}{\xref{\task{wcsattrib}}{sun95}{WCSATTRIB}}
\newcommand{\fitslist}{\xref{\task{fitslist}}{sun95}{FITSLIST}}
\newcommand{\display}{\xref{\task{display}}{sun95}{DISPLAY}}
\newcommand{\topcat}{\xref{\textsc{topcat}}{sun253}{}}


% macros for typesetting parameters
\newcommand{\aparam}[1]{\texttt{#1}}     % ADAM parameter
\newcommand{\cparam}[1]{\texttt{#1}}     % CONFIG parameter
\newcommand{\ndfcomp}[1]{\texttt{#1}}    % NDF component

%% Definitions imported from SUN/95

% A kind of list item, like description, but with an easily adjustable
% item separation.  Note that the paragraph and fount-size change are
% needed to make the revised \baselinestretch work.
\newlength{\menuwidth}
\newlength{\menuindent}
\newcommand{\menuitem}[2]
  {{\bf #1} \settowidth{\menuwidth}{{\bf #1} }
  \setlength{\menuindent}{-0.5em}
  \addtolength{\menuwidth}{-2\menuwidth}
  \addtolength{\menuwidth}{\textwidth}
  \addtolength{\menuwidth}{\menuindent}
  \hspace{\menuindent}\parbox[t]{\menuwidth}{
  \renewcommand{\baselinestretch}{0.75}\small
  #2 \par \vspace{1.0ex}
  \renewcommand{\baselinestretch}{1.0}\normalsize} \\ }
\begin{htmlonly}
\newcommand{\menuitem}[2]
  {\item [\htmlref{#1}{#1}] #2}
\end{htmlonly}

\newcommand{\classitem}[1]{\item [\htmlref{#1}{#1}]}

% an environment for references (for the SST sstdiytopic command).
\newenvironment{refs}{\vspace{-4ex} % normally 3ex
                      \begin{list}{}{\setlength{\topsep}{0mm}
                                     \setlength{\partopsep}{0mm}
                                     \setlength{\itemsep}{0mm}
                                     \setlength{\parsep}{0mm}
                                     \setlength{\leftmargin}{1.5em}
                                     \setlength{\itemindent}{-\leftmargin}
                                     \setlength{\labelsep}{0mm}
                                     \setlength{\labelwidth}{0mm}}
                    }{\end{list}}

%+
%  Name:
%     SST.TEX

%  Purpose:
%     Define LaTeX commands for laying out Starlink routine descriptions.

%  Language:
%     LaTeX

%  Type of Module:
%     LaTeX data file.

%  Description:
%     This file defines LaTeX commands which allow routine documentation
%     produced by the SST application PROLAT to be processed by LaTeX and
%     by LaTeX2html. The contents of this file should be included in the
%     source prior to any statements that make of the sst commnds.

%  Notes:
%     The style file html.sty provided with LaTeX2html needs to be used.
%     This must be before this file.

%  Authors:
%     RFWS: R.F. Warren-Smith (STARLINK)
%     PDRAPER: P.W. Draper (Starlink - Durham University)

%  History:
%     10-SEP-1990 (RFWS):
%        Original version.
%     10-SEP-1990 (RFWS):
%        Added the implementation status section.
%     12-SEP-1990 (RFWS):
%        Added support for the usage section and adjusted various spacings.
%     8-DEC-1994 (PDRAPER):
%        Added support for simplified formatting using LaTeX2html.
%     21-JUL-2009 (TIMJ):
%        Added \sstdiylist{}{} as used when a Parameters section is located that
%        is not "ADAM Parameters".
%     {enter_further_changes_here}

%  Bugs:
%     {note_any_bugs_here}

%-

%  Define length variables.
\newlength{\sstbannerlength}
\newlength{\sstcaptionlength}
\newlength{\sstexampleslength}
\newlength{\sstexampleswidth}

%  Define a \tt font of the required size.
\latex{\newfont{\ssttt}{cmtt10 scaled 1095}}
\html{\newcommand{\ssttt}{\tt}}

%  Define a command to produce a routine header, including its name,
%  a purpose description and the rest of the routine's documentation.
\newcommand{\sstroutine}[3]{
   \goodbreak
   \rule{\textwidth}{0.5mm}
   \vspace{-7ex}
   \newline
   \settowidth{\sstbannerlength}{{\Large {\bf #1}}}
   \setlength{\sstcaptionlength}{\textwidth}
   \setlength{\sstexampleslength}{\textwidth}
   \addtolength{\sstbannerlength}{0.5em}
% Modifications to only print title once: see SUN232 for info
   \addtolength{\sstcaptionlength}{-1.0\sstbannerlength}
   \addtolength{\sstcaptionlength}{-2.5pt}
   \settowidth{\sstexampleswidth}{{\bf Examples:}}
   \addtolength{\sstexampleslength}{-\sstexampleswidth}
   \parbox[t]{\sstbannerlength}{\flushleft{\Large {\bf #1}}}
   \parbox[t]{\sstcaptionlength}{\center{\Large #2}}
%   \parbox[t]{\sstbannerlength}{\flushright{\Large {\bf #1}}}
   \begin{description}
      #3
   \end{description}
}

%  Format the description section.
\newcommand{\sstdescription}[1]{\item[Description:] #1}

%  Format the usage section.
\newcommand{\sstusage}[1]{\item[Usage:] \mbox{}
\\[1.3ex]{\raggedright \ssttt #1}}

%  Format the invocation section.
\newcommand{\sstinvocation}[1]{\item[Invocation:]\hspace{0.4em}{\tt #1}}

%  Format the arguments section.
\newcommand{\sstarguments}[1]{
   \item[Arguments:] \mbox{} \\
   \vspace{-3.5ex}
   \begin{description}
      #1
   \end{description}
}

%  Format the returned value section (for a function).
\newcommand{\sstreturnedvalue}[1]{
   \item[Returned Value:] \mbox{} \\
   \vspace{-3.5ex}
   \begin{description}
      #1
   \end{description}
}

%  Format the parameters section (for an application).
\newcommand{\sstparameters}[1]{
   \item[Parameters:] \mbox{} \\
   \vspace{-3.5ex}
   \begin{description}
      #1
   \end{description}
}

%  Format the examples section.
\newcommand{\sstexamples}[1]{
   \item[Examples:] \mbox{} \\
   \vspace{-3.5ex}
   \begin{description}
      #1
   \end{description}
}

%  Define the format of a subsection in a normal section.
\newcommand{\sstsubsection}[1]{ \item[{#1}] \mbox{} \\}

%  Define the format of a subsection in the examples section.
\newcommand{\sstexamplesubsection}[2]{\sloppy
\item[\parbox{\sstexampleslength}{\ssttt #1}] \mbox{} \vspace{1.0ex}
\\ #2 }

%  Format the notes section.
\newcommand{\sstnotes}[1]{\item[Notes:] \mbox{} \\[1.3ex] #1}

%  Provide a general-purpose format for additional (DIY) sections.
\newcommand{\sstdiytopic}[2]{\item[{\hspace{-0.35em}#1\hspace{-0.35em}:}]
\mbox{} \\[1.3ex] #2}

%  Format the a generic section as a list
\newcommand{\sstdiylist}[2]{
   \item[#1:] \mbox{} \\
   \vspace{-3.5ex}
   \begin{description}
      #2
   \end{description}
}

%  Format the implementation status section.
\newcommand{\sstimplementationstatus}[1]{
   \item[{Implementation Status:}] \mbox{} \\[1.3ex] #1}

%  Format the bugs section.
\newcommand{\sstbugs}[1]{\item[Bugs:] #1}

%  Format a list of items while in paragraph mode.
\newcommand{\sstitemlist}[1]{
  \mbox{} \\
  \vspace{-3.5ex}
  \begin{itemize}
     #1
  \end{itemize}
}

%  Define the format of an item.
\newcommand{\sstitem}{\item}

%% Now define html equivalents of those already set. These are used by
%  latex2html and are defined in the html.sty files.
\begin{htmlonly}

%  sstroutine.
   \newcommand{\sstroutine}[3]{
      \subsection{#1\xlabel{#1}-\label{#1}#2}
      \begin{description}
         #3
      \end{description}
   }

%  sstdescription
   \newcommand{\sstdescription}[1]{\item[Description:]
      \begin{description}
         #1
      \end{description}
      \\
   }

%  sstusage
   \newcommand{\sstusage}[1]{\item[Usage:]
      \begin{description}
         {\ssttt #1}
      \end{description}
      \\
   }

%  sstinvocation
   \newcommand{\sstinvocation}[1]{\item[Invocation:]
      \begin{description}
         {\ssttt #1}
      \end{description}
      \\
   }

%  sstarguments
   \newcommand{\sstarguments}[1]{
      \item[Arguments:] \\
      \begin{description}
         #1
      \end{description}
      \\
   }

%  sstreturnedvalue
   \newcommand{\sstreturnedvalue}[1]{
      \item[Returned Value:] \\
      \begin{description}
         #1
      \end{description}
      \\
   }

%  sstparameters
   \newcommand{\sstparameters}[1]{
      \item[Parameters:] \\
      \begin{description}
         #1
      \end{description}
      \\
   }

%  sstexamples
   \newcommand{\sstexamples}[1]{
      \item[Examples:] \\
      \begin{description}
         #1
      \end{description}
      \\
   }

%  sstsubsection
   \newcommand{\sstsubsection}[1]{\item[{#1}]}

%  sstexamplesubsection
   \newcommand{\sstexamplesubsection}[2]{\item[{\ssttt #1}] #2}

%  sstnotes
   \newcommand{\sstnotes}[1]{\item[Notes:] #1 }

%  sstdiytopic
   \newcommand{\sstdiytopic}[2]{\item[{#1}] #2 }

%  sstimplementationstatus
   \newcommand{\sstimplementationstatus}[1]{
      \item[Implementation Status:] #1
   }

%  sstitemlist
   \newcommand{\sstitemlist}[1]{
      \begin{itemize}
         #1
      \end{itemize}
      \\
   }
%  sstitem
   \newcommand{\sstitem}{\item}

\end{htmlonly}

%  End of "sst.tex" layout definitions.
%.



% ? End of document specific commands
% -----------------------------------------------------------------------------
%  Title Page.
%  ===========
\renewcommand{\thepage}{\roman{page}}
\begin{document}
\thispagestyle{empty}

%  Latex document header.
%  ======================
\begin{latexonly}
   \textsc{University of British Columbia} / \textsc{Joint Astronomy Centre} \hfill \textbf{\stardocname}\\
   {\large Science \& Technology Facilities Council}\\
   {\large Starlink Software Collection\\}
   {\large \stardoccategory\ \stardocnumber}
   \begin{flushright}
   \stardocauthors\\
   \stardocdate
   \end{flushright}
   \vspace{-4mm}
   \rule{\textwidth}{0.5mm}
   \vspace{5mm}
   \begin{center}
   {\Huge\textbf{\stardoctitle \\ [2.5ex]}}
   {\LARGE\textbf{\stardocversion \\ [4ex]}}
   {\Huge\textbf{\stardocmanual}}
   \end{center}
   \vspace{5mm}

% ? Add picture here if required for the LaTeX version.
%   e.g. \includegraphics[scale=0.3]{filename.ps}
\begin{center}
\includegraphics[scale=0.3]{sun265_logo}
\end{center}
% ? End of picture

% ? Heading for abstract if used.
   \vspace{10mm}
   \begin{center}
      {\Large\textbf{Abstract}}
   \end{center}
% ? End of heading for abstract.
\end{latexonly}

%  HTML documentation header.
%  ==========================
\begin{htmlonly}
   \xlabel{}
   \begin{rawhtml} <H1> \end{rawhtml}
      \stardoctitle\\
      \stardocversion\\
      \stardocmanual
   \begin{rawhtml} </H1> <HR> \end{rawhtml}

% ? Add picture here if required for the hypertext version.
%   e.g. \includegraphics[scale=0.7]{filename.ps}
\includegraphics[scale=0.7]{sun258_logo}
% ? End of picture

   \begin{rawhtml} <P> <I> \end{rawhtml}
   \stardoccategory\ \stardocnumber \\
   \stardocauthors \\
   \stardocdate
   \begin{rawhtml} </I> </P> <H3> \end{rawhtml}
      \htmladdnormallink{University of British Columbia}
                        {http://www.ubc.ca} \\
      \htmladdnormallink{Joint Astronomy Centre}
                        {http://www.jach.hawaii.edu}\\
      \htmladdnormallink{Science \& Technology Facilities Council}
                        {http://www.stfc.ac.uk} \\
   \begin{rawhtml} </H3> <H2> \end{rawhtml}
      \htmladdnormallink{Starlink Software Collection}{http://starlink.jach.hawaii.edu/}
   \begin{rawhtml} </H2> \end{rawhtml}
   \htmladdnormallink{\htmladdimg{source.gif} Retrieve hardcopy}
      {http://starlink.jach.hawaii.edu/cgi-bin/hcserver?\stardocsource}\\

%  HTML document table of contents.
%  ================================
%  Add table of contents header and a navigation button to return to this
%  point in the document (this should always go before the abstract \section).
  \label{stardoccontents}
  \begin{rawhtml}
    <HR>
    <H2>Contents</H2>
  \end{rawhtml}
  \htmladdtonavigation{\htmlref{\htmladdimg{contents_motif.gif}}
        {stardoccontents}}

% ? New section for abstract if used.
  \section{\xlabel{abstract}Abstract}
% ? End of new section for abstract
\end{htmlonly}

% -----------------------------------------------------------------------------
% ? Document Abstract. (if used)
%  ==================
\stardocabstract
% ? End of document abstract

% -----------------------------------------------------------------------------
% ? Latex Copyright Statement
%  =========================
\begin{latexonly}
\newpage
\vspace*{\fill}
\stardoccopyright
\end{latexonly}
% ? End of Latex copyright statement

% -----------------------------------------------------------------------------
% ? Latex document Table of Contents (if used).
%  ===========================================
  \newpage
  \begin{latexonly}
    \setlength{\parskip}{0mm}
    \tableofcontents
    \setlength{\parskip}{\medskipamount}
    \markboth{\stardocname}{\stardocname}
  \end{latexonly}
% ? End of Latex document table of contents
% -----------------------------------------------------------------------------

\cleardoublepage
\renewcommand{\thepage}{\arabic{page}}
\setcounter{page}{1}

% Main text

\section{\xlabel{introduction}Introduction\label{se:intro}}

The \oracdr\ pipeline (\oracsun) is a suite of recipes and primitives
for the automated processing of raw instrument data into
scientifically-useable products. However, this is only the start point
for the analysis and further operations on these data is
inevitable. The PIpeline for Combining and Analyzing Reduced Data
(\picard) is a modification to \oracdr\ which allows
pipeline-processed data to be manipulated using generic,
instrument-independent methods. Furthermore, it is inefficient to
begin at the computationally-expensive raw data stage again for every
minor adjustment to the analysis. Thus \oracdr\ and \picard\ together
represent two halves of the data reduction and analysis workflow.

This document will describe the basics of using \picard\ and contains
a summary of available processing recipes.

\subsection{Document conventions}

In an attempt to make this document clearer to read, different fonts
are used for specific structures.

Starlink package names are shown in small caps (e.g.\ \SMURF);
individual task names are shown in sans-serif
(e.g.\ \makemap). \picard\ recipe and primitive names are also shown
in sans-serif and are always upper case (e.g.\ \task{REMOVE\_BACKGROUND}).

Text relating to filenames (including suffices for data products), key
presses or entries typed at the command line are also denoted by
fixed-width type (e.g.\ \texttt{\% smurf}), as are parameters for
tasks which are displayed in upper case (e.g.\ \aparam{METHOD}).

References to Starlink documents, i.e., Starlink User Notes (SUN),
Starlink General documents (SG) and Starlink Cookbooks (SC), are given
in the text using the document type and the corresponding number
(e.g.\ SUN/95). Non-Starlink documents are cited in the text and
listed in the bibliography.

File name suffices represent the text between the final underscore
character and the three-letter \verb+.sdf+ extension. For example, a
file named \verb+s4a20101020_00002_0001_cal.sdf+ has the suffix
\verb+_cal+.

\section{\xlabel{picard}PICARD overview\label{se:picard}}

\picard\ is a tool for analyzing and combining a batch of astronomical
data files that have previously had their instrumental signatures
removed (for example by running \oracdr\ on the raw data). It is
designed to be instrument-independent. \picard\ uses the same
infrastructure as \oracdr, where data are processed by recipes which
contain a series of processing steps called primitives.

\picard\ is designed to be easy to use. It needs no initialization,
has few options and, by default, assumes that all input/output occurs
in the current working directory.

\subsection{Requirements for running PICARD}

\oracdr\ (and thus \picard) requires a recent Starlink
installation. The latest release may be obtained from
\htmladdnormallink{\texttt{http://starlink.jach.hawaii.edu/starlink}}{http://starlink.jach.hawaii.edu/starlink}. Since
\oracdr\ development is an ongoing process, it is recommended that the
newest builds be used. These builds can be obtained from:
\htmladdnormallink{\texttt{http://starlink.jach.hawaii.edu/starlink/rsyncStarlink}}{http://starlink.jach.hawaii.edu/starlink/rsyncStarlink}
and may be kept up-to-date with rsync.

The Starlink Perl installation (Starperl) must be used to run the
pipeline due to the module requirements. The Starlink environment
should be initialized as usual before running \picard.

\subsection{Important environment variables}

\picard\ does not need to have specific environment variables defined
(other than those initialized as part of Starlink). Data are read from
and written to the current working directory by default. However, it
is possible to define an alternative location for the output data via
\verb+ORAC_DATA_OUT+ (which is used by \oracdr).

Two other specialized environment variables may be defined by users
who wish to write their own processing routines: see Section
\ref{se:write} for more information.

\subsection{Running PICARD}

The only mandatory arguments are the name of the recipe and a list of
the files to process. Running \picard\ is as easy as typing
\begin{myquote}
\begin{verbatim}
% picard <options> RECIPE *.sdf
\end{verbatim}
\end{myquote}
where \task{RECIPE} is the name of the processing recipe to use and
\verb+*.sdf+ is the list of files to process. In practice, everything
after the recipe name is treated as an input file. The recipe will be
applied to all input files, which must be in NDF format. Currently
there is no automated conversion from FITS.

More generally:
\begin{myquote}
\begin{verbatim}
% picard [options] RECIPE FILES
\end{verbatim}
\end{myquote}
where \verb+[options]+ are command-line options of the form
\verb+-option+ or \verb+-option value+.  Note that the options must be
given before the recipe. The options are described in more detail
below.

\subsection{PICARD options}

\picard\ has a number of command-line options which may be used to
control the processing and feedback.

\begin{description}
\item[-help] \mbox{}

  Lists help text summarizing \picard\ usage

\item[-version] \mbox{}

  Prints out the version information

\item[-man] \mbox{}

  Displays the help text as a manual page

\item[-verbose] \mbox{}

  Enable verbose output from algorithm engines (e.g.\ \SMURF\ \task{makemap})

\item[-debug] \mbox{}

  Enable debugging output, listing primitive entry and exits points,
  timing and calls to algorithm engines.

\item[-log sfhx] \mbox{}

  Control where text output is displayed either on the terminal screen
  (\texttt{s}), a log file (\texttt{f}), HTML log file (\texttt{h}) or
  to an X-window (\texttt{x}). Default is \texttt{fx}; for most
  recipes, \texttt{sf} is recommended.

\item[-nodisplay] \mbox{}

  Do not launch the display system. No data will be displayed and GWM,
  \GAIA\ etc windows will not be opened.

\item[-recsuffix SUFFIX] \mbox{}

  Modify the recipe search algorithm such that a recipe variant can be
  selected if available. For example with \texttt{-recsuffix QL} a
  recipe named \task{MYRECIPE\_QL} would be picked up in preference to
  \task{MYRECIPE}.

  Multiple suffices can be supplied using a comma separator, e.g.\
  \texttt{-recsuffix QL1,QL2}

\item[-recpars filename] \mbox{}

  Recipe behaviour can be controlled by specifying a recipe parameters
  file. This is a file in INI format with a block per recipe name.
\begin{myquote}
\begin{verbatim}
[RECIPE_NAME]
param1 = value1
param2 = value2
\end{verbatim}
\end{myquote}
See the documentation for individual recipes in Appendix~\ref{ap:full}
for supported parameters.

\end{description}

\section{Hints and tips\label{se:hints}}

This section lists a handful of useful hints and tips for running
\picard.

\begin{itemize}

\item
Make sure the \verb+.sdf+ extension is included in the filename if
passing in a single file.

\item A single recipe parameter file can be used for multiple recipes:
\begin{myquote}
\begin{verbatim}
[RECIPE1]
PARAM1 = VALUE1
PARAM2 = VALUE2

[RECIPE2]
PARAM_A = VALUE_A
PARAM_B = VALUE_B
\end{verbatim}
\end{myquote}

\item If the environment variable \verb+ORAC_DATA_OUT+ is defined, any
  files created by \picard\ will be written in that location. Check
  there if new files are expected but do not appear in the current
  directory.

\item The list of files can be the output from \texttt{cat}: e.g.\
\begin{myquote}
\begin{verbatim}
% picard -log s RECIPE_NAME `cat filestoprocess.lis`
\end{verbatim}
\end{myquote}
Remember to include the \verb+.sdf+ for each file in the list in the
file \verb+filestoprocess.lis+.

\item All data should be from the same instrument. For SCUBA-2 users,
  this means data from a single wavelength.

\item By default \picard\ does not know how to display the files
  produced as part of processing, especially if relying on
  instrument-specific features. If display is required, make a copy of
  the file \texttt{disp.dat} located in
  \verb+$ORAC_CAL_ROOT/inst_name+ where \verb+inst_name+ is the
  (lower-case) name of the instrument from which the data originated
  (e.g.\,\verb+scuba2+).

\end{itemize}

\section{Writing PICARD recipes and primitives\label{se:write}}

\picard\ allows end-users to write their own primitives and recipes
for processing reduced data, although a number of (mostly SCUBA-2)
recipes exist. Interested users are advised to read \oracsun\ and
\oracprogsun\ for further details. The user can specify the
environment variables \verb+ORAC_RECIPE_DIR+ and
\verb+ORAC_PRIMITIVE_DIR+ to point to the locations containing recipes
and primitives.

While \picard\ is designed to be instrument-independent, it is
possible to access methods from supported instrument classes, provided
that all the input data are from the same instrument. This is
particularly useful for accessing instrument-specific values and
methods provided by the calibration class, for example. It also
provides access to instrument-specific recipes and primitives.

\newpage
\appendix
\begin{small}

\section{\xlabel{ap_list}Alphabetical list of PICARD recipes\label{ap:list}}
\begin{htmlonly}
\begin{description}
\end{htmlonly}

\menuitem{CALC\_SCUBA2\_AVPSPEC}{
  Calculate average bolometer power spectra from SCUBA-2 data}
\menuitem{CALC\_SCUBA2\_FCF}{
  Calculate FCFs from SCUBA-2 calibrators}
\menuitem{CALC\_SCUBA2\_NEFD}{
  Calculate NEFDs from SCUBA-2 images}
\menuitem{CREATE\_MOMENTS\_MAP}{
  Creates a moments map from a spectral line cube}
\menuitem{CREATE\_PNG}{
  Create a PNG from the current Frame object.}
\menuitem{CROP\_JCMT\_IMAGES}{
  Trim images to the defined map area}
\menuitem{MOSAIC\_JCMT\_IMAGES}{
  Coadd images produced by JCMT instruments}
\menuitem{PICARD\_DEMONSTRATOR}{
  Simple recipe to test Picard infrastructure}
\menuitem{REMOVE\_BACKGROUND}{
  Remove a background from images}
\menuitem{SCUBA2\_CHECK\_CAL}{
  perform SCUBA-2 calibration checks on standard sources}
\menuitem{SCUBA2\_DISPLAY\_PCA}{
  Calculate and display properties of PCA components}
\menuitem{SCUBA2\_JACKKNIFE}{
  calculate optimal map using jack-knife noise estimator}
\menuitem{SCUBA2\_MATCHED\_FILTER}{
  Apply a matched filter to input images}
\menuitem{SCUBA2\_PHOTOM}{
  perform aperture photometry on SCUBA-2 images}
\menuitem{SCUBA2\_REGISTER\_IMAGES}{
  Register SCUBA-2 images to a common position}
\menuitem{SCUBA2\_SASSY}{
  analyze a single SASSy field}
\menuitem{STACK\_JCMT\_FRAMES}{
  stack images produced by JCMT instruments}
\menuitem{UNCALIBRATE\_SCUBA2\_DATA}{
  Undo the default SCUBA-2 calibration}

\begin{htmlonly}
\end{description}
\end{htmlonly}


\end{small}
\newpage

\section{\xlabel{ap_full}Specifications of PICARD recipes\label{ap:full}}

The following pages describe the current \picard\ recipes in detail.

\newpage
\sstroutine{
   CALC\_SCUBA2\_AVPSPEC
}{
   Calculate average bolometer power spectra from SCUBA-2 data
}{
   \sstdescription{
      A simple PICARD recipe to calculate the average bolometer power
      spectra from raw SCUBA-2 data.
   }
   \sstnotes{
      \sstitemlist{

         \sstitem
         The input data must be raw SCUBA-2 data

         \sstitem
         Produces one output file per subarray with suffix \_avpspec.
      }
   }
   \sstdiylist{
      Available Parameters
   }{
      \sstsubsection{
         The following parameter can be set via the -recpar option:
      }{
      }
      \sstsubsection{
         DISPLAY
      }{
         Flag to control the display of power spectra. The recipe will
         attempt to display spectra by default.
      }
   }
   \sstdiytopic{
      Display
   }{
      The power spectrum for each file is displayed if desired, up to a
      maximum of four. Note that a suitable disp.dat must be present in
      the output directory, or the environment variable ORAC\_DATA\_CAL
      must point to the location of the SCUBA-2 version.
   }
}
\newpage
\sstroutine{
   CALC\_SCUBA2\_FCF
}{
   Calculate FCFs from SCUBA-2 calibrators
}{
   \sstdescription{
      A simple PICARD recipe to calculate the FCF from reduced SCUBA-2
      images of calibration sources.
   }
   \sstnotes{
      \sstitemlist{

         \sstitem
         The input data should be uncalibrated.

         \sstitem
         The results of the calculation are printed to the screen and
         written to a log file, log.fcf.
      }
   }
   \sstdiytopic{
      Display
   }{
      No display is used by this recipe.
   }
}
\newpage
\sstroutine{
   CALC\_SCUBA2\_NEFD
}{
   Calculate NEFDs from SCUBA-2 images
}{
   \sstdescription{
      A simple PICARD recipe to calculate the noise equivalent flux
      density (NEFD) from reduced SCUBA-2 images.
   }
   \sstnotes{
      \sstitemlist{

         \sstitem
         The input data are calibrated if necessary using standard FCFs.

         \sstitem
         The results of the calculation are printed to the screen and
         written to a log file, log.nefd.
      }
   }
   \sstdiytopic{
      Display
   }{
      No display is used by this recipe.
   }
}
\newpage
\sstroutine{
   CREATE\_MOMENTS\_MAP
}{
   Creates a moments map from a spectral line cube
}{
   \sstdescription{
      This recipe is used to create a moments map (or multiple moments
      maps) from a cube. It smooths the cube in frequency and spatial
      extents, then finds clumps of emission. Everything in the cube not
      found in a clump is masked out, then the masked cube is collapsed
      to form the moments map.
   }
   \sstdiylist{
      Available Parameters
   }{
      \sstsubsection{
         The following parameters can be set via the -recpars option:
      }{
      }
      \sstsubsection{
         MOMENTS
      }{
         The moment maps to create. These are any of the values allowed
         for the ESTIMATOR parameter to the COLLAPSE task, but in
         reality this should probably be {\tt '}integ{\tt '}, {\tt '}iwc{\tt '}, and/or {\tt '}itd{\tt '}.
         Any number of moments can be given in a comma-separated string.
         [{\tt '}integ{\tt '}]
      }
      \sstsubsection{
         MOMENTS\_LOWER
      }{
         An optional lower velocity in km/s, below which no data will be
         used when creating the moments map. When it is undefined, the
         full velocity range is used. [undef]
      }
      \sstsubsection{
         MOMENTS\_SNR
      }{
         Whether or not to do clump detection on an signal-to-noise cube
         instead of the signal cube. Enabling this is useful for data
         taken in varying conditions. [0]
      }
      \sstsubsection{
         MOMENTS\_UPPER
      }{
         An optional upper velocity in km/s, above which no data will be
         used when creating the moments map. When it is undefined, the
         full velocity range is used. [undef]
      }
   }
}
\newpage
\sstroutine{
   CREATE\_PNG
}{
   Create a PNG from the current Frame object
}{
   \sstdescription{
      This recipe creates a 256x256 pixel PNG file for each file in the
      current Frame object. It will only work properly on 1-D or 2-D
      images, throwing a warning if the input file is neither 1-D nor
      2-D.
   }
   \sstnotes{
      \sstitemlist{

         \sstitem
         Creates output files with same name as input, but with
         extension .png.
      }
   }
}
\newpage
\sstroutine{
   CROP\_JCMT\_IMAGES
}{
   Trim images to the defined map area
}{
   \sstdescription{
      A simple PICARD recipe to trim images from SCUBA-2 or ACSIS to the
      same size as defined by the map parameters in the FITS header. The
      map width and height may be overridden with recipe parameters.
      Note that if a map radius is given, in order to produce a circular
      output image, the height and width are ignored.

      Uses the JCMT::MapArea Perl module to define a (rectangular) AST
      Region using the map parameters in the FITS header.
   }
   \sstnotes{
      \sstitemlist{

         \sstitem
         Creates output file with suffix \_crop, one for each input file.
      }
   }
   \sstdiylist{
      Available Parameters
   }{
      \sstsubsection{
         The following parameters can be set via the -recpars option:
      }{
      }
      \sstsubsection{
         MAP\_HEIGHT
      }{
         Height of output image in arcsec.
      }
      \sstsubsection{
         MAP\_RADIUS
      }{
         Radius of output image in arcsec. Overrides existence of
         MAP\_HEIGHT and MAP\_WIDTH.
      }
      \sstsubsection{
         MAP\_WIDTH
      }{
         Width of output image in arcsec.
      }
   }
   \sstdiytopic{
      Display
   }{
      No display is used by this recipe.
   }
}
\newpage
\sstroutine{
   MOSAIC\_JCMT\_IMAGES
}{
   Coadd images produced by JCMT instruments
}{
   \sstdescription{
      A simple PICARD recipe combine SCUBA-2 or ACSIS images taking into
      account the EXP\_TIME NDF component.
   }
   \sstnotes{
      \sstitemlist{

         \sstitem
         All the input images should be of the same source as defined by
         the OBJECT FITS header.

         \sstitem
         Creates a single output file with suffix \_mos.
      }
   }
   \sstdiylist{
      Available Parameters
   }{
      \sstsubsection{
         The following parameters can be set via the -recpars option:
      }{
      }
      \sstsubsection{
         MOSAIC\_TASK
      }{
         The mosaicking task to use either wcsmosaic (default) or
         makemos.
      }
      \sstsubsection{
         MAKEMOS\_METHOD
      }{
         The image combination method for makemos.
      }
      \sstsubsection{
         MAKEMOS\_SIGMAS
      }{
         The sigma-clipping threshold if MAKEMOS\_METHOD is SIGMAS.
         Default is 4.
      }
      \sstsubsection{
         WCSMOSAIC\_METHOD
      }{
         Rebinning method for wcsmosaic and/or wcsalign. Default is
         nearest.
      }
      \sstsubsection{
         WCSMOSAIC\_PARAMS
      }{
         Additional parameters required for certain choices of
         WCSMOSAIC\_METHOD.
      }
   }
   \sstdiytopic{
      Display
   }{
      No display is used by this recipe.
   }
}
\newpage
\sstroutine{
   PICARD\_DEMONSTRATOR
}{
   Simple recipe to test Picard infrastructure
}{
   \sstdescription{
      Write out the name of each file.
   }
}
\newpage
\sstroutine{
   REMOVE\_BACKGROUND
}{
   Remove a background from images
}{
   \sstdescription{
      A simple PICARD recipe to fit and remove a background from an
      image.
   }
   \sstnotes{
      \sstitemlist{

         \sstitem
         The images should not be of extended sources.

         \sstitem
         Creates an output file for each input file with a suffix
         $<$\_back$>$.
      }
   }
   \sstdiylist{
      Available Parameters
   }{
      \sstsubsection{
         The following parameters can be set via the --recpars option:
      }{
      }
      \sstsubsection{
         MASK\_SOURCE
      }{
         Flag to denote whether to mask the source before removing the
         background. Default is 0 (do not mask the source).
      }
      \sstsubsection{
         APERTURE\_RADIUS
      }{
         Radius of aperture (in arcsec) used to mask out source. Default
         is about twice the beamsize.
      }
      \sstsubsection{
         BACKGROUND\_FITMETHOD
      }{
         Method to use for removing background. May be fitsurface,
         findback, plane or dc. Default is fitsurface.
      }
      \sstsubsection{
         FITSURFACE\_FITTYPE
      }{
         Type of fit to use with fitsurface. May be polynomial or
         spline. Default is polynomial.
      }
      \sstsubsection{
         FITSURFACE\_FITPAR
      }{
         Up to two values which define either the order of the
         polynomial (for polynomial) or the number of knots (for spline)
         in the X and Y directions respectively. A single number means
         the same value is used for both axes. Default is 2 for
         polynomial, 4 for spline.
      }
      \sstsubsection{
         FITSURFACE\_KEEPSURFACE
      }{
         A flag to denote whether or not to keep the fitted surface on
         disk. Useful for debugging purposes. Default is 0 (do not keep
         on disk).
      }
      \sstsubsection{
         FINDBACK\_BOX
      }{
         Size of the box (in pixels) used by findback. Default is 11.

         Default values are those used if the parameter is not
         specified.
      }
   }
   \sstdiytopic{
      Display
   }{
      No display is used by this recipe.
   }
}
\newpage
\sstroutine{
   SCUBA2\_CHECK\_CAL
}{
   perform SCUBA-2 calibration checks on standard sources
}{
   \sstdescription{
      A simple PICARD recipe to calculate fluxes, FCFs and beam size
      from a given uncalibrated map of a point source. The results are
      written to a log file called log.checkcal if desired.

      Procedure:

      \sstitemlist{

         \sstitem
         The images are cropped to the given size (as specified in the
         FITS headers or via the MAP\_HEIGHT and MAP\_WIDTH recipe
         parameters, and must be at least twice the diameter of the
         aperture).

         \sstitem
         A background may be fitted and removed. (Optional - only if the
         REMOVE\_BACKGROUND recipe parameter is true.)

         \sstitem
         The beam size is determined using KAPPA beamfit.

         \sstitem
         FCFs are calculated from the cropped (background-subtracted)
         image.

         \sstitem
         The source flux and its uncertainty are derived from aperture
         photometry on these images. The background is estimate from an
         annulus with inner and outer radii of 1.25 and 2.0 times the
         aperture radius.

         \sstitem
         The map is calibrated using either the standard FCF or the one
         derived above (if the USEFCF recipe parameter is true).

         \sstitem
         The noise is calculated from the calibrated map.

         \sstitem
         The matched filter is applied to the calibrated map.

         \sstitem
         Results are written to a log file, log.checkcal.

      }
      By default this recipe only works on known calibration sources.
      However, the user may specify the source flux at 850 and/or 450 um
      by using recipe parameters called FLUX\_850 and FLUX\_450
      respectively. The fluxes for different sources may be specified by
      appending the target name (in upper case with spaces removed),
      e.g. FLUX\_850.HLTAU.

      By default a log file is written containing a variety of
      information about the data and the values calculated.
   }
   \sstnotes{
      \sstitemlist{

         \sstitem
         The input data must be uncalibrated in order to calculate an
         FCF from calibrator observations. (The PICARD recipe
         UNCALIBRATE\_SCUBA2\_DATA can be used to undo the default
         calibration.)

         \sstitem
         The default behaviour is to leave every file created during the
         recipe on disk. This may not be desireable - see the KEEPFILES
         recipe parameter above to reduce the number of output files.

         \sstitem
         Re-processing data already processed by this recipe is not
         recommended.

         \sstitem
         If the recipe parameter FITSURFACE\_KEEPSURFACE is true, then a
         file will be created (for each input file) with suffix \_surface.

         \sstitem
         Documentation for other recipes may list other recipe
         parameters that appear to be applicable to some of the stesp in
         this recipe, but are not shown due to the possibility of adverse
         interactions.
      }
   }
   \sstdiylist{
      Available Parameters
   }{
      \sstsubsection{
         The following parameters can be set via the -recpars option:
      }{
      }
      \sstsubsection{
         APERTURE\_RADIUS
      }{
         Radius of aperture in arcsec for calculating total flux. The
         default is 30 arcsec.
      }
      \sstsubsection{
         BACKGROUND\_FITMETHOD
      }{
         Method to use for removing background. May be fitsurface,
         findback, plane or dc. Default is fitsurface.
      }
      \sstsubsection{
         FINDBACK\_BOX
      }{
         Size of the box (in pixels) used by findback. Default is 11.
      }
      \sstsubsection{
         FIT\_GAUSSIAN
      }{
         Flag to indicate whether or not to force a Gaussian fit to the
         source when estimating the beam parameters. Default is 1 (fit
         Gaussian).
      }
      \sstsubsection{
         FIT\_FIXAMP
      }{
         A flag to denote that the amplitude of the fit to the source
         should be fixed as the peak value in the map. Default is 0
         (amplitude is a free parameter).
      }
      \sstsubsection{
         FIT\_FIXBACK
      }{
         Specifies the background level to be used in the fit to the
         source. May be ! to allow the background to float. If not
         given, the default is either a fixed level of 0 for known
         calibrators, or the background is left as a free parameter.
      }
      \sstsubsection{
         FITSURFACE\_FITPAR
      }{
         Up to two values which define either the order of the
         polynomial (for polynomial) or the number of knots (for spline)
         in the X and Y directions respectively. A single number means
         the same value is used for both axes. Default is 2 for
         polynomial, 4 for spline.
      }
      \sstsubsection{
         FITSURFACE\_FITTYPE
      }{
         Type of fit to use with fitsurface. May be polynomial or
         spline. Default is polynomial.
      }
      \sstsubsection{
         FITSURFACE\_KEEPSURFACE
      }{
         A flag to denote whether or not to keep the fitted surface on
         disk. Useful for debugging purposes. Default is 0 (do not keep
         on disk).
      }
      \sstsubsection{
         FLUX\_450
      }{
         Source flux density at 450 um in Jy. Source-specific values may
         be given by dot-appending the source name in upper case with
         spaces removed. For example, FLUX\_450.DGTAU.
      }
      \sstsubsection{
         FLUX\_850
      }{
         Source flux density at 850 um in Jy. Source-specific values may
         be given by dot-appending the source name in upper case with
         spaces removed (see above).
      }
      \sstsubsection{
         KEEPFILES
      }{
         A flag to indicate whether or not to keep all files produced by
         the recipe. May be 0 to keep no files, or $+$1 to keep only files
         with suffix \_crop, \_back and \_mf. Default is -1 (keep all
         files).
      }
      \sstsubsection{
         LOGFILE
      }{
         Flag to denote whether to write results to a log file at the
         end of processing. Default is 1 (write log file).
      }
      \sstsubsection{
         MAP\_HEIGHT
      }{
         Height of map in arcsec after cropping. Must be at least twice
         the aperture diameter. Default is that in the FITS header.
      }
      \sstsubsection{
         MAP\_RADIUS
      }{
         Radius in arcsec of the circular region to define the map. Must
         be at least twice the aperture radius. Overrides the use of
         MAP\_HEIGHT and MAP\_WIDTH.
      }
      \sstsubsection{
         MAP\_WIDTH
      }{
         Width of map in arcsec after cropping. Must be at least twice
         the aperture diameter. Default is that in the FITS header.
      }
      \sstsubsection{
         MASK\_SOURCE
      }{
         Flag to denote whether to mask the source before removing the
         background. Default is 0 (do not mask the source).
      }
      \sstsubsection{
         NOISE\_METHOD
      }{
         Method used to calculate the noise in the calibrated image. May
         be VARIANCE to use the variance, MASK to mask out the source
         and calculate the image-plane standard deviation, or MINIMUM to
         determine the lowest standard deviation in a series of
         apertures placed on the image. Default is VARIANCE, and minimum
         match is supported.
      }
      \sstsubsection{
         PSF\_MATCHFILTER
      }{
         Name of a file to use as the PSF when applying the matched
         filter.
      }
      \sstsubsection{
         REMOVE\_BACKGROUND
      }{
         A flag to indicate whether or not a background should be
         estimated and removed from the image. Default is 0 (do not
         remove a background).
      }
      \sstsubsection{
         USEFCF
      }{
         Flag to denote whether to calibrate the data using the FCFs
         derived in this recipe (1) or use standard FCFs (0). Standard
         FCFs will be used if not specified.
      }
      \sstsubsection{
         USEFCF\_CALTYPE
      }{
         Calibration type to use if USEFCF is 1. May be ARCSEC, BEAM or
         BEAMMATCH. Default is BEAM.
      }
   }
   \sstdiytopic{
      Display
   }{
      None.
   }
}
\newpage
\sstroutine{
   SCUBA2\_DISPLAY\_PCA
}{
   Calculate and display properties of PCA components
}{
   \sstdescription{
      A simple PICARD recipe to apply PCA processing to raw SCUBA-2
      data.

      The input data should contain a fast-ramp flatfield (taken prior
      to the target data).
   }
   \sstnotes{
      Input data should be from a single subarray only, and for a single
      observation. However, no checks are made that this is actually the
      case.
      If results are to be calculated by the recipe, then all the input
      data are used. This could lead to long run times with a large
      number of files as the data are pre-processed with SMURF sc2clean.
   }
   \sstdiylist{
      Available Parameters
   }{
      \sstsubsection{
         The following parameters can be set via the -recpars option:
      }{
      }
      \sstsubsection{
         LOGFILE
      }{
         Flag to denote whether to write results to a log file at the
         end of processing. Default is 1 (write a log file).
      }
      \sstsubsection{
         PCA\_COMP
      }{
         PCA components to analyze and display. Default is 0 to 5. The
         components may be specified either as a comma-separated list
         (e.g. 0,1,2,3 etc - they need not be contiguous or in order),
         or as a Perl array slice (e.g. 0..3). The number of components
         must be no more than 8. If more than 8 are given, only the
         first 8 are used.
      }
      \sstsubsection{
         PCA\_KEEPFILES
      }{
         Flag to indicate which files should be kept on disk. Default is
         1 which keeps the PCA amplitude, component and power spectrum
         files on disk. A value of 0 deletes all files, while a value
         of 1 indicates that all files should be kept on disk.
      }
      \sstsubsection{
         PCA\_REUSE
      }{
         Flag to indicate that existing data should be used if present.
         Default is 1 (reuse).
      }
   }
   \sstdiytopic{
      Display
   }{
      The results for each chosen PCA component are displayed in up to
      two KAPVIEW windows. The left-hand column displays the amplitude
      scaled between $+$/-2 sigma, the next column displays the component
      as a function of time and the third column shows the power
      spectrum of each component. Each KAPVIEW window can display
      results for up to 4 PCA components.
   }
}
\newpage
\sstroutine{
   SCUBA2\_JACKKNIFE
}{
   Calculate optimal map using jack-knife noise estimator
}{
   \sstdescription{
      This recipe uses a jack-knife method to remove residual
      low-spatial frequency noise and create an optimal match-filtered
      output map. The recipe proceeds as follows:

      \sstitemlist{

         \sstitem
         The input images are coadded to produce a total signal map.

         \sstitem
         The observations are divided into two groups which are coadded
         separately. These coadds are subtracted from one another to create
         the jack-knife map.

         \sstitem
         The angular power spectrum of the jack-knife map (which should
         consist purely of noise) is calculated and used to remove residual
         low-spatial frequency noise from the signal map and the given
         (map-filtered) psf. This is the so-called whitening step (because
         it produces a map which has a noise power spectrum that is white).

         \sstitem
         The whitened signal map is processed with a matched filter
         using the whitened psf image as the psf.

         \sstitem
         The jack-knife map is also whitened and processed with the
         matched filter. This map should consist purely of noise.

         \sstitem
         Signal-to-noise ratio maps are created for the filtered
         versions of the signal map and the jack-knife map.

      }
      The outcome (the match-filtered whitened signal map) should be the
      optimal map with white noise properties. This is the map to be
      used for science goals.
   }
   \sstnotes{
      \sstitemlist{

         \sstitem
         Ideally there should be an even number of observations, but
         this is not important if the number of input files is large.

         \sstitem
         See the documentation for the SCUBA-2 recipe\\
         REDUCE\_SCAN\_FAINT\_POINT\_SOURCES\_JACKKNIFE for a more complete
         description of the procedure.
      }
   }
   \sstdiylist{
      Available Parameters
   }{
      \sstsubsection{
         The following recipe parameters can be set via the --recpars
      }{
      }
      \sstsubsection{
         option:
      }{
      }
      \sstsubsection{
         FAKEMAP\_SCALE
      }{
         Amplitude of the fake source (in Jy) added to the timeseries to
         assess the map-making response to a point source.
      }
      \sstsubsection{
         MAKEMAP\_CONFIG
      }{
         Name of a config file for use with the SMURF makemap task. The
         file must exist in the current working directory,
         \$MAKEMAP\_CONFIG\_DIR, \$ORAC\_DATA\_OUT, \$ORAC\_DATA\_CAL or
         \$STARLINK\_DIR/share/smurf.
      }
      \sstsubsection{
         MAKEMAP\_PIXSIZE
      }{
         Pixel size in arcsec for the output map. Default is wavelength
         dependent (4 arcsec at 850 um, 2 arcsec at 450 um).
      }
      \sstsubsection{
         PSF\_MATCHFILTER
      }{
         Name of a file to use as the map-filtered PSF.
      }
      \sstsubsection{
         WHITEN\_BOX
      }{
         Size of the region used to calculate the angular power spectrum
         for removing residual low-frequency noise in the data. Default
         is a square region bounded by the noise being less than twice
         the minimum value.
      }
   }
   \sstdiytopic{
      Display
   }{
      None.
   }
}
\newpage
\sstroutine{
   SCUBA2\_MATCHED\_FILTER
}{
   Apply a matched filter to input images
}{
   \sstdescription{
      A simple PICARD recipe to apply a matched filter to input SCUBA-2
      images with the aim of detecting point sources. The given images
      are convolved with a PSF, which the user can supply or is created
      by the recipe. Before the convolution, the maps and the PSF are
      smoothed with a Gaussian, and these smoothed versions are
      subtracted from the unsmoothed versions.
   }
   \sstnotes{
      \sstitemlist{

         \sstitem
         It may be worth cropping the images before applying this filter
         to remove large-scale junk around the edge.

         \sstitem
         Input data should all be able to use the same PSF image (if
         specified).

         \sstitem
         Creates an output file for each input file with suffix \_mf.

         \sstitem
         Creates an PSF file for each input file with suffix \_psf if the
         KEEPFILES recipe parameter is true.
      }
   }
   \sstdiylist{
      Available Parameters
   }{
      \sstsubsection{
         The following parameters can be set via the -recpars option:
      }{
      }
      \sstsubsection{
         KEEPFILES
      }{
         A flag to indicate that the PSF created by this recipe should
         remain on disk after processing. If not specified, the PSF will
         be deleted if one is created. This parameter is ignored if a
         PSF file is given (see PSF\_MATCHFILTER).
      }
      \sstsubsection{
         PSF\_MATCHFILTER
      }{
         Name of an NDF file containing a suitable PSF. Must exist in
         the current working directory. If not specified, the recipe
         will calculate one itself for each input file.
      }
      \sstsubsection{
         PSF\_NORM
      }{
         Normalization scheme used for the PSF created by this recipe if
         one is not specified using the above parameter. Maybe be PEAK
         or SUM to indicate whether the Gaussian PSF should have a peak
         of unity or a sum of unity. If not specified, the recipe
         assumes PEAK.
      }
      \sstsubsection{
         SMOOTH\_DATA
      }{
         Flag to denote whether or not the image and PSF should be
         smoothed and have the smoothed version subtracted from the
         original. If not specified, the recipe assumes a value of 1
         (smooth and subtract).
      }
      \sstsubsection{
         SMOOTH\_FWHM
      }{
         FWHM of Gaussian used to smooth data and PSF images before
         convolving with the PSF. If not specified the recipe assumes 30
         arcsec.
      }
   }
   \sstdiytopic{
      Display
   }{
      No display is used by this recipe.
   }
}
\newpage
\sstroutine{
   SCUBA2\_PHOTOM
}{
   perform aperture photometry on SCUBA-2 images
}{
   \sstdescription{
      Perform aperture photometry on SCUBA-2 images using the chosen
      method. By default (i.e. with no recipe parameters), this recipe
      will calculate the flux within a 60-arcsec diameter aperture,
      corrected for any DC offsets by using the rest of the image to
      estimate the background. Alternatively the background may be
      estimated from an annulus centred on the source. Finally, the user
      may request that the AUTOPHOTOM packaged be used. Note, however,
      in that case that no uncertainties are returned.

      The results are written to a log file called log.flux.
   }
   \sstnotes{
      It is assumed that the images can be used as is with no further
      requirement for cropping or background removal.
   }
   \sstdiylist{
      Available Parameters
   }{
      \sstsubsection{
         The following parameters can be set via the -recpars option:
      }{
      }
      \sstsubsection{
         ANNULUS
      }{
         Flag to denote whether to use an annulus for background
         estimation.
      }
      \sstsubsection{
         ANNULUS\_INNER
      }{
         Inner radius for annulus as a multiplier of the aperture
         radius.
      }
      \sstsubsection{
         ANNULUS\_OUTER
      }{
         Outer radius for annulus as a multiplier of the aperture
         radius.
      }
      \sstsubsection{
         APERTURE\_RADIUS
      }{
         Radius of aperture in arcsec for calculating total flux.
      }
      \sstsubsection{
         AUTOPHOTOM
      }{
         Flag to denote whether to use the autophotom package for
         photometry.
      }
      \sstsubsection{
         REGISTER\_DEC
      }{
         Declination of position of aperture (DD:MM:SS format).
      }
      \sstsubsection{
         REGISTER\_RA
      }{
         Right ascension of position of aperture (HH:MM:SS format).
      }
      \sstsubsection{
         STATS\_ESTIMATOR
      }{
         Background estimator for aperture photometry. Default is
         median.
      }
   }
   \sstdiytopic{
      Display
   }{
      No display is used by this recipe.
   }
}
\newpage
\sstroutine{
   SCUBA2\_REGISTER\_IMAGES
}{
   Register SCUBA-2 images to a common position
}{
   \sstdescription{
      A PICARD recipe to register SCUBA-2 images to a common position.
      The position may be specified, or the WCS SkyRef attrib is used if
      the source is a calibrator, or (0,0) is used for images in offset
      coordinate systems.
   }
   \sstnotes{
      \sstitemlist{

         \sstitem
         A reference position should always be given for
         non-calibrators.

         \sstitem
         The reference position should be that of a known source in each
         image, and that source must be present in all images.

         \sstitem
         Creates an output file for each input file with suffix \_reg
      }
   }
   \sstdiylist{
      Available Parameters
   }{
      \sstsubsection{
         The following parameters can be set via the -recpars option:
      }{
      }
      \sstsubsection{
         REGISTER\_IMAGES
      }{
         Flag to indicate that the given images should all be shifted to
         a common position. No action will be taken if this flag is
         false (0).
      }
      \sstsubsection{
         REGISTER\_RA
      }{
         Right Ascension (in HH:MM:SS.S format) of reference position.
      }
      \sstsubsection{
         REGISTER\_DEC
      }{
         Declination (in DD:MM:SS.S format) of reference position.
      }
   }
   \sstdiytopic{
      Display
   }{
      No display is used by this recipe.
   }
}
\newpage
\sstroutine{
   SCUBA2\_SASSY
}{
   Analyze a single SASSy field
}{
   \sstdescription{
      A PICARD recipe to analyze individual maps of SASSy fields,
      combine them into a single coadd and apply a matched filter before
      running a source-detection algorithm. Detected sources are written
      to a CUPID catalogue file with suffix \_cat. Statistics are written
      to a log file called log.sassy.

      The statistics are calculated within the area defined by the
      MAP\_HGHT and MAP\_WDTH FITS headers, or by equivalent recipe
      parameters (below).

      See the documentation for the SCUBA2\_MATCHED\_FILTER recipe for
      matched-filter-specific parameters which may also be specified.
   }
   \sstnotes{
      None.
   }
   \sstdiylist{
      Available Parameters
   }{
      \sstsubsection{
         The following parameters can be set via the -recpars option:
      }{
      }
      \sstsubsection{
         LOGFILE
      }{
         A flag to indicate whether or not a log file (called log.sassy)
         should be written to disk. Default is 1 (yes).
      }
      \sstsubsection{
         MAP\_HEIGHT
      }{
         Map height in arcsec. Default is to use the value in the FITS
         header.
      }
      \sstsubsection{
         MAP\_WIDTH
      }{
         Map width in arcsec. Default is to use the value in the FITS
         header.
      }
   }
   \sstdiytopic{
      Display
   }{
      No display is used by this recipe.
   }
}
\newpage
\sstroutine{
   STACK\_JCMT\_FRAMES
}{
   Stack images produced by JCMT instruments
}{
   \sstdescription{
      A simple PICARD recipe to stack SCUBA-2 or ACSIS images into a 3-d
      cube with time as the third axis.

      By default the recipe will write out a separate file for each UT
      date given. SCUBA-2 data will also be sorted by the shutter
      setting. The user may give a list of additional FITS headers for
      collating the input files.

      The user may also provide the name of an NDF extension which will
      be stacked instead of the top-level data component (e.g. NEP).
   }
   \sstnotes{
      \sstitemlist{

         \sstitem
         Creates output files based on the name of the first file in the
         stack with suffix \_stack, unless there is only 1 file to stack.

         \sstitem
         The given FITS header keywords must exist in every file, and
         are not validated before accessing.
      }
   }
   \sstdiylist{
      Available Parameters
   }{
      \sstsubsection{
         The following parameters can be set via the -recpars option:
      }{
      }
      \sstsubsection{
         NDF\_EXTEN
      }{
         The name of an NDF extension to stack, rather than the
         top-level data structure. It must be located under the
         .more.smurf hierarchy, and no check is made that it exists
         before attempting to access it.
      }
      \sstsubsection{
         STACK\_KEYS
      }{
         A list of FITS header keywords to be used to sort the files
         before stacking. Only files with matching FITS header values
         will be used in the stack.
      }
   }
   \sstdiytopic{
      Display
   }{
      No display is used by this recipe.
   }
}
\newpage
\sstroutine{
   UNCALIBRATE\_SCUBA2\_DATA
}{
   Undo the default SCUBA-2 calibration
}{
   \sstdescription{
      A simple PICARD recipe to undo the default SCUBA-2 calibration.
      The units of the input data are checked and the appropriate
      default FCF chosen. The output files have a suffix of \_uncal.
   }
   \sstnotes{
      \sstitemlist{

         \sstitem
         Creates an output file for each calibrated input file with
         suffix \_uncal.
      }
   }
   \sstdiytopic{
      Display
   }{
      No display is used by this recipe.
   }
}

\end{document}

