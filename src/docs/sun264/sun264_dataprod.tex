\section{\xlabel{dataproducts}Data products (output files)\label{se:dataprod}}

The output maps from the pipeline are Starlink NDF files with a number
of NDF components, which may be listed with \task{hdstrace}. The
full hierarchy is shown in the example trace below.

The files contain standard the DATA and VARIANCE components. Images
have a three-dimensional WCS added, namely RA-Dec-wavelength. The
third axis is a single-pixel wide and is wavelength in m (sometimes
referred to as a non-significant axis, on the basis that it is not
always necessary to specify it). A FITS header extension is written
with values collated from all the files that went in to producing the
image. The FITS extension may be viewed with
\KAPPA\ \task{fitslist}. The UNITS component contains the current units
of the file, usually mJy\,beam$^{-1}$, along with the LABEL (a text
description of the units). The source name is written into the TITLE
field.

\SMURF\ \task{makemap} also writes additional NDF extensions under the
\verb+.MORE.SMURF+ hierarchy called \verb+EXP_TIME+ and
\verb+WEIGHTS+. The former is the exposure time image, the latter is
the number of `hits' on a given pixel (i.e.\ the number of
measurements that went into that pixel). Each of these NDFs has its
own WCS, UNITS and LABEL. If the pipeline calculated the NEFD for this
file then an \verb+NEFD+ image will also exist.

\KAPPA\ \task{hislist} will print the HISTORY embedded in the file,
which also shows the full list of arguments to \task{makemap}. The
PROVENANCE my be examined with \KAPPA\ \task{provshow}.

Example trace:
\begin{terminalv}
GS20091205_22_8  <NDF>

   DATA_ARRAY     <ARRAY>         {structure}
      DATA(990,1036,1)  <_DOUBLE>    *,*,*,*,*,*,*,*,*,*,*,*,*,*,*,*,*,*,
                                     *,*,*,*,*,*,*,*,*,*,*,*,*,*,*,*,*,*,*
      ORIGIN(3)      <_INTEGER>      -494,-522,1

   TITLE          <_CHAR*3>       'W5E'
   LABEL          <_CHAR*12>      'Flux Density'
   UNITS          <_CHAR*8>       'mJy/beam'
   VARIANCE       <ARRAY>         {structure}
      DATA(990,1036,1)  <_DOUBLE>    *,*,*,*,*,*,*,*,*,*,*,*,*,*,*,*,*,*,
                                     *,*,*,*,*,*,*,*,*,*,*,*,*,*,*,*,*,*,*
      ORIGIN(3)      <_INTEGER>      -494,-522,1

   WCS            <WCS>           {structure}
      DATA(221)      <_CHAR*32>      ' Begin FrameSet',' Nframe = 5',' Cu...'
                                     ... ' E...',' End CmpMap',' End
FrameSet'

   HISTORY        <HISTORY>       {structure}
      CREATED        <_CHAR*24>      '2011-JAN-10 21:41:27.349'
      CURRENT_RECORD  <_INTEGER>     12
      RECORDS(15)    <HIST_REC>      {array of structures}

      Contents of RECORDS(1)
         DATE           <_CHAR*24>      '2011-JAN-10 21:41:28.417'
         COMMAND        <_CHAR*30>      'MAKEMAP         (SMURF V1.1.0)'
         USER           <_CHAR*5>       'agibb'
         HOST           <_CHAR*7>       'aspersa'
         DATASET        <_CHAR*77>
'/scuba2/jcmtdata/reduced/scuba2_8...'
         TEXT(70)       <_CHAR*72>      'Parameters: ALIGNSYS=FALSE BBM=',
                                        ... '   s8d20091205_00022_0042,
s8...'

   MORE           <EXT>           {structure}
      SMURF          <SMURF_EXT>     {structure}
         WEIGHTS        <NDF>           {structure}
            DATA_ARRAY     <ARRAY>         {structure}
               DATA(990,1036)  <_DOUBLE>      *,*,*,*,*,*,*,*,*,*,*,*,*,*,
                                              *,*,*,*,*,*,*,*,*,*,*,*,*,*
               ORIGIN(2)      <_INTEGER>      -494,-522

            LABEL          <_CHAR*6>       'Weight'
            WCS            <WCS>           {structure}
               DATA(175)      <_CHAR*32>      ' Begin FrameSet',' Nframe =
5',
                                              ... ' End Cm...',' End
FrameSet'

            UNITS          <_CHAR*6>       'pW**-2'

         EXP_TIME       <NDF>           {structure}
            DATA_ARRAY     <ARRAY>         {structure}
               DATA(990,1036)  <_DOUBLE>      *,*,*,*,*,*,*,*,*,*,*,*,*,*,
                                              *,*,*,*,*,*,*,*,*,*,*,*,*,*
               ORIGIN(2)      <_INTEGER>      -494,-522

            LABEL          <_CHAR*19>      'Total exposure time'
            UNITS          <_CHAR*1>       's'
            WCS            <WCS>           {structure}
               DATA(175)      <_CHAR*32>      ' Begin FrameSet',' Nframe =
5',
                                              ... ' End Cm...',' End
FrameSet'

      PROVENANCE     <PROVENANCE>    {structure}
         ANCESTORS(45)  <PROV>          {array of structures}

         Contents of ANCESTORS(1)
            PATH           <_CHAR*74>
'/scuba2/jcmtdata/reduced/scuba...'
            DATE           <_CHAR*19>      '2011-01-10 21:48:49'
            CREATOR        <_CHAR*13>      'KAPPA:NDFCOPY'
            HISTORY(1)     <HISREC>        {structure}
               DATE           <_CHAR*24>      '2011-JAN-10 21:49:14.648'
               COMMAND        <_CHAR*30>      'NDFCOPY         (KAPPA
1.12-4)'
               USER           <_CHAR*5>       'agibb'
               TEXT           <_CHAR*228>     'Parameters: COMP='DATA'
EXT...'

            PARENTS(1)     <_INTEGER>      2

         DATE           <_CHAR*19>      '2011-01-10 21:48:49'
         CREATOR        <_CHAR*13>      'KAPPA:NDFCOPY'
         HASH           <_INTEGER>      789255632
         PARENTS(1)     <_INTEGER>      1

      FITS(201)      <_CHAR*80>      'TELESCOP= 'JCMT    '           / Na...'
                                     ... 'NBOLOEFF=     529.152083333333
/...'

End of Trace.
\end{terminalv}

\subsection{Log files}

The pipeline writes out a number of log files with the results of
various calculations made during the data reduction. The main science
recipe produces log files for the image noise and the noise-equivalent
flux density (NEFD), called \texttt{log.noise} and \texttt{log.nefd}
respectively. Both values are calculated for the individual
observation images and the final coadd. The noise is calculated as the
median error value (i.e.\ square-root of the variance) and given in
mJy\,beam$^{-1}$. The NEFD is calculated from the square-root of the
product of the variance and the exposure time NDF components, and the
median value taken as the representative NEFD (in mJy\,$\sqrt{\rm s}$).
