\section{\xlabel{procdetails}Processing and analysis details\label{procdetails}}

This section covers some aspects of the SCUBA-2 pipeline in detail.

\subsection{Matched filter}

The standard matched filter used by the SCUBA-2 pipeline is based on a
compensated PSF or Mexican-Hat wavelet technique
(e.g.\ \cite{mhwpaper}). The filter employs a two-component gaussian
based on the telescope beam determined in \cite{scuba2calpaper} to
determine the detection scale of the PSF. Both the map and PSF are
smoothed using a single, larger gaussian to remove a local background,
and the smoothed versions subtracted from each. The smoothing gaussian
has a FWHM of 30$''$ at 850\,$\mu$m, 20$''$ at 450\,$\mu$m. (For the
relevant \picard\ recipes, the FWHM of the smoothing gaussian may be
given as a recipe parameter.) The smoothed-and-subtracted input image
is convolved with the identically processed PSF to produce the output
image.

For those recipes that assess the point-source response as part of the
processing (e.g.\ the jack-knife-based recipes), the matched filter
will use a PSF derived from the images that include the artificial
source. In these cases, the map (and PSF) will not be smoothed by a
larger gaussian before the convolution.

\subsection{NEFD image calculation}

For image data, the pipeline calculates a corresponding image of the
noise equivalent flux density (NEFD), defined as the square-root of
the product of the exposure time and variance components. Thus each
pixel, $i$, in the NEFD image is given by:
\begin{equation}
{\rm NEFD}_i = \sqrt{(t_{{\rm exp},i} \sigma_i^2)}
\end{equation}
Since this image is calculated from components internal to the image,
the NEFD image is written as an additional NDF component under the
same extension as the exposure time and weights, i.e.\,
\verb+MORE.SMURF.NEFD+. Note that the calculation will overwrite any
existing component of the same name.

\subsection{Source-fitting}

\KAPPA\ \task{beamfit} is the main task used for fitting sources in
order to calculate beam size, pointing offsets and flux conversion
factors (FCFs). The facility exists (within \picard) to attempt to fit
a realistic beam using two (circular) gaussian components as
determined in \cite{scuba2calpaper}. The criterion is that the peak
signal-to-noise ratio (SNR) must exceed \snrmin. See the documentation
for \task{SCUBA2\_CHECK\_CAL} in \picardsun\ for further details.

When estimating the beam size, \task{beamfit} always assumes a
gaussian profile whether or not it is fitting two components. Fits to
the beam are always carried out in an Az-El coordinate frame so that
fits may be analyzed for systematic elongations.

For calculating pointing offsets, the peak position is most important
and the choice of profile has no effect on the result. FCF
calculations will use a single-component fit and the profile is left
as a free parameter.

\subsection{FCF calculations}

The pipeline calculates three FCFs to convert the uncalibrated data in
pW to astronomically-meaningful units:
\begin{itemize}
\item ARCSEC -- calibrate maps in surface brightness units,
  Jy\,arcsec$^{-2}$;
\item BEAM -- calibrate maps in Jy\,beam$^{-1}$;
\item BEAMMATCH -- calibrate maps processed with the matched filter in
  Jy.
\end{itemize}
All three of these FCFs are calculated in the
\verb+_FIND_CALIBRATION_MAP_+ primitive with the detailed calculation
of each carried out in the primitives specified below.

The combination of these FCFs can be used to assess telescope
performance. The ratio of the BEAM FCF to the ARCSEC FCF provides an
estimate of the effective solid angle of the telescope beam which can
be compared with the standard value derived in
\cite{scuba2calpaper}. If the telescope is well focussed, the two
should agree to within the calibration uncertainty. However, if the
focus is not optimal, the BEAM/ARCSEC ratio will yield a larger value.

Maps of calibrators are made with 1$''$ pixels at both 850- and
450\,$\mu$m which allows the fitting areas to be defined in terms of pixels.

\subsubsection{ARCSEC}

The ARCSEC FCF is calculated using aperture photometry
(\task{autophotom}) on a calibrator (using the
\verb+_APERTURE_PHOTOMETRY_+ primitive). The primary aperture is
30$''$ in radius (at both wavelengths) with a sky annulus defined
within 1.25--2.0 times the aperture radius. The known total flux of
the source is divided by the measured background-corrected flux to
yield the ARCSEC FCF in Jy\,arcsec$^{-2}$\,pW$^{-1}$.

The \task{autophotom} task is called with the following parameters:
\begin{terminalv}
biasle=0 centro padu=1 photon=3 positive skyest=2 nousemags nousemask
\end{terminalv}
The input file defining the source position and aperture properties
contains the following lines:
\begin{quote}
\verb+1 +$x$ $y$\verb+ 0.0 0.0 0.0 0.0 OK +$r_{\rm ap}$\verb+ 0.0 0.0 annulus circle+\\
\verb+#ANN 1 1.25 2.0+
\end{quote}
where $x$ and $y$ are the RA and Dec of the source (obtained from the
\verb+skyref+ WCS attribute) and $r_{\rm ap}$ is the radius of the
aperture in pixels.

The signal sum, $S$, is obtained from the \verb+SIGNAL+ entry (column
7) in the output file, which is converted to a total flux
(pW\,arcsec$^2$) using the pixel area, $F = S A_{\rm pix}$. The
uncertainty in this flux is derived from the \verb+MAG+ and
\verb+MAGERR+ entries (columns 4 and 5 respectively). With the
\verb+nousemags+ parameter, these values are counts, rather than
magnitudes and are thus a mean count ($\mu$) and uncertainty in that
value ($\delta \mu$). Then $\delta F = F \mu/\delta\mu$ (also in
pW\,arcsec$^2$).

\subsubsection{BEAM}

The BEAM FCF is obtained from the ratio of the known peak flux to the
fitted source peak to give the FCF in units of
Jy\,beam$^{-1}$\,pW$^{-1}$. The fitted peak is derived from \KAPPA\
\task{beamfit} called from the \verb+_FIT_SOURCE_+ primitive. If the
source has a SNR exceeding \snrmin\ the map is fitted by two
superposed gaussians to mimic the realistic telescope beam. The
fallback position is that a single (non-gaussian) component is fitted
if the SNR is less than \snrmin.

The arguments to \task{beamfit} for a single component fit are:
\begin{terminalv}
gauss=false mode=interface variance=false fitarea=15 fixback=0
\end{terminalv}
The \verb+pos+ parameter is set to either (0,0) for planets or the RA
and Dec of the reference position for stationary sources. (For a
two-component fit, the \verb+pos2+ parameter is the same as
\verb+pos+.)

The peak of the fit and its uncertainty are used to estimate the FCF
and the corresponding uncertainty directly. \task{beamfit} also
returns an estimate of the RMS deviation between the map and the fit;
however, since the FCF is derived from the peak of the fit, the
uncertainty in that value is preferred for estimating the uncertainty
in the FCF (although in practice the two tend to be similar).

\subsubsection{BEAMMATCH}

The BEAMMATCH FCF is obtained from the ratio of the known total flux
to the fitted source peak in an image processed by the matched filter,
to give the FCF in units of Jy\,pW$^{-1}$. \KAPPA\ \task{beamfit} is
used to fit a single component, though the fit is not constrained to
be a gaussian in order to estimate the peak as accurately as
possible. For point sources it should yield the same value as the BEAM
FCF. However, it is rarely used to calibrate data directly; the
application of a matched filter is usually carried out on images
calibrated with the BEAM FCF.

The \task{beamfit} arguments for deriving the BEAMMATCH FCF are:
\begin{terminalv}
gauss=false mode=interface variance=false fitarea=15 fixback=0
\end{terminalv}
where \verb+fitarea+ is the smaller of 1.5$\times$FWHM or 15 pixels. A
smaller fit area is used in order to limit the influence on the fit of
the negative dip associated with the matched filter. The \verb+pos+
parameter is the same as that used for the BEAM FCF.

As with the BEAM FCF, the uncertainty in the peak of the fit is used
to directly estimate the uncertainty in this FCF.

\subsection{Error beam}

The SCUBA-2 error beam is defined as the fraction of the total power
that lies outside of an aperture defined by the FWHM, i.e.: $E = 1 -
(S_{\rm main}/S_{\rm total})$. For the model JCMT beam, these values
are 0.57 and 0.67 at 850- and 450\,$\mu$m respectively.

The fluxes are calculated within apertures of radii equal to half the
FWHM and the standard radius for calculating the ARCSEC FCF above
(i.e.\ 30\,arcsec). The annulus used for the background estimate is
kept the same in both cases at 1.25 and 2.0 times the standard
aperture radius.

