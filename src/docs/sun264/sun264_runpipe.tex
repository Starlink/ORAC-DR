\section{\xlabel{runpipeline}Running the SCUBA-2 pipeline at the JCMT\label{se:runpipe}}

At the summit the pipeline is normally started by the telescope
support specialist (TSS) as normal user accounts do not have the
access privileges to write to the data output directories.

There are four pipelines running at the telescope, a QL and summit
version for each wavelength. Each pipeline runs on a separate data
reduction (DR) machine (\verb+sc2dr#+ where \verb+#+ is 1--4). Raw
data are stored in \verb+/jcmtdata/raw/scuba2/sXX/YYYYMMDD+, where
\verb+sXX+ is the subarray and \verb+YYYYMMDD+ is the current UT
date. Reduced data are written
to\\ \verb+/jcmtdata/reduced/dr#/scuba2_XXX/YYYYMMDD+ where \verb+dr#+
is the number of the machine running the pipeline and \verb+XXX+ is
either 850 or 450. The directory \verb+/jac_sw/oracdr-locations+
contains files that list the locations of the output directories for
each pipeline (and therefore which DR machine is processing which
pipeline). Note that the output directories are local to their host
computers (though they are NFS-mounted by the other DR machines).

Each pipeline waits for new data to appear before processing, and
processes all data automatically choosing the correct recipe based on
the observation type (which may be modified by the particular pipeline
being run).

\subsection{Prerequisites}

DRAMA must be running on the QL DR machines, and the DRAMA task names
must be defined. The task names are communicated through the
\verb+ORAC_REMOTE_TASK+ environment variable, which contains a
comma-separated list of names. The usual form of an individual task
name is \verb+TASK@server+, e.g., \verb+SC2DA8D@sc2da8d+. The task
name is in upper case; the machine name serving the parameter in lower
case.

\subsection{Running the QL pipeline}

The QL pipeline is started with the following commands (substitute 850
for 450 for the short wave pipeline in this and the summit pipeline
examples):
\begin{terminalv}
% oracdr_scuba2_850_ql
% oracdr &
\end{terminalv}
The QL pipeline is fed data via DRAMA parameters and must be told the
names of the tasks to expect data from, as described
above. QL-specific recipes will be used if present. A stripchart
program, which plots a number of quantities derived by the QL pipeline
as a function of time, is made available once the QL pipeline has been
initialized. Type \verb+xstripchart+ to run. (Note that the stripchart
is a separate task and is not part of the pipeline itself.)

\subsection{Running the summit pipeline}

The summit pipeline is started by:
\begin{terminalv}
% oracdr_scuba2_850_summit
% oracdr -loop flag -skip &
\end{terminalv}
The summit pipeline reads the data files from flag files, and skips
non-existent observations. Summit-specific recipes will be used if
present. Should the pipeline need restarting, the \verb+-from+
argument must be given to tell the pipeline the observation number it
should begin processing.

%\subsection{Monitoring the pipeline from a non-privileged account}
% Note: This doesn't work any more because the pipeline deletes some
% of the files it displays - the monitor relies on those files being
% persistent.
