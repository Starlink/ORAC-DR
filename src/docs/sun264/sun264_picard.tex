\section{\xlabel{picard}Analysis of processed data\label{picard}}

A variety of simple \picard\ recipes (see \picardsun) exist to perform
post-processing analysis.

SCUBA-2 specific \picard\ recipes begin with or contain the word
\verb+SCUBA2+; recipes specific to processing JCMT data contain
\verb+JCMT+. Documentation for each recipe is given in \picardsun, and
on the \picard\ home page,
\htmladdnormallink{\texttt{http://www.oracdr.org/oracdr/PICARD}}{http://www.oracdr.org/oracdr/PICARD}
where each recipe is fully documented.

Running \picard\ is simple. For example:
\begin{terminalv}
% picard -recpars <recipe_param_file> RECIPE <list_of_files>
\end{terminalv}
where \verb+<recipe_param_file>+ is a text file containing the
relevant recipe parameters, \verb+RECIPE+ is the name of the
recipe to run and \verb+<list_of_files>+ is the list of NDF files to
process, which must exist in the current directory. The output files
are written to the current directory, or the directory defined by
\verb+ORAC_DATA_OUT+.

Most recipes have one or more recipe parameters which can be specified
using the \texttt{-recpars} option. Recipe parameters are given in a
text file with the following format:
\begin{terminalv}
[RECIPE_NAME]
RECIPE_PARAMETER1 = value1
RECIPE_PARAMETER2 = value2
\end{terminalv}
The available recipe parameters are listed in the documentation on the
\picard\ home page above and in \picardsun.

The recommended approach for a few common tasks is detailed below.

\subsection{Coadding/mosaicking multiple images}

Although the pipeline will mosaic observations of the same target from
the same night, it is clearly desirable to combine data from multiple
nights. Alternatively, the user may wish to exert some additional
control over the mosaicking parameters.

The \task{MOSAIC\_JCMT\_IMAGES} recipe deals with processed JCMT data
(including ACSIS data cubes) and takes into account the
instrument-specific NDF components such as the exposure time
(\verb+EXP_TIME+). The choice of
coadding task is left to the user and may be either
\CCDPACK\ \task{makemos} or \KAPPA\ \task{wcsmosaic} (the default). If
using \task{makemos}, images are first aligned using
\KAPPA\ \task{wcsalign}. By default, the images (and additional NDF
components) are combined using a nearest-neighbour scheme but this may
be overridden by specifying the relevant parameter for
\task{wcsmosaic} or \task{wcsalign}.

The output mosaic takes its name from the last input file in the list
and has the suffix \verb+_mos+. The user should take care to ensure
this file does not already exist otherwise it will be overwritten.

\subsection{Registering images to a common centre\label{se:reg}}

Random pointing offsets and drifts between observations on a given
night (and over different nights) mean that the final mosaic of a
point source will not be optimal, and any faint surrounding structure
may be masked entirely.

The recipe \task{SCUBA2\_REGISTER\_IMAGES} is specific to SCUBA-2
data. The approach is to find the position of a given source in each
image and apply a shift to the WCS so that the peak fitted positions
are coincident for each image. If a suitable source exists in each
image, this recipe should be used before mosaicking (above).

Several recipe parameters are required, namely the coordinates of the
reference position. Currently only equatorial coordinates are
supported and must be written in sexagesimal format. The registered
images have the suffix \verb+_reg+.

As with the mosaicking recipe, this recipe knows about and takes care
of applying the same shift to the \verb+EXP_TIME+ and \verb+WEIGHTS+
(and \verb+NEFD+ if it exists) components, so the combined results are
accurate.

\subsection{Comparing noise with integration-time calculator\label{se:checkrms}}

A \picard\ recipe called \task{SCUBA2\_CHECK\_RMS} exists for making
the same assessments as the \oracdr\ recipe
\task{REDUCE\_SCAN\_CHECKRMS}. However it should be noted that this
recipe should be run \textit{only} on maps created from individual
observations: it will not give the correct answer for coadds. This is
because the coadds do not preserve the elapsed time, which makes it
impossible to use the SCUBA-2 integration time calculator (ITC).

A minor difference is that, with the exception of running at EAO, it
is not possible to determine NEP-based results. However, these are
generally less useful for comparing with the ITC.

The recipe produces the same output log file, \texttt{log.checkrms}
with the identical format. In order to calculate the NEFD, this recipe
will also create an \verb+NEFD+ image within the given map unless one
exists already.

%One enhancement that the \picard\ recipe has is the ability to take a
%matched-filtered image as input (provided it has been processed with
%the \picard\ recipe \task{SCUBA2\_MATCHED\_FILTER}). The recipe will
%adjust the regridding factors for the ITC as necessary.
