\documentclass[twoside,11pt]{article}

%+
%  Name:
%     sun232.tex

%  Purpose:
%     SUN documentation for ORAC-DR imaging (SUN/232)

%  Notes:
%     -  Has definitions to give small capitals for CCDPACK,
%     CONVERT, CURSA, GAIA, KAPPA, PHOTOM, PISA, and POLPACK.
%     Each command is \package.
%     -  Definition of \ORACDR as ORAC-DR in small capitals.
%     -  Definition of \FITSref to point to FITS Home Page.
%     -  SST has singled-sided name.
%     -  Added \pagebreak[3] and \goodbreaks to SST to prevent
%     section headings at the foot of a page.
%     -  \markboth in definition of \sstroutine to get running
%     header in reference section.
%     -  Used SUN/95 version of sstitemlist to remove
%     superfluous space after headings.  Added ssthitemlist
%     where the spacing is correct.
%     -  Uses some definitions of degrees, arcminutes,
%     arcsecond, underscore, and a centred asterisk.

%  Things to do:
%     - Example of -calib.

%  Authors:
%     MJC: Malcolm J. Currie (JACH)

%  History:
%     2000 February 2 (MJC):
%        Original version 232.1.
%     2000 August 30 (MJC):
%        232.2: Replaced deprecated recipes with generic (jitter
%        size and instruments) versions.
%     2000 October 26 (MJC):
%        232.3: Added descriptions of the photometry and results file
%        to the *_APHOT recipes' notes.  Updated the file suffices,
%        changes, and the Features of the Primitives since the original
%        submission of V2.0-0.  Move notes about using the pre-ORAC
%        data to an appendix for improved visibility.
%     2001 March 20 (MJC):
%        232.4: Updated for v2.1-0, including new recipes.

%  Copyright:
%     Copyright (C) 1998--2001 Particle Physics and Astronomy
%     Research Council.  All Rights Reserved.

%-

% ? Specify used packages
\usepackage{graphicx}        %  Use this one for final production.
%\usepackage[draft]{graphicx} %  Use this one for drafting.
% ? End of specify used packages

\pagestyle{myheadings}

% -----------------------------------------------------------------------------
% ? Document identification
% Fixed part
\newcommand{\stardoccategory}  {Starlink User Note}
\newcommand{\stardocinitials}  {SUN}
\newcommand{\stardocsource}    {sun\stardocnumber}

% Variable part - replace [xxx] as appropriate.
\newcommand{\stardocnumber}    {232.4}
\newcommand{\stardocauthors}   {Malcolm J. Currie\\
                               Joint Astronomy Centre, Hilo, Hawaii}
\newcommand{\stardocdate}      {2001 March 20}
\newcommand{\stardoccopyright} {Copyright \copyright\ 2001 Particle Physics and Astronomy Research Council}
\newcommand{\stardoctitle}     {ORAC-DR -- imaging data reduction}
\newcommand{\stardocversion}   {2.1-0}
\newcommand{\stardocmanual}    {User Guide}
\newcommand{\stardocabstract}  {{\footnotesize ORAC-DR} is a
general-purpose automatic data-reduction pipeline environment.  This
document describes its use to reduce imaging data collected at the
United Kingdom Infrared Telescope (UKIRT) with the UFTI and IRCAM
instruments.  It outlines the algorithms used and how to make minor
modifications of them, and how to correct for errors made at the
telescope.}

% ? End of document identification
% -----------------------------------------------------------------------------

% +
%  Name:
%     sun.tex
%
%  Purpose:
%     Template for Starlink User Note (SUN) documents.
%     Refer to SUN/199
%
%  Authors:
%     AJC: A.J.Chipperfield (Starlink, RAL)
%     BLY: M.J.Bly (Starlink, RAL)
%     PWD: Peter W. Draper (Starlink, Durham University)
%
%  History:
%     17-JAN-1996 (AJC):
%        Original with hypertext macros, based on MDL plain originals.
%     16-JUN-1997 (BLY):
%        Adapted for LaTeX2e.
%        Added picture commands.
%     13-AUG-1998 (PWD):
%        Converted for use with LaTeX2HTML version 98.2 and
%        Star2HTML version 1.3.
%     {Add further history here}
%
% -

\newcommand{\stardocname}{\stardocinitials /\stardocnumber}
\markboth{\stardocname}{\stardocname}
\setlength{\textwidth}{160mm}
\setlength{\textheight}{230mm}
\setlength{\topmargin}{-2mm}
\setlength{\oddsidemargin}{0mm}
\setlength{\evensidemargin}{0mm}
\setlength{\parindent}{0mm}
\setlength{\parskip}{\medskipamount}
\setlength{\unitlength}{1mm}

% -----------------------------------------------------------------------------
%  Hypertext definitions.
%  ======================
%  These are used by the LaTeX2HTML translator in conjunction with star2html.

%  Comment.sty: version 2.0, 19 June 1992
%  Selectively in/exclude pieces of text.
%
%  Author
%    Victor Eijkhout                                      <eijkhout@cs.utk.edu>
%    Department of Computer Science
%    University Tennessee at Knoxville
%    104 Ayres Hall
%    Knoxville, TN 37996
%    USA

%  Do not remove the %begin{latexonly} and %end{latexonly} lines (used by 
%  LaTeX2HTML to signify text it shouldn't process).
%begin{latexonly}
\makeatletter
\def\makeinnocent#1{\catcode`#1=12 }
\def\csarg#1#2{\expandafter#1\csname#2\endcsname}

\def\ThrowAwayComment#1{\begingroup
    \def\CurrentComment{#1}%
    \let\do\makeinnocent \dospecials
    \makeinnocent\^^L% and whatever other special cases
    \endlinechar`\^^M \catcode`\^^M=12 \xComment}
{\catcode`\^^M=12 \endlinechar=-1 %
 \gdef\xComment#1^^M{\def\test{#1}
      \csarg\ifx{PlainEnd\CurrentComment Test}\test
          \let\html@next\endgroup
      \else \csarg\ifx{LaLaEnd\CurrentComment Test}\test
            \edef\html@next{\endgroup\noexpand\end{\CurrentComment}}
      \else \let\html@next\xComment
      \fi \fi \html@next}
}
\makeatother

\def\includecomment
 #1{\expandafter\def\csname#1\endcsname{}%
    \expandafter\def\csname end#1\endcsname{}}
\def\excludecomment
 #1{\expandafter\def\csname#1\endcsname{\ThrowAwayComment{#1}}%
    {\escapechar=-1\relax
     \csarg\xdef{PlainEnd#1Test}{\string\\end#1}%
     \csarg\xdef{LaLaEnd#1Test}{\string\\end\string\{#1\string\}}%
    }}

%  Define environments that ignore their contents.
\excludecomment{comment}
\excludecomment{rawhtml}
\excludecomment{htmlonly}

%  Hypertext commands etc. This is a condensed version of the html.sty
%  file supplied with LaTeX2HTML by: Nikos Drakos <nikos@cbl.leeds.ac.uk> &
%  Jelle van Zeijl <jvzeijl@isou17.estec.esa.nl>. The LaTeX2HTML documentation
%  should be consulted about all commands (and the environments defined above)
%  except \xref and \xlabel which are Starlink specific.

\newcommand{\htmladdnormallinkfoot}[2]{#1\footnote{#2}}
\newcommand{\htmladdnormallink}[2]{#1}
\newcommand{\htmladdimg}[1]{}
\newcommand{\hyperref}[4]{#2\ref{#4}#3}
\newcommand{\htmlref}[2]{#1}
\newcommand{\htmlimage}[1]{}
\newcommand{\htmladdtonavigation}[1]{}

\newenvironment{latexonly}{}{}
\newcommand{\latex}[1]{#1}
\newcommand{\html}[1]{}
\newcommand{\latexhtml}[2]{#1}
\newcommand{\HTMLcode}[2][]{}

%  Starlink cross-references and labels.
\newcommand{\xref}[3]{#1}
\newcommand{\xlabel}[1]{}

%  LaTeX2HTML symbol.
\newcommand{\latextohtml}{\LaTeX2\texttt{HTML}}

%  Define command to re-centre underscore for Latex and leave as normal
%  for HTML (severe problems with \_ in tabbing environments and \_\_
%  generally otherwise).
\renewcommand{\_}{\texttt{\symbol{95}}}

% -----------------------------------------------------------------------------
%  Debugging.
%  =========
%  Remove % on the following to debug links in the HTML version using Latex.

% \newcommand{\hotlink}[2]{\fbox{\begin{tabular}[t]{@{}c@{}}#1\\\hline{\footnotesize #2}\end{tabular}}}
% \renewcommand{\htmladdnormallinkfoot}[2]{\hotlink{#1}{#2}}
% \renewcommand{\htmladdnormallink}[2]{\hotlink{#1}{#2}}
% \renewcommand{\hyperref}[4]{\hotlink{#1}{\S\ref{#4}}}
% \renewcommand{\htmlref}[2]{\hotlink{#1}{\S\ref{#2}}}
% \renewcommand{\xref}[3]{\hotlink{#1}{#2 -- #3}}
%end{latexonly}
% -----------------------------------------------------------------------------
% ? Document specific \newcommand or \newenvironment commands.

% degrees symbol
\newcommand{\dgs}{\hbox{$^\circ$}} 
\begin{htmlonly}
\newcommand{\dgs}{{\rawhtml &deg;}} 
\end{htmlonly}

% arcminute symbol
\newcommand{\arcm}{\hbox{$^\prime$}} 
\begin{htmlonly}
\newcommand{\arcm}{{\rawhtml &acute;}} 
\end{htmlonly}

% arcsec symbol
\newcommand{\arcsec}{\arcm\hskip -0.1em\arcm}
\begin{htmlonly}
\newcommand{\arcsec}{{\rawhtml &quot;}} 
\end{htmlonly}

% decimal-degree symbol
\newcommand{\udeg}{\hskip-0.3em\dgs\hskip-0.08em}
\begin{htmlonly}
\newcommand{\udeg}{{\rawhtml &deg;}} 
\end{htmlonly}

% decimal-arcsecond symbol
\newcommand{\uarcs}{\hskip-0.27em\arcsec\hskip-0.02em}  
\begin{htmlonly}
\newcommand{\uarcs}{{\rawhtml &quot;}}
\end{htmlonly}

% centre an asterisk
\newcommand{\lsk}{\raisebox{-0.4ex}{\rm *}}

\newcommand{\CCDPACK}{{\footnotesize CCDPACK}}
\newcommand{\CONVERT}{{\footnotesize CONVERT}}
\newcommand{\CURSA}{{\footnotesize CURSA}}
\newcommand{\GAIA}{{\footnotesize GAIA}}
\newcommand{\KAPPA}{{\footnotesize KAPPA}}
\newcommand{\ORACDR}{{\footnotesize ORAC-DR}}
\newcommand{\PHOTOM}{{\footnotesize PHOTOM}}
\newcommand{\PISA}{{\footnotesize PISA}}
\newcommand{\POLPACK}{{\footnotesize POLPACK}}
\newcommand{\FITSref}{\htmladdnormallink{FITS}{http://fits.gsfc.nasa.gov/}}

% +
%  Name:
%     SST.TEX

%  Purpose:
%     Define LaTeX commands for laying out Starlink routine descriptions.

%  Language:
%     LaTeX

%  Type of Module:
%     LaTeX data file.

%  Description:
%     This file defines LaTeX commands which allow routine documentation
%     produced by the SST application PROLAT to be processed by LaTeX and
%     by LaTeX2html. The contents of this file should be included in the
%     source prior to any statements that make of the sst commnds.

%  Notes:
%     The commands defined in the style file html.sty provided with LaTeX2html
%     are used. These should either be made available by using the appropriate
%     sun.tex (with hypertext extensions) or by putting the file html.sty
%     on your TEXINPUTS path (and including the name as part of the
%     documentstyle declaration).

%  Authors:
%     RFWS: R.F. Warren-Smith (STARLINK)
%     PDRAPER: P.W. Draper (Starlink - Durham University)

%  History:
%     10-SEP-1990 (RFWS):
%        Original version.
%     10-SEP-1990 (RFWS):
%        Added the implementation status section.
%     12-SEP-1990 (RFWS):
%        Added support for the usage section and adjusted various spacings.
%     8-DEC-1994 (PDRAPER):
%        Added support for simplified formatting using LaTeX2html.
%     {enter_further_changes_here}

%  Bugs:
%     {note_any_bugs_here}

% -

%  Define length variables.
\newlength{\sstbannerlength}
\newlength{\sstcaptionlength}
\newlength{\sstexampleslength}
\newlength{\sstexampleswidth}

%  Define a \tt font of the required size.
\latex{\newfont{\ssttt}{cmtt10 scaled 1095}}
\html{\newcommand{\ssttt}{\tt}}

%  Define a command to produce a routine header, including its name,
%  a purpose description and the rest of the routine's documentation.
\newcommand{\sstroutine}[3]{
   \goodbreak
   \markboth{{\stardocname}~ --- #1}{{\stardocname}~ --- #1}
   \rule{\textwidth}{0.5mm}
   \vspace{-7ex}
   \newline
   \settowidth{\sstbannerlength}{{\Large {\bf #1}}}
   \setlength{\sstcaptionlength}{\textwidth}
   \setlength{\sstexampleslength}{\textwidth}
   \addtolength{\sstbannerlength}{0.5em}
% Changed to -1.0 from -2.0 because there is only space for one banner.
   \addtolength{\sstcaptionlength}{-1.0\sstbannerlength}
% Changed to -2.5 from -5.0 because there is only space for one banner.
   \addtolength{\sstcaptionlength}{-2.5pt}
   \settowidth{\sstexampleswidth}{{\bf Examples:}}
   \addtolength{\sstexampleslength}{-\sstexampleswidth}
   \parbox[t]{\sstbannerlength}{\flushleft{\Large {\bf #1}}}
   \parbox[t]{\sstcaptionlength}{\center{\Large #2}}
%   \parbox[t]{\sstbannerlength}{\flushright{\Large {\bf #1}}}
   \begin{description}
      #3
   \end{description}
}

%  Format the description section.
\newcommand{\sstdescription}[1]{\item[Description:] #1}

%  Format the usage section.
\newcommand{\sstusage}[1]{\pagebreak[3] \item[Usage:] \mbox{} \\[1.3ex] {\ssttt #1}}

%  Format the invocation section.
\newcommand{\sstinvocation}[1]{\item[Invocation:]\hspace{0.4em}{\tt #1}}

%  Format the arguments section.
\newcommand{\sstarguments}[1]{
   \item[Arguments:] \mbox{} \\
   \vspace{-3.5ex}
   \begin{description}
      #1
   \end{description}
}

%  Format the returned value section (for a function).
\newcommand{\sstreturnedvalue}[1]{
   \item[Returned Value:] \mbox{} \\
   \vspace{-3.5ex}
   \begin{description}
      #1
   \end{description}
}

%  Format the parameters section (for a ORAC-DR recipe).
\newcommand{\sstparameters}[1]{
   \goodbreak
   \item[Configurable Steering Parameters:] \mbox{} \\
   \vspace{-3.5ex}
   \begin{description}
      #1
   \end{description}
}

%  Format the examples section.
\newcommand{\sstexamples}[1]{
   \goodbreak
   \item[Examples:] \mbox{} \\
   \vspace{-3.5ex}
   \begin{description}
      #1
   \end{description}
}

%  Define the format of a subsection in a normal section.
\newcommand{\sstsubsection}[1]{ \item[{#1}] \mbox{} \\}

%  Define the format of a subsection in the examples section.
\newcommand{\sstexamplesubsection}[2]{\sloppy \item[\parbox{\sstexampleslength}{\ssttt #1}] \mbox{} \\ #2 }

%  Format the notes section.
\newcommand{\sstnotes}[1]{\pagebreak[3] \item[Notes:] \mbox{} \\[1.3ex] #1}

%  Provide a general-purpose format for additional (DIY) sections.
\newcommand{\sstdiytopic}[2]{\goodbreak \item[{\hspace{-0.35em}#1\hspace{-0.35em}:}] \mbox{} \\[1.3ex] #2}

%  Format the implementation status section.
\newcommand{\sstimplementationstatus}[1]{
   \pagebreak[3] \item[{Implementation Status:}] \mbox{} \\[1.3ex] #1}

%  Format the bugs section.
\newcommand{\sstbugs}[1]{\item[Bugs:] #1}

%  Specify a variant of the itemize environment where the top separation
%  is reduced.  It is needed because a \vspace is ignored in the
%  \sstitemlist command.
\newenvironment{sstitemize}{%
  \vspace{-4.3ex}\begin{itemize}}{\end{itemize}}

%  Format a list of items while in paragraph mode.
\newcommand{\sstitemlist}[1]{
  \mbox{} \\
  \vspace{-3.5ex}
  \begin{sstitemize}
     #1
  \end{sstitemize}
}

%  Format a list of items while in paragraph mode, and where there
%  is a heading, thus the negative vertical space is not needed.
\newcommand{\ssthitemlist}[1]{
  \mbox{} \\
  \vspace{-3.5ex}
  \begin{itemize}
     #1
  \end{itemize}
}

%  Define the format of an item.
\newcommand{\sstitem}{\item}

%  Now define html equivalents of those already set. These are used by
%  latex2html and are defined in the html.sty files.
\begin{htmlonly}

%  sstroutine.
   \newcommand{\sstroutine}[3]{
      \subsection{#1\xlabel{#1}-\label{#1}#2}
      \begin{description}
         #3
      \end{description}
   }

%  sstdescription
   \newcommand{\sstdescription}[1]{\item[Description:]
      \begin{description}
         #1
      \end{description}
      \\
   }

%  sstusage
   \newcommand{\sstusage}[1]{\item[Usage:]
      \begin{description}
         {\ssttt #1}
      \end{description}
      \\
   }

%  sstinvocation
   \newcommand{\sstinvocation}[1]{\item[Invocation:]
      \begin{description}
         {\ssttt #1}
      \end{description}
      \\
   }

%  sstarguments
   \newcommand{\sstarguments}[1]{
      \item[Arguments:] \\
      \begin{description}
         #1
      \end{description}
      \\
   }

%  sstreturnedvalue
   \newcommand{\sstreturnedvalue}[1]{
      \item[Returned Value:] \\
      \begin{description}
         #1
      \end{description}
      \\
   }

%  sstparameters
   \newcommand{\sstparameters}[1]{
      \item[Configurable Steering Parameters:] \\
      \begin{description}
         #1
      \end{description}
      \\
   }

%  sstexamples
   \newcommand{\sstexamples}[1]{
      \item[Examples:] \\
      \begin{description}
         #1
      \end{description}
      \\
   }

%  sstsubsection
   \newcommand{\sstsubsection}[1]{\item[{#1}]}

%  sstexamplesubsection
   \newcommand{\sstexamplesubsection}[2]{\item[{\ssttt #1}] #2}

%  sstnotes
   \newcommand{\sstnotes}[1]{\item[Notes:]
      \begin{description}
         #1
      \end{description}
      \\
   }

%  sstdiytopic
   \newcommand{\sstdiytopic}[2]{\item[{#1}:]
      \begin{description}
         #2
      \end{description}
      \\
   }

%  sstimplementationstatus
   \newcommand{\sstimplementationstatus}[1]{\item[Implementation Status:]
      \begin{description}
         #1
      \end{description}
      \\
   }

%  sstitemlist
   \newcommand{\sstitemlist}[1]{
      \begin{itemize}
         #1
      \end{itemize}
      \\
   }

%  ssthitemlist
   \newcommand{\ssthitemlist}[1]{
      \begin{itemize}
         #1
      \end{itemize}
      \\
   }
\end{htmlonly}

%  End of "sst.tex" layout definitions.

% ? End of document specific commands
% -----------------------------------------------------------------------------
%  Title Page.
%  ===========
\renewcommand{\thepage}{\roman{page}}
\begin{document}
\thispagestyle{empty}

%  Latex document header.
%  ======================
\begin{latexonly}
   CCLRC / \textsc{Rutherford Appleton Laboratory} \hfill \textbf{\stardocname}\\
   {\large Particle Physics \& Astronomy Research Council}\\
   {\large Starlink Project\\}
   {\large \stardoccategory\ \stardocnumber}
   \begin{flushright}
   \stardocauthors\\
   \stardocdate
   \end{flushright}
   \vspace{-4mm}
   \rule{\textwidth}{0.5mm}
   \vspace{5mm}
   \begin{center}
   {\Huge\textbf{\stardoctitle \\ [2.5ex]}}
   {\LARGE\textbf{\stardocversion \\ [4ex]}}
   {\Huge\textbf{\stardocmanual}}
   \end{center}
   \vspace{5mm}

% ? Add picture here if required for the LaTeX version.
%   e.g. \includegraphics[scale=0.3]{filename.ps}
\begin{center}
\includegraphics[width=1.0in]{sun232_logo.eps}
\end{center}
% ? End of picture

% ? Heading for abstract if used.
   \vspace{10mm}
   \begin{center}
      {\Large\textbf{Abstract}}
   \end{center}
% ? End of heading for abstract.
\end{latexonly}

%  HTML documentation header.
%  ==========================
\begin{htmlonly}
   \xlabel{}
   \begin{rawhtml} <H1> \end{rawhtml}
      \stardoctitle\\
      \stardocversion\\
      \stardocmanual
   \begin{rawhtml} </H1> <HR> \end{rawhtml}

% ? Add picture here if required for the hypertext version.
\includegraphics[width=1.0in]{sun232_logo.eps}
%   e.g. \includegraphics[scale=0.7]{filename.ps}
% ? End of picture

   \begin{rawhtml} <P> <I> \end{rawhtml}
   \stardoccategory\ \stardocnumber \\
   \stardocauthors \\
   \stardocdate
   \begin{rawhtml} </I> </P> <H3> \end{rawhtml}
      \htmladdnormallink{CCLRC / Rutherford Appleton Laboratory}
                        {http://www.cclrc.ac.uk} \\
      \htmladdnormallink{Particle Physics \& Astronomy Research Council}
                        {http://www.pparc.ac.uk} \\
   \begin{rawhtml} </H3> <H2> \end{rawhtml}
      \htmladdnormallink{Starlink Project}{http://www.starlink.rl.ac.uk/}
   \begin{rawhtml} </H2> \end{rawhtml}
   \htmladdnormallink{\htmladdimg{source.gif} Retrieve hardcopy}
      {http://www.starlink.rl.ac.uk/cgi-bin/hcserver?\stardocsource}\\

%  HTML document table of contents. 
%  ================================
%  Add table of contents header and a navigation button to return to this 
%  point in the document (this should always go before the abstract \section). 
  \label{stardoccontents}
  \begin{rawhtml} 
    <HR>
    <H2>Contents</H2>
  \end{rawhtml}
  \htmladdtonavigation{\htmlref{\htmladdimg{contents_motif.gif}}
        {stardoccontents}}

% ? New section for abstract if used.
  \section{\xlabel{abstract}Abstract}
% ? End of new section for abstract
\end{htmlonly}

% -----------------------------------------------------------------------------
% ? Document Abstract. (if used)
%  ==================
\stardocabstract
% ? End of document abstract

% -----------------------------------------------------------------------------
% ? LateX Copyright Statement
%  =========================
\begin{latexonly}
\newpage
\vspace*{\fill}
\stardoccopyright
\end{latexonly}
% ? End of Latex copyright statement

% -----------------------------------------------------------------------------
% ? Latex document Table of Contents (if used).
%  ===========================================
  \newpage
  \begin{latexonly}
    \setlength{\parskip}{0mm}
    \tableofcontents
    \setlength{\parskip}{\medskipamount}
    \markboth{\stardocname}{\stardocname}
  \end{latexonly}
% ? End of Latex document table of contents
% -----------------------------------------------------------------------------
\cleardoublepage
\renewcommand{\thepage}{\arabic{page}}
\setcounter{page}{1}

\section{\xlabel{introduction}Introduction\label{introduction}}

\ORACDR\ is a data-reduction pipeline at UKIRT, part of the
\htmladdnormallink{ORAC system}{http://www.stsci.edu/stsci/meetings/adassVII/bridgera.html}
The pipeline reduces and displays multi-frame
observations soon after they are read from the detector.  This allows
observers to assess the quality and suitability of their data in near
real time.  Yet \ORACDR\ is capable of producing publication-quality
results.  

\ORACDR\ is suitable for `offline' data reduction at your home
institution too.  There are many reasons why you may wish to use
\ORACDR\ in this fashion.  For instance, you may have come back from
UKIRT with only the raw observations; or there was an error in a
telescope `exec' mixing the groups of observations; or some data were
reduced with a basic algorithm for speed at the telescope, and now you
want to do a more-careful job.  Hence \ORACDR\ is now available on
Starlink.

\xref{SUN/230}{sun230}{} presents an overview of \ORACDR,
general facilities like its display system, and it explains the
differences between a pipeline and a traditional reduction package.
Put briefly, \ORACDR\ uses a few data headers to direct the data
reduction.  Amongst these headers is the name of a {\em recipe}.  A
recipe is a series of high-level instructions such as ``make a
mosaic'' or ``divide by a flat'' that reduces an {\em observation\/}
comprising one or more data frames.  The implementation of each of
these instructions is through a Perl script---called a
primitive---which calls Starlink packages such as \CCDPACK\ and \KAPPA,
to actually do the processing of the bulk data.

This document describes how to use \ORACDR\ software on Starlink to
reduce data from the UKIRT imaging instruments: UFTI and IRCAM. It
outlines the various algorithms used in the recipes, and includes
detailed recipe documentation in the appendix.  Besides the standard
reduction recipes, this manual describes how you can customise recipes
to suit your preferences, and how to correct errors in the headers of
your data frames.

There is a complementary document \xref{SUN/231}{sun231}{} which
addresses the reduction of SCUBA data with \ORACDR.

% When SUN/234 is ready, remove the above sentence and these comments,
% and uncomment the three lines below.

%\ORACDR\ is not limited to processing UKIRT imaging data.  Complementary
%documents \xref{SUN/231}{sun231}{} and \xref{SUN/234}{sun234}{} address
%the reduction of SCUBA and CGS4 data respectively.

Those wishing wishing to write their own recipes from scratch, or wanting
to apply ORAC-DR to new instruments should consult \xref{SUN/233}{sun233}{}.

\section{\xlabel{using_the_pipeline}Using the pipeline\label{using_the_pipeline}}

\subsection{\xlabel{setting_up_orac-dr}Setting up \ORACDR\label{setting_up_orac-dr}}

Before you can run the pipeline you have to tell \ORACDR\ for which
instrument you wish to reduce data, the observation date, and the
directory containing the raw data, and wher you want the processed
data to be written.  There are two options.

\begin{itemize}

\item  The first needs your data to conform to the directory-naming
convention of the instrument at UKIRT.  This will be the case if you
simply unpack the archive written by the {\bf uktape} utility.  In
this case enter

\begin{verbatim}
      % setenv ORAC_DATA_ROOT <root_data_directory>
      % oracdr_<instrument> <date>
\end{verbatim}
where {\tt $<$root\_data\_directory$>$} is the directory in which you
unpacked the data from the tape, {\tt $<$instrument$>$} is either {\tt
ufti} or {\tt ircam}, and {\tt$<$date$>$} is the UT date in the format
YYYYMMDD.  Note that the \texttt{\%} represent the UNIX shell's
prompt, which you do not type.  The commands must be entered in the
above order.  

For example, the standard location for raw UFTI data is {\tt
raw/ufti/YYYYMMDD/}, and {\tt reduced/ufti/YYYYMMDD/} for the
corresponding reduced data.  So if your data are stored in {\tt
/home/users/abc/data/UKIRT/raw/ufti/20001108/} you should enter the
following.

\begin{verbatim}
      % setenv ORAC_DATA_ROOT  /home/users/abc/data/UKIRT/
      % oracdr_ufti 20001108
\end{verbatim}
to enable the pipeline for UFTI data taken on 2000 November 8.

\item The second option is where your raw and reduced data are to be
in arbitrary directories.  Type the following

\begin{verbatim}
      % oracdr_<instrument> <date>
      % setenv ORAC_DATA_IN <raw_data_directory>
      % setenv ORAC_DATA_OUT <reduced_data_directory>
\end{verbatim}

Here is an example for IRCAM data.
\begin{verbatim}
      % oracdr_ircam 19990328
      % setenv ORAC_DATA_IN /export/data/mjc/asteroid/night1
      % setenv ORAC_DATA_OUT /home/scratch/mjc/reduced
\end{verbatim}

\end{itemize}
In the first case {\tt \$ORAC\_DATA\_IN} and {\tt \$ORAC\_DATA\_OUT} are
still defined, but in terms of the root directory.  For instance, re-using
the earlier example with UFTI for UT date 2000 November 8,
{\tt \$ORAC\_DATA\_IN} points to {\tt \$ORAC\_DATA\_ROOT/raw/ufti/19991108/}.

\ORACDR\ operates in {\tt \$ORAC\_DATA\_OUT}, irrespective of
what your current directory is when you invoke it.  Raw data take the
form of multiple NDFs within an HDS container file: one NDF for the
data array and dynamic headers, such as the start time of the
exposure, and another for static headers.  Each container file is
converted to a single NDF in {\tt \$ORAC\_DATA\_OUT} with a merged set
of headers.  Your current directory remains unchanged.

It is highly recommended to work in directories on discs local to
the computer running the pipeline.  Processing over NFS-served drives
can many times slower and degrades the performance seen by other users.

\subsection{\xlabel{running_the_pipeline}Running the
pipeline\label{running_the_pipeline}}

To run the pipeline, you use the {\bf oracdr} command.  This has a
number of qualifiers described fully in \xref{SUN/230}{sun230}{oracdr}.
There is online help too; enter

\begin{verbatim}
      % oracdr -h
\end{verbatim}
for a list of the options.

Unlike using \ORACDR\ at UKIRT, you are unlikely to need the looping ({\tt -loop}
option) for offline processing, as all the data exist.  Thus the most
important qualifiers are {\tt -list} and {\tt -from}, which specify
the frames to process; and the recipe name.

\begin{verbatim}
      % oracdr -from 42
\end{verbatim}
will process frames f19991108\_00042 until the end of the night's data
(assuming the earlier {\bf orac\_ufti} command), running the recipes
given by each frame's header (RECIPE keyword).  More likely is that
you provide a list of selected observations.  The following example

\begin{verbatim}
      % oracdr -list 41:49,51:59 JITTER_SELF_FLAT
\end{verbatim}
processes frames from 41 to 49 inclusive and 51 to 59 inclusive,
invoking the {\tt JITTER\_SELF\_FLAT} recipe, and overriding
the RECIPE header.

\begin{verbatim}
      % oracdr -list 5:7,23,33
\end{verbatim}
woild reduce the frames 5, 6, 7, 23, and 33.  This is most likely
to be applicable to a series of dark frames.

There is a hazard with the {\tt -list} option.  Take care to select a
complete set of frames associated with an observation.  A common error
is to include accidently a dark frame not part of the group.  Check
the log in the raw data directory; it has file extension {\tt
.nightlog}.  If you do not have a log, it is easy to create one.

\begin{verbatim}
      % oracdr -noeng -from 1 -skip -nodisplay NIGHT_LOG
\end{verbatim}
This will create a log called {\tt \$ORAC\_DATA\_IN/$<$date$>$.nightlog}
for the current UT date.

\subsection{\xlabel{display}Display\label{display}}

\ORACDR\ optionally lets you inspect the raw frames, and the processed 
data as they are created happens.  There is a variety 
of graphics methods available, including histograms and contour plots,
if you choose a \KAPPA\ GWM widget.  Most people prefer a simple scaled
image display with \GAIA.  The selection of frame types to display, 
where they should appear, and how they are scaled are configurable
using a simple text file or a special GUI tool {\bf oracdisp}.  See
\xref{SUN/230}{sun230}{display_system} for details and examples.

Processing offline, there is less need to see the data displayed in real time.
If you wish to accelerate the processing switch off the display option.

\begin{verbatim}
      % oracdr -nodisplay ...
\end{verbatim}

If you do want to display a recommendation is to create two GAIA windows
displaying images using autoscaled limits.  This first could be for
raw and flat-fielded data, and the second for the mosaics.  You are
likely to want to interact with the latter, using \GAIA's toolboxes.
Your {\tt \$ORAC\_DATA\_OUT/disp.dat} could look like this.

\begin{verbatim}
      # Send raw frame to first GAIA window 
      num type=image tool=gaia region=0 window=0 autoscale=1 zautoscale=1

      # Send flatfielded frame to first GAIA window.
      ff  type=image tool=gaia region=0 window=0 autoscale=1 zautoscale=1

      # Send mosaic frame to second GAIA window.
      mos type=image tool=gaia region=0 window=1 autoscale=1 zautoscale=1
\end{verbatim}

\subsection{\xlabel{log_files}Log files\label{log_files}}

In addition to presenting the processing to the \ORACDR\ X-window,
\ORACDR, by default, retains a copy of the processing steps in a log
file.  These logs are important if something has gone wrong, and the
X-window has disappeared.  Information from the applications software
can be included if you run the pipeline with the {\tt -verbose}
command-line option.  Logs also serve as a record of the data
processing.  Yet the log files are often overlooked because the are
hidden.  The log file is called
{\tt\$ORAC\_DATA\_OUT/.oracdr\_$<$number$>$}, where {\tt$<$number$>$} is
the current process identification.  The {\tt -log f} option to the
{\bf oracdr} command enables log-file creation.

\section{\xlabel{features_of_the_primitives}Features of the
Primitives\label{features_of_the_primitives}}

Primitives are the Perl scripts which actually call the applications
to do most of the data processing.  All of the imaging recipes are
independent of the instrument.  Appropriate header information is
extracted for IRCAM and UFTI.

\subsection{\xlabel{preliminaries}Preliminaries\label{preliminaries}}

There a few operations applied to all frames.  The first sets the
origin of the frame so that frame pixels retain the detector pixel
indices.  It then becomes possible to use a full-sized bad-pixel mask
or flat field on any subset of a detector's pixel grid.

The second is to switch on history recording.  It is recommended to
leave this enabled, since it provides a record of the processing steps
of your final mosaics.  Otherwise the pipeline becomes something of 
a black box.  Use the \KAPPA\ command
\xref{{\bf hislist}}{sun95}{HISLIST} to list the history records.

IRCAM has two additional steps.  The pipeline has to copy the files
because the original files are normally write protected.  It finally
corrects for detector non-linearity.  The correction is $3.3~10^{-6}$
times the square of the bias-subtracted signal. 

\subsection{\xlabel{fits_headers}FITS headers\label{fits_headers}}

For historical reasons, IRCAM headers prior to ORAC were somewhat
jumbled, disorganised, and some violated the 
\htmladdnormallink{FITS standard}{http://archive.stsci.edu/fits/fits_standard/}.
The pipeline corrects, orders and structures the headers to help
the human reader locate information quickly, and allow complete 
conversion to FITS.

For both IRCAM and UFTI, even since ORAC came online, the raw headers
do not provide a sky co-ordinate system.  Using information in the raw
headers, the pipeline creates a FITS world co-ordinate system using
the AIPS convention.  This is quite adequate given the small fields of
view.  \GAIA\ and \KAPPA\ can display these co-ordinates on overlay
grids and axes.  For the latter package, say to run {\bf display}
with axes, you may need to select the appropriate astrometric Frame
like this.

\begin{verbatim}
     % wcsframe <your_NDF> frame=sky
\end{verbatim}
where you substitute your NDF's name for {\tt$<$your\_NDF$>$}.

Note that the reference pixel of the equatorial co-ordinates in the
raw headers is only known to a few arcseconds.  The pipeline sets
empirical reference pixels; see primitive \_CREATE\_WCS\_ for
details.  For critical work, you should tie in your frames with online
catalogues, as available through \GAIA.

\subsection{\xlabel{bad_pixels}Bad pixels\label{bad_pixels}}

The recipes apply a predetermined bad-pixel mask with the aim of
removing the bulk of `hot' and `cold' pixels.  This flags
approximately 0.4\% of UFTI pixels and 0.1\% of IRCAM pixels.
However, there are two problems.  First, the pre-calculated mask
only accounts for 95\% of UFTI's problem pixels.  The
\htmladdnormallink{other 5\% are occasionally deviant on timescales of
days}{http://www.jach.hawaii.edu/JACpublic/UKIRT/instruments/ufti/badpixels.html}.
The variability of IRCAM bad pixels is unknown at the time of writing.
In addition the bad-pixel masks have not been regularly monitored
prior to 2000 August.  The result is that non-physical values could
appear in the processed data, some as extreme as $-10^{-31}$ causing
automatic registration and image display to go awry.

Therefore, after dark subtraction, recipes apply thresholding which
flags non-phyical values as bad, meaning undefined.  This is
just augmenting the bad-pixel mask, and no valid data are lost.  The
upper limit is above the nominal saturation levels: 15000 for UFTI,
20000 for IRCAM in STARE or NDSTARE mode, and 33000 for IRCAM using
the Deepwell.  The lower limit is the 2-, 3-, 3-$\sigma$-clipped mean,
approximating to the mode, less five times the clipped standard
deviation.  While a positive threshold looks attractive, small
negative values, while appearing non-physical, can arise through noise.
Therefore, to avoid a bias (mainly in the $J$ band), a further
constraint is that the lower limit lies in the range $-$100 to 1.  

\subsection{\xlabel{flat_creation}Flat creation\label{flat_creation}}

Frames are optionally cleaned to remove extreme outliers ($\pm$3 or
6$\sigma$ about the mean in 15$\times$15-pixel neighbourhood) iterated
three times.  The data are then normalised, combined pixel by pixel
using a median, and the combined array normalised to a have a mean of
one.  It is possible to use another statistic for the combination,
such as a clipped median.

\subsection{\xlabel{object_masking}Object masking\label{object_masking}}

In recipes which make a flat using the frames taken of the targets,
the so-called self flat, any sources present can bias the flat field, and
result in blotchy mosaics.  The full versions of such recipes, as
opposed to the \_BASIC versions, and SKY\_MASKED greatly reduce these
artifacts using the following algorithm.  After the application of the
approximate self-flat field, an inventory is made of objects having at
least 12 connected pixels above one sigma above sky.  The locations,
shapes, orientations and sizes are used to make a mask.  The mask is
applied to the dark-subtracted frames and a new flat created.  As
the outer parts of bright objects often leave residual unmasked
blobs, a circular central occulting mask is used.  The diameter is
normally 7 arcseconds, but it can be adjusted through the OCCULT argument
of primitive \_MAKE\_OBJECTS\_MASK\_.  In the QUADRANT\_JITTER recipe
the central mask's diameter equals the length of the shorter side
of a quadrant.

The improved flat typically shows a uniformity at $\sim$0.02\% of the
sky. It is this flat which produces the flat-field frames for
mosaicking.  Systematic errors in the sky---a major uncertainty in
infra-red point-source photometry---are also reduced significantly by
this algorithm.

\subsection{\xlabel{automatic_registration}Automatic
Registration\label{automatic_registration}}

This makes an inventory of the sources above a threshold in each
frame.  It then performs a \xref{pattern
recognition}{sun139}{mosaicing} to identify common features in
jittered frames.  If the fraction of common objects is under 40\% or
the total is fewer than three, the registration fails, and so the
script resorts to reading the offsets stored in the \FITSref\ headers, or
matching a central bright object in certain recipes.  Using telescope
offsets can lead to trailed sources, as occurred with the
\xref{{\footnotesize IRCAMDR} package}{sun41}{}.  The improved
registration leads to the detection of fainter sources and more-accurate
measurement thereof.

To make use of the best information, registration using more than one
of the above methods is permitted.

\subsection{\xlabel{mosaicking}Mosaicking\label{mosaicking}}

This is fully automated.  No longer do you have to measure
painstakingly centroids and manually tile to form mosaics from the
jittered frames.  The jitter offsets are sufficiently small to permit
shift of Cartesian co-ordinates to register.  The offsets are derived
from the automatic registration.  No correction for the detector
orientation is applied, since it degrades the quality of the data,
despite the small rotation 
\begin{latexonly}
($\sim$1$\dgs$)
\end{latexonly}
\begin{htmlonly}
($\sim$1 degree)
\end{htmlonly}
normally involved.  Allowance for this misalignment with the cardinal
directions will be through the \htmladdnormallink{FITS world co-ordinate
system}{http://fits.gsfc.nasa.gov/documents.html#WCS}. In the meantime
should you need the rotation angle, pipe the output of KAPPA's
\xref{{\bf fitslist}}{sun95}{FITSLIST} function into grep.

\begin{verbatim}
      % fitslist <your_frame> | grep CROTA2
\end{verbatim}

The flat-fielded frame can either be resampled to give sub-pixel
registration, or to the nearest pixel. The latter is much faster and
is adequate for most uses of UFTI and IRCAM/TUFTI.  It also has the
advantage of not smoothing the data and introducing covariance
into the data errors.
For older
undersampled IRCAM3 data
\begin{latexonly}
(0.$\uarcs$286
\end{latexonly}
\begin{htmlonly}
(0.286 arcseconds
\end{htmlonly}
per pixel) resampling is recommended.

The mosaicking uses the \xref{\CCDPACK\ algorithm}{sun139}{mosaicing}
of its \xref{{\bf makemos}}{sun139}{MAKEMOS} command.  Only zero-point
shifts of intensity are applied to the resampled frames to create the
mosaic.  For most cases the comparison is of the sky levels as sky
pixels dominate.  This comparison is repeated for all pairs.  {\bf
makemos} then finds the most mutually consistent set of additive
corrections, weighting appropriately, to make the smoothest mosaic
given the data. The first frame, which for the UFTI execs is the
central (offset 0,0) frame, has no additive correction applied.  The
mosaic generation adjusts the zero-point of the jittered frames.
Another way of looking at it is that mosaicking attempts to remove the
sky variations.  The additive corrections are normally quite small,
like a few tenths of a count to a few counts.  However, over longer
integrations or in the thermal regime they can amount to a few score.
A mosaic pixel value is the mean of all the adjusted contributing
pixel values at that location.  It is possible to select other
statistics for the contributing pixels, such as the median, through
the METHOD argument of the \_MAKE\_MOSAIC\_ family of primitives.

There is no normalisation to counts per second in the mosaic. The
mosaic's signal corresponds to that of the first frame, thus the
exposure time of the mosaic is that of one individual frame.  The
recipes assume that you have used their corresponding `execs'
or`sequences', and hence have not changed the exposure time during a
jitter.  The exposure time (header EXP\_TIME [UFTI] or DEXPTIME
[IRCAM]) is propagated from the first frame to the mosaic.  Where
multiple frames combined to create a mosaic pixel the signal-to-noise
ratio corresponds to the combined integration time.  The integration
time (keyword INT\_TIME [UFTI] or EXPOSED [IRCAM]) is the number of
coadds times the exposure time per coadd.

Depending on the recipe, the mosaic may be trimmed to the dimensions
of a raw frame.  Mosaicking removes virtually all the bad pixels for
standard stars where the jitter offsets are small.

A mosaic forms for each cycle of the recipe, e.g. all four frames in a
QUADRANT\_JITTER.  For multiple cycles, an integrated `grand' mosaic
forms of improving signal-to-noise.  To avoid the build up of bad
pixels from cosmic rays, bad pixels are interpolated before the
addition.  This may result in some strange stripes in the top-left
corner of UFTI frames where no interpolation can occur.  Those pixels
are bad in all frames and should be ignored.  The EXP\_TIME or
DEXPTIME for the integrating mosaic is the sum of the exposure times
of the contributing mosaics.  Again the signal is not divided by the
exposure time.

\subsection{\xlabel{aperture_photometry}Aperture
Photometry\label{aperture_photometry}}

The recipes with an \_APHOT suffix perform aperture photometry on
the mosaic and the contributing flat-fielded frames.  The method
assumes that the target, usually a standard star, is approximately
centrally located after allowing for the jitter offsets.  If you have
data where the star lies outside the aperture, it is possible to
apply an offset.  See the XOFF and YOFF arguments of primitive
\_FIXED\_APERTURE\_PHOTOMETRY\_ in \_APHOT\_MAG\_ to adjust the
aperture's position.

The photometry is through a circular aperture located at the centroid
of the source, with the sky measured from a concentric annulus outside
the aperture.  The typical aperture size is 5 or 6 arcseconds,
depending on the recipe.  The annulus is typically 7 to 13 arcseconds
diameter.  The default estimator of the sky flux is the mode.  The
photometry accounts for fractional pixels at the aperture edge but
without allowance for the local gradient.

The magnitudes are given by the expression $-2.5 $\lsk$ $log10(abs(counts)
per second exposure time).  Therefore negative sources can be
measured too, as presented by the NOD recipes.  The photometry also
yields an internal error determined from the sky variance. 

A case- and space-insensitive comparison of the object name with the
entries in a table provides a catalogue magnitude for a standard star
in $I$, $Z$, $J$, $H$, or $K$ for both instruments, and in $L$ or $M$
for IRCAM.  Also a mean extinction is applied for the mean
of the start and end airmasses.  Thus the primitive calculates an
approximate zero point.  For accurate photometry the actual extinction
coefficients should be determined.  As the output from the photometry
is a \xref{small text list}{sun190}{STLREF}, you can use
\xref{{\bf catphotomfit} command}{sun190}{PHOTCAL} of the \CURSA\ package
to achieve this.  The units and meanings of the columns are documented
within each small text list.

The seeing is estimated for each frame and the mosaic by fitting a
two-dimensional Gaussian to the star, although in good seeing the
images are more centrally concentrated than a Gaussian.  The
full-width-half-maximum so derived is also tabulated in the small text
list.

\subsection{\xlabel{tidying}Tidying\label{tidying}}

Each recipe has a tidy procedure, which removes unnecessary
intermediate frames when the recipe no longer requireds them. Retained
are the raw data, flat-fielded frames, and the mosaics.  Most of the
intermediate small text files are removed in individual primitives,
but some do persist despite efforts to remove them.

\section{\xlabel{customising_recipes}Customising
Recipes\label{customising_recipes}}

If you wish to write your own data-reduction recipes, you should
consult the {\em \ORACDR\ Programmer's Guide\/}
(\xref{SUN/233}{sun233}{}).  However, for most purposes, observers
wishing to modify existing scripts can get by without this document.

\subsection{\xlabel{search_paths}Search paths\label{search_paths}}

\ORACDR\ allows you to create your own recipes and primitives, or
modify those provided as part of the package.  In either case you must
tell \ORACDR\ where your recipes and/or primitives are stored.  This
is achieved through two environment variables.  {\tt\$ORAC\_RECIPE\_DIR}
should equate to the directory containing your recipes.
{\tt \$ORAC\_PRIMITIVE\_DIR} specifies the directory containing your
primitives.  Here's an example.

\begin{verbatim}
      % setenv ORAC_RECIPE_DIR /home/user/drmoan/recipes
      % setenv ORAC_PRIMITIVE_DIR /home/user/drmoan/primitives
\end{verbatim}

Once these environment variables are defined, \ORACDR\ first looks in
{\tt \$ORAC\_RECIPE\_DIR} or {\tt \$ORAC\_PRIMITIVE\_DIR} to find a recipe or
primitive respectively.  If the script is absent, \ORACDR\ looks in
the standard {\tt \$ORAC\_DIR} directories.

\subsection{\xlabel{anatomy_of_an_imaging_recipe}Anatomy of an imaging recipe}

There are documentation modules---a Starlink style between \verb/#+/ and
\verb/#-/ delimiters at the head, and a Perl POD (Plain Old
Documentation) at the foot.  Between these is the code.  This consists
of calls to primitives, sometimes with arguments.  Primitives have
uppercase names preceded and terminated by underscores, such as
\_DIVIDE\_BY\_FLAT\_.

\subsubsection{\xlabel{hello_primitives}Hello primitives}

The first of these primitives is \_IMAGING\_HELLO\_.
It contains instrument-specific code and initialisation.  It is
best left alone.  See 
\begin{htmlonly}
\htmlref{preliminaries}{preliminaries}
\end{htmlonly}
\begin{latexonly}
Section~\ref{preliminaries}
\end{latexonly}
for a description of what these primitives do.

Second there is a recipe-specific primitive such as
\_JITTER\_SELF\_FLAT\_HELLO\_.  This sets up \CCDPACK.  Two things you
might wish to change in the recipe hello script are listed below.

\begin{itemize}

\item Change the extent of the images.  If there is an instrumental defect
in some peripheral rows you might not want to use the full bounds
as given by headers RDOUT\_X1 ($x$ lower bound), RDOUT\_X2
($x$ upper bound), and RDOUT\_Y1 and RDOUT\_Y2 for $y$.  Suppose you
wanted to trim off the top three rows you could change the line.

\begin{verbatim}
    my $y2 = $Frm->hdr( "RDOUT_Y2" );
\end{verbatim}
to
\begin{verbatim}
    my $y2 = $Frm->hdr( "RDOUT_Y2" ) - 3;
\end{verbatim}

\item Switch on error propagation.  To save time at the telescope, the pipeline
does not keep track of the errors per pixel (except for the
polarimetry recipes with names starting ``POL'').  If you wish to know
realistic errors for your data, in the line defining the hidden
variables for the \CCDPACK\ setup, change the parameter GENVAR's
value from {\tt{no}} to {\tt{yes}}.

\end{itemize}

\subsubsection{\xlabel{steering_primitive}Steering
primitive\label{steering_primitive}}

Third comes a `steering' primitive.  These are best left well alone.
They control when the various operations are performed.  What you can
safely adjust are the configurable steering parameters listed in the
recipe documentation.  In the main these parameters set the number of
frames processed in a pass through the recipe. 

Suppose that for some reason an observation of a nine-point jitter
self flat was aborted after seven positions.  If you try the recipe
stored in the headers some processing will occur, but it will not
include mosaic creation.  The final steps inclusing mosaic creation
occur once all nine frames are dark subtracted.  Now there are no
seven-point recipes to substitute on the command line.  You could make
your own seven-point recipe to reduce those data.  First make a new
recipe by copying the standard one.  Recipes are stored in
{\tt\$ORAC\_DIR/recipes/$<$instrument$>$}, where {\tt$<$instrument$>$} is
either {\tt UFTI} or {\tt IRCAM}.

\begin{verbatim}
    % cd $ORAC_RECIPE_DIR
    % cp $ORAC_DIR/recipes/UFTI/JITTER_SELF_FLAT ./JITTER7_SELF_FLAT
\end{verbatim}
Next edit JITTER7\_SELF\_FLAT and alter the line

\begin{verbatim}
      _JITTER_SELF_FLAT_STEER_
\end{verbatim}
to become

\begin{verbatim}
      _JITTER_SELF_FLAT_STEER_ NUMBER=7
\end{verbatim}
The recipe will then generate the self flat, flat field and make the
mosaic once the seventh frame is dark-subtracted.

\subsubsection{\xlabel{recipe_primitives}Recipe primitives\label{recipe_primitives}}

After the steering primitive we come to the recipe-specific scripts
that actually perform the recipe.  The most likely and easiest things
you would change are to add arguments or modify argument values of the
primitives in the recipes.  For instance, you might wish to change the
aperture diameter for the aperture photometry.  To alter to a
10-arcsecond aperture, change the APERTURE argument's value of the
\_APHOT\_MAG\_ primitive like below.

\begin{verbatim}
      _APHOT_MAG_ APERTURE=10
\end{verbatim}

To obtain details of a primitive's arguments, use the {\bf oracman}
command.  Thus

\begin{verbatim}
      % oracman _MAKE_MOSAIC_NO_BOUND_
\end{verbatim}
will display the documentation for primitive
\_MAKE\_MOSAIC\_NO\_BOUND\_.  Space does not permit inclusion of the
documentation of the many primitives in this manual.  The sources are
stored in {\tt \$ORAC\_DIR/primitives/$<$instrument$>$}, where
{\tt$<$instrument$>$} is either {\tt UFTI} or {\tt IRCAM}.

While the simplest primitives just invoke a Starlink task, and updates
just that and are amenable to customisation, some are quite complex
especially for the registration.  They may invoke other primitives,
manipulate parameters and small data files so that the various tasks
connect to cope with a variety of circumstances.  The most likely
change you will want to make is to change the parameter values of a
Starlink task.  Armed with the reference documentation for the
application, say with a {\tt findme $<$application$>$}, it is easy
to change values or append further parameters.  

Here is an example.  Let us suppose you wanted to combine frames to
make a self flat, not with the median, since you have heard that a mean
trimmed of the most-extreme tenth of the values gives better results.
First copy \_MAKE\_FLAT\_FROM\_GROUP\_ to your primitives directory.

\begin{verbatim}
      % cp $ORAC_DIR/primitives/UFTI/_MAKE_FLAT_FROM_GROUP_ $ORAC_PRIMITIVE_DIR
\end{verbatim}
Using an editor, find the first line in your copy of
\_MAKE\_FLAT\_FROM\_GROUP\_ commencing {\tt \$hidden}.  It should be
as follows.

\begin{verbatim}
      $hidden = "method=median sigmas=2.0 reset accept"; 
\end{verbatim}
Change this to

\begin{verbatim}
      $hidden = "method=trimmed alpha=0.1 sigmas=2.0 reset accept"; 
\end{verbatim}
to effect the change of statistic.  There is in fact a second line
assigning variable {\tt \$hidden} depending on argument CLEAN, and you
should make the same alteration there too.

If there is demand, additional arguments could be provided for
primitives, to simplify control.  Please contact the author if you
have suggestions for arguments and new recipes, or need help
customising your \ORACDR\ scripts.

\subsection{\xlabel{index_files}Index files\label{index_files}}

Once the pipeline has run for a bit you will find text files in {\tt
\$ORAC\_DATA\_OUT} called {\tt index.flat}, {\tt index.dark} amongst
others.  These list the calibration frames.  \ORACDR\ uses these to
find the most-recent, appropriate calibration.  For example, a flat
requires that the filter of the flat matches that of the frame being
flat fielded, and a dark must have the same exposure time as the
target frame; and both must have been taken in the same instrument
mode.

Here is an example of a flat index.

\begin{verbatim}
#FILTER MODE ORACTIME RDOUT_X1 RDOUT_X2 RDOUT_Y1 RDOUT_Y2 WPLANGLE
flat_Lp98_23 Lp98 STARE 13.3484 1 256 1 256 0
flat_K98_pol0_62 K98+pol NDSTARE 7.971 1 1024 1 1024 0.000
flat_K98_pol22_62 K98+pol NDSTARE 7.971 1 1024 1 1024 22.5
flat_K98_pol45_62 K98+pol NDSTARE 7.971 1 1024 1 1024 45
flat_K98_pol67_62 K98+pol NDSTARE 7.971 1 1024 1 1024 67.5
flat_J98_88 J98 flush_read 7.46111E+00 1 1024 1 1024 0
flat_H98_93 H98 flush_read 7.58330E+00 1 1024 1 1024 0
flat_K98_98 K98 flush_read 7.70559E+00 1 1024 1 1024 0
flat_H98_133 H98 flush_read 8.99034E+00 1 1024 1 1024 0
flat_H98_138 H98 flush_read 9.07139E+00 1 1024 1 1024 0
flat_H98_138_c1 H98 flush_read 9.12074E+00 1 1024 1 1024 0
flat_K98_290_row0 K98 flush_read 1.13094E+01 1 1024 1 1024 0
flat_K98_290_row1 K98 flush_read 1.15080E+01 1 1024 1 1024 0
\end{verbatim}
The first line contains the column headings.  ORACTIME is the UT in
decimal hours, and WPLANGLE is the polarisation waveplate angle.

In general you should not manipulate these files.  Mis-editing can
lead to the calibration system breaking down.  If you must edit this
file, say to exclude a poor dark or an uneven flat, restrict yourself
deleting the line corresponding to that calibration file.  It's safer
to remove the calibration file and recreate a new one with the
calibration frames you want by running the pipeline.

If you want to nominate specific calibration frames, overriding those
selected from the calibration indices, there is a {\tt -calib} option
for the {\bf oracdr} command to do this.  See
\begin{latexonly}
the section on
\end{latexonly}
\xref{calibration options}{sun230}{calibration_options}
\begin{latexonly}
in SUN/230
\end{latexonly}
for examples.

\section{\xlabel{correcting_headers}Correcting headers}

There are reasons why you may need to edit some of the FITS headers
used by \ORACDR.

\begin{itemize}
\item  At the summit of Mauna Kea, it's easy to make mistakes.  One of
the common ones is to make an error in the `exec' or `sequence'.  This
can cause, for example, frames to be in the wrong observation groups
or be assigned the wrong data-reduction recipe.  While the ORAC
Observation Tool has reduced the frequency of such errors, they will
not be eliminated.

\item You may have made some trial observations before taking making
a longer integration through several cycles of a recipe.  Now you wish
to combine all the observations of a target to obtain the best
signal-to-noise.  
\end{itemize}

The main headers to change are
\begin{description}
\item [{\tt RECIPE}]---the data-reduction recipe;
\item [{\tt NOFFSETS}]---the number of offsets;
\item [{\tt OBSNUM}]---the number of the frame, starting from 1 on
each night;
\item [{\tt GRPNUM}]---the group number, and should be
given by the frame number (OBSNUM) of its first member; and
\item [{\tt GRPMEM}]---whether or not the frame participates in group
processing.
\end{description}

For UFTI you can edit the raw FITS files.  {\tt \$ORAC\_DIR/bin/fitsmod.pl}
is a documented example Perl script to edit FITS headers.  The intention
is for you to make a copy and edit to suit your particular header-editing
requirements.

For IRCAM and UFTI, it's possible to edit the NDF's
\xref{FITS extension}{sun95}{se_fitsairlock} using \KAPPA's
\xref{{\bf fitsmod}}{sun95}{FITSMOD} command.  The command is a bit
long and the author regrets not defining a {\bf fitsupdate} synonym.
The following changes the GRPNUM keyword to have value 36 in the raw
NDF frame f19991108\_00042.

\begin{verbatim}
      % fitsmod f19991108_00042 grpnum u 36 \$C !
\end{verbatim}

It's possible to edit many files using a \xref{C-shell}{sc4}{} or Perl
script to edit a series of files very quickly.  If you do, it's better
to specify the values by keyword instead of position, like this

\begin{verbatim}
      % fitsmod ndf=f19991108_00042 edit=update keyword=grpnum value=36 \
                position=! comment=\$C
\end{verbatim}
because it is better insulated against change to {\bf fitsmod}.

\section{Acknowledgments}

\ORACDR\ was developed at the Joint Astronomy Centre by Frossie
Economou and Tim Jenness in collaboration with the UK Astronomy
Technology Centre as part of the ORAC project.  I should like to
thanks to members of the ORAC team, UKIRT staff and observers who made
suggestions for new or improved recipes.  Special thanks go to Gillian
Wright and Sandy Leggett for defining the initial specifications of
the UFTI scripts, and for subsequent discussions.  Chris Davis kindly
supplied the specifications of the polarimetry and Fabry-Perot
recipes.  Thanks also to Frossie Economou and Tim Jenness for
answering my \ORACDR\ and Perl questions, and for incorporating my
requested enhancements into \ORACDR.

The application engines used in \ORACDR\ were supplied by the Starlink
Project, which is run by CCLRC on behalf of PPARC.  I should like to
thank the Starlink programmers for their excellent support, especially
for quickly providing enhancements to tasks.

\section{Copyright and License}

\ORACDR\ is copyright \copyright 1998--2001 PPARC (the UK Particle Physics
and Astronomy Research Council).  It is distributed by Starlink
under the GNU General Public License as published by the Free Software
Foundation.

Whenever you have used \ORACDR\ as part of a publication, please give
an acknowledgment to \ORACDR\ in the paper.  This will help us assess
the usage of \ORACDR.

\newpage
\appendix
\section{\xlabel{processing_data_obtained_before_2000_august}Processing data obtained before 2000 August\label{
processing_data_obtained_before_2000_august}}%
\index{processing_data_obtained_before_2000_august}

Before the introduction of
\htmladdnormallink{ORAC}{http://www.stsci.edu/stsci/meetings/adassVII/bridgera.html}
on 2000 August 1, UFTI raw data were in \FITSref\ format, IRCAM data had a
different naming convention, and there were different default paths for
{\tt\$ORAC\_DATA\_IN} and {\tt\$ORAC\_DATA\_OUT}.  IRCAM's NDFs were
copied to {\tt \$ORAC\_DATA\_OUT}, and UFTI's raw FITS files were
converted to NDFs in that directory.

To process data from this era, follow the instructions 
\begin{htmlonly}
\htmlref{for using the pipeline}{setting_up_orac-dr}
\end{htmlonly}
\begin{latexonly}
of Section~\ref{setting_up_orac-dr}
\end{latexonly}
except you should invoke 
\begin{verbatim}
      % oracdr_<instrument>_old <date>
\end{verbatim}
instead of
\begin{verbatim}
      % oracdr_<instrument> <date>
\end{verbatim}
to set up the necessary environment variables.  The rest is the same.

You can use modern jitter-generic recipes too, provided they know how
many frames to process.  The easiest way to do that is make your own
copy of the recipe and set the number frames as an argument to the
steering primitive.  See
\begin{htmlonly}
\htmlref{here}{steering_primitive}
\end{htmlonly}
\begin{latexonly}
Section~\ref{steering_primitive}
\end{latexonly}
for details.

The standard raw and reduced directories prior to 2000 August were
\linebreak {\tt $<$instrument$>$\_data/YYYYMMDD/raw/}, and {\tt
$<$instrument$>$\_data/YYYYMMDD/reduced/} respectively, where
{\tt$<$instrument$>$} was either {\tt ufti} or {\tt ircam}.

Details of the former naming convention for IRCAM frames is given under
the ``Output Data'' headings in the \htmlref{reference section}{recipes}.

\newpage
\section{\xlabel{file_suffices}File suffices}\label{file_suffices}%
\index{file_suffices}

\subsection*{Description}%
\index{Description}

Files generated during \ORACDR\ imaging data reduction have suffices
denoting the processing step that created them.  This appendix
contains a list with short descriptions of what they mean.  Most will
be removed once a recipe has finished using them.  So you will
probably only see these files if you list the contents of directory
{\tt \$ORAC\_DIR} while the pipeline is running, or you interrupt the
pipeline with {\tt CTRL/C}, or something has gone wrong with the the
recipe and pipeline has aborted.

\subsection*{Frame suffices}%
\index{UFTI Frame suffices}

\begin{tabular}{llp{90mm}}
\hline
  Suffix  & Stands for        & Description \\ \hline
{\tt\_bp} & Bad Pixel         & Co-added with the bad-pixel mask \\
{\tt\_cl} & CLone             & Modifiable copy of IRCAM raw data \\
{\tt\_db} & De-Biassed        & The bias is actually zero, but it sets
                                up various \xref{\CCDPACK}{sun139}{}
                                ancillary data for later processing \\
{\tt\_dg} & De-Glitched       & Bad pixels replaced by median of neighbours \\
{\tt\_dk} & DarK              & Dark subtracted \\
{\tt\_ess} & E-beam Sky Subtracted & Polarimetry target e-beam after sky subtraction \\
{\tt\_ff} & Flat Field        & Divided by the flat field \\
{\tt\_fm} & Flat Masked       & This has the flagged deviant pixels
                                detected by the initial flat-field creation
                                restored after object masking \\
{\tt\_md} & Masked Deviants   & Deviant pixels from the neighbourhood (usually
                                3~$\sigma$ in 15$\times$15-pixel region) flagged as bad \\
{\tt\_nl} & Non-Linearity     & The standard non-linearity correction has
                                been applied (IRCAM only) \\
{\tt\_nm} & Normalised to Mode& Normalised masked frames combined to make the flat field \\
{\tt\_om} & Objects Masked    & This has sources masked with bad values
                                so that they do not bias the self flat field \\
{\tt\_oss} & O-beam Sky Subtracted & Polarimetry target o-beam after sky subtraction \\
{\tt\_qm} & Quadrant Masked   & One of the quadrants is masked with bad
                                pixels, created in QUADRANT\_JITTER \\
{\tt\_sbp}& Substitute Bad Pixels & Bad pixels replaced (needed for \PISA) \\
{\tt\_sc} & SCaled            & Data scaled to lie within the range of values
                                allowed by \xref{\PISA}{sun109}{} for the object 
                                masking \\
{\tt\_ss} & Sky Subtracted    & Global or local sky subtraction applied \\
{\tt\_th} & THresholded       & Non-physical values set to bad \\
{\tt\_trn}& TRaNsform         & The transformed or resampled data immediately 
                                prior to making a mosaic \\
\end{tabular}
\vspace*{17ex}

\subsection*{UFTI Group suffices}%
\index{UFTI Group suffices}
\begin{tabular}{llp{80mm}}
\hline
  Suffix   & Stands for             & Description \\ \hline
{\tt\_I}   & Intensity              & Polarisation intensity \\
{\tt\_mos} & Mosaic                 & Final mosaic \\
{\tt\_mu}  & Mosaic (Unfiltered)~~~ & Intermediate mosaic (could contain
                                      bad/hot pixels) \\
{\tt\_P}   & Percentage             & Percentage polarisation \\
{\tt\_PI}  & Polarisation Intensity & \\
{\tt\_Q}   & Stokes Q               & Stokes $Q$ parameter \\
{\tt\_sp}  & Stokes Parameters      & Data cube of Stokes parameters \\
{\tt\_TH}  & THeta                  & Polarisation angle \\
{\tt\_U}   & Stokes U               & Stokes $U$ parameter \\
\end{tabular}

\newpage
\section{\xlabel{recipes}Recipes\label{recipes}}

The original set of recipes and names were prescribed in
G.S. Wright \& S.K. Leggett, 1997, {\em Scripts for UFTI}, 
\htmladdnormallink{orac009-ufts, v01}
{http://www.jach.hawaii.edu/JACpublic/UKIRT/software/orac/docs/orac009-ufts-1.html}.

\subsection{\xlabel{classified_recipes}Classified Recipes}

In hindsight you may decide that there was a better recipe for your
data than stored in the RECIPE header.  Here is a classified list so
that you can select an alternative.  Magnitudes and dimensions
apply to UFTI, except for the NOD recipes, whose magnitude ranges
are for IRCAM/TUFTI.  For IRCAM dimensions are 22\% smaller.  The
magnitude ranges are courtesy of Sandy Leggett.
\bigskip\bigskip

\begin{center}
\begin{tabular}{|l|l|p{77mm}|}
\multicolumn{3}{c}{\large{\bf Calibration Recipes}} \vspace*{1ex} \\
\hline
Recipe Name & Type of Data & Function and Comments \\ \hline
\htmlref{ARRAY\_TESTS}{ARRAY\_TESTS} & Array check &
   Calculates read noise and dark current \\ \hline
\htmlref{REDUCE\_DARK}{REDUCE\_DARK} & Dark &
   Files a frame as the current dark. \\ \hline
\htmlref{SKY\_FLAT}{SKY\_FLAT} & Flat &
   Creates and files a flat field derived from five jittered frames.
   Mostly for use with BRIGHT\_POINT\_SOURCE recipes. Requires a
   dark.\\ \hline
\htmlref{SKY\_FLAT\_FP}{SKY\_FLAT\_FP} & Flat &
   Creates a Fabry-Perot sky flat (from jittered blank-sky exposures) 
   FP at on- and off-line wavelengths.  Requires a dark. \\ \hline
\htmlref{SKY\_FLAT\_MASKED}{SKY\_FLAT\_MASKED} & Flat &
   As SKY\_FLAT but masks objects to give a better flat field. \\ \hline
\htmlref{SKY\_FLAT\_POL}{SKY\_FLAT\_POL} & Flat &
   Obtain a `master' polarimetry flat-field from the median average
   of eight jittered frames; the waveplate is cycled after
   every second frame.  Makes a copy of the flat for each
   waveplate angle. Requires a dark.\\ \hline
\htmlref{SKY\_FLAT\_POL\_ANGLE}{SKY\_FLAT\_POL\_ANGLE} & Flat &
   Obtain four polarimetry flat-fields, one for each waveplate angle,
   from the median average of jittered frames. Requires a dark.\\ \hline
\end{tabular}
\end{center}
\bigskip

\begin{center}
\begin{tabular}{|p{37mm}|l|p{86mm}|}
\multicolumn{3}{c}{\large{\bf Miscellaneous recipes}} \vspace*{1ex} \\
\hline
Recipe Name & Type of Data & Function and Comments \\ \hline
\htmlref{NIGHT\_LOG}{NIGHT\_LOG} & &
   Generates a text-file log of a series of observations.\\ \hline
\end{tabular}
\end{center}

\begin{center}
\begin{tabular}{|l|p{24mm}|p{57mm}|}
\multicolumn{3}{c}{\large{\bf Very-bright-point-source recipes}} \vspace*{1ex} \\
\hline
Recipe Name & Type of Data & Function and Comments \\ \hline
\htmlref{BRIGHT\_POINT\_SOURCE}{BRIGHT\_POINT\_SOURCE} & 
   \mbox{$IZ<13$}, \mbox{$JHK<9$}, and bright \mbox{$13<IZ<17$}, \mbox{$9<JHK<15$} & 
   Normally a 5-point jitter but would be usable as 3-point.  Requires a
   separate flat as the background is too low to self flat, and a dark. \\ \hline
\htmlref{BRIGHT\_POINT\_SOURCE\_APHOT}{BRIGHT\_POINT\_SOURCE\_APHOT} & &
   As BRIGHT\_POINT\_SOURCE, but also performs aperture photometry of the
   source. \\ \hline
\end{tabular}
\end{center}
\bigskip

\begin{center}
\begin{tabular}{|l|p{25mm}|p{61mm}|}
\multicolumn{3}{c}{\large{\bf Bright-point-source recipes}} \vspace*{1ex} \\
\hline
Recipe Name & Type of Data & Function and Comments \\ \hline
\htmlref{JITTER\_SELF\_FLAT}{JITTER\_SELF\_FLAT} &
   \mbox{$13<I$,$Z<17$}, \mbox{$9<JHK<15$} & Standard jitter, self
   flats. Normally 5-point jitter.  Requires a dark. \\ \hline
\htmlref{JITTER\_SELF\_FLAT\_APHOT}{JITTER\_SELF\_FLAT\_APHOT} & &
   As JITTER\_SELF\_FLAT, but also performs aperture photometry of the
   source. \\ \hline
\htmlref{JITTER\_SELF\_FLAT\_NO\_MASK}{JITTER\_SELF\_FLAT\_NO\_MASK} & &
   As JITTER\_SELF\_FLAT but faster as it lacks object masking.  It
   only suitable for uncrowded fields. \\ \hline
\htmlref{SKY\_AND\_JITTER}{SKY\_AND\_JITTER} & &
   A sky frame and jitter on target.  The sky is subtracted from the
   target frame before flat fielding.  Requires a separate flat,
   as the background is too low to self flat, and a dark.
   No longer recommended as sky varies too quickly. \\ \hline
\htmlref{SKY\_AND\_JITTER\_APHOT}{SKY\_AND\_JITTER\_APHOT} & &
   As SKY\_AND\_JITTER, but also performs aperture photometry of the
   source.  \\ \hline
\end{tabular}
\end{center}
\bigskip

\begin{center}
\begin{tabular}{|l|p{26mm}|p{61mm}|}
\multicolumn{3}{c}{\large{\bf Faint-point-source recipes}} \vspace*{1ex} \\
\hline
Recipe Name & Type of Data & Function and Comments \\ \hline
\htmlref{JITTER\_SELF\_FLAT}{JITTER\_SELF\_FLAT} & Faint
   $I$,$Z>17$, $JHK>15$ & Standard jitter, self flats.  Normally 9-point
   jitter.  Requires a dark. \\ \hline
\htmlref{JITTER\_SELF\_FLAT\_BASIC}{JITTER\_SELF\_FLAT\_BASIC} & &
   Fastest JITTER\_SELF\_FLAT recipe as it lacks object masking, automatic
   registration and resampling. \\ \hline
\htmlref{JITTER\_SELF\_FLAT\_NO\_MASK}{JITTER\_SELF\_FLAT\_NO\_MASK} & &
   As JITTER\_SELF\_FLAT, but faster as it lacks object masking.  It
   only suitable for uncrowded fields. \\ \hline
\end{tabular}
\end{center}

\newpage
\begin{center}
\begin{tabular}{|l|p{25mm}|p{51mm}|}
\multicolumn{3}{c}{\large{\bf Point-source recipes---thermal}} \vspace*{1ex} \\
\hline
Recipe Name & Type of Data & Function and Comments \\ \hline
\htmlref{NOD\_SELF\_FLAT\_NO\_MASK}{NOD\_SELF\_FLAT\_NO\_MASK} &
   Bright $L<10$; or all $M$, faint $L>10$ &
   Nod jitter, self flats of differenced pairs of frames.  Has superior
   and fast sky subtraction.  No object masking.  Requires a dark.
   Use 4-point jitter for $L<10$, and 8-point for fainter $L$ and all
   $M$. \\ \hline
\htmlref{NOD\_SELF\_FLAT\_NO\_MASK\_APHOT}{NOD\_SELF\_FLAT\_NO\_MASK\_APHOT} & &
   As the previous recipe, but also performs aperture photometry of the
   positive and negative sources.  \\ \hline
\end{tabular}
\end{center}

\begin{center}
\begin{tabular}{|l|p{32mm}|p{54mm}|}
\multicolumn{3}{c}{\large{\bf Extended-source recipes}} \vspace*{1ex} \\
\hline
Recipe Name & Type of Data & Function and Comments \\ \hline
\htmlref{QUADRANT\_JITTER}{QUADRANT\_JITTER} & Galaxies, quasars and
   nebulae of small ($<$45 arcsec) angular extent & 4-point jitter; masks
   the quadrant containing the target to make the flat, then
   masks the objects. Requires a dark.\\ \hline
\htmlref{QUADRANT\_JITTER\_NO\_MASK}{QUADRANT\_JITTER\_NO\_MASK} & &
   As QUADRANT\_JITTER but without object masking. \\ \hline
\htmlref{QUADRANT\_JITTER\_BASIC}{QUADRANT\_JITTER\_BASIC} & &
   Fastest QUADRANT\_JITTER variant as it lacks object masking, automatic
   registration and resampling. \\ \hline
\htmlref{MOVING\_QUADRANT\_JITTER}{MOVING\_QUADRANT\_JITTER} & 
   Compact comets ($<$45 arcsec) &
   As QUADRANT\_JITTER, but uses ephemeris data to track the
   non-sidereal source.\\ \hline
\htmlref{QUADRANT\_JITTER\_TELE}{QUADRANT\_JITTER\_TELE} & &
   As QUADRANT\_JITTER but uses telescope offsets for registration.
   Telescope tracks object. \\ \hline
\htmlref{EXTENDED\_3x3}{EXTENDED\_3x3} & Galaxies and nebulae with
   angular extent $<$2 arcminutes & Sky-subtracted 3$\times$3 grid
   mosaic on target.  Frames alternate between sky and target. 
   Requires a dark.\\ \hline
\htmlref{EXTENDED\_3x3\_BASIC}{EXTENDED\_3x3\_BASIC} & &
   As EXTENDED\_3x3 but lacks resampling and registers using telescope
   offsets. \\ \hline
\htmlref{EXTENDED\_5x5}{EXTENDED\_5x5} & Galaxies and nebulae with
   angular extent $<$3 arcminutes & Sky-subtracted 5$\times$5 grid 
   mosaic of the target.  Frames alternate between sky and target.
   Requires a dark. \\ \hline
\htmlref{EXTENDED\_5x5\_BASIC}{EXTENDED\_5x5\_BASIC} & &
   As EXTENDED\_5x5 but lacks resampling and registers using telescope
   offsets. \\ \hline
\end{tabular}
\end{center}

\begin{center}
\begin{tabular}{|l|p{14mm}|p{61mm}|}
\multicolumn{3}{c}{\large{\bf Moving (non-sidereal) source recipes}} \vspace*{1ex} \\
\hline
Recipe Name & Type of Data & Function and Comments \\ \hline
\htmlref{MOVING\_JITTER\_SELF\_FLAT}{MOVING\_JITTER\_SELF\_FLAT} &
   Minor planets, comets &
   As JITTER\_SELF\_FLAT, but uses ephemeris data to track the non-sidereal
   source. \\ \hline
\htmlref{MOVING\_JITTER\_SELF\_FLAT\_BASIC}{MOVING\_JITTER\_SELF\_FLAT\_BASIC} &  &
   As JITTER\_SELF\_FLAT\_BASIC, but uses ephemeris data to track the
   non-sidereal source.\\ \hline
\htmlref{JITTER\_SELF\_FLAT\_TELE}{JITTER\_SELF\_FLAT\_TELE} & &
   Standard jitter, using telescope offsets.  This is needed
   when the telescope has tracked on the non-sidereal target. Requires
   a dark.\\ \hline
\htmlref{MOVING\_QUADRANT\_JITTER}{MOVING\_QUADRANT\_JITTER} & Compact comets &
   As QUADRANT\_JITTER, but uses ephemeris data to track the
   non-sidereal source.\\ \hline
\htmlref{QUADRANT\_JITTER\_TELE}{QUADRANT\_JITTER\_TELE} & &
   As QUADRANT\_JITTER but uses telescope offsets for registration.  This is
   needed when the telescope has tracked on the non-sidereal target.\\ \hline
\end{tabular}
\end{center}
\bigskip

\begin{center}
\begin{tabular}{|l|p{25mm}|p{81mm}|}
\multicolumn{3}{c}{\large{\bf Fabry-Perot recipes}} \vspace*{1ex} \\
\hline
Recipe Name & Type of Data & Function and Comments \\ \hline
\htmlref{FP}{FP} & Fabry-Perot &
   Uses a sequence of eight frames; object-sky pairs at on-line,
   off-line (blue), on-line and off-line (red) FP settings to make
   a mosaic.  Requires a separate flat-field (made by SKY\_FLAT\_FP)
   and a dark. \\ \hline
\htmlref{FP\_JITTER}{FP\_JITTER} & &
   On/Off-line images with nodding to blank sky (as FP), and spatial
   jittering on-source.  Requires a separate flat-field and a dark. \\ \hline
\htmlref{FP\_JITTER\_NO\_SKY}{FP\_JITTER\_NO\_SKY} & &
   On/Off-line images without nodding to blank sky (i.e. sequence of
   four frames), and spatial jittering on-source. 
   Requires a separate flat-field and a dark.  \\ \hline
\htmlref{SKY\_FLAT\_FP}{SKY\_FLAT\_FP} & Flat &
   Creates a Fabry-Perot sky flat (from jittered blank-sky exposures) 
   FP at on- and off-line wavelengths.  Requires a dark. \\ \hline
\end{tabular}
\end{center}
\bigskip

\begin{center}
\begin{tabular}{|l|p{29mm}|p{77mm}|}
\multicolumn{3}{c}{\large{\bf Polarimetry recipes}} \vspace*{1ex} \\
\hline
Recipe Name & Type of Data & Function and Comments \\ \hline
\htmlref{POL\_ANGLE\_JITTER}{POL\_ANGLE\_JITTER} & 
   Polarimetry of point or small ($<\sim35$ arcsec) extended sources &
   Makes a polarisation map from frames at the four waveplate angles at
   each of at least three jittered positions; the waveplate is moved
   before the telescope.  An appropriate dark and separate flat-fields
   at each waveplate angle (using SKY\_FLAT\_POL or SKY\_FLAT\_POL\_ANGLE)
   must be obtained \\ \hline
\htmlref{POL\_EXTENDED}{POL\_EXTENDED} &
   Polarimetry of extended sources &
   Makes a polarisation map of an extended source from frames
   nodded between object and blank sky.  The object-sky pairs
   must be taken at each of the four waveplate angles.  Requires
   an appropriate dark and separate flat-fields at each waveplate
   angle.\\ \hline
\htmlref{POL\_JITTER}{POL\_JITTER} & 
   Polarimetry of point or small ($<\sim35$ arcsec) extended sources &
   Makes a polarimetry map from frames at the four waveplate angles at
   each of at least three jittered positions; the telescope is moved
   before the waveplate.  An appropriate dark and flat-fields at each
   waveplate angle must be obtained.\\ \hline
\htmlref{SKY\_FLAT\_POL}{SKY\_FLAT\_POL} & Flat &
   Obtain a `master' polarimetry flat-field from the median average
   of eight jittered frames; the waveplate is cycled after
   every second frame.  Makes a copy of the flat for each
   waveplate angle.  Requires a dark.\\ \hline
\end{tabular}
\end{center}

\newpage
\subsection{Reference documentation}

The following recipes apply to both UFTI and IRCAM.  Where there are
processing differences for the two instruments, they are noted in
the reference specification.

In the {\bf Configurable Steering Parameters} sections the defaults
appear at the end of the parameter's description between {\tt [~]}.

The non-generic recipes of the original release are not documented
here, but are still available for reducing pre-ORAC (2000 August)
data.  They are listed in the {\bf Deprecated variants} section of
their generic counterpart.  Each behaves as the generic counterpart,
except the number of jitter points is fixed.  Thus the
JITTER9\_SELF\_FLAT reduces a nine-point jitter.
\bigskip\bigskip\bigskip

\sstroutine{
   ARRAY\_TESTS
}{
   Calculates the readout noises and dark current
}{
   \sstdescription{
      This script calculates for UFTI or IRCAM, the ND\_STARE readout
      noise, and the dark current from a series of engineering data taken
      with the sequence called array\_tests.  In addition for IRCAM the
      STARE readout noise is derived.  The results are compared with
      the nominal values, and you are notified whether or not the values
      obtained are within limits.  At UKIRT, the results are also logged
      to an engineering file for archival purposes.
   }
   \sstnotes{
      \sstitemlist{

         \sstitem
         Intermediate frames are deleted.

         \sstitem
         The engineering log contains the UT date and time, the STARE
         and ND\_STARE readout noises, and the dark current.  The results are
         normally appended to the log.  If for some reason it does not exist,
         a new log is created containing the column headings.

         \sstitem
         Multiple array tests are permitted.  A new set of results is
         reported and logged for each cycle.
      }
   }
   \sstdiytopic{
      Output Data
   }{
      \sstitemlist{

         \sstitem
         The engineering log.\\
         UFTI: {\tt/ukirt\_sw/logs/ufti\_array\_tests.log}.\\
         IRCAM: {\tt/ukirt\_sw/logs/ircam3\_array\_tests.log}.
      }
   }
   \sstimplementationstatus{
      \sstitemlist{

         \sstitem
         The processing engines are from the Starlink package \xref{\KAPPA}{sun95}{}.

         \sstitem
         Uses the Starlink NDF format.
      }
   }
}

\newpage
\sstroutine{
   BRIGHT\_POINT\_SOURCE
}{
   Reduces a bright-point-source photometry observation
}{
   \sstdescription{
      This recipe reduces a ``bright standard'' photometry observation
      with UFTI or IRCAM data.  It takes an imaging observation comprising
      a series of jittered object frames and a dark frame, and a
      predetermined flat-field frame to make a calibrated, trimmed
      mosaic automatically.

      This recipe performs bad-pixel masking, null debiassing, dark
      subtraction, flat-field division, feature detection and matching
      between object frames, and resampling.  See the ``Notes'' for details.

      As the name implies, it is intended for bright point sources,
      such as standard stars.
   }
   \sstnotes{
      \sstitemlist{

         \sstitem
         You may use \htmlref{SKY\_FLAT}{SKY\_FLAT} or
         \htmlref{SKY\_FLAT\_MASKED }{SKY\_FLAT\_MASKED} to make the flat field.

         \sstitem
         A World Co-ordinate System (WCS) using the AIPS convention is
         created in the headers should no WCS already exist.

         \sstitem
         For IRCAM, old headers are reordered and structured with
         headings before groups of related keywords.  The comments have
         units added or appear in a standard format.  Four deprecated
         deprecated are removed.  FITS-violating headers are corrected.
         Spurious instrument names are changed to IRCAM3.

         \sstitem
         The bad pixel mask applied is {\tt\$ORAC\_DATA\_CAL/bpm}.

         \sstitem
         Each dark-subtracted frame has thresholds applied beyond which
         pixels are flagged as bad.  The lower limit is 5 standard
         deviations below the mode, but constrained to the range $-$100 to 1.
         The upper limit is 1000 above the saturation limit for the detector
         in the mode used.

         \sstitem
         Where automatic registration is not possible, the recipe matches
         the centroid of central source, and should that fail, it resorts
         to using the telescope offsets transformed to pixels.

         \sstitem
         The resampling applies non-integer shifts of origin using
         bilinear interpolation.  There is no rotation to align the
         Cartesian axes with the cardinal directions.

         \sstitem
         The recipe makes the mosaic by applying offsets in intensity to
         give the most consistent result amongst the overlapping regions.
         The mosaic is trimmed to the dimensions of an input frame.  The
         mosaic is not normalised by its exposure time (that being the
         exposure time of a single frame).

         \sstitem
         For each cycle of jittered frames, the recipe creates a mosaic,
         which is then added into a master mosaic of improving signal to
         noise.  The exposure time is also summed and stored in the mosaic's
         EXP\_TIME (UFTI) or DEXPTIME (IRCAM) header.  Likewise the end
         airmass header, AMEND, is updated to match that of the last-observed
         frame contributing to the mosaic.

         \sstitem
         Intermediate frames are deleted except for the flat-fielded ({\tt\_ff}
         suffix) frames.

         \sstitem
         Sub-arrays are supported.
      }
   }
   \sstdiytopic{
      Output Data
   }{
      \sstitemlist{

         \sstitem
         The resultant mosaic in {\tt$<$m$>$$<$date$>$\_$<$group\_number$>$\_mos}, where {\tt$<$m$>$}
         is {\tt{gf}} for UFTI and {\tt{gi}} for IRCAM.  Before 2000 August these
         were {\tt{g}} and {\tt{rg}} respectively.

         \sstitem
         The individual flat-fielded frames in {\tt$<$i$>$$<$date$>$\_$<$obs\_number$>$\_ff},
         where {\tt$<$i$>$} is {\tt{f}} for UFTI and {\tt{i}} for IRCAM.  Before 2000 August
         IRCAM frames had prefix {\tt{ro}}.
      }
   }
   \sstparameters{
      \sstsubsection{
         NUMBER = INTEGER
      }{
         The number of frames in the jitter pattern.  If this is not
         set, the number of offsets, as given by FITS header NOFFSETS,
         minus one is used.  If neither is available, 5 is the default.
         An error state arises if the number of jittered frames is fewer
         than 3.  For observations prior to the availability of full
         ORAC, header NOFFSETS will be absent.  {\tt[]}
     }
   }
   \sstdiytopic{
      Related Recipes
   }{
      \htmlref{BRIGHT\_POINT\_SOURCE\_APHOT}{BRIGHT\_POINT\_SOURCE\_APHOT},
      \htmlref{JITTER\_SELF\_FLAT}{JITTER\_SELF\_FLAT},
      \htmlref{SKY\_FLAT}{SKY\_FLAT},\\
      \htmlref{SKY\_FLAT\_MASKED}{SKY\_FLAT\_MASKED}.
   }
   \sstimplementationstatus{
      \sstitemlist{

         \sstitem
         The processing engines are from the Starlink packages: \xref{\CCDPACK}{sun139}{}
         and \xref{\KAPPA}{sun95}{}.

         \sstitem
         Uses the Starlink NDF format.

         \sstitem
         History is recorded within the data files.

         \sstitem
         The title of the data is propagated through intermediate files
         to the mosaic.

         \sstitem
         Error propagation is not used.
      }
   }
}

%\newpage
\sstroutine{
   BRIGHT\_POINT\_SOURCE\_APHOT
}{
   Reduces a bright-point-source photometry observation and performs
   aperture photometry
}{
   \sstdescription{
      This recipe reduces a ``bright standard'' photometry observation
      with UFTI or IRCAM data.  It takes an imaging observation comprising
      a series of jittered object frames and a dark frame with a
      predetermined flat-field frame to make a calibrated, trimmed mosaic
      automatically.

      This recipe performs a null debiassing, bad-pixel masking, dark
      subtraction, flat-field division, feature detection and matching
      between object frames, and resampling.  See the ``Notes'' for details.

      Photometry of the point source using a fixed 5-arcsecond aperture
      is calculated for each jitter frame and the mosaic.  The results
      appear in {\tt\$ORAC\_DATA\_OUT/aphot\_results.txt} in the form of a Starlink
      small text list.  The analysis of each star is appended to this file.

      As the name implies, it is intended for bright point sources,
      such as standard stars.
   }
   \sstnotes{
      \sstitemlist{

         \sstitem
         You may use \htmlref{SKY\_FLAT}{SKY\_FLAT} or \htmlref{SKY\_FLAT\_MASKED}{SKY\_FLAT\_MASKED} to make the flat field.

         \sstitem
         A World Co-ordinate System (WCS) using the AIPS convention is
         created in the headers should no WCS already exist.

         \sstitem
         For IRCAM, old headers are reordered and structured with
         headings before groups of related keywords.  The comments have
         units added or appear in a standard format.  Four deprecated
         deprecated are removed.  FITS-violating headers are corrected.
         Spurious instrument names are changed to IRCAM3.

         \sstitem
         The bad pixel mask applied is {\tt\$ORAC\_DATA\_CAL/bpm}.

         \sstitem
         Each dark-subtracted frame has thresholds applied beyond which
         pixels are flagged as bad.  The lower limit is 5 standard
         deviations below the mode, but constrained to the range $-$100 to 1.
         The upper limit is 1000 above the saturation limit for the detector
         in the mode used.

         \sstitem
         Where automatic registration is not possible, the recipe matches
         the centroid of central source, and should that fail, it resorts
         to using the telescope offsets transformed to pixels.

         \sstitem
         The resampling applies non-integer shifts of origin using
         bilinear interpolation.  There is no rotation to align the
         Cartesian axes with the cardinal directions.

         \sstitem
         The recipe makes the mosaic by applying offsets in intensity to
         give the most consistent result amongst the overlapping regions.
         The mosaic is trimmed to the dimensions of an input frame.  The
         mosaic is not normalised by its exposure time (that being the
         exposure time of a single frame).

         \sstitem
         For each cycle of jittered frames, the recipe creates a mosaic,
         which is then added into a master mosaic of improving signal to
         noise.  The exposure time is also summed and stored in the mosaic's
         EXP\_TIME (UFTI) or DEXPTIME (IRCAM) header.  Likewise the end
         airmass header, AMEND, is updated to match that of the last-observed
         frame contributing to the mosaic.

         \sstitem
         The photometry tabulation includes the file name, source name,
         time, filter, airmass, the catalogue magnitude and estimates of
         the zero-point with and without the application of a mean
         extinction.  There are headings at the top of each column.

         \sstitem

         The photometry uses the mode calculated from
         \mbox{3 $\lsk$ median $-$ 2 $\lsk$ mean} and Chauvenet's
         rejection criterion to estimate the sky level in an annulus
         about the source. The inner annulus diameter is 1.3 times
         that of the aperture (6.5 arcsec); the outer annulus is 2.5
         times (12.5 arcsec) for UFTI and twice the aperture (10 arcsec)
         for IRCAM. The errors are internal, based on the sky noise.

         \sstitem
         Intermediate frames are deleted except for the flat-fielded ({\tt\_ff}
         suffix) frames.

         \sstitem
         Sub-arrays are supported.
      }
   }
   \sstdiytopic{
      Output Data
   }{
      \sstitemlist{

         \sstitem
         The resultant mosaic in {\tt$<$m$>$$<$date$>$\_$<$group\_number$>$\_mos}, where {\tt$<$m$>$}
         is {\tt{gf}} for UFTI and {\tt{gi}} for IRCAM.  Before 2000 August these
         were {\tt{g}} and {\tt{rg}} respectively.

         \sstitem
         The individual flat-fielded frames in {\tt$<$i$>$$<$date$>$\_$<$obs\_number$>$\_ff},
         where {\tt$<$i$>$} is {\tt{f}} for UFTI and {\tt{i}} for IRCAM.  Before 2000 August
         IRCAM frames had prefix {\tt{ro}}.

         \sstitem
         Results tabulation to log {\tt\$ORAC\_DATA\_OUT/aphot\_results.txt}.
      }
   }
   \sstparameters{
      \sstsubsection{
         NUMBER = INTEGER
      }{
         The number of frames in the jitter pattern.  If this is not
         set, the number of offsets, as given by FITS header NOFFSETS,
         minus one is used.  If neither is available, 5 is the default.
         An error state arises if the number of jittered frames is fewer
         than 3.  For observations prior to the availability of full
         ORAC, header NOFFSETS will be absent.  {\tt[]}
      }
   }
   \sstdiytopic{
      Related Recipes
   }{
      \htmlref{BRIGHT\_POINT\_SOURCE}{BRIGHT\_POINT\_SOURCE},
      \htmlref{JITTER\_SELF\_FLAT\_APHOT}{JITTER\_SELF\_FLAT\_APHOT},
      \htmlref{SKY\_FLAT}{SKY\_FLAT},\\
      \htmlref{SKY\_FLAT\_MASKED}{SKY\_FLAT\_MASKED}.
   }
   \sstimplementationstatus{
      \sstitemlist{

         \sstitem
         The processing engines are from the Starlink packages: \xref{\CCDPACK}{sun139}{}
         \xref{\KAPPA}{sun95}{}, and \xref{\PHOTOM}{sun45}{}.

         \sstitem
         Uses the Starlink NDF format.

         \sstitem
         History is recorded within the data files.

         \sstitem
         The title of the data is propagated through intermediate files
         to the mosaic.

         \sstitem
         Error propagation is not used.
      }
   }
}

%\newpage
\sstroutine{
   CHOP\_SKY\_JITTER
}{
   Reduction of alternating sky-target jitters using interpolated sky
   subtraction
}{
   \sstdescription{
      This recipe reduces a moderately extended source using UFTI or IRCAM
      data.  The data comprise alternating blank-sky and target frames
      commencing and ending with a blank sky.  Both the sky and target
      frames are jittered.  The recipe makes a sky-subtracted untrimmed
      mosaic automatically.

      The script performs bad-pixel masking, null debiassing, dark
      subtraction, flat-field division, sky subtraction, registration,
      resampling, and mosaicking.  The ``Notes'' give more details.

      It is suitable for extended objects where the object fills or nearly
      fills the frame, so sky estimation within the frame is impossible or
      unreliable, but the extended mapping of the target is not required.
   }
   \sstnotes{
      \sstitemlist{

         \sstitem
         A World Co-ordinate System (WCS) using the AIPS convention is
         created in the headers should no WCS already exist.

         \sstitem
         For IRCAM, old headers are reordered and structured with
         headings before groups of related keywords.  The comments have
         units added or appear in a standard format.  Four deprecated
         deprecated are removed.  FITS-violating headers are corrected.
         Spurious instrument names are changed to IRCAM3.

         \sstitem
         The bad pixel mask applied is {\tt\$ORAC\_DATA\_CAL/bpm}.

         \sstitem
         Each dark-subtracted frame has thresholds applied beyond which
         pixels are flagged as bad.  The lower limit is 5 standard
         deviations below the mode, but constrained to the range $-$100 to 1.
         The upper limit is 1000 above the saturation limit for the detector
         in the mode used.

         \sstitem
         The flat field is derived from the sky frames as follows.  The
         mode (sigma-clipped mean) is used to offset each sky frame's mode
         to that of the first sky frame.  The corrected sky frames are
         combined pixel by pixel using a median of the values in each
         frame.  The resultant frame is normalised by its median to form
         the flat field.  This frame median is subtracted from the source
         frames after they have been flat-fielded.  A flat field is created
         from all the jittered sky frames, and applied to all the target
         frames.

         \sstitem
         The sky subtraction comes from linear interpolation of the sky
         modal values of the two sky frames which immediately bracket the
         target frame.

         \sstitem
         Registration is performed using common point sources in the
         overlap regions.  If the recipe cannot identify sufficient common
         objects, it then tries the crosshead offsets.  If these are null,
         the script resorts to the telescope offsets.

         \sstitem
         The resampling applies non-integer shifts of origin using
         bilinear interpolation.  There is no rotation to align the
         Cartesian axes with the cardinal directions.

         \sstitem
         The recipe makes the mosaics by applying offsets in intensity
         to give the most consistent result amongst the overlapping regions.
         The noise will be greater in the mosaic's peripheral areas, having
         received less exposure time.  The mosaic is not normalised by its
         exposure time (that being the exposure time of a single frame).

         \sstitem
         At the end of each cycle of sky and object frames the full
         mosaic of target frames is created and displayed.  The mosaic has
         its bad pixels filled by interpolation.  On the second and
         subsequent cycles the full mosaic is added into a master mosaic of
         improving signal to noise.  The exposure time is also summed and
         stored in the mosaic's EXP\_TIME (UFTI) or DEXPTIME (IRCAM) header.
         Likewise the end airmass header, AMEND, is updated to match that of
         the last-observed frame contributing to the mosaic.

         \sstitem
         Intermediate frames are deleted except for the flat-fielded ({\tt\_ff}
         suffix) frames.
      }
   }
   \sstdiytopic{
      Output Data
   }{
      \sstitemlist{

         \sstitem
         The integrated mosaic in {\tt$<$m$>$$<$date$>$\_$<$group\_number$>$\_mos}, where {\tt$<$m$>$}
         is {\tt{gf}} for UFTI and {\tt{gi}} for IRCAM.  Before 2000 August these
         were {\tt{g}} and {\tt{rg}} respectively.

         \sstitem
         A mosaic for each cycle of jittered frames in\\
         {\tt$<$m$>$$<$date$>$\_$<$group\_number$>$\_mos$<$cycle\_number$>$}, where {\tt$<$cycle\_number$>$}\\
         counts from 0.

         \sstitem
         The individual flat-fielded frames in {\tt$<$i$>$$<$date$>$\_$<$obs\_number$>$\_ff},
         where {\tt$<$i$>$} is {\tt{f}} for UFTI and {\tt{i}} for IRCAM.  Before 2000 August
         IRCAM had prefix {\tt{ro}}.

         \sstitem
         The created flat fields in {\tt{flat\_$<$filter$>$\_$<$group\_number$>$}} for the
         first or only cycle, and {\tt{{\tt{flat\_$<$filter$>$\_$<$group\_number$>$}}\_c$<$cycle\_number$>$}}
         for subsequent cycles.
      }
   }
   \sstparameters{
      \sstsubsection{
         NUMBER = INTEGER
      }{
         The number of target frames in the jitter pattern.  If this
         is not set, a value is derived from the number of offsets, as
         given by header NOFFSETS.  The formula is NOFFSETS / 2 $-$ 1.
         An error results should NOFFSETS be odd.  If neither is 
         available, 9 is the default.  An error state arises if the
         number of jittered frames is fewer than 3.  For observations
         prior to the availability of full ORAC, header NOFFSETS will
         be absent.  {\tt[]}
      }
   }
   \sstdiytopic{
      Related Recipes
   }{
      \htmlref{CHOP\_SKY\_JITTER\_BASIC}{CHOP\_SKY\_JITTER\_BASIC},
      \htmlref{EXTENDED\_3x3}{EXTENDED\_3x3},
      \htmlref{QUADRANT\_JITTER}{QUADRANT\_JITTER}.
   }
   \sstimplementationstatus{
      \sstitemlist{

         \sstitem
         The processing engines are from the Starlink packages: \xref{\CCDPACK}{sun139}{}
         \xref{\KAPPA}{sun95}{}, and \xref{\PISA}{sun109}{}.

         \sstitem
         Uses the Starlink NDF format.

         \sstitem
         History is recorded within the data files.

         \sstitem
         The title of the data is propagated through intermediate files
         to the mosaic.

         \sstitem
         Error propagation is not used.
      }
   }
}

%\newpage
\sstroutine{
   CHOP\_SKY\_JITTER\_BASIC
}{
   Basic reduction of alternating sky-target jitters using interpolated
   sky subtraction
}{
   \sstdescription{
      This recipe reduces a moderately extended source using UFTI or IRCAM
      data.  The data comprise alternating blank-sky and target frames
      commencing and ending with a blank sky.  Both the sky and target
      frames are jittered.  The recipe makes a sky-subtracted untrimmed
      mosaic automatically.

      The script performs bad-pixel masking, null debiassing, dark
      subtraction, flat-field division, sky subtraction, registration
      using telescope offsets, and mosaicking.  The ``Notes'' give more
      details.

      It is suitable for extended objects where the object fills or nearly
      fills the frame, so sky estimation within the frame is impossible or
      unreliable, but the extended mapping of the target is not required.
   }
   \sstnotes{
      \sstitemlist{

         \sstitem
         A World Co-ordinate System (WCS) using the AIPS convention is
         created in the headers should no WCS already exist.

         \sstitem
         For IRCAM, old headers are reordered and structured with
         headings before groups of related keywords.  The comments have
         units added or appear in a standard format.  Four deprecated
         deprecated are removed.  FITS-violating headers are corrected.
         Spurious instrument names are changed to IRCAM3.

         \sstitem
         The bad pixel mask applied is {\tt\$ORAC\_DATA\_CAL/bpm}.

         \sstitem
         Each dark-subtracted frame has thresholds applied beyond which
         pixels are flagged as bad.  The lower limit is 5 standard
         deviations below the mode, but constrained to the range $-$100 to 1.
         The upper limit is 1000 above the saturation limit for the detector
         in the mode used.

         \sstitem
         The flat field is derived from the sky frames as follows.  The
         mode (sigma-clipped mean) is used to offset each sky frame's mode
         to that of the first sky frame.  The corrected sky frames are
         combined pixel by pixel using a median of the values in each
         frame.  The resultant frame is normalised by its median to form
         the flat field.  This frame median is subtracted from the source
         frames after they have been flat-fielded.  A flat field is created
         from all the jittered sky frames, and applied to all the target
         frames.

         \sstitem
         The sky subtraction comes from linear interpolation of the sky
         modal values of the two sky frames which immediately bracket the
         target frame.

         \sstitem
         Registration is performed using the telescope offsets
         transformed to pixels.

         \sstitem
         There is no resampling, merely integer shifts of origin.

         \sstitem
         The recipe makes the mosaics by applying offsets in intensity
         to give the most consistent result amongst the overlapping regions.
         The noise will be greater in the mosaic's peripheral areas, having
         received less exposure time.  The mosaic is not normalised by its
         exposure time (that being the exposure time of a single frame).

         \sstitem
         At the end of each cycle of sky and object frames the full
         mosaic of target frames is created and displayed.  On the second and
         subsequent cycles the full mosaic is added into a master mosaic of
         improving signal to noise.  The exposure time is also summed and
         stored in the mosaic's EXP\_TIME (UFTI) or DEXPTIME (IRCAM) header.
         Likewise the end airmass header, AMEND, is updated to match that of
         the last-observed frame contributing to the mosaic.

         \sstitem
         Intermediate frames are deleted except for the flat-fielded ({\tt\_ff}
         suffix) frames.
      }
   }
   \sstdiytopic{
      Output Data
   }{
      \sstitemlist{

         \sstitem
         The integrated mosaic in {\tt$<$m$>$$<$date$>$\_$<$group\_number$>$\_mos}, where {\tt$<$m$>$}
         is {\tt{gf}} for UFTI and {\tt{gi}} for IRCAM.  Before 2000 August these
         were {\tt{g}} and {\tt{rg}} respectively.

         \sstitem
         A mosaic for each cycle of jittered frames in\\
         {\tt$<$m$>$$<$date$>$\_$<$group\_number$>$\_mos$<$cycle\_number$>$}, where {\tt$<$cycle\_number$>$}\\
         counts from 0.

         \sstitem
         The individual flat-fielded frames in {\tt$<$i$>$$<$date$>$\_$<$obs\_number$>$\_ff},
         where {\tt$<$i$>$} is {\tt{f}} for UFTI and {\tt{i}} for IRCAM.  Before 2000 August
         IRCAM had prefix {\tt{ro}}.

         \sstitem
         The created flat fields in {\tt{flat\_$<$filter$>$\_$<$group\_number$>$}} for the
         first or only cycle, and {\tt{{\tt{flat\_$<$filter$>$\_$<$group\_number$>$}}\_c$<$cycle\_number$>$}}
         for subsequent cycles.
      }
   }
   \sstparameters{
      \sstsubsection{
         NUMBER = INTEGER
      }{
         The number of target frames in the jitter pattern.  If this
         is not set, a value is derived from the number of offsets, as
         given by header NOFFSETS.  The formula is NOFFSETS / 2 $-$ 1.
         An error results should NOFFSETS be odd.  If neither is 
         available, 9 is the default.  An error state arises if the
         number of jittered frames is fewer than 3.  For observations
         prior to the availability of full ORAC, header NOFFSETS will
         be absent.  {\tt[]}
      }
   }
   \sstdiytopic{
      Related Recipes
   }{
      \htmlref{CHOP\_SKY\_JITTER\_BASIC}{CHOP\_SKY\_JITTER\_BASIC},
      \htmlref{EXTENDED\_3x3\_BASIC}{EXTENDED\_3x3\_BASIC},\\
      \htmlref{QUADRANT\_JITTER\_BASIC}{QUADRANT\_JITTER\_BASIC}.
   }
   \sstimplementationstatus{
      \sstitemlist{

         \sstitem
         The processing engines are from the Starlink packages: \xref{\CCDPACK}{sun139}{}
         \xref{\KAPPA}{sun95}{}, and \xref{\PISA}{sun109}{}.

         \sstitem
         Uses the Starlink NDF format.

         \sstitem
         History is recorded within the data files.

         \sstitem
         The title of the data is propagated through intermediate files
         to the mosaic.

         \sstitem
         Error propagation is not used.
      }
   }
}

%\newpage
\sstroutine{
   EXTENDED\_3x3
}{
   Extended-source standard reduction using interpolated sky subtraction
}{
   \sstdescription{
      This recipe reduces an extended source using UFTI or IRCAM data.
      The data comprise alternating blank-sky and target frames commencing
      and ending with a blank sky.  The target frames are arranged in an
      overlapping (30--50\%) grid of 3$\times$3 frames from which the recipe
      makes a sky-subtracted untrimmed mosaic automatically.

      The script performs bad-pixel masking, null debiassing, dark
      subtraction, flat-field division, sky subtraction, registration,
      resampling, and mosaicking.  The ``Notes'' give more details.

      It is suitable for extended objects up to 2 arcminutes across
      with UFTI and 28 arcseconds with IRCAM.
   }
   \sstnotes{
      \sstitemlist{

         \sstitem
         A World Co-ordinate System (WCS) using the AIPS convention is
         created in the headers should no WCS already exist.

         \sstitem
         For IRCAM, old headers are reordered and structured with
         headings before groups of related keywords.  The comments have
         units added or appear in a standard format.  Four deprecated
         deprecated are removed.  FITS-violating headers are corrected.
         Spurious instrument names are changed to IRCAM3.

         \sstitem
         The bad-pixel mask applied is {\tt\$ORAC\_DATA\_CAL/bpm}.

         \sstitem
         Each dark-subtracted frame has thresholds applied beyond which
         pixels are flagged as bad.  The lower limit is 5 standard
         deviations below the mode, but constrained to the range $-$100 to 1.
         The upper limit is 1000 above the saturation limit for the detector
         in the mode used.

         \sstitem
         The flat field is derived from the sky frames as follows.  The
         mode (sigma-clipped mean) is used to offset each sky frame's mode
         to that of the first sky frame.  The corrected sky frames are
         combined pixel by pixel using a median of the values in each
         frame.  The resultant frame is normalised by its median to form
         the flat field.  This frame median is subtracted from the source
         frames after they have been flat-fielded.  A flat field is created
         for each row of the grid of target frames, and applied only to
         that row of target frames.

         \sstitem
         The sky subtraction comes from linear interpolation of the sky
         modal values of the two sky frames which immediately bracket the
         target frame.

         \sstitem
         Registration is performed using common point sources in the
         overlap regions.  If the recipe cannot identify sufficient common
         objects, it then tries the crosshead offsets.  If these are null,
         the script resorts to the telescope offsets.

         \sstitem
         The resampling applies non-integer shifts of origin using
         bilinear interpolation.  There is no rotation to align the
         Cartesian axes with the cardinal directions.

         \sstitem
         The recipe makes the mosaics by applying offsets in intensity
         to give the most consistent result amongst the overlapping regions.
         The noise will be greater in the mosaic's peripheral areas, having
         received less exposure time.  The mosaic is not normalised by its
         exposure time (that being the exposure time of a single frame).

         \sstitem
         Mosaics are made and displayed for each row, except the last.
         At the end of each cycle of 19 frames the full mosaic of nine target
         frames is created and displayed instead.  On the second and
         subsequent cycles the full mosaic is added into a master mosaic of
         improving signal to noise.  The exposure time is also summed and
         stored in the mosaic's EXP\_TIME (UFTI) or DEXPTIME (IRCAM) header.
         Likewise the end airmass header, AMEND, is updated to match that of
         the last-observed frame contributing to the mosaic.

         \sstitem
         Intermediate frames are deleted except for the flat-fielded ({\tt\_ff}
         suffix) frames.
      }
   }
   \sstdiytopic{
      Output Data
   }{
      \sstitemlist{

         \sstitem
         The full mosaic in {\tt$<$m$>$$<$date$>$\_$<$group\_number$>$\_mos}, where {\tt$<$m$>$}
         is {\tt{gf}} for UFTI and {\tt{gi}} for IRCAM.  Before 2000 August these
         were {\tt{g}} and {\tt{rg}} respectively.

         \sstitem
         A mosaic for each row in {\tt$<$m$>$$<$date$>$\_$<$group\_number$>$\_mos$<$row\_number$>$},
         where {\tt$<$row\_number$>$} is 0 or 1.

         \sstitem
         The individual flat-fielded frames in {\tt$<$i$>$$<$date$>$\_$<$obs\_number$>$\_ff},
         where {\tt$<$i$>$} is {\tt{f}} for UFTI and {\tt{i}} for IRCAM.  Before 2000 August
         IRCAM frames had prefix {\tt{ro}}.
      }
   }
   \sstparameters{
      \sstsubsection{
         NROW = INTEGER
      }{
         The number of target frames in a row of the mosaic.  Its
         minimum is 3 because this number of blank skies are needed to
         form a flat field.  {\tt[3]}
      }
      \sstsubsection{
         NCOL = INTEGER
      }{
         The number of target frames in a column of the mosaic.  Its
         minimum is 2.  {\tt[3]}
      }
   }
   \sstdiytopic{
      Related Recipes
   }{
      \htmlref{EXTENDED\_3x3\_BASIC}{EXTENDED\_3x3\_BASIC},
      \htmlref{EXTENDED\_5x5}{EXTENDED\_5x5},
      \htmlref{QUADRANT\_JITTER}{QUADRANT\_JITTER}.
   }
   \sstimplementationstatus{
      \sstitemlist{

         \sstitem
         The processing engines are from the Starlink packages: \xref{\CCDPACK}{sun139}{}
         \xref{\KAPPA}{sun95}{}, and \xref{\PISA}{sun109}{}.

         \sstitem
         Uses the Starlink NDF format.

         \sstitem
         History is recorded within the data files.

         \sstitem
         The title of the data is propagated through intermediate files
         to the mosaic.

         \sstitem
         Error propagation is not used.
      }
   }
}

%\newpage
\sstroutine{
   EXTENDED\_3x3\_BASIC
}{
   Basic extended-source standard reduction using interpolated sky
   subtraction
}{
   \sstdescription{
      This recipe reduces an extended source using UFTI or IRCAM data.
      The data comprise alternating blank-sky and target frames commencing
      and ending with a blank sky.  The target frames are arranged in an
      overlapping (30--50\%) grid of 3$\times$3 frames from which the recipe
      makes a sky-subtracted untrimmed mosaic automatically.

      The script performs bad-pixel masking, null debiassing, dark
      subtraction, flat-field division, sky subtraction, registration
      using telescope offsets, and mosaicking.  The ``Notes'' give more
      details.

      It is suitable for extended objects up to 2 arcminutes across
      with UFTI and 28 arcseconds with IRCAM.
   }
   \sstnotes{
      \sstitemlist{

         \sstitem
         A World Co-ordinate System (WCS) using the AIPS convention is
         created in the headers should no WCS already exist.

         \sstitem
         For IRCAM, old headers are reordered and structured with
         headings before groups of related keywords.  The comments have
         units added or appear in a standard format.  Four deprecated
         deprecated are removed.  FITS-violating headers are corrected.
         Spurious instrument names are changed to IRCAM3.

         \sstitem
         The bad-pixel mask applied is {\tt\$ORAC\_DATA\_CAL/bpm}.

         \sstitem
         Each dark-subtracted frame has thresholds applied beyond which
         pixels are flagged as bad.  The lower limit is 5 standard
         deviations below the mode, but constrained to the range $-$100 to 1.
         The upper limit is 1000 above the saturation limit for the detector
         in the mode used.

         \sstitem
         The flat field is derived from the sky frames as follows.  The
         mode (sigma-clipped mean) is used to offset each sky frame's mode
         to that of the first sky frame.  The corrected sky frames are
         combined pixel by pixel using a median of the values in each
         frame.  The resultant frame is normalised by its median to form
         the flat field.  This frame median is subtracted from the source
         frames after they have been flat-fielded.  A flat field is created
         for each row of the grid of target frames, and applied only to
         that row of target frames.

         \sstitem
         The sky subtraction comes from linear interpolation of the sky
         modal values of the two sky frames which immediately bracket the
         target frame.

         \sstitem
         Registration is performed using the telescope offsets
         transformed to pixels.

         \sstitem
         There is no resampling, merely integer shifts of origin.

         \sstitem
         The recipe makes the mosaics by applying offsets in intensity
         to give the most consistent result amongst the overlapping regions.
         The noise will be greater in the mosaic's peripheral areas, having
         received less exposure time.  The mosaic is not normalised by its
         exposure time (that being the exposure time of a single frame).

         \sstitem
         Mosaics are made and displayed for each row, except the last.
         At the end of each cycle of 19 frames the full mosaic of nine target
         frames is created and displayed instead.  On the second and
         subsequent cycles the full mosaic is added into a master mosaic of
         improving signal to noise.  The exposure time is also summed and
         stored in the mosaic's EXP\_TIME (UFTI) or DEXPTIME (IRCAM) header.
         Likewise the end airmass header, AMEND, is updated to match that of
         the last-observed frame contributing to the mosaic.

         \sstitem
         Intermediate frames are deleted except for the flat-fielded ({\tt\_ff}
         suffix) frames.
      }
   }
   \sstdiytopic{
      Output Data
   }{
      \sstitemlist{

         \sstitem
         The full mosaic in {\tt$<$m$>$$<$date$>$\_$<$group\_number$>$\_mos}, where {\tt$<$m$>$}
         is {\tt{gf}} for UFTI and {\tt{gi}} for IRCAM.  Before 2000 August these
         were {\tt{g}} and {\tt{rg}} respectively.

         \sstitem
         A mosaic for each row in {\tt$<$m$>$$<$date$>$\_$<$group\_number$>$\_mos$<$row\_number$>$},
         where {\tt$<$row\_number$>$} is 0 or 1.

         \sstitem
         The individual flat-fielded frames in {\tt$<$i$>$$<$date$>$\_$<$obs\_number$>$\_ff},
         where {\tt$<$i$>$} is {\tt{f}} for UFTI and {\tt{i}} for IRCAM.  Before 2000 August
         IRCAM frames had prefix {\tt{ro}}.
      }
   }
   \sstparameters{
      \sstsubsection{
         NROW = INTEGER
      }{
         The number of target frames in a row of the mosaic.  Its
         minimum is 3 because this number of blank skies are needed to
         form a flat field.  {\tt[3]}
      }
      \sstsubsection{
         NCOL = INTEGER
      }{
         The number of target frames in a column of the mosaic.  Its
         minimum is 2.  {\tt[3]}
      }
   }
   \sstdiytopic{
      Related Recipes
   }{
      \htmlref{EXTENDED\_3x3}{EXTENDED\_3x3},
      \htmlref{EXTENDED\_5x5\_BASIC}{EXTENDED\_5x5\_BASIC},
      \htmlref{QUADRANT\_JITTER\_BASIC}{QUADRANT\_JITTER\_BASIC}.
   }
   \sstimplementationstatus{
      \sstitemlist{

         \sstitem
         The processing engines are from the Starlink packages: \xref{\CCDPACK}{sun139}{}
         \xref{\KAPPA}{sun95}{}, and \xref{\PISA}{sun109}{}.

         \sstitem
         Uses the Starlink NDF format.

         \sstitem
         History is recorded within the data files.

         \sstitem
         The title of the data is propagated through intermediate files
         to the mosaic.

         \sstitem
         Error propagation is not used.
      }
   }
}

%\newpage
\sstroutine{
   EXTENDED\_5x5
}{
   Extended-source standard reduction using interpolated sky subtraction
}{
   \sstdescription{
      This recipe reduces an extended source using UFTI or IRCAM data.
      The data comprise alternating blank-sky and target frames commencing
      and ending with a blank sky.  The target frames are arranged in an
      overlapping (30--50\%) grid of 5$\times$5 frames from which the recipe
      makes a sky-subtracted untrimmed mosaic automatically.

      The script performs bad-pixel masking, null debiassing, dark
      subtraction, flat-field division, sky subtraction, registration,
      resampling, and mosaicking.  The ``Notes'' give more details.

      It is suitable for extended objects up to 3 arcminutes across
      with UFTI and 42 arcseconds with IRCAM.
   }
   \sstnotes{
      \sstitemlist{

         \sstitem
         A World Co-ordinate System (WCS) using the AIPS convention is
         created in the headers should no WCS already exist.

         \sstitem
         For IRCAM, old headers are reordered and structured with
         headings before groups of related keywords.  The comments have
         units added or appear in a standard format.  Four deprecated
         deprecated are removed.  FITS-violating headers are corrected.
         Spurious instrument names are changed to IRCAM3.

         \sstitem
         The bad-pixel mask applied is {\tt\$ORAC\_DATA\_CAL/bpm}.

         \sstitem
         Each dark-subtracted frame has thresholds applied beyond which
         pixels are flagged as bad.  The lower limit is 5 standard
         deviations below the mode, but constrained to the range $-$100 to 1.
         The upper limit is 1000 above the saturation limit for the detector
         in the mode used.

         \sstitem
         The flat field is derived from the sky frames as follows.  The
         mode (sigma-clipped mean) is used to offset each sky frame's mode
         to that of the first sky frame.  The corrected sky frames are
         combined pixel by pixel using a median of the values in each
         frame.  The resultant frame is normalised by its median to form
         the flat field.  This frame median is subtracted from the source
         frames after they have been flat-fielded.  A flat field is created
         for each row of the grid of target frames, and applied only to
         that row of target frames.

         \sstitem
         The sky subtraction comes from linear interpolation of the sky
         modal values of the two sky frames which immediately bracket the
         target frame.

         \sstitem
         Registration is performed using common point sources in the
         overlap regions.  If the recipe cannot identify sufficient common
         objects, it then tries the crosshead offsets.  If these are null,
         the script resorts to the telescope offsets.

         \sstitem
         The resampling applies non-integer shifts of origin using
         bilinear interpolation.  There is no rotation to align the
         Cartesian axes with the cardinal directions.

         \sstitem
         The recipe makes the mosaics by applying offsets in intensity
         to give the most consistent result amongst the overlapping regions.
         The noise will be greater in the mosaic's peripheral areas, having
         received less exposure time.  The mosaic is not normalised by its
         exposure time (that being the exposure time of a single frame).

         \sstitem
         Mosaics are made and displayed for each row, except the last.
         At the end of each cycle of 51 frames the full mosaic of 25 target
         frames is created and displayed instead.  On the second and
         subsequent cycles the full mosaic is added into a master mosaic of
         improving signal to noise.  The exposure time is also summed and
         stored in the mosaic's EXP\_TIME (UFTI) or DEXPTIME (IRCAM) header.
         Likewise the end airmass header, AMEND, is updated to match that of
         the last-observed frame contributing to the mosaic.

         \sstitem
         Intermediate frames are deleted except for the flat-fielded ({\tt\_ff}
         suffix) frames.
      }
   }
   \sstdiytopic{
      Output Data
   }{
      \sstitemlist{

         \sstitem
         The full mosaic in {\tt$<$m$>$$<$date$>$\_$<$group\_number$>$\_mos}, where {\tt$<$m$>$}
         is {\tt{gf}} for UFTI and {\tt{gi}} for IRCAM.  Before 2000 August these
         were {\tt{g}} and {\tt{rg}} respectively.

         \sstitem
         A mosaic for each row in {\tt$<$m$>$$<$date$>$\_$<$group\_number$>$\_mos$<$row\_number$>$},
         where {\tt$<$row\_number$>$} is 0 to 3.

         \sstitem
         The individual flat-fielded frames in {\tt$<$i$>$$<$date$>$\_$<$obs\_number$>$\_ff},
         where {\tt$<$i$>$} is {\tt{f}} for UFTI and {\tt{i}} for IRCAM.  Before 2000 August
         IRCAM frames had prefix {\tt{ro}}.
      }
   }
   \sstparameters{
      \sstsubsection{
         NROW = INTEGER
      }{
         The number of target frames in a row of the mosaic.  Its
         minimum is 3 because this number of blank skies are needed to
         form a flat field.  {\tt[5]}
      }
      \sstsubsection{
         NCOL = INTEGER
      }{
         The number of target frames in a column of the mosaic.  Its
         minimum is 2.  {\tt[5]}
      }
   }
   \sstdiytopic{
      Related Recipes
   }{
      \htmlref{EXTENDED\_3x3}{EXTENDED\_3x3},
      \htmlref{EXTENDED\_5x5\_BASIC}{EXTENDED\_5x5\_BASIC},
      \htmlref{QUADRANT\_JITTER}{QUADRANT\_JITTER}.
   }
   \sstimplementationstatus{
      \sstitemlist{

         \sstitem
         The processing engines are from the Starlink packages: \xref{\CCDPACK}{sun139}{}
         \xref{\KAPPA}{sun95}{}, and \xref{\PISA}{sun109}{}.

         \sstitem
         Uses the Starlink NDF format.

         \sstitem
         History is recorded within the data files.

         \sstitem
         The title of the data is propagated through intermediate files
         to the mosaic.

         \sstitem
         Error propagation is not used.
      }
   }
}

%\newpage
\sstroutine{
   EXTENDED\_5x5\_BASIC
}{
   Basic extended-source standard reduction using interpolated sky
   subtraction
}{
   \sstdescription{
      This recipe reduces an extended source using UFTI or IRCAM data.
      The data comprise alternating blank-sky and target frames commencing
      and ending with a blank sky.  The target frames are arranged in an
      overlapping (30--50\%) grid of 5$\times$5 frames from which the recipe
      makes a sky-subtracted untrimmed mosaic automatically.

      The script performs bad-pixel masking, null debiassing, dark
      subtraction, flat-field division, sky subtraction, registration
      using telescope offsets, and mosaicking.  The ``Notes'' give more
      details.

      It is suitable for extended objects up to 3 arcminutes across
      with UFTI and 42 arcseconds with IRCAM.
   }
   \sstnotes{
      \sstitemlist{

         \sstitem
         A World Co-ordinate System (WCS) using the AIPS convention is
         created in the headers should no WCS already exist.

         \sstitem
         For IRCAM, old headers are reordered and structured with
         headings before groups of related keywords.  The comments have
         units added or appear in a standard format.  Four deprecated
         deprecated are removed.  FITS-violating headers are corrected.
         Spurious instrument names are changed to IRCAM3.

         \sstitem
         The bad-pixel mask applied is {\tt\$ORAC\_DATA\_CAL/bpm}.

         \sstitem
         Each dark-subtracted frame has thresholds applied beyond which
         pixels are flagged as bad.  The lower limit is 5 standard
         deviations below the mode, but constrained to the range $-$100 to 1.
         The upper limit is 1000 above the saturation limit for the detector
         in the mode used.

         \sstitem
         The flat field is derived from the sky frames as follows.  The
         mode (sigma-clipped mean) is used to offset each sky frame's mode
         to that of the first sky frame.  The corrected sky frames are
         combined pixel by pixel using a median of the values in each
         frame.  The resultant frame is normalised by its median to form
         the flat field.  This frame median is subtracted from the source
         frames after they have been flat-fielded.  A flat field is created
         for each row of the grid of target frames, and applied only to
         that row of target frames.

         \sstitem
         The sky subtraction comes from linear interpolation of the sky
         modal values of the two sky frames which immediately bracket the
         target frame.

         \sstitem
         Registration is performed using the telescope offsets
         transformed to pixels.

         \sstitem
         There is no resampling, merely integer shifts of origin.

         \sstitem
         The recipe makes the mosaics by applying offsets in intensity
         to give the most consistent result amongst the overlapping regions.
         The noise will be greater in the mosaic's peripheral areas, having
         received less exposure time.  The mosaic is not normalised by its
         exposure time (that being the exposure time of a single frame).

         \sstitem
         Mosaics are made and displayed for each row, except the last.
         At the end of each cycle of 51 frames the full mosaic of 25 target
         frames is created and displayed instead.  On the second and
         subsequent cycles the full mosaic is added into a master mosaic of
         improving signal to noise.  The exposure time is also summed and
         stored in the mosaic's EXP\_TIME (UFTI) or DEXPTIME (IRCAM) header.
         Likewise the end airmass header, AMEND, is updated to match that of
         the last-observed frame contributing to the mosaic.

         \sstitem
         Intermediate frames are deleted except for the flat-fielded ({\tt\_ff}
         suffix) frames.
      }
   }
   \sstdiytopic{
      Output Data
   }{
      \sstitemlist{

         \sstitem
         The full mosaic in {\tt$<$m$>$$<$date$>$\_$<$group\_number$>$\_mos}, where {\tt$<$m$>$}
         is {\tt{gf}} for UFTI and {\tt{gi}} for IRCAM.  Before 2000 August these
         were {\tt{g}} and {\tt{rg}} respectively.

         \sstitem
         A mosaic for each row in {\tt$<$m$>$$<$date$>$\_$<$group\_number$>$\_mos$<$row\_number$>$},
         where {\tt$<$row\_number$>$} is 0 to 3.

         \sstitem
         The individual flat-fielded frames in {\tt$<$i$>$$<$date$>$\_$<$obs\_number$>$\_ff},
         where {\tt$<$i$>$} is {\tt{f}} for UFTI and {\tt{i}} for IRCAM.  Before 2000 August
         IRCAM frames had prefix {\tt{ro}}.
      }
   }
   \sstparameters{
      \sstsubsection{
         NROW = INTEGER
      }{
         The number of target frames in a row of the mosaic.  Its
         minimum is 3 because this number of blank skies are needed to
         form a flat field. {\tt[5]}
      }
      \sstsubsection{
         NCOL = INTEGER
      }{
         The number of target frames in a column of the mosaic.  Its
         minimum is 2. {\tt[5]}
      }
   }
   \sstdiytopic{
      Related Recipes
   }{
      \htmlref{EXTENDED\_3x3\_BASIC}{EXTENDED\_3x3\_BASIC},
      \htmlref{EXTENDED\_5x5}{EXTENDED\_5x5},
      \htmlref{QUADRANT\_JITTER\_BASIC}{QUADRANT\_JITTER\_BASIC}.
   }
   \sstimplementationstatus{
      \sstitemlist{

         \sstitem
         The processing engines are from the Starlink packages: \xref{\CCDPACK}{sun139}{}
         \xref{\KAPPA}{sun95}{}, and \xref{\PISA}{sun109}{}.

         \sstitem
         Uses the Starlink NDF format.

         \sstitem
         History is recorded within the data files.

         \sstitem
         The title of the data is propagated through intermediate files
         to the mosaic.

         \sstitem
         Error propagation is not used.
      }
   }
}

%\newpage
\sstroutine{
   FP
}{
   Reduces an 8-frame Fabry-Perot observation
}{
   \sstdescription{
      This script reduces a Fabry-Perot observation with UFTI data.  It takes
      an imaging observation comprising eight object frames and a dark frame
      to make a continuum-subtracted and sky-subtracted, untrimmed mosaic
      automatically.

      The sequence of frames expected in the observations are tabulated
      below.

      \begin{center}
      \begin{tabular}{cll}
      Frame & \multicolumn{1}{c}{Position} & \multicolumn{1}{c}{Wavelength} \\ \hline
        1 & On  source & On  line \\
        2 & Off source & On  line \\
        3 & Off source & Off line, positive offset \\
        4 & On  source & Off line, positive offset \\
        5 & On  source & On  line \\
        6 & Off source & On  line \\
        7 & Off source & Off line, negative offset \\
        8 & On  source & Off line, negative offset \\
      \end{tabular}
      \end{center}

      It performs a null debiassing, bad-pixel masking, dark subtraction,
      pairwise frame differencing, flat-field division, integer shifts of
      origin to register, and mosaicking.  The desired result is given by

      \begin{center}
      \[ \frac{[(F1 - F2)-(F4 - F3)] + [(F5 - F6)-(F8 - F7)]}{\rm Flat field} \]
      \end{center}

      where $Fn$ is the bad-pixel masked and dark subtracted frame $n$.
      In practice, the flat field is applied to each differenced pair,
      such as ($F4 - F3$), when the pair becomes available, rather than
      waiting until all eight frames have been observed.
   }
   \sstnotes{
      \sstitemlist{

         \sstitem
         A World Co-ordinate System (WCS) using the AIPS convention is
         created in the headers should no WCS already exist.

         \sstitem
         The bad-pixel mask applied is {\tt\$ORAC\_DATA\_CAL/bpm}.

         \sstitem
         Each dark-subtracted frame has thresholds applied beyond which
         pixels are flagged as bad.  The lower limit is 5 standard
         deviations below the mode, but constrained to the range $-$100 to 1.
         The upper limit is 1000 above the saturation limit for the detector
         in the mode used.

         \sstitem
         You should use \htmlref{SKY\_FLAT\_FP}{SKY\_FLAT\_FP} to make the flat field.

         \sstitem
         Registration is performed using the telescope offsets
         transformed to pixels.

         \sstitem
         There is no resampling, merely integer shifts of origin.

         \sstitem
         The recipe makes the mosaics by applying offsets in intensity
         to give the most consistent result amongst the overlapping regions.
         The mosaic is not trimmed to the dimensions of a single frame, thus
         the noise will be greater in the peripheral areas having received
         less exposure time.  The mosaic is not normalised by its exposure
         time (that being the exposure time of a single frame).

         \sstitem
         For each cycle of eight, the recipe creates a mosaic, which is
         then added into a master mosaic of improving signal to noise.  The
         exposure time is also summed and stored in the mosaic's EXP\_TIME 
         header.  Likewise the end airmass header, AMEND, is updated to match
         that of the last-observed frame contributing to the mosaic.

         \sstitem
         Intermediate frames are deleted except for the flat-fielded ({\tt\_ff}
         suffix) frames.
      }
   }
   \sstdiytopic{
      Output Data
   }{
      \sstitemlist{

         \sstitem
         The integrated mosaic in {\tt{g}$<$date$>$\_$<$group\_number$>$\_mos}, where
         {\tt$<$date$>$} is the UT date in yyyymmdd format.  Before 2000 August the
         prefix was {\tt{g}}.

         \sstitem
         A mosaic for each cycle of eight in \\
         {\tt{gf}$<$date$>$\_$<$group\_number$>$\_mos$<$cycle\_number$>$}, where {\tt$<$cycle\_number$>$}\\
         counts from 0.

         \sstitem
         The individual flat-fielded frames in {\tt{f}$<$date$>$\_$<$obs\_number$>$\_ff}.
      }
   }
   \sstparameters{
      \sstsubsection{
         NPAIRS = INTEGER
      }{
         The number of frame pairs to be differenced.  It must be a multiple
         of 2 otherwise 4 is assumed.  A value of four or more is assumed to
         indicate sky subtraction.  {\tt[4]}
      }
      \sstsubsection{
         NUMBER = INTEGER
      }{
         The number of spatial jitter positions.  For each spatial position 
         there are NPAIRS pairs of frames.  A value of 1 also dictates
         that no jittering has occurred.  To make a master mosaic combining
         spatial positions, NUMBER should be at least 3.  {\tt[1]}
      }
   }
   \sstdiytopic{
      Related Recipes
   }{
      \htmlref{SKY\_FLAT\_FP}{SKY\_FLAT\_FP},
      \htmlref{FP\_JITTER}{FP\_JITTER},
      \htmlref{FP\_JITTER\_NO\_SKY}{FP\_JITTER\_NO\_SKY}.
   }
   \sstimplementationstatus{
      \sstitemlist{

         \sstitem
         The processing engines are from the Starlink packages: \xref{\CCDPACK}{sun139}{}
         and \xref{\KAPPA}{sun95}{}.

         \sstitem
         Uses the Starlink NDF format.

         \sstitem
         History is recorded within the data files.

         \sstitem
         The title of the data is propagated through intermediate files
         to the mosaic.

         \sstitem
         Error propagation is not used.
      }
   }
}

\newpage
\sstroutine{
   FP\_JITTER
}{
   Reduces spatially jittered sets of 8-frame Fabry-Perot observations
}{
   \sstdescription{
      This script reduces a Fabry-Perot observation with UFTI data.  It
      takes an imaging observation comprising at least three sets of eight
      object frames, each set being for a different telescope position.
      The recipe combines these with a dark frame and a separate flat, to
      make a continuum-subtracted and sky-subtracted, untrimmed mosaic
      automatically.

      Each sequence of eight frames expected in each spatial position are
      tabulated below.

      \begin{center}
      \begin{tabular}{cll}
      Frame & \multicolumn{1}{c}{Position} & \multicolumn{1}{c}{Wavelength} \\ \hline
        1 & On  source & On  line \\
        2 & Off source & On  line \\
        3 & Off source & Off line, positive offset \\
        4 & On  source & Off line, positive offset \\
        5 & On  source & On  line \\
        6 & Off source & On  line \\
        7 & Off source & Off line, negative offset \\
        8 & On  source & Off line, negative offset \\
      \end{tabular}
      \end{center}

      For each spatial set, the recipe performs a null debiassing, bad-pixel
      masking, dark subtraction, pairwise frame differencing, flat-field
      division, integer shifts of origin to register, and mosaicking.  The
      wavelength-shifted mosaic is given by

      \begin{center}
      \[ \frac{[(F1 - F2)-(F4 - F3)] + [(F5 - F6)-(F8 - F7)]}{\rm Flat field} \]
      \end{center}

      where $Fn$ is the bad-pixel masked and dark subtracted frame $n$.
      In practice, the flat field is applied to each differenced pair,
      such as ($F4 - F3$), when the pair becomes available, rather than
      waiting until all eight frames have been observed.

      Finally the recipe registers all the wavelength mosaics spatially, and a
      forms a untrimmed mosaic, combined using the median to reduce stellar
      artifacts.
   }
   \sstnotes{
      \sstitemlist{

         \sstitem
         A World Co-ordinate System (WCS) using the AIPS convention is
         created in the headers should no WCS already exist.

         \sstitem
         The bad-pixel mask applied is {\tt\$ORAC\_DATA\_CAL/bpm}.

         \sstitem
         Each dark-subtracted frame has thresholds applied beyond which
         pixels are flagged as bad.  The lower limit is 5 standard
         deviations below the mode, but constrained to the range $-$100 to 1.
         The upper limit is 1000 above the saturation limit for the detector
         in the mode used.

         \sstitem
         You should use \htmlref{SKY\_FLAT\_FP}{SKY\_FLAT\_FP} to make the flat field.

         \sstitem
         Registration is performed using the telescope offsets
         transformed to pixels.

         \sstitem
         There is no resampling, merely integer shifts of origin.

         \sstitem
         For each set of eight, the recipe creates a wavelength mosaic.
         For each cycle of spatial positions the wavelength mosaics are
         registered to form a spatial mosaic.  For repeat cycles the spatial
         mosaic is then added into a master mosaic of improving signal to
         noise.  The exposure time is also summed and stored in the master
         mosaic's EXP\_TIME header.  Likewise the end airmass header, AMEND,
         is updated to match that of the last-observed frame contributing to
         the mosaic.

         \sstitem
         The recipe makes the mosaics by applying offsets in intensity
         to give the most consistent result amongst the overlapping regions.
         No mosaic is trimmed to the dimensions of a single frame, thus the
         noise will be greater in the peripheral areas of the spatial having
         received less exposure time.  Each mosaic is not normalised by its
         exposure time (that being the exposure time of a single frame).

         \sstitem
         Intermediate frames are deleted except for the flat-fielded ({\tt\_ff}
         suffix) frames.
      }
   }
   \sstdiytopic{
      Output Data
   }{
      \sstitemlist{

         \sstitem
         The integrated mosaic in {\tt{g}$<$date$>$\_$<$group\_number$>$\_mos}, where
         {\tt$<$date$>$} is the UT date in yyyymmdd format. 

         \sstitem
         A mosaic for each cycle of eight in \\
         {\tt{gf}$<$date$>$\_$<$group\_number$>$\_mos$<$cycle\_number$>$}, where {\tt$<$cycle\_number$>$}\\
         counts from 0.

         \sstitem
         The individual flat-fielded frames in {\tt{f}$<$date$>$\_$<$obs\_number$>$\_ff}.
      }
   }
   \sstparameters{
      \sstsubsection{
         NPAIRS = INTEGER
      }{
         The number of frame pairs to be differenced.  It must be a multiple
         of 2 otherwise 4 is assumed.  A value of four or more is assumed to
         indicate sky subtraction.  {\tt[4]}
      }
      \sstsubsection{
         NUMBER = INTEGER
      }{
         The number of spatial jitter positions.  For each spatial position 
         there are NPAIRS pairs of frames.  A value of 1 also dictates
         that no jittering has occurred.  To make a master mosaic combining
         spatial positions, NUMBER should be at least 3.  {\tt[1]}
      }
   }
   \sstdiytopic{
      Related Recipes
   }{
      \htmlref{FP}{FP},
      \htmlref{FP\_JITTER\_NO\_SKY}{FP\_JITTER\_NO\_SKY},
      \htmlref{SKY\_FLAT\_FP}{SKY\_FLAT\_FP}.
   }
   \sstimplementationstatus{
      \sstitemlist{

         \sstitem
         The processing engines are from the Starlink packages: \xref{\CCDPACK}{sun139}{}
         and \xref{\KAPPA}{sun95}{}.

         \sstitem
         Uses the Starlink NDF format.

         \sstitem
         History is recorded within the data files.

         \sstitem
         The title of the data is propagated through intermediate files
         to the mosaic.

         \sstitem
         Error propagation is not used.
      }
   }
}

\newpage
\sstroutine{
   FP\_JITTER\_NO\_SKY
}{
   Reduces a spatially jittered 4-frame Fabry-Perot observation
}{
   \sstdescription{
      This script reduces a Fabry-Perot observation with UFTI data.  It
      takes an imaging observation comprising at least three sets of
      four object frames, each set being for a different telescope position.
      The recipe combines these with a dark frame and a separate flat, to
      make a continuum-subtracted, untrimmed mosaic automatically.

      Each sequence of four frames expected in each spatial position are
      tabulated below.

      \begin{center}
      \begin{tabular}{cll}
      Frame & \multicolumn{1}{c}{Position} & \multicolumn{1}{c}{Wavelength} \\ \hline
        1 & On  source & On  line \\
        2 & On  source & Off line, positive offset \\
        3 & On  source & On  line \\
        4 & On  source & Off line, negative offset \\
      \end{tabular}
      \end{center}

      For each spatial set, the recipe performs a null debiassing, bad-pixel
      masking, dark subtraction, pairwise frame differencing, flat-field
      division, integer shifts of origin to register, and mosaicking.  The
      wavelength-shifted mosaic is given by

      \begin{center}
      \[ \frac{[(F1 - F2) - (F4 - F3)]}{\rm Flat field} \]
      \end{center}

      where $Fn$ is the bad-pixel masked and dark subtracted frame $n$.
      In practice, the flat field is applied to each differenced pair,
      such as ($F1 - F2$), when the pair becomes available, rather than
      waiting until all four frames have been observed.

      Finally the recipe registers all the wavelength mosaics spatially, and a
      forms a untrimmed mosaic, combined using the median to reduce stellar
      artifacts.
   }
   \sstnotes{
      \sstitemlist{

         \sstitem
         A World Co-ordinate System (WCS) using the AIPS convention is
         created in the headers should no WCS already exist.

         \sstitem
         The bad-pixel mask applied is {\tt\$ORAC\_DATA\_CAL/bpm}.

         \sstitem
         Each dark-subtracted frame has thresholds applied beyond which
         pixels are flagged as bad.  The lower limit is 5 standard
         deviations below the mode, but constrained to the range $-$100 to 1.
         The upper limit is 1000 above the saturation limit for the detector
         in the mode used.

         \sstitem
         You should use \htmlref{SKY\_FLAT\_FP}{SKY\_FLAT\_FP} to make the flat field.

         \sstitem
         Registration is performed using the telescope offsets
         transformed to pixels.

         \sstitem
         There is no resampling, merely integer shifts of origin.

         \sstitem
         For each set of four, the recipe creates a wavelength mosaic.
         For each cycle of spatial positions the wavelength mosaics are
         registered to form a spatial mosaic.  For repeat cycles the spatial
         mosaic is then added into a master mosaic of improving signal to
         noise.  The exposure time is also summed and stored in the master
         mosaic's EXP\_TIME header. Likewise the end airmass header, AMEND,
         is updated to match that of the last-observed frame contributing to
         the mosaic.

         \sstitem
         The recipe makes the mosaics by applying offsets in intensity
         to give the most consistent result amongst the overlapping regions.
         No mosaic is trimmed to the dimensions of a single frame, thus the
         noise will be greater in the peripheral areas of the spatial having
         received less exposure time.  Each mosaic is not normalised by its
         exposure time (that being the exposure time of a single frame).

         \sstitem
         Intermediate frames are deleted except for the flat-fielded ({\tt\_ff}
         suffix) frames.
      }
   }
   \sstdiytopic{
      Output Data
   }{
      \sstitemlist{

         \sstitem
         The integrated mosaic in {\tt{g}$<$date$>$\_$<$group\_number$>$\_mos}, where
         {\tt$<$date$>$} is the UT date in yyyymmdd format.

         \sstitem
         A mosaic for each cycle of eight in \\
         {\tt{gf}$<$date$>$\_$<$group\_number$>$\_mos$<$cycle\_number$>$}, where {\tt$<$cycle\_number$>$}\\
         counts from 0.

         \sstitem
         The individual flat-fielded frames in {\tt{f}$<$date$>$\_$<$obs\_number$>$\_ff}.
      }
   }
   \sstparameters{
      \sstsubsection{
         NUMBER = INTEGER
      }{
         The number of spatial jitter positions.  For each spatial position 
         there are NPAIRS pairs of frames.  A value of 1 also dictates
         that no jittering has occurred.  To make a master mosaic combining
         spatial positions, NUMBER should be at least 3.  {\tt[1]}
      }
   }
   \sstdiytopic{
      Related Recipes
   }{
      \htmlref{SKY\_FLAT\_FP}{SKY\_FLAT\_FP},
      \htmlref{FP}{FP},
      \htmlref{FP\_JITTER}{FP\_JITTER}.
   }
   \sstimplementationstatus{
      \sstitemlist{

         \sstitem
         The processing engines are from the Starlink packages: \xref{\CCDPACK}{sun139}{}
         and \xref{\KAPPA}{sun95}{}.

         \sstitem
         Uses the Starlink NDF format.

         \sstitem
         History is recorded within the data files.

         \sstitem
         The title of the data is propagated through intermediate files
         to the mosaic.

         \sstitem
         Error propagation is not used.
      }
   }
}

%\newpage
\sstroutine{
   JITTER\_SELF\_FLAT
}{
   Reduces a ``standard jitter'' photometry observation using object
   masking
}{
   \sstdescription{
      This script reduces a ``standard jitter'' photometry observation
      with UFTI or IRCAM data.  It takes an imaging observation
      comprising jittered object frames and a dark frame to make a
      calibrated, untrimmed mosaic automatically.

      It performs a null debiassing, bad-pixel masking, dark
      subtraction, flat-field creation and division, feature
      detection and matching between object frames, and resampling.
      See the ``Notes'' for further information.

      This recipe works well for faint sources and for moderately
      crowded fields.
   }
   \sstnotes{
      \sstitemlist{

         \sstitem
         A World Co-ordinate System (WCS) using the AIPS convention is
         created in the headers should no WCS already exist.

         \sstitem
         For IRCAM, old headers are reordered and structured with
         headings before groups of related keywords.  The comments have
         units added or appear in a standard format.  Four deprecated
         deprecated are removed.  FITS-violating headers are corrected.
         Spurious instrument names are changed to IRCAM3.

         \sstitem
         The bad-pixel mask applied is {\tt\$ORAC\_DATA\_CAL/bpm}.

         \sstitem
         The dark-subtracted frame has thresholds applied beyond which
         pixels are flagged as bad.  The lower limit is 5 standard
         deviations below the mode, but constrained to the range $-$100 to 1.
         The upper limit is 1000 above the saturation limit for the detector
         in the mode used.

         \sstitem
         The flat field is created iteratively.  First an approximate
         flat-field is created by combining normalised object frames using
         the median at each pixel.  This flat field is applied to the object
         frames.  Sources within the flat-fielded frames are detected, and
         masked in the dark-subtracted frames.  The first stage is repeated
         but applied to the masked frames to create the final flat field.

         \sstitem
         Registration is performed using common point sources in the
         overlap regions.  If the recipe cannot identify sufficient common
         objects, the script resorts to using the telescope offsets
         transformed to pixels.

         \sstitem
         The resampling applies non-integer shifts of origin using
         bilinear interpolation.  There is no rotation to align the
         Cartesian axes with the cardinal directions.

         \sstitem
         The recipe makes the mosaics by applying offsets in intensity
         to give the most consistent result amongst the overlapping regions.
         The mosaic is not trimmed to the dimensions of a single frame, thus
         the noise will be greater in the peripheral areas having received
         less exposure time.  The mosaic is not normalised by its exposure
         time (that being the exposure time of a single frame).

         \sstitem
         For each cycle of jittered frames, the recipe creates a mosaic,
         which has its bad pixels filled and is then added into a master
         mosaic of improving signal to noise.  The exposure time is also
         summed and stored in the mosaic's EXP\_TIME (UFTI) or DEXPTIME
         (IRCAM) header.  Likewise the end airmass header, AMEND, is updated
         to match that of the last-observed frame contributing to the mosaic.

         \sstitem
         Intermediate frames are deleted except for the flat-fielded ({\tt\_ff}
         suffix) frames.
      }
   }
   \sstdiytopic{
      Output Data
   }{
      \sstitemlist{

         \sstitem
         The integrated mosaic in {\tt$<$m$>$$<$date$>$\_$<$group\_number$>$\_mos}, where {\tt$<$m$>$}
         is {\tt{gf}} for UFTI and {\tt{gi}} for IRCAM.  Before 2000 August these
         were {\tt{g}} and {\tt{rg}} respectively.

         \sstitem
         A mosaic for each cycle of jittered frames in\\
         {\tt$<$m$>$$<$date$>$\_$<$group\_number$>$\_mos$<$cycle\_number$>$}, where {\tt$<$cycle\_number$>$}\\
         counts from 0.

         \sstitem
         The individual flat-fielded frames in {\tt$<$i$>$$<$date$>$\_$<$obs\_number$>$\_ff},
         where {\tt$<$i$>$} is {\tt{f}} for UFTI and {\tt{i}} for IRCAM.  Before 2000 August
         IRCAM had prefix {\tt{ro}}.

         \sstitem
         The created flat fields in {\tt{flat\_$<$filter$>$\_$<$group\_number$>$}} for the
         first or only cycle, and {\tt{{\tt{flat\_$<$filter$>$\_$<$group\_number$>$}}\_c$<$cycle\_number$>$}}
         for subsequent cycles.
      }
   }
   \sstparameters{
      \sstsubsection{
         NUMBER = INTEGER
      }{
         The number of frames in the jitter pattern.  If not supplied
         the number of offsets, as given by FITS header NOFFSETS, minus
         one is used.  If neither is available, 9 is the default.  An
         error state arises if the number of jittered frames is fewer
         than 3.  For observations prior to the availability of full
         ORAC, header NOFFSETS will be absent.  {\tt[]}
      }
   }
   \sstdiytopic{
      Related Recipes
   }{
      \htmlref{JITTER\_SELF\_FLAT\_APHOT}{JITTER\_SELF\_FLAT\_APHOT},
      \htmlref{JITTER\_SELF\_FLAT\_BASIC}{JITTER\_SELF\_FLAT\_BASIC},\\
      \htmlref{JITTER\_SELF\_FLAT\_NO\_MASK}{JITTER\_SELF\_FLAT\_NO\_MASK},
      \htmlref{JITTER\_SELF\_FLAT\_TELE}{JITTER\_SELF\_FLAT\_TELE},\\
      \htmlref{MOVING\_JITTER\_SELF\_FLAT}{MOVING\_JITTER\_SELF\_FLAT},
      \htmlref{QUADRANT\_JITTER}{QUADRANT\_JITTER}.
   }
   \sstimplementationstatus{
      \sstitemlist{

         \sstitem
         The processing engines are from the Starlink packages: \xref{\CCDPACK}{sun139}{}
         \xref{\KAPPA}{sun95}{}, and \xref{\PISA}{sun109}{}.

         \sstitem
         Uses the Starlink NDF format.

         \sstitem
         History is recorded within the data files.

         \sstitem
         The title of the data is propagated through intermediate files
         to the mosaic.

         \sstitem
         Error propagation is not used.
      }
   }
   \sstdiytopic{
      Deprecated Variants
   }{
      JITTER3\_SELF\_FLAT, JITTER5\_SELF\_FLAT, JITTER9\_SELF\_FLAT.
   }
}

%\newpage
\sstroutine{
   JITTER\_SELF\_FLAT\_APHOT
}{
   Reduces a ``standard jitter'' photometry observation using
   object masking, and performs aperture photometry
}{
   \sstdescription{
      This script reduces a ``standard jitter'' photometry observation
      with UFTI or IRCAM data.  It takes an imaging observation
      comprising jittered object frames and a dark frame to make a
      calibrated, untrimmed mosaic automatically.

      It performs a null debiassing, bad-pixel masking, dark
      subtraction, flat-field creation and division, feature
      detection and matching between object frames, and resampling.
      See the ``Notes'' for further information.

      Photometry of the point source using a fixed 5-arcsecond aperture
      is calculated for each jitter frame and the mosaic.  The results
      appear in {\tt\$ORAC\_DATA\_OUT/aphot\_results.txt} in the form of a Starlink
      small text list.  The analysis of each star is appended to this file.

      This recipe works well for faint sources in moderately crowded fields.
   }
   \sstnotes{
      \sstitemlist{

         \sstitem
         A World Co-ordinate System (WCS) using the AIPS convention is
         created in the headers should no WCS already exist.

         \sstitem
         For IRCAM, old headers are reordered and structured with
         headings before groups of related keywords.  The comments have
         units added or appear in a standard format.  Four deprecated
         deprecated are removed.  FITS-violating headers are corrected.
         Spurious instrument names are changed to IRCAM3.

         \sstitem
         The bad-pixel mask applied is {\tt\$ORAC\_DATA\_CAL/bpm}.

         \sstitem
         Each dark-subtracted frame has thresholds applied beyond which
         pixels are flagged as bad.  The lower limit is 5 standard
         deviations below the mode, but constrained to the range $-$100 to 1.
         The upper limit is 1000 above the saturation limit for the detector
         in the mode used.

         \sstitem
         The flat field is created iteratively.  First an approximate
         flat-field is created by combining normalised object frames using
         the median at each pixel.  This flat field is applied to the object
         frames.  Sources within the flat-fielded frames are detected, and
         masked in the dark-subtracted frames.  The first stage is repeated
         but applied to the masked frames to create the final flat field.

         \sstitem
         Registration is performed using common point sources in the
         overlap regions.  If the recipe cannot identify sufficient common
         objects, the script resorts to using the telescope offsets
         transformed to pixels.

         \sstitem
         The resampling applies non-integer shifts of origin using
         bilinear interpolation.  There is no rotation to align the
         Cartesian axes with the cardinal directions.

         \sstitem
         The recipe makes the mosaics by applying offsets in intensity
         to give the most consistent result amongst the overlapping regions.
         The mosaic is not trimmed to the dimensions of a single frame, thus
         the noise will be greater in the peripheral areas having received
         less exposure time.  The mosaic is not normalised by its exposure
         time (the exposure time of a single frame).

         \sstitem
         For each cycle of jittered frames, the recipe creates a mosaic,
         which has its bad pixels filled and is then added into a master
         mosaic of improving signal to noise.  The exposure time is also
         summed and stored in the mosaic's EXP\_TIME (UFTI) or DEXPTIME
         (IRCAM) header.  Likewise the end airmass header, AMEND, is updated
         to match that of the last-observed frame contributing to the mosaic.

         \sstitem
         The photometry tabulation includes the file name, source name,
         time, filter, airmass, the catalogue magnitude and estimates of
         the zero-point with and without the application of a mean
         extinction.  There are headings at the top of each column.

         \sstitem
         The photometry uses a multiply clipped (2,2,2.5,3 standard 
         deviations) mean to estimate the sky mode in an annulus about the
         source.  This is not unduly biased by the presence of the self-flat
         artifact in the pixel histogram.  The inner annulus diameter is 1.3 
         times that of the aperture (6.5 arcsec); the outer annulus is 2.5
         times (12.5 arcsec) for UFTI and twice the aperture (10 arcsec) for
         IRCAM.  The errors are internal, based on the sky noise.

         \sstitem
         Intermediate frames are deleted except for the flat-fielded ({\tt\_ff}
         suffix) frames.
      }
   }
   \sstdiytopic{
      Output Data
   }{
      \sstitemlist{

         \sstitem
         The integrated mosaic in {\tt$<$m$>$$<$date$>$\_$<$group\_number$>$\_mos}, where {\tt$<$m$>$}
         is {\tt{gf}} for UFTI and {\tt{gi}} for IRCAM.  Before 2000 August these
         were {\tt{g}} and {\tt{rg}} respectively.

         \sstitem
         A mosaic for each cycle of jittered frames in\\
         {\tt$<$m$>$$<$date$>$\_$<$group\_number$>$\_mos$<$cycle\_number$>$}, where {\tt$<$cycle\_number$>$}\\
         counts from 0.

         \sstitem
         The individual flat-fielded frames in {\tt$<$i$>$$<$date$>$\_$<$obs\_number$>$\_ff},
         where {\tt$<$i$>$} is {\tt{f}} for UFTI and {\tt{i}} for IRCAM.  Before 2000 August
         IRCAM frames had prefix {\tt{ro}}.

         \sstitem
         The created flat fields in {\tt{flat\_$<$filter$>$\_$<$group\_number$>$}} for the
         first or only cycle, and {\tt{{\tt{flat\_$<$filter$>$\_$<$group\_number$>$}}\_c$<$cycle\_number$>$}}
         for subsequent cycles.
      }
   }
   \sstparameters{
      \sstsubsection{
         NUMBER = INTEGER
      }{
         The number of frames in the jitter pattern.  If not supplied
         the number of offsets, as given by FITS header NOFFSETS, minus
         one is used.  If neither is available, 9 is the default.  An
         error state arises if the number of jittered frames is fewer
         than 3.  For observations prior to the availability of full
         ORAC, header NOFFSETS will be absent.  {\tt[]}
      }
   }
   \sstdiytopic{
      Related Recipes
   }{
      \htmlref{BRIGHT\_POINT\_SOURCE\_APHOT}{BRIGHT\_POINT\_SOURCE\_APHOT},
      \htmlref{JITTER\_SELF\_FLAT}{JITTER\_SELF\_FLAT},\\
      \htmlref{JITTER\_SELF\_FLAT\_BASIC}{JITTER\_SELF\_FLAT\_BASIC},
      \htmlref{JITTER\_SELF\_FLAT\_NO\_MASK}{JITTER\_SELF\_FLAT\_NO\_MASK},\\
      \htmlref{QUADRANT\_JITTER}{QUADRANT\_JITTER}.
   }
   \sstimplementationstatus{
      \sstitemlist{

         \sstitem
         The processing engines are from the Starlink packages: \xref{\CCDPACK}{sun139}{}
         \xref{\KAPPA}{sun95}{}, and \xref{\PISA}{sun109}{}.

         \sstitem
         Uses the Starlink NDF format.

         \sstitem
         History is recorded within the data files.

         \sstitem
         The title of the data is propagated through intermediate files
         to the mosaic.

         \sstitem
         Error propagation is not used.
      }
   }
   \sstdiytopic{
      Deprecated Variants
   }{
      JITTER5\_SELF\_FLAT\_APHOT.
   }
}

%\newpage
\sstroutine{
   JITTER\_SELF\_FLAT\_BASIC
}{
   Reduces a ``standard jitter'' photometry observation using
   just the basic operations for speed
}{
   \sstdescription{
      This script reduces a ``standard jitter'' photometry observation
      with UFTI or IRCAM data.  It takes an imaging observation
      comprising jittered object frames and a dark frame to make a
      calibrated, untrimmed mosaic automatically.

      It performs a null debiassing, bad-pixel masking, dark
      subtraction, flat-field creation, flat-field division, integer
      shifts of origin to register, and mosaicking.  See the ``Notes''
      for further information.

      This recipe aims to keep pace with the pipeline's incoming
      data, and is intended for faint sources and for moderately
      crowded fields.
   }
   \sstnotes{
      \sstitemlist{

         \sstitem
         A World Co-ordinate System (WCS) using the AIPS convention is
         created in the headers should no WCS already exist.

         \sstitem
         For IRCAM, old headers are reordered and structured with
         headings before groups of related keywords.  The comments have
         units added or appear in a standard format.  Four deprecated
         deprecated are removed.  FITS-violating headers are corrected.
         Spurious instrument names are changed to IRCAM3.

         \sstitem
         The bad-pixel mask applied is {\tt\$ORAC\_DATA\_CAL/bpm}.

         \sstitem
         Each dark-subtracted frame has thresholds applied beyond which
         pixels are flagged as bad.  The lower limit is 5 standard
         deviations below the mode, but constrained to the range $-$100 to 1.
         The upper limit is 1000 above the saturation limit for the detector
         in the mode used.

         \sstitem
         The flat field is created by combining normalised object
         frames using the clipped median at each pixel.

         \sstitem
         Registration is performed using the telescope offsets
         transformed to pixels.

         \sstitem
         There is no resampling, merely integer shifts of origin.

         \sstitem
         The recipe makes the mosaics by applying offsets in intensity
         to give the most consistent result amongst the overlapping regions.
         The mosaic is not trimmed to the dimensions of a single frame, thus
         the noise will be greater in the peripheral areas having received
         less exposure time.  The mosaic is not normalised by its exposure
         time (that being the exposure time of a single frame).

         \sstitem
         For each cycle of jittered frames, the recipe creates a mosaic,
         which is then added into a master mosaic of improving signal to noise.
         The exposure time is also summed and stored in the mosaic's EXP\_TIME
         (UFTI) or DEXPTIME (IRCAM) header.  Likewise the end airmass header,
         AMEND, is updated to match that of the last-observed frame
         contributing to the mosaic.

         \sstitem
         Intermediate frames are deleted except for the flat-fielded ({\tt\_ff}
         suffix) frames.
      }
   }
   \sstdiytopic{
      Output Data
   }{
      \sstitemlist{

         \sstitem
         The integrated mosaic in {\tt$<$m$>$$<$date$>$\_$<$group\_number$>$\_mos}, where {\tt$<$m$>$}
         is {\tt{gf}} for UFTI and {\tt{gi}} for IRCAM.  Before 2000 August these
         were {\tt{g}} and {\tt{rg}} respectively.

         \sstitem
         A mosaic for each cycle of jittered frames in \\
         {\tt$<$m$>$$<$date$>$\_$<$group\_number$>$\_mos$<$cycle\_number$>$}, where {\tt$<$cycle\_number$>$}\\
         counts from 0.

         \sstitem
         The individual flat-fielded frames in {\tt$<$i$>$$<$date$>$\_$<$obs\_number$>$\_ff},
         where {\tt$<$i$>$} is {\tt{f}} for UFTI and {\tt{i}} for IRCAM.  Before 2000 August
         IRCAM frames had prefix {\tt{ro}}.

         \sstitem
         The created flat fields in {\tt{flat\_$<$filter$>$\_$<$group\_number$>$}} for the
         first or only cycle, and {\tt{{\tt{flat\_$<$filter$>$\_$<$group\_number$>$}}\_c$<$cycle\_number$>$}}
         for subsequent cycles.
      }
   }
   \sstparameters{
      \sstsubsection{
         NUMBER = INTEGER
      }{
         The number of frames in the jitter pattern.  If not supplied
         the number of offsets, as given by FITS header NOFFSETS, minus
         one is used.  If neither is available, 9 is the default.  An
         error state arises if the number of jittered frames is fewer
         than 3.  For observations prior to the availability of full
         ORAC, header NOFFSETS will be absent.  {\tt[]}
      }
   }
   \sstdiytopic{
      Related Recipes
   }{
      \htmlref{JITTER\_SELF\_FLAT}{JITTER\_SELF\_FLAT}, 
      \htmlref{JITTER\_SELF\_FLAT\_APHOT}{JITTER\_SELF\_FLAT\_APHOT},\\
      \htmlref{JITTER\_SELF\_FLAT\_NO\_MASK}{JITTER\_SELF\_FLAT\_NO\_MASK},
      \htmlref{JITTER\_SELF\_FLAT\_TELE}{JITTER\_SELF\_FLAT\_TELE},\\
      \htmlref{MOVING\_JITTER\_SELF\_FLAT\_BASIC}{MOVING\_JITTER\_SELF\_FLAT\_BASIC},
      \htmlref{QUADRANT\_JITTER\_BASIC}{QUADRANT\_JITTER\_BASIC}.
   }
   \sstimplementationstatus{
      \sstitemlist{

         \sstitem
         The processing engines are from the Starlink packages: \xref{\CCDPACK}{sun139}{}
         \xref{\KAPPA}{sun95}{}, and \xref{\PISA}{sun109}{}.

         \sstitem
         Uses the Starlink NDF format.

         \sstitem
         History is recorded within the data files.

         \sstitem
         The title of the data is propagated through intermediate files
         to the mosaic.

         \sstitem
         Error propagation is not used.
      }
   }
   \sstdiytopic{
      Deprecated Variants
   }{
      JITTER5\_SELF\_FLAT\_BASIC, JITTER9\_SELF\_FLAT\_BASIC.
   }
}

%\newpage
\sstroutine{
   JITTER\_SELF\_FLAT\_NCOLOUR
}{
   Reduces a multi-colour ``standard jitter'' photometry observation
   using object masking
}{
   \sstdescription{
      This script reduces a ``standard jitter'' photometry observation
      with UFTI or IRCAM data observed through one or more filters.  For
      each filter it takes an UFTI observation comprising jittered object
      frames and a dark frame to make a calibrated, untrimmed mosaic
      automatically.

      It performs a null debiassing, bad-pixel masking, dark
      subtraction, flat-field creation and division, feature
      detection and matching between object frames, and resampling.
      See the ``Notes'' for further information.

      This recipe works well for faint sources and for moderately
      crowded fields.
   }
   \sstnotes{
      \sstitemlist{

         \sstitem
         A World Co-ordinate System (WCS) using the AIPS convention is
         created in the headers should no WCS already exist.

         \sstitem
         For IRCAM, old headers are reordered and structured with
         headings before groups of related keywords.  The comments have
         units added or appear in a standard format.  Four deprecated
         deprecated are removed.  FITS-violating headers are corrected.
         Spurious instrument names are changed to IRCAM3.

         \sstitem
         The bad pixel mask applied is {\tt\$ORAC\_DATA\_CAL/bpm}.

         \sstitem
         Each dark-subtracted frame has thresholds applied beyond which
         pixels are flagged as bad.  The lower limit is 5 standard
         deviations below the mode, but constrained to the range $-$100 to 1.
         The upper limit is 1000 above the saturation limit for the detector
         in the mode used.

         \sstitem
         The flat field is created iteratively.  First an approximate
         flat-field is created by combining normalised object frames using
         the median at each pixel.  This flat field is applied to the object
         frames.  Sources within the flat-fielded frames are detected, and
         masked in the dark-subtracted frames.  The first stage is repeated
         but applied to the masked frames to create the final flat field.

         \sstitem
         Registration is performed using common point sources in the
         overlap regions.  If the recipe cannot identify sufficient common
         objects, the script resorts to using the telescope offsets
         transformed to pixels.

         \sstitem
         The resampling applies non-integer shifts of origin using
         bilinear interpolation.  There is no rotation to align the
         Cartesian axes with the cardinal directions.

         \sstitem
         The recipe makes the mosaics by applying offsets in intensity
         to give the most consistent result amongst the overlapping regions.
         The mosaic is not trimmed to the dimensions of a single frame, thus
         the noise will be greater in the peripheral areas having received
         less exposure time.

         \sstitem
         For each cycle of jittered frames, the recipe creates a mosaic,
         which has its bad pixels filled and is then added into a master
         mosaic of improving signal to noise.  The exposure time is also
         summed and stored in the mosaic's EXP\_TIME (UFTI) or DEXPTIME
         (IRCAM) header.  Likewise the end airmass header, AMEND, is updated
         to match that of the last-observed frame contributing to the mosaic.

         \sstitem
         Intermediate frames are deleted except for the flat-fielded ({\tt\_ff}
         suffix) frames.
      }
   }
   \sstdiytopic{
      Output Data
   }{
      \sstitemlist{

         \sstitem
         Integrated mosaics in {\tt$<$m$>$$<$date$>$\_$<$group\_number$>$\_$<$filter$>$\_mos}, where
         {\tt$<$m$>$} is {\tt{gf}} for UFTI and {\tt{gi}} for IRCAM.  Before 2000 August these
         were {\tt{g}} and {\tt{rg}} respectively,. and $<$filter$>$ is the filter
         name.

         \sstitem
         A mosaic for each cycle of jittered frames for each filter in \\
         {\tt$<$m$>$$<$date$>$\_$<$group\_number$>$\_$<$filter$>$\_mos$<$cycle\_number$>$}, where\\
         {\tt$<$cycle\_number$>$} counts from 0.

         \sstitem
         The individual flat-fielded frames in {\tt$<$i$>$$<$date$>$\_$<$obs\_number$>$\_ff},
         where {\tt$<$i$>$} is {\tt{f}} for UFTI and {\tt{i}} for IRCAM.  Before 2000 August
         IRCAM frames had prefix {\tt{ro}}.

         \sstitem
         The created flat fields in {\tt{flat\_$<$filter$>$\_$<$group\_number$>$}}
         for the first or only cycle, and
         {\tt{{\tt{flat\_$<$filter$>$\_$<$group\_number$>$}}\_c$<$cycle\_number$>$}} for subsequent
         cycles.
      }
   }
   \sstparameters{
      \sstsubsection{
         NUMBER = INTEGER
      }{
         The number of frames in the jitter.  If absent, the number of
         offsets, as given by header NOFFSETS, minus one is used.  If
         neither is available, 5 is used.  An error state arises if
         the number of jittered frames is fewer than 3.  {\tt[]}
      }
   }
   \sstdiytopic{
      Related Recipes
   }{
      \htmlref{JITTER\_SELF\_FLAT}{JITTER\_SELF\_FLAT}.
   }
   \sstimplementationstatus{
      \sstitemlist{

         \sstitem
         The processing engines are from the Starlink packages: \xref{\CCDPACK}{sun139}{}
         \xref{\KAPPA}{sun95}{}, and \xref{\PISA}{sun109}{}.

         \sstitem
         Uses the Starlink NDF format.

         \sstitem
         History is recorded within the data files.

         \sstitem
         The title of the data is propagated through intermediate files
         to the mosaic.

         \sstitem
         Error propagation is not used.
      }
   }
   \sstdiytopic{
      Deprecated Variants
   }{
      JITTER5\_SELF\_FLAT\_NCOLOUR.
   }
}

%\newpage
\sstroutine{
   JITTER\_SELF\_FLAT\_NO\_MASK
}{
   Reduces a ``standard jitter'' photometry observation without object
   masking
}{
   \sstdescription{
      This script reduces a ``standard jitter'' photometry observation
      with UFTI or IRCAM data.  It takes an imaging observation
      comprising jittered object frames and a dark frame to make a
      calibrated, untrimmed mosaic automatically.

      It performs a null debiassing, bad-pixel masking, dark
      subtraction, flat-field creation and division, feature
      detection and matching between object frames, and resampling.
      See the ``Notes'' for further information.

      This recipe works well for faint sources and sparse fields.
   }
   \sstnotes{
      \sstitemlist{

         \sstitem
         A World Co-ordinate System (WCS) using the AIPS convention is
         created in the headers should no WCS already exist.

         \sstitem
         For IRCAM, old headers are reordered and structured with
         headings before groups of related keywords.  The comments have
         units added or appear in a standard format.  Four deprecated
         deprecated are removed.  FITS-violating headers are corrected.
         Spurious instrument names are changed to IRCAM3.

         \sstitem
         The bad-pixel mask applied is {\tt\$ORAC\_DATA\_CAL/bpm}.

         \sstitem
         Each dark-subtracted frame has thresholds applied beyond which
         pixels are flagged as bad.  The lower limit is 5 standard
         deviations below the mode, but constrained to the range $-$100 to 1.
         The upper limit is 1000 above the saturation limit for the detector
         in the mode used.

         \sstitem
         The flat field is created by combining normalised object
         frames using the clipped median at each pixel.

         \sstitem
         Registration is performed using common point sources in the
         overlap regions.  If the recipe cannot identify sufficient common
         objects, the script resorts to using the telescope offsets
         transformed to pixels.

         \sstitem
         The resampling applies non-integer shifts of origin using
         bilinear interpolation.  There is no rotation to align the
         Cartesian axes with the cardinal directions.

         \sstitem
         The recipe makes the mosaics by applying offsets in intensity
         to give the most consistent result amongst the overlapping regions.
         The mosaic is not trimmed to the dimensions of a single frame, thus
         the noise will be greater in the peripheral areas having received
         less exposure time.  The mosaic is not normalised by its exposure
         time (that being the exposure time of a single frame).

         \sstitem
         For each cycle of jittered frames, the recipe creates a mosaic,
         which has its bad pixels filled and is then added into a master
         mosaic of improving signal to noise.  The exposure time is also
         summed and stored in the mosaic's EXP\_TIME (UFTI) or DEXPTIME
         (IRCAM) header.  Likewise the end airmass header, AMEND, is updated
         to match that of the last-observed frame contributing to the mosaic.

         \sstitem
         Intermediate frames are deleted except for the flat-fielded ({\tt\_ff}
         suffix) frames.
      }
   }
   \sstdiytopic{
      Output Data
   }{
      \sstitemlist{

         \sstitem
         The integrated mosaic in {\tt$<$m$>$$<$date$>$\_$<$group\_number$>$\_mos}, where {\tt$<$m$>$}
         is {\tt{gf}} for UFTI and {\tt{gi}} for IRCAM.  Before 2000 August these
         were {\tt{g}} and {\tt{rg}} respectively.

         \sstitem
         A mosaic for each cycle of jittered frames in \\
         {\tt$<$m$>$$<$date$>$\_$<$group\_number$>$\_mos$<$cycle\_number$>$}, where {\tt$<$cycle\_number$>$}\\
         counts from 0.

         \sstitem
         The individual flat-fielded frames in {\tt$<$i$>$$<$date$>$\_$<$obs\_number$>$\_ff},
         where {\tt$<$i$>$} is {\tt{f}} for UFTI and {\tt{i}} for IRCAM.  Before 2000 August
         IRCAM frames had prefix {\tt{ro}}.

         \sstitem
         The created flat fields in {\tt{flat\_$<$filter$>$\_$<$group\_number$>$}} for the
         first or only cycle, and {\tt{{\tt{flat\_$<$filter$>$\_$<$group\_number$>$}}\_c$<$cycle\_number$>$}}
         for subsequent cycles.
      }
   }
   \sstparameters{
      \sstsubsection{
         NUMBER = INTEGER
      }{
         The number of frames in the jitter pattern.  If not supplied
         the number of offsets, as given by FITS header NOFFSETS, minus
         one is used.  If neither is available, 9 is the default.  An
         error state arises if the number of jittered frames is fewer
         than 3.  For observations prior to the availability of full
         ORAC, header NOFFSETS will be absent.  {\tt[]}
      }
   }
   \sstdiytopic{
      Related Recipes
   }{
      \htmlref{JITTER\_SELF\_FLAT}{JITTER\_SELF\_FLAT},
      \htmlref{JITTER\_SELF\_FLAT\_APHOT}{JITTER\_SELF\_FLAT\_APHOT},\\
      \htmlref{JITTER\_SELF\_FLAT\_BASIC}{JITTER\_SELF\_FLAT\_BASIC},
      \htmlref{JITTER\_SELF\_FLAT\_TELE}{JITTER\_SELF\_FLAT\_TELE},\\
      \htmlref{QUADRANT\_JITTER\_NO\_MASK}{QUADRANT\_JITTER\_NO\_MASK}.
   }
   \sstimplementationstatus{
      \sstitemlist{

         \sstitem
         The processing engines are from the Starlink packages: \xref{\CCDPACK}{sun139}{}
         \xref{\KAPPA}{sun95}{}, and \xref{\PISA}{sun109}{}.

         \sstitem
         Uses the Starlink NDF format.

         \sstitem
         History is recorded within the data files.

         \sstitem
         The title of the data is propagated through intermediate files
         to the mosaic.

         \sstitem
         Error propagation is not used.
      }
   }
   \sstdiytopic{
      Deprecated Variants
   }{
      JITTER5\_SELF\_FLAT\_NO\_MASK, JITTER9\_SELF\_FLAT\_NO\_MASK.
   }
}

%\newpage
\sstroutine{
   JITTER\_SELF\_FLAT\_TELE
}{
   Reduces a ``standard jitter'' photometry observation using
   object masking, and telescope offsets for registration
}{
   \sstdescription{
      This script reduces a ``standard jitter'' photometry observation
      with UFTI or IRCAM data.  It takes an observation comprising jittered
      object frames and a dark frame to make a calibrated, untrimmed
      mosaic automatically.

      It performs a null debiassing, bad-pixel masking, dark
      subtraction, flat-field creation and division, registration
      using telescope offsets, and resampling.  See the ``Notes'' for
      further information.

      This recipe works well for faint sources and for moderately
      crowded fields.  It is also used for observations that track a
      moving object.
   }
   \sstnotes{
      \sstitemlist{

         \sstitem
         A World Co-ordinate System (WCS) using the AIPS convention is
         created in the headers should no WCS already exist.

         \sstitem
         For IRCAM, old headers are reordered and structured with
         headings before groups of related keywords.  The comments have
         units added or appear in a standard format.  Four deprecated
         deprecated are removed.  FITS-violating headers are corrected.
         Spurious instrument names are changed to IRCAM3.

         \sstitem
         The bad-pixel mask applied is {\tt\$ORAC\_DATA\_CAL/bpm}.

         \sstitem
         Each dark-subtracted frame has thresholds applied beyond which
         pixels are flagged as bad.  The lower limit is 5 standard
         deviations below the mode, but constrained to the range $-$100 to 1.
         The upper limit is 1000 above the saturation limit for the detector
         in the mode used.

         \sstitem
         The flat field is created iteratively.  First an approximate
         flat-field is created by combining normalised object frames using
         the median at each pixel.  This flat field is applied to the object
         frames.  Sources within the flat-fielded frames are detected, and
         masked in the dark-subtracted frames.  The first stage is repeated
         but applied to the masked frames to create the final flat field.

         \sstitem
         Registration is performed using the telescope offsets
         transformed to pixels.

         \sstitem
         The resampling applies non-integer shifts of origin using
         bilinear interpolation.  There is no rotation to align the
         Cartesian axes with the cardinal directions.

         \sstitem
         The recipe makes the mosaics by applying offsets in intensity
         to give the most consistent result amongst the overlapping regions.
         The mosaic is not trimmed to the dimensions of a single frame, thus
         the noise will be greater in the peripheral areas having received
         less exposure time.  The mosaic is not normalised by its exposure
         time (that being the exposure time of a single frame).

         \sstitem
         For each cycle of jittered frames, the recipe creates a mosaic,
         which has its bad pixels filled and is then added into a master
         mosaic of improving signal to noise.  The exposure time is also
         summed and stored in the mosaic's EXP\_TIME (UFTI) or DEXPTIME
         (IRCAM) header.  Likewise the end airmass header, AMEND, is updated
         to match that of the last-observed frame contributing to the mosaic.

         \sstitem
         Intermediate frames are deleted except for the flat-fielded ({\tt\_ff}
         suffix) frames.
      }
   }
   \sstdiytopic{
      Output Data
   }{
      \sstitemlist{

         \sstitem
         The integrated mosaic in {\tt$<$m$>$$<$date$>$\_$<$group\_number$>$\_mos}, where {\tt$<$m$>$}
         is {\tt{gf}} for UFTI and {\tt{gi}} for IRCAM.  Before 2000 August these
         were {\tt{g}} and {\tt{rg}} respectively.

         \sstitem
         A mosaic for each cycle of jittered frames in\\
         {\tt$<$m$>$$<$date$>$\_$<$group\_number$>$\_mos$<$cycle\_number$>$}, where {\tt$<$cycle\_number$>$}\\
         counts from 0.

         \sstitem
         The individual flat-fielded frames in {\tt$<$i$>$$<$date$>$\_$<$obs\_number$>$\_ff},
         where {\tt$<$i$>$} is {\tt{f}} for UFTI and {\tt{i}} for IRCAM.  Before 2000 August
         IRCAM frames had prefix {\tt{ro}}.

         \sstitem
         The created flat fields in {\tt{flat\_$<$filter$>$\_$<$group\_number$>$}} for the
         first or only cycle, and {\tt{{\tt{flat\_$<$filter$>$\_$<$group\_number$>$}}\_c$<$cycle\_number$>$}}
         for subsequent cycles.
      }
   }
   \sstparameters{
      \sstsubsection{
         NUMBER = INTEGER
      }{
         The number of frames in the jitter pattern.  If not supplied
         the number of offsets, as given by FITS header NOFFSETS, minus
         one is used.  If neither is available, 9 is the default.  An
         error state arises if the number of jittered frames is fewer
         than 3.  For observations prior to the availability of full
         ORAC, header NOFFSETS will be absent.  {\tt[]}
      }
   }
   \sstdiytopic{
      Related Recipes
   }{
      \htmlref{JITTER\_SELF\_FLAT}{JITTER\_SELF\_FLAT},
      \htmlref{JITTER\_SELF\_FLAT\_APHOT}{JITTER\_SELF\_FLAT\_APHOT},\\
      \htmlref{JITTER\_SELF\_FLAT\_BASIC}{JITTER\_SELF\_FLAT\_BASIC},
      \htmlref{JITTER\_SELF\_FLAT\_NO\_MASK}{JITTER\_SELF\_FLAT\_NO\_MASK},\\
      \htmlref{MOVING\_JITTER\_SELF\_FLAT}{MOVING\_JITTER\_SELF\_FLAT},
      \htmlref{QUADRANT\_JITTER}{QUADRANT\_JITTER}.
   }
   \sstimplementationstatus{
      \sstitemlist{

         \sstitem
         The processing engines are from the Starlink packages: \xref{\CCDPACK}{sun139}{}
         \xref{\KAPPA}{sun95}{}, and \xref{\PISA}{sun109}{}.

         \sstitem
         Uses the Starlink NDF format.

         \sstitem
         History is recorded within the data files.

         \sstitem
         The title of the data is propagated through intermediate files
         to the mosaic.

         \sstitem
         Error propagation is not used.
      }
   }
   \sstdiytopic{
      Deprecated Variants
   }{
      JITTER9\_SELF\_FLAT\_TELE.
   }
}

%\newpage
\sstroutine{
   MOVING\_JITTER\_SELF\_FLAT
}{
   Reduces a ``standard jitter'' photometry observation of a
   moving target using object masking
}{
   \sstdescription{
      This script reduces a ``standard jitter'' photometry observation
      with UFTI or IRCAM data.  It takes an observation comprising
      jittered object frames and a dark frame to make a calibrated,
      untrimmed mosaic automatically.

      It performs a null debiassing, bad-pixel masking, dark
      subtraction, flat-field creation and division, feature
      detection and matching between object frames, and resampling.
      See the ``Notes'' for further information.

      Registration is adjusted to track the motion of the moving
      target using ephemeris data stored in file {\tt target\_ephem.dat}.
      See \htmlref{``Ephemeris-file Format''}{ephem_format} for details
      of this file's format.

      This recipe works well for faint moving sources and in moderately
      crowded fields.  It should not be used for frames where the
      telescope guided on the moving object.  In that case reduction
      should be performed by \htmlref{JITTER\_SELF\_FLAT\_TELE}{JITTER\_SELF\_FLAT\_TELE} which registers
      using the telescope offsets alone.
   }
   \sstnotes{
      \sstitemlist{
         \sstitem
         A World Co-ordinate System (WCS) using the AIPS convention is
         created in the headers should no WCS already exist.

         \sstitem
         For IRCAM, old headers are reordered and structured with
         headings before groups of related keywords.  The comments have
         units added or appear in a standard format.  Four deprecated
         deprecated are removed.  FITS-violating headers are corrected.
         Spurious instrument names are changed to IRCAM3.

         \sstitem
         The bad pixel mask applied is {\tt\$ORAC\_DATA\_CAL/bpm}.

         \sstitem
         Each dark-subtracted frame has thresholds applied beyond which
         pixels are flagged as bad.  The lower limit is 5 standard
         deviations below the mode, but constrained to the range $-$100 to 1.
         The upper limit is 1000 above the saturation limit for the detector
         in the mode used.

         \sstitem
         The flat field is created iteratively.  First an approximate
         flat-field is created by combining normalised object frames using
         the median at each pixel.  This flat field is applied to the object
         frames.  Sources within the flat-fielded frames are detected, and
         masked in the dark-subtracted frames.  The first stage is repeated
         but applied to the masked frames to create the final flat field.

         \sstitem
         Registration is performed using common point sources in the
         overlap regions.  If the recipe cannot identify sufficient common
         objects, the script resorts to using the telescope offsets
         transformed to pixels.  Once the offsets are determined, they
         are adjusted for the motion of the target, so that the final
         mosaic registers the target, not the background stars.

         \sstitem
         The ephemeris file is specified by environment variable
         {\tt{ORAC\_EPHEMERIS}}, defaulting to {\tt\$ORAC\_DATA\_OUT/target\_ephem.dat}.

         \sstitem
         The resampling applies non-integer shifts of origin using
         bilinear interpolation.  There is no rotation to align the
         Cartesian axes with the cardinal directions.

         \sstitem
         The recipe makes the mosaics by applying offsets in intensity
         to give the most consistent result amongst the overlapping regions.
         The mosaic is not trimmed to the dimensions of a single frame, thus
         the noise will be greater in the peripheral areas having received
         less exposure time.  The mosaic is not normalised by its exposure
         time (that being the exposure time of a single frame).

         \sstitem
         For each cycle of jittered frames, the recipe creates a mosaic,
         which has its bad pixels filled and is then added into a master
         mosaic of improving signal to noise.  The exposure time is also
         summed and stored in the mosaic's EXP\_TIME (UFTI) or DEXPTIME
         (IRCAM) header.  Likewise the end airmass header, AMEND, is updated
         to match that of the last-observed frame contributing to the mosaic.

         \sstitem
         Intermediate frames are deleted except for the flat-fielded ({\tt\_ff}
         suffix) frames.
      }
   }
   \label{ephem_format}
   \sstdiytopic{
      Ephemeris-file Format
   }{
      The current format of the ephemeris file is one line per object
      comprising three space-separated fields in the following order:
      \ssthitemlist{

         \sstitem
           the objectname, which may contain embedded spaces;

         \sstitem
           the motion in the plane of the sky in arcsec/second for right
           ascension then declination.

      }
      Note that the right ascension motion is the change in right ascension
      multiplied by the cosine of the declination.  The format will change
      to include UT and possibly date.
   }
   \sstdiytopic{
      Output Data
   }{
      \sstitemlist{

         \sstitem
         The integrated mosaic in {\tt$<$m$>$$<$date$>$\_$<$group\_number$>$\_mos}, where {\tt$<$m$>$}
         is {\tt{gf}} for UFTI and {\tt{gi}} for IRCAM.  Before 2000 August these
         were {\tt{g}} and {\tt{rg}} respectively.

         \sstitem
         A mosaic for each cycle of jittered frames in \\
         {\tt$<$m$>$$<$date$>$\_$<$group\_number$>$\_mos$<$cycle\_number$>$}, where {\tt$<$cycle\_number$>$}\\
         counts from 0.

         \sstitem
         The individual flat-fielded frames in {\tt$<$i$>$$<$date$>$\_$<$obs\_number$>$\_ff},
         where {\tt$<$i$>$} is {\tt{f}} for UFTI and {\tt{i}} for IRCAM.  Before 2000 August
         IRCAM frames had prefix {\tt{ro}}.

         \sstitem
         The created flat fields in {\tt{flat\_$<$filter$>$\_$<$group\_number$>$}} for the
         first or only cycle, and {\tt{{\tt{flat\_$<$filter$>$\_$<$group\_number$>$}}\_c$<$cycle\_number$>$}}
         for subsequent cycles.
      }
   }
   \sstparameters{
      \sstsubsection{
         NUMBER = INTEGER
      }{
         The number of frames in the jitter pattern.  If not supplied
         the number of offsets, as given by FITS header NOFFSETS, minus
         one is used.  If neither is available, 9 is the default.  An
         error state arises if the number of jittered frames is fewer
         than 3.  For observations prior to the availability of full
         ORAC, header NOFFSETS will be absent.  {\tt[]}
      }
   }
   \sstdiytopic{
      Related Recipes
   }{
      \htmlref{JITTER\_SELF\_FLAT}{JITTER\_SELF\_FLAT},
      \htmlref{JITTER\_SELF\_FLAT\_TELE}{JITTER\_SELF\_FLAT\_TELE},\\
      \htmlref{MOVING\_JITTER\_SELF\_FLAT\_BASIC}{MOVING\_JITTER\_SELF\_FLAT\_BASIC}.
   }
   \sstimplementationstatus{
      \sstitemlist{

         \sstitem
         The processing engines are from the Starlink packages: \xref{\CCDPACK}{sun139}{}
         \xref{\KAPPA}{sun95}{}, and \xref{\PISA}{sun109}{}.

         \sstitem
         Uses the Starlink NDF format.

         \sstitem
         History is recorded within the data files.

         \sstitem
         The title of the data is propagated through intermediate files
         to the mosaic.

         \sstitem
         Error propagation is not used.
      }
   }
   \sstdiytopic{
      Deprecated Variants
   }{
      MOVING\_JITTER9\_SELF\_FLAT.
   }
}

%\newpage
\sstroutine{
   MOVING\_JITTER\_SELF\_FLAT\_BASIC
}{
   Reduces a ``standard jitter'' photometry observation of a
   moving target using just the basic operations for speed
}{
   \sstdescription{
      This script reduces a ``standard jitter'' photometry observation
      with UFTI or IRCAM data.  It takes an observation comprising
      jittered object frames and a dark frame to make a calibrated,
      untrimmed mosaic automatically.

      It performs a null debiassing, bad-pixel masking, dark
      subtraction, flat-field creation and division, amd integer shifts
      of pixel origin to register to fixed sky co-ordinates.   See the
      ``Notes'' for further information.

      The registration is adjusted to track the motion of the moving
      target using ephemeris data stored in file {\tt target\_ephem.dat}.
      See \htmlref{``Ephemeris-file Format''}{ephem_format} for details
      of this file's format.

      This recipe aims to keep pace with the pipeline's incoming data.
      It works well for faint moving sources and in moderately
      crowded fields.  It should not be used for frames where the
      telescope guided on the moving object.  In that case reduction
      should be performed by \htmlref{JITTER\_SELF\_FLAT\_TELE}{JITTER\_SELF\_FLAT\_TELE} which registers
      using the telescope offsets alone.
   }
   \sstnotes{
      \sstitemlist{

         \sstitem
         A World Co-ordinate System (WCS) using the AIPS convention is
         created in the headers should no WCS already exist.

         \sstitem
         For IRCAM, old headers are reordered and structured with
         headings before groups of related keywords.  The comments have
         units added or appear in a standard format.  Four deprecated
         deprecated are removed.  FITS-violating headers are corrected.
         Spurious instrument names are changed to IRCAM3.

         \sstitem
         The bad pixel mask applied is {\tt\$ORAC\_DATA\_CAL/bpm}.

         \sstitem
         Each dark-subtracted frame has thresholds applied beyond which
         pixels are flagged as bad.  The lower limit is 5 standard
         deviations below the mode, but constrained to the range $-$100 to 1.
         The upper limit is 1000 above the saturation limit for the detector
         in the mode used.

         \sstitem
         The flat field is created by combining normalised object
         frames using the median at each pixel.

         \sstitem
         Registration is performed using the telescope offsets
         transformed to pixels.  Once the offsets are determined, they
         are adjusted for the motion of the target, so that the final
         mosaic registers the target, not the background stars.

         \sstitem
         There is no resampling, merely integer shifts of origin.

         \sstitem
         The ephemeris file is specified by environment variable
         {\tt{ORAC\_EPHEMERIS}}, defaulting to {\tt \$ORAC\_DATA\_OUT/target\_ephem.dat}.

         \sstitem
         The recipe makes the mosaics by applying offsets in intensity
         to give the most consistent result amongst the overlapping regions.
         The mosaic is not trimmed to the dimensions of a single frame, thus
         the noise will be greater in the peripheral areas having received
         less exposure time.  The mosaic is not normalised by its exposure
         time (that being the exposure time of a single frame).

         \sstitem
         For each cycle of jittered frames, the recipe creates a mosaic,
         which is then added into a master mosaic of improving signal to
         noise.  The exposure time is also summed and stored in the mosaic's
         EXP\_TIME (UFTI) or DEXPTIME (IRCAM) header.  Likewise the end
         airmass header, AMEND, is updated to match that of the last-observed
         frame contributing to the mosaic.

         \sstitem
         Intermediate frames are deleted except for the flat-fielded ({\tt\_ff}
         suffix) frames.
      }
   }
   \sstdiytopic{
      Ephemeris-file Format
   }{
      The current format of the ephemeris file is one line per object
      comprising three space-separated fields in the following order:
      \ssthitemlist{

         \sstitem
           the objectname, which may contain embedded spaces; and

         \sstitem
           the motion in the plane of the sky in arcsec/second for right
         ascension then declination.

      }
      Note that the right-ascension motion is the change in right ascension
      multiplied by the cosine of the declination.  The format may change
      to include UT and possibly date.
   }
   \sstdiytopic{
      Output Data
   }{
      \sstitemlist{

         \sstitem
         The integrated mosaic in {\tt$<$m$>$$<$date$>$\_$<$group\_number$>$\_mos}, where {\tt$<$m$>$}
         is {\tt{gf}} for UFTI and {\tt{gi}} for IRCAM.  Before 2000 August these
         were {\tt{g}} and {\tt{rg}} respectively.

         \sstitem
         A mosaic for each cycle of jittered frames in \\
         {\tt$<$m$>$$<$date$>$\_$<$group\_number$>$\_mos$<$cycle\_number$>$}, where {\tt$<$cycle\_number$>$}\\
         counts from 0.

         \sstitem
         The individual flat-fielded frames in {\tt$<$i$>$$<$date$>$\_$<$obs\_number$>$\_ff},
         where {\tt$<$i$>$} is {\tt{f}} for UFTI and {\tt{i}} for IRCAM.  Before 2000 August
         IRCAM frames had prefix {\tt{ro}}.

         \sstitem
         The created flat fields in {\tt{flat\_$<$filter$>$\_$<$group\_number$>$}} for the
         first or only cycle, and {\tt{{\tt{flat\_$<$filter$>$\_$<$group\_number$>$}}\_c$<$cycle\_number$>$}}
         for subsequent cycles.
      }
   }
   \sstparameters{
      \sstsubsection{
         NUMBER = INTEGER
      }{
         The number of frames in the jitter pattern.  If not supplied
         the number of offsets, as given by FITS header NOFFSETS, minus
         one is used.  If neither is available, 9 is the default.  An
         error state arises if the number of jittered frames is fewer
         than 3.  For observations prior to the availability of full
         ORAC, header NOFFSETS will be absent.  {\tt[]}
      }
   }
   \sstdiytopic{
      Related Recipes
   }{
      \htmlref{JITTER\_SELF\_FLAT\_BASIC}{JITTER\_SELF\_FLAT\_BASIC},
      \htmlref{JITTER\_SELF\_FLAT\_TELE}{JITTER\_SELF\_FLAT\_TELE},\\
      \htmlref{MOVING\_JITTER\_SELF\_FLAT}{MOVING\_JITTER\_SELF\_FLAT}.
   }
   \sstimplementationstatus{
      \sstitemlist{

         \sstitem
         The processing engines are from the Starlink packages: \xref{\CCDPACK}{sun139}{}
         and \xref{\KAPPA}{sun95}{}.

         \sstitem
         Uses the Starlink NDF format.

         \sstitem
         History is recorded within the data files.

         \sstitem
         The title of the data is propagated through intermediate files
         to the mosaic.

         \sstitem
         Error propagation is not used.
      }
   }
   \sstdiytopic{
      Deprecated Variants
   }{
      MOVING\_JITTER9\_SELF\_FLAT\_BASIC.
   }
}

%\newpage
\sstroutine{
   MOVING\_QUADRANT\_JITTER
}{
   Reduces a ``Quadrant Jitter'' observation, of a moving target including
   object masking
}{
   \sstdescription{
      This script reduces a ``quadrant jitter'' photometry observation
      with UFTI or IRCAM data.  It takes an imaging observation comprising
      one or more series of four object frames where the target is
      approximately centred in each quadrant; and a dark frame to make
      a calibrated, untrimmed mosaic automatically.

      It performs bad-pixel masking, null debiassing, dark subtraction,
      flat-field creation and division, feature detection and matching
      between object frames, and resampling.   See the ``Notes'' for
      further information.

      Registration is adjusted to track the motion of the moving
      target using ephemeris data stored in file {\tt target\_ephem.dat}.
      See \htmlref{``Ephemeris-file Format''}{ephem_format} for details
      of this file's format.

      This recipe works well for extended moving sources (comets), whose
      extent does not exceed 45 arcseconds for UFTI or 10 arcseconds
      for IRCAM, in moderately crowded fields.  Sources may include
      those with a comparatively bright core embedded in faint extended
      emission.  The object need not be isolated, as the recipe masks
      objects within the other quadrants, and hence does not introduce
      significant artifacts into the flat field.  This recipe should not
      be used for frames where the telescope guided on the moving object.
      In that case reduction should be performed by
      \htmlref{QUADRANT\_JITTER\_TELE}{QUADRANT\_JITTER\_TELE},
      which registers using the telescope offsets alone.
   }
   \sstnotes{
      \sstitemlist{

         \sstitem
         A World Co-ordinate System (WCS) using the AIPS convention is
         created in the headers should no WCS already exist.

         \sstitem
         For IRCAM, old headers are reordered and structured with
         headings before groups of related keywords.  The comments have
         units added or appear in a standard format.  Four deprecated
         deprecated are removed.  FITS-violating headers are corrected.
         Spurious instrument names are changed to IRCAM3.

         \sstitem
         The bad pixel mask applied is {\tt\$ORAC\_DATA\_CAL/bpm}.

         \sstitem
         Each dark-subtracted frame has thresholds applied beyond which
         pixels are flagged as bad.  The lower limit is 5 standard
         deviations below the mode, but constrained to the range $-$100 to 1.
         The upper limit is 1000 above the saturation limit for the detector
         in the mode used.

         \sstitem
         The flat field is created iteratively.  First the quadrant
         containing the object is masked in each object frame.  Second an
         approximate flat field is created by combining the normalised
         and masked object frames using the clipped median at each pixel.
         This flat field is applied to the object frames.  Sources within
         the flat-fielded frames are detected, and masked in the
         dark-subtracted frames.  The second stage is repeated but applied
         to the masked frames to create the final flat field.

         \sstitem
         Registration is performed using common point sources in the
         overlap regions.  If the recipe cannot identify sufficient common
         objects, it matches the centroid of the central source.  If this
         fails, the script resorts to using the telescope offsets
         transformed to pixels.  Once the offsets are determined, they
         are adjusted for the motion of the target, so that the final
         mosaic registers the target, not the background stars.

         \sstitem
         The ephemeris file is specified by environment variable
         {\tt{ORAC\_EPHEMERIS}}, defaulting to {\tt\$ORAC\_DATA\_OUT/target\_ephem.dat}.

         \sstitem
         The resampling applies non-integer shifts of origin using
         bilinear interpolation.  There is no rotation to align the
         Cartesian axes with the cardinal directions.

         \sstitem
         The recipe makes the mosaics by applying offsets in intensity
         to give the most consistent result amongst the overlapping regions.
         The mosaic is not trimmed to the dimensions of a single frame.  Thus
         the noise will be greater in the peripheral areas having received
         less exposure time.  The full signal will be in the central ninth
         containing the main object.  The mosaic is not normalised by its
         exposure time (that being the exposure time of a single frame).

         \sstitem
         For each cycle of four, the recipe creates a mosaic, which has
         its bad pixels filled and is then added into a master mosaic of
         improving signal to noise.  The exposure time is also summed and
         stored in the mosaic's EXP\_TIME (UFTI) or DEXPTIME (IRCAM) header.
         Likewise the end airmass header, AMEND, is updated to match that of
         the last-observed frame contributing to the mosaic.

         \sstitem
         Intermediate frames are deleted except for the flat-fielded ({\tt\_ff}
         suffix) frames.
      }
   }
   \sstdiytopic{
      Ephemeris-file Format
   }{
      The current format of the ephemeris file is one line per object
      comprising three space-separated fields in the following order:
      \ssthitemlist{

         \sstitem
           the objectname, which may contain embedded spaces;

         \sstitem
           the motion in the plane of the sky in arcsec/second for right
           ascension then declination.

      }
      Note that the right ascension motion is the change in right ascension
      multiplied by the cosine of the declination.  The format will change
      to include UT and possibly date.
   }
   \sstdiytopic{
      Output Data
   }{
      \sstitemlist{

         \sstitem
         The integrated mosaic in {\tt$<$m$>$$<$date$>$\_$<$group\_number$>$\_mos}, where {\tt$<$m$>$}
         is {\tt{gf}} for UFTI and {\tt{gi}} for IRCAM.  

         \sstitem
         A mosaic for each cycle of four in \\
         {\tt$<$m$>$$<$date$>$\_$<$group\_number$>$\_mos$<$cycle\_number$>$}, where {\tt$<$cycle\_number$>$}\\
         counts from 0.

         \sstitem
         The individual flat-fielded frames in {\tt$<$i$>$$<$date$>$\_$<$obs\_number$>$\_ff},
         where {\tt$<$i$>$} is {\tt{f}} for UFTI and {\tt{i}} for IRCAM.

         \sstitem
         The created flat fields in {\tt{flat\_$<$filter$>$\_$<$group\_number$>$}} for the
         first or only cycle, and {\tt{{\tt{flat\_$<$filter$>$\_$<$group\_number$>$}}\_c$<$cycle\_number$>$}}
         for subsequent cycles.
      }
   }
   \sstdiytopic{
      Related Recipes
   }{
      \htmlref{MOVING\_JITTER\_SELF\_FLAT}{MOVING\_JITTER\_SELF\_FLAT},
      \htmlref{QUADRANT\_JITTER}{QUADRANT\_JITTER},
      \htmlref{QUADRANT\_JITTER\_TELE}{QUADRANT\_JITTER\_TELE}.
   }
   \sstimplementationstatus{
      \sstitemlist{

         \sstitem
         The processing engines are from the Starlink packages: \xref{\CCDPACK}{sun139}{}
         \xref{\KAPPA}{sun95}{}, and \xref{\PISA}{sun109}{}.

         \sstitem
         Uses the Starlink NDF format.

         \sstitem
         History is recorded within the data files.

         \sstitem
         The title of the data is propagated through intermediate files
         to the mosaic.

         \sstitem
         Error propagation is not used.
      }
   }
}

%\newpage
\sstroutine{
   NIGHT\_LOG
}{
   Produces a text file log of a night's observations
}{
   \sstdescription{
      This recipe takes a night's observations, and creates a text file
      containing a headed tabulation of parameters for each frame.

      The parameters are: observation number, object name, observation type,
      UT start time, exposure time, number of coadds, read mode and speed,
      filter, start airmass, frame dimensions in pixels, base equatorial
      co-ordinates, and data-reduction recipe name.
   }
   \sstnotes{
      \sstitemlist{

         \sstitem
         Run with {\tt oracdr -noeng -nodisplay -from 1 -skip}  for efficiency.

         \sstitem
         The {\tt $<$date$>$} comes from the header keyword DATE.

         \sstitem
         Specification provided by Sandy Leggett.
      }
   }
   \sstdiytopic{
      Output Data
   }{
      \sstitemlist{

         \sstitem
         The text log file {\tt \$ORAC\_DATA\_IN/$<$date$>$.nightlog}, where
         {\tt $<$date$>$} is the UT date.
      }
   }
}

%\newpage
\sstroutine{
   NOD\_SELF\_FLAT\_NO\_MASK
}{
   Reduces a ``nod jitter'' observation
}{
   \sstdescription{
      This script reduces a ``nod jitter'' observation with UFTI or IRCAM
      data.  It takes an imaging observation comprising a
      multiple-of-four object frames and a dark frame to make a
      calibrated, untrimmed mosaic automatically.

      It performs a null debiassing, bad-pixel masking, dark
      subtraction, difference adjacent pairs, flat-field creation and
      division, feature detection and matching between object frames,
      and resampling.  See the ``Notes'' for further information.

      This recipe works well for faint sources in moderately crowded fields.
   }
   \sstnotes{
      \sstitemlist{

         \sstitem
         A World Co-ordinate System (WCS) using the AIPS convention is
         created in the headers should no WCS already exist.

         \sstitem
         For IRCAM, old headers are reordered and structured with
         headings before groups of related keywords.  The comments have
         units added or appear in a standard format.  Four deprecated
         deprecated are removed.  FITS-violating headers are corrected.
         Spurious instrument names are changed to IRCAM3.

         \sstitem
         The bad pixel mask applied is {\tt\$ORAC\_DATA\_CAL/bpm}.

         \sstitem
         Each dark-subtracted frame has thresholds applied beyond which
         pixels are flagged as bad.  The lower limit is 5 standard
         deviations below the mode, but constrained to the range $-$100 to 1.
         The upper limit is 1000 above the saturation limit for the detector
         in the mode used.

         \sstitem
         The flat field is created by combining normalised object
         frames using the median at each pixel.

         \sstitem
         Registration is performed using common point sources in the
         overlap regions.  If the recipe cannot identify sufficient common
         objects, the script resorts to using the telescope offsets
         transformed to pixels.

         \sstitem
         The resampling applies non-integer shifts of origin using
         bilinear interpolation.  There is no rotation to align the
         Cartesian axes with the cardinal directions.

         \sstitem
         The recipe makes the mosaics by applying offsets in intensity
         to give the most consistent result amongst the overlapping regions.
         The mosaic is not trimmed to the dimensions of a single frame, thus
         the noise will be greater in the peripheral areas having received
         less exposure time.  The mosaic is not normalised by its exposure
         time (that being the exposure time of a single frame).

         \sstitem
         For each cycle of object frames, the recipe creates a mosaic,
         which has its bad pixels filled and is then added into a master
         mosaic of improving signal to noise.  The exposure time is also
         summed and stored in the mosaic's EXP\_TIME (UFTI) or DEXPTIME
         (IRCAM) header.  Likewise the end airmass header, AMEND, is updated
         to match that of the last-observed frame contributing to the mosaic.

         \sstitem
         Intermediate frames are deleted except for the flat-fielded ({\tt\_ff}
         suffix) frames.
      }
   }
   \sstdiytopic{
      Output Data
   }{
      \sstitemlist{

         \sstitem
         The integrated mosaic in {\tt$<$m$>$$<$date$>$\_$<$group\_number$>$\_mos}, where {\tt$<$m$>$}
         is {\tt{gf}} for UFTI and {\tt{gi}} for IRCAM.  Before 2000 August these
         were {\tt{g}} and {\tt{rg}} respectively.

         \sstitem
         A mosaic for each cycle of object frames in \\
         {\tt$<$m$>$$<$date$>$\_$<$group\_number$>$\_mos$<$cycle\_number$>$}, where {\tt$<$cycle\_number$>$}\\
         counts from 0.

         \sstitem
         The individual flat-fielded frames in {\tt$<$i$>$$<$date$>$\_$<$obs\_number$>$\_ff},
         where {\tt$<$i$>$} is {\tt{f}} for UFTI and {\tt{i}} for IRCAM.  Before 2000 August
         IRCAM frames had prefix {\tt{ro}}.

         \sstitem
         The created flat fields in {\tt{flat\_$<$filter$>$\_$<$group\_number$>$}} for the
         first or only cycle, and {\tt{{\tt{flat\_$<$filter$>$\_$<$group\_number$>$}}\_c$<$cycle\_number$>$}}
         for subsequent cycles.
      }
   }
   \sstparameters{
      \sstsubsection{
         NUMBER = INTEGER
      }{
         The number of frames in the nod pattern.  If absent, the number
         of offsets, as given by header NOFFSETS, minus one is used.  If
         neither is available, 4 is used.  An error state arises if
         the number of jittered frames is fewer than 4 and not a
         multiple of 4.  {\tt[]}
      }
   }
   \sstdiytopic{
      Related Recipes
   }{
      \htmlref{BRIGHT\_POINT\_SOURCE}{BRIGHT\_POINT\_SOURCE},
      \htmlref{NOD\_SELF\_FLAT\_NO\_MASK\_APHOT}{NOD\_SELF\_FLAT\_NO\_MASK\_APHOT}.
   }
   \sstimplementationstatus{
      \sstitemlist{

         \sstitem
         The processing engines are from the Starlink packages: \xref{\CCDPACK}{sun139}{}
         \xref{\KAPPA}{sun95}{}, and \xref{\PISA}{sun109}{}.

         \sstitem
         Uses the Starlink NDF format.

         \sstitem
         History is recorded within the data files.

         \sstitem
         The title of the data is propagated through intermediate files
         to the mosaic.

         \sstitem
         Error propagation is not used.
      }
   }
   \sstdiytopic{
      Deprecated Variants
   }{
      NOD4\_SELF\_FLAT\_NO\_MASK, NOD8\_SELF\_FLAT\_NO\_MASK.
   }
}

%\newpage
\sstroutine{
   NOD\_SELF\_FLAT\_NO\_MASK\_APHOT
}{
   Reduces a ``nod jitter'' photometry observation, and performs
   aperture photometry
}{
   \sstdescription{
      This script reduces a ``nod jitter'' photometry observation with
      UFTI or IRCAM data.  It takes an imaging observation comprising a
      multiple-of-four object frames and a dark frame to make a
      calibrated, untrimmed mosaic automatically.

      It performs a null debiassing, bad-pixel masking, dark
      subtraction, difference adjacent pairs, flat-field creation and
      division, feature detection and matching between object frames,
      and resampling.  See the ``Notes'' for further information.

      Photometry of the point source using a fixed 5-arcsecond aperture
      is calculated for each jitter frame and the mosaic.  The results
      appear in {\tt\$ORAC\_DATA\_OUT/aphot\_results.txt} in the form of a Starlink
      small text list.  The analysis of each star is appended to this file.

      This recipe works well for faint sources in moderately crowded fields.
   }
   \sstnotes{
      \sstitemlist{

         \sstitem
         A World Co-ordinate System (WCS) using the AIPS convention is
         created in the headers should no WCS already exist.

         \sstitem
         For IRCAM, old headers are reordered and structured with
         headings before groups of related keywords.  The comments have
         units added or appear in a standard format.  Four deprecated
         deprecated are removed.  FITS-violating headers are corrected.
         Spurious instrument names are changed to IRCAM3.

         \sstitem
         The bad pixel mask applied is {\tt\$ORAC\_DATA\_CAL/bpm}.

         \sstitem
         Each dark-subtracted frame has thresholds applied beyond which
         pixels are flagged as bad.  The lower limit is 5 standard
         deviations below the mode, but constrained to the range $-$100 to 1.
         The upper limit is 1000 above the saturation limit for the detector
         in the mode used.

         \sstitem
         The flat field is created by combining normalised object
         frames using the median at each pixel.

         \sstitem
         Registration is performed using common point sources in the
         overlap regions.  If the recipe cannot identify sufficient common
         objects, the script resorts to using the telescope offsets
         transformed to pixels.

         \sstitem
         The resampling applies non-integer shifts of origin using
         bilinear interpolation.  There is no rotation to align the
         Cartesian axes with the cardinal directions.

         \sstitem
         The recipe makes the mosaics by applying offsets in intensity
         to give the most consistent result amongst the overlapping regions.
         The mosaic is not trimmed to the dimensions of a single frame, thus
         the noise will be greater in the peripheral areas having received
         less exposure time.  The mosaic is not normalised by its exposure
         time (that being the exposure time of a single frame).

         \sstitem
         For each cycle of object frames, the recipe creates a mosaic,
         which has its bad pixels filled and is then added into a master
         mosaic of improving signal to noise.  The exposure time is also
         summed and stored in the mosaic's EXP\_TIME (UFTI) or DEXPTIME
         (IRCAM) header.  Likewise the end airmass header, AMEND, is updated
         to match that of the last-observed frame contributing to the mosaic.

         \sstitem
         The photometry tabulation includes the file name, source
         name, time, filter, airmass, the catalogue magnitude and
         estimates of the zero-point with and without the application
         of a mean extinction.  To discriminate between the various
         results, the positive images have suffix {\tt \_pos} after the
         frame name and the negative images have a {\tt \_neg} suffix.
         There are headings at the top of each column.

         \sstitem
         The photometry uses a multiply clipped (2,2,2.5,3 standard
         deviations) mean to estimate the sky mode in an annulus about
         the source.  This is not unduly biased by the presence of the
         self-flat artifact in the pixel histogram. The inner annulus
         diameter is 1.3 times that of the aperture (6.5 arcsec); the
         outer annulus is 2.5 times (12.5 arcsec) for UFTI and twice
         the aperture (10 arcsec) for IRCAM.  The errors are internal,
         based on the sky noise.

         \sstitem
         Intermediate frames are deleted except for the flat-fielded ({\tt\_ff}
         suffix) frames.
      }
   }
   \sstdiytopic{
      Output Data
   }{
      \sstitemlist{

         \sstitem
         The integrated mosaic in {\tt$<$m$>$$<$date$>$\_$<$group\_number$>$\_mos}, where {\tt$<$m$>$}
         is {\tt{gf}} for UFTI and {\tt{gi}} for IRCAM.  Before 2000 August these
         were {\tt{g}} and {\tt{rg}} respectively.

         \sstitem
         A mosaic for each cycle of object frames in \\
         {\tt$<$m$>$$<$date$>$\_$<$group\_number$>$\_mos$<$cycle\_number$>$}, where {\tt$<$cycle\_number$>$}\\
         counts from 0.

         \sstitem
         The individual flat-fielded frames in {\tt$<$i$>$$<$date$>$\_$<$obs\_number$>$\_ff},
         where {\tt$<$i$>$} is {\tt{f}} for UFTI and {\tt{i}} for IRCAM.  Before 2000 August
         IRCAM frames had prefix {\tt{ro}}.

         \sstitem
         The created flat fields in {\tt{flat\_$<$filter$>$\_$<$group\_number$>$}} for the
         first or only cycle, and {\tt{{\tt{flat\_$<$filter$>$\_$<$group\_number$>$}}\_c$<$cycle\_number$>$}}
         for subsequent cycles.
      }
   }
   \sstparameters{
      \sstsubsection{
         NUMBER = INTEGER
      }{
         The number of frames in the nod pattern.  If absent, the number
         of offsets, as given by header NOFFSETS, minus one is used.  If
         neither is available, 4 is used.  An error state arises if
         the number of jittered frames is fewer than 4 and not a
         multiple of 4.  {\tt[]}
      }
   }
   \sstdiytopic{
      Related Recipes
   }{
      \htmlref{BRIGHT\_POINT\_SOURCE\_APHOT}{BRIGHT\_POINT\_SOURCE\_APHOT},
      \htmlref{NOD\_SELF\_FLAT\_NO\_MASK}{NOD\_SELF\_FLAT\_NO\_MASK}.
   }
   \sstimplementationstatus{
      \sstitemlist{

         \sstitem
         The processing engines are from the Starlink packages: \xref{\CCDPACK}{sun139}{}
         \xref{\KAPPA}{sun95}{}, and \xref{\PISA}{sun109}{}.

         \sstitem
         Uses the Starlink NDF format.

         \sstitem
         History is recorded within the data files.

         \sstitem
         The title of the data is propagated through intermediate files
         to the mosaic.

         \sstitem
         Error propagation is not used.
      }
   }
   \sstdiytopic{
      Deprecated Variants
   }{
      NOD4\_SELF\_FLAT\_NO\_MASK\_APHOT, NOD8\_SELF\_FLAT\_NO\_MASK\_APHOT.
   }
}

%\newpage
\sstroutine{
   POL\_ANGLE\_JITTER
}{
   Reduces an imaging polarimetry observation, where waveplate angle
   iterates at each jitter position
}{
   \sstdescription{
      This script reduces a polarimetry observation with UFTI or IRCAM
      data.  It takes an imaging observation comprising object frames
      at the four waveplate angles 0, 45, 22.5, 67.5 degrees for each of
      of a series of jitter positions offset in Right Ascension; and a
      dark frame to make calibrated polarisation images and vectors
      automatically.  See \htmlref{``Output Data''}{paj_data} for a
      list of these images.

      It performs a null debiassing, bad-pixel masking, dark subtraction
      and flat-field division on all frames.  Next the sections of the frame
      representing the e- and o-beam target and sky regions are extracted,
      and the target frames sky-subtracted.  The resultant frames undergo
      registration and resampling to form a mosaic for each waveplate angle
      and beam.  Once all eight mosaics are formed they are registered and
      resampled, and then combined to form the various polarisation images.
      The polarisation data are binned and noisy data excluded from
      a final catalogue of vectors.  See the ``Notes'' for details.

      This recipe works well for point sources, and for extended sources
      whose sizes in Right Ascension and Declination are less than about
      35 and 15 arcseconds respectively for UFTI, or 9 and 4 arcseconds
      for IRCAM.  Objects which would appear in both the target and
      sky regions, {\em{i.e.}}\ Declination extents south of the centre
      larger than 35 arcseconds (UFTI) or 8 arcseconds (IRCAM), should
      use recipe \htmlref{POL\_EXTENDED}{POL\_EXTENDED} for best results.
   }
   \sstnotes{
      \sstitemlist{

         \sstitem
         A World Co-ordinate System (WCS) using the AIPS convention is
         created in the headers should no WCS already exist.

         \sstitem
         For IRCAM, old headers are reordered and structured with
         headings before groups of related keywords.  The comments have
         units added or appear in a standard format.  Four deprecated
         deprecated are removed.  FITS-violating headers are corrected.
         Spurious instrument names are changed to IRCAM3.

         \sstitem
         Data errors are propagated through all processing steps.
         The initial values are found by applying the nominal ADU conversion
         and read noise.

         \sstitem
         The bad-pixel mask applied is {\tt\$ORAC\_DATA\_CAL/bpm}.

         \sstitem
         Each dark-subtracted frame has thresholds applied beyond which
         pixels are flagged as bad.  The lower limit is 5 standard
         deviations below the mode, but constrained to the range $-$100 to 1.
         The upper limit is 1000 above the saturation limit for the detector
         in the mode used.

         \sstitem
         You should use \htmlref{SKY\_FLAT\_POL}{SKY\_FLAT\_POL} or
         \htmlref{SKY\_FLAT\_POL\_ANGLE}{SKY\_FLAT\_POL\_ANGLE} to make the
         flat fields.

         \sstitem
         The target regions are 30\% to 70\% of the frame width about
         the Right-ascension centre, {\em{i.e.}}\ roughly centred on the source.
         The current sky limits are 1\% to 99\% of the frame width along the
         Right-ascension axis.  The Declination pixel limits are instrument
         dependent, and are as follows.  For UFTI, o sky: 69--264;
         e sky: 320--484; o target: 601--764; e target: 824--988.  For
         IRCAM, o sky: 12--52; e sky: 67--107; o target: 152--192;
         e target: 207--247.

         \sstitem
         The sky subtraction for a beam uses a constant modal sky level
         from the corresponding sky region.

         \sstitem
         Registration is performed using common point sources in the
         overlap regions.  If the recipe cannot identify sufficient common
         objects for automatic registration, the recipe matches the centroid
         of central source within an 8-arcsecond box.  Should that fail for
         the jittered e- and o-beam sections, the recipe resorts to using the
         telescope offsets transformed to pixels.  However, the final option
         for registering the e and o-beam mosaics at different waveplate
         angles, uses the beam offsets in arcseconds for the current filter
         converted to pixels.

         \sstitem
         The resampling applies non-integer shifts of origin using
         bilinear interpolation.  There is no rotation to align the
         Cartesian axes with the cardinal directions.

         \sstitem
         The recipe makes the mosaics by applying offsets in intensity
         to give the most consistent result amongst the overlapping regions.
         The mosaic is not trimmed to the dimensions of a single frame, thus
         the noise will be greater in the few pixels in the peripheral areas
         having received less exposure time.  The mosaic is not normalised by
         its exposure time (that being the exposure time of a single frame).

         \sstitem
         For each cycle of twelve frames, the recipe creates mosaics
         for each beam and waveplate angle.  Each mosaic has its bad pixels
         filled and after the first cycle is then added into its own master
         mosaic of improving signal to noise.  The exposure time is also
         summed and stored in each master mosaic's EXP\_TIME (UFTI) or
         DEXPTIME (IRCAM) header.  Likewise the end airmass header, AMEND,
         is updated to match that of the last-observed frame contributing to
         the mosaic.

         \sstitem
         The polarised intensity is corrected for the statistical bias
         of the noise by subtracting the variance of $Q$ or $U$.

         \sstitem
         An offset of 6.3 degrees clockwise is applied to the rotation
         angle for the orientation of the analyser with respect to north.

         \sstitem
         The polarisation data for each pixel are also stored in
         catalogues.  See \htmlref{``Output Data''}{paj_data}.

         \sstitem
         The intensity image may be displayed with vectors overlaid.
         Steps are taken to reduce the number of noisy or insignificant
         pixels, as well as clutter.  First, the polarisation catalogue data
         are averaged in 3-by-3-pixel bins.  Second, a binned pixel is
         rejected if its polarisation is greater than 50\% or is not positive,
         or its polarisation signal to noise less than 3, or its polarisation
         error is greater 5\%.  The bin size and thresholds can readily be
         changed by supplying arguments to the \_CALC\_STOKES\_ primitive.

         \sstitem
         At the end of each cycle, the grand mosaics are registered, and
         new polarisation maps and catalogues constructed.

         \sstitem
         Intermediate frames are deleted except for the flat-fielded ({\tt\_ff}
         suffix) frames and the mosaics ({\tt\_mos} or
         {\tt\_mos\_c$<$cycle\_number$>$} suffix).
      }
   }
   \label{paj_data}
   \sstdiytopic{
      Output Data
   }{
      \sstitemlist{

         \sstitem
         The integrated mosaic in {\tt$<$m$>$$<$date$>$\_$<$group\_number$>$\_$<$beam$>$$<$angle$>$\_mos}, 
         where {\tt$<$m$>$} is {\tt{gf}} for UFTI and {\tt{gi}} for IRCAM.  Before 2000 August these 
         were {\tt{g}} and {\tt{rg}} respectively.  Token {\tt$<$beam$>$} is {\tt{e}} or {\tt{o}};
         and {\tt$<$angle$>$} is {\tt{0}}, {\tt{22}}, {\tt{45}}, or {\tt{67}}.

         \sstitem
         A mosaic for each cycle of jittered frames per beam and angle in \\
         {\tt$<$m$>$$<$date$>$\_$<$group\_number$>$\_$<$beam$>$$<$angle$>$\_mos\_c$<$cycle\_number$>$}, where
         {\tt$<$cycle\_number$>$} counts from 0.

         \sstitem
         The individual flat-fielded frames in {\tt$<$i$>$$<$date$>$\_$<$obs\_number$>$\_ff},
         where {\tt$<$i$>$} is {\tt{f}} for UFTI and {\tt{i}} for IRCAM.  Before 2000 August
         IRCAM frames had prefix {\tt{ro}}.

         \sstitem
         Polarisation frames {\tt$<$m$>$$<$date$>$\_$<$group\_number$>$\_$<$suffix$>$}, each with a
         different suffix for the each parameter.  The suffices are:\\
         \begin{tabular}{cl}
             I &  intensity \\
             P &  percentage polarisation \\
             PI &  polarisation intensity \\
             Q  &  Stokes $Q$ \\
             TH & polarisation angle \\
             U  &  Stokes $U$ \\
         \end{tabular}

         \sstitem
         A FITS binary-table catalogue of the binned and culled
         polarisation data, called {\tt$<$m$>$$<$date$>$\_$<$group\_number$>$\_I.FIT}.  For
         each point it tabulates the $x$-$y$ co-ordinates, the total intensity,
         the Stokes parameters, the percentage polarisation, the polarisation
         angle and intensity.  There are additional columns giving the
         standard deviation on each of the tabulated values (excluding the
         co-ordinates).  Likewise \\
         {\tt$<$m$>$$<$date$>$\_$<$group\_number$>$\_all.FIT} and \\
         {\tt$<$m$>$$<$date$>$\_$<$group\_number$>$\_bin.FIT} store the
         full and binned catalogues respectively.
      }
   }
   \sstparameters{
      \sstsubsection{
         NUMBER = INTEGER
      }{
         The number of frames in the jitter pattern, per waveplate angle.
         If this is not set, the number of offsets, as given by FITS
         header NOFFSETS, minus one is used.  If neither is available, 3
         is the default.  An error state arises if the number of jittered
         frames is fewer than 3.  For observations prior to the
         availability of full ORAC, header NOFFSETS will be absent. {\tt[]}
      }
   }
   \sstdiytopic{
      Related Recipes
   }{
      \htmlref{POL\_EXTENDED}{POL\_EXTENDED},
      \htmlref{POL\_JITTER}{POL\_JITTER},
      \htmlref{SKY\_FLAT\_POL}{SKY\_FLAT\_POL},
      \htmlref{SKY\_FLAT\_POL\_ANGLE}{SKY\_FLAT\_POL\_ANGLE}.
   }
   \sstimplementationstatus{
      \sstitemlist{

         \sstitem
         The processing engines are from the Starlink packages: \xref{\CCDPACK}{sun139}{}
         \xref{\KAPPA}{sun95}{}, \xref{\POLPACK}{sun223}{}, and \xref{\CURSA}{sun190}{}.

         \sstitem
         Uses the Starlink NDF format.

         \sstitem
         History is recorded within the data files.

         \sstitem
         The title of the data is propagated through intermediate files
         to the mosaics.  The polarisation maps have new titles as follows
         using the suffices described in {\tt{Output Data}}.  I: {\tt{Intensity}};
         P: {\tt{Polarisation}}; PI: {\tt{Polarised Intensity}}; Q: {\tt{Stokes Q}};
         TH: {\tt{Polarisation Angle}}; U: {\tt{Stokes U}}.

         \sstitem
         The origins of the generated polarisation maps are set to [1,1].
         The WCS current frame is unchanged.

         \sstitem
         The units are set for the frames with suffices (see \htmlref{``Output Data''}{paj_data})
         {\tt{P}} to {\tt{\%}}, and {\tt{TH}} to {\tt{degrees}}.
      }
   }
}

%\newpage
\sstroutine{
   POL\_EXTENDED
}{
   Reduces an imaging polarimetry observation of an extended source
}{
   \sstdescription{
      This script reduces a polarimetry observation with UFTI or IRCAM
      data.  It takes an imaging observation comprising alternating
      object and sky frames at the four waveplate angles 0, 45,
      22.5, 67.5 degrees in turn, then jittered to at least three
      positions; and a dark frame to make calibrated polarisation images
      and vectors automatically.  See \htmlref{``Output Data''}{pe_data}
      for a list of these images.

      It performs a null debiassing, bad-pixel masking, dark subtraction
      and flat-field division on all frames.  Next the sections of the
      target frame representing the e- and o-beam target regions are
      extracted and sky-subtracted.  The sky levels are determined from
      the two corresponding regions for each beam in the following sky
      frame.   The resultant frames undergo registration and resampling
      to form a mosaic for each waveplate angle and beam.  Once all eight
      mosaics are formed they are registered and resampled, and then
      combined to form the various polarisation images.  The polarisation
      data are binned and noisy data excluded from a final catalogue of
      vectors.  See the ``Notes'' for details.

      This recipe is intended for extended sources whose sizes are more
      than about 35 arcseconds respectively for UFTI, or 8 arcseconds for
      IRCAM.
   }
   \sstnotes{
      \sstitemlist{

         \sstitem
         A World Co-ordinate System (WCS) using the AIPS convention is
         created in the headers should no WCS already exist.

         \sstitem
         For IRCAM, old headers are reordered and structured with
         headings before groups of related keywords.  The comments have
         units added or appear in a standard format.  Four deprecated
         deprecated are removed.  FITS-violating headers are corrected.
         Spurious instrument names are changed to IRCAM3.

         \sstitem
         Data errors are propagated through all processing steps.
         The initial values are found by applying the nominal ADU conversion
         and read noise.

         \sstitem
         The bad-pixel mask applied is {\tt\$ORAC\_DATA\_CAL/bpm}.

         \sstitem
         Each dark-subtracted frame has thresholds applied beyond which
         pixels are flagged as bad.  The lower limit is 5 standard
         deviations below the mode, but constrained to the range $-$100 to 1.
         The upper limit is 1000 above the saturation limit for the detector
         in the mode used.

         \sstitem
         You should use \htmlref{SKY\_FLAT\_POL}{SKY\_FLAT\_POL} or
         \htmlref{SKY\_FLAT\_POL\_ANGLE}{SKY\_FLAT\_POL\_ANGLE} to make the
         flat fields.

         \sstitem
         The target regions are 10\% to 90\% of the frame width about
         the Right-ascension centre, {\em{i.e.}}\ roughly centred on the source.
         The current sky limits are 1\% to 99\% of the frame width along the
         Right-ascension axis.  The Declination pixel limits are instrument
         dependent, and are as follows.  For UFTI, o sky: 69--264;
         e sky: 320--484; o target: 601--764; e target: 824--988.  For
         IRCAM, o sky: 12--52; e sky: 67--107; o target: 152--192;
         e target: 207--247.

         \sstitem
         The sky subtraction for a beam uses a constant modal sky level
         from the corresponding sky regions.

         \sstitem
         Registration is performed using common point sources in the
         overlap regions.  If the recipe cannot identify sufficient common
         objects for automatic registration, the recipe matches the centroid
         of central source within an 8-arcsecond box.  Should that fail for
         the jittered e- and o-beam sections, the recipe resorts to using the
         telescope offsets transformed to pixels.  However, the final option
         for registering the e and o-beam mosaics at different waveplate
         angles, uses the beam offsets in arcseconds for the current filter
         converted to pixels.

         \sstitem
         The resampling applies non-integer shifts of origin using
         bilinear interpolation.  There is no rotation to align the
         Cartesian axes with the cardinal directions.

         \sstitem
         The recipe makes the mosaics by applying offsets in intensity
         to give the most consistent result amongst the overlapping regions.
         The mosaic is not trimmed to the dimensions of a single frame, thus
         the noise will be greater in the few pixels in the peripheral areas
         having received less exposure time.  The mosaic is not normalised by
         its exposure time (that being the exposure time of a single frame).

         \sstitem
         For each cycle of twelve frames, the recipe creates mosaics
         for each beam and waveplate angle.  Each mosaic has its bad pixels
         filled and after the first cycle is then added into its own master
         mosaic of improving signal to noise.  The exposure time is also
         summed and stored in each master mosaic's EXP\_TIME (UFTI) or
         DEXPTIME (IRCAM) header.  Likewise the end airmass header, AMEND,
         is updated to match that of the last-observed frame contributing to
         the mosaic.

         \sstitem
         The polarised intensity is corrected for the statistical bias
         of the noise by subtracting the variance of $Q$ or $U$.

         \sstitem
         An offset of 6.3 degrees clockwise is applied to the rotation
         angle for the orientation of the analyser with respect to north.

         \sstitem
         The polarisation data for each pixel are also stored in
         catalogues.  See \htmlref{``Output Data''}{pe_data}.

         \sstitem
         The intensity image may be displayed with vectors overlaid.
         Steps are taken to reduce the number of noisy or insignificant
         pixels, as well as clutter.  First, the polarisation catalogue data
         are averaged in 3-by-3-pixel bins.  Second, a binned pixel is
         rejected if its polarisation is greater than 50\% or is not positive,
         or its polarisation signal to noise less than 3, or its polarisation
         error is greater 5\%.  The bin size and thresholds can readily be
         changed by supplying arguments to the \_CALC\_STOKES\_ primitive.

         \sstitem
         At the end of each cycle, the grand mosaics are registered, and
         new polarisation maps and catalogues constructed.

         \sstitem
         Intermediate frames are deleted except for the flat-fielded ({\tt\_ff}
         suffix) frames and the mosaics ({\tt\_mos} or
         {\tt\_mos\_c$<$cycle\_number$>$} suffix).
      }
   }
   \label{pe_data}
   \sstdiytopic{
      Output Data
   }{
      \sstitemlist{

         \sstitem
         The integrated mosaic in $<$m$>$$<$date$>$\_$<$group\_number$>$\_$<$beam$>$$<$angle$>$\_mos,
         where {\tt$<$m$>$} is {\tt{gf}} for UFTI and {\tt{gi}} for IRCAM.  Before 2000 August these
         were {\tt{g}} and {\tt{rg}} respectively.  Token $<$beam$>$ is {\tt{e}} or {\tt{o}};
         and $<$angle$>$ is {\tt{0}}, {\tt{22}}, {\tt{45}}, or {\tt{67}}.

         \sstitem
         A mosaic for each cycle of jittered frames per beam and angle in \\
         {\tt$<$m$>$$<$date$>$\_$<$group\_number$>$\_$<$beam$>$$<$angle$>$\_mos\_c$<$cycle\_number$>$}, where
         {\tt$<$cycle\_number$>$} counts from 0.

         \sstitem
         The individual flat-fielded frames in {\tt$<$i$>$$<$date$>$\_$<$obs\_number$>$\_ff},
         where {\tt$<$i$>$} is {\tt{f}} for UFTI and {\tt{i}} for IRCAM.  Before 2000 August
         IRCAM frames had prefix {\tt{ro}}.

         \sstitem
         Polarisation frames {\tt$<$m$>$$<$date$>$\_$<$group\_number$>$\_$<$suffix$>$}, each with a
         different suffix for the each parameter.  The suffices are:\\
         \begin{tabular}{cl}
             I &  intensity \\
             P &  percentage polarisation \\
             PI &  polarisation intensity \\
             Q  &  Stokes $Q$ \\
             TH & polarisation angle \\
             U  &  Stokes $U$ \\
         \end{tabular}

         \sstitem
         A FITS binary-table catalogue of the binned and culled
         polarisation data, called {\tt$<$m$>$$<$date$>$\_$<$group\_number$>$\_I.FIT}.  For
         each point it tabulates the $x$-$y$ co-ordinates, the total intensity,
         the Stokes parameters, the percentage polarisation, the polarisation
         angle and intensity.  There are additional columns giving the
         standard deviation on each of the tabulated values (excluding the
         co-ordinates).  Likewise \\
         {\tt$<$m$>$$<$date$>$\_$<$group\_number$>$\_all.FIT} and \\
         {\tt$<$m$>$$<$date$>$\_$<$group\_number$>$\_bin.FIT} store the full
         and binned catalogues respectively.
      }
   }
   \sstparameters{
      \sstsubsection{
         NUMBER = INTEGER
      }{
         The number of frames in the jitter pattern, per waveplate angle.
         If this is not set, the number of offsets, as given by FITS
         header NOFFSETS, minus one is used.  If neither is available, 3
         is the default.  An error state arises if the number of jittered
         frames is fewer than 3.  For observations prior to the
         availability of full ORAC, header NOFFSETS will be absent. {\tt[]}
      }
   }
   \sstdiytopic{
      Related Recipes
   }{
      \htmlref{POL\_ANGLE\_JITTER}{POL\_ANGLE\_JITTER},
      \htmlref{SKY\_FLAT\_POL}{SKY\_FLAT\_POL},
      \htmlref{SKY\_FLAT\_POL\_ANGLE}{SKY\_FLAT\_POL\_ANGLE}.
   }
   \sstimplementationstatus{
      \sstitemlist{

         \sstitem
         The processing engines are from the Starlink packages: \xref{\CCDPACK}{sun139}{}
         \xref{\KAPPA}{sun95}{}, \xref{\POLPACK}{sun223}{}, and \xref{\CURSA}{sun190}{}.

         \sstitem
         Uses the Starlink NDF format.

         \sstitem
         History is recorded within the data files.

         \sstitem
         The title of the data is propagated through intermediate files
         to the mosaics.  The polarisation maps have new titles as follows
         using the suffices described in {\tt{Output Data}}.  I: {\tt{Intensity}};
         P: {\tt{Polarisation}}; PI: {\tt{Polarised Intensity}}; Q: {\tt{Stokes Q}};
         TH: {\tt{Polarisation Angle}}; U: {\tt{Stokes U}}.

         \sstitem
         The origins of the generated polarisation maps are set to [1,1].
         The WCS current frame is unchanged.

         \sstitem
         The units are set for the frames with suffices (see \htmlref{``Output Data''}{pe_data})
         {\tt{P}} to {\tt{\%}}, and {\tt{TH}} to {\tt{degrees}}.
      }
   }
}

%\newpage
\sstroutine{
   POL\_JITTER
}{
   Reduces an imaging polarimetry observation jittered at each angle
}{
   \sstdescription{
      This script reduces a polarimetry observation with UFTI or IRCAM
      data.  It takes an imaging observation comprising object frames
      jittered in Right Ascension at the four waveplate angles 0, 45,
      22.5, 67.5 degrees in turn; and a dark frame to make calibrated
      polarisation images and vectors automatically.  See
      \htmlref{``Output Data''}{pj_data} for a list of these images.

      It performs a null debiassing, bad-pixel masking, dark subtraction
      and flat-field division on all frames.  Next the sections of the frame
      representing the e- and o-beam target and sky regions are extracted,
      and the target frames sky-subtracted.  The resultant frames undergo
      registration and resampling to form a mosaic for each waveplate angle
      and beam.  Once all eight mosaics are formed they are registered and
      resampled, and then combined to form the various polarisation images.
      The polarisation data are binned and noisy data excluded from
      a final catalogue of vectors.  See the ``Notes'' for details.

      This recipe works well for point sources, and for extended sources
      whose sizes in Right Ascension and Declination are less than about
      35 and 15 arcseconds respectively for UFTI, or 9 and 4 arcseconds
      for IRCAM.  Objects which would appear in both the target and
      sky regions, {\em{i.e.}}\ Declination extents south of the centre
      larger than 35 arcseconds (UFTI) or 8 arcseconds (IRCAM), should
      use recipe \htmlref{POL\_EXTENDED}{POL\_EXTENDED} for best results.
   }
   \sstnotes{
      \sstitemlist{

         \sstitem
         A World Co-ordinate System (WCS) using the AIPS convention is
         created in the headers should no WCS already exist.

         \sstitem
         For IRCAM, old headers are reordered and structured with
         headings before groups of related keywords.  The comments have
         units added or appear in a standard format.  Four deprecated
         deprecated are removed.  FITS-violating headers are corrected.
         Spurious instrument names are changed to IRCAM3.

         \sstitem
         Data errors are propagated through all processing steps.
         The initial values are found by applying the nominal ADU conversion
         and read noise.

         \sstitem
         The bad-pixel mask applied is {\tt\$ORAC\_DATA\_CAL/bpm}.

         \sstitem
         Each dark-subtracted frame has thresholds applied beyond which
         pixels are flagged as bad.  The lower limit is 5 standard
         deviations below the mode, but constrained to the range $-$100 to 1.
         The upper limit is 1000 above the saturation limit for the detector
         in the mode used.

         \sstitem
         You should use \htmlref{SKY\_FLAT\_POL}{SKY\_FLAT\_POL} or
         \htmlref{SKY\_FLAT\_POL\_ANGLE}{SKY\_FLAT\_POL\_ANGLE} to make the
         flat fields.

         \sstitem
         The target regions are 30\% to 70\% of the frame width about
         the Right-ascension centre, {\em{i.e.}}\ roughly centred on the source.
         The current sky limits are 1\% to 99\% of the frame width along the
         Right-ascension axis.  The Declination pixel limits are instrument
         dependent, and are as follows.  For UFTI, o sky: 69--264;
         e sky: 320--484; o target: 601--764; e target: 824--988.  For
         IRCAM, o sky: 12--52; e sky: 67--107; o target: 152--192;
         e target: 207--247.

         \sstitem
         The sky subtraction for a beam uses a constant modal sky level
         from the corresponding sky region.

         \sstitem
         Registration is performed using common point sources in the
         overlap regions.  If the recipe cannot identify sufficient common
         objects for automatic registration, the recipe matches the centroid
         of central source within an 8-arcsecond box.  Should that fail for
         the jittered e- and o-beam sections, the recipe resorts to using the
         telescope offsets transformed to pixels.  However, the final option
         for registering the e and o-beam mosaics at different waveplate
         angles, uses the beam offsets in arcseconds for the current filter
         converted to pixels.

         \sstitem
         The resampling applies non-integer shifts of origin using
         bilinear interpolation.  There is no rotation to align the
         Cartesian axes with the cardinal directions.

         \sstitem
         The recipe makes the mosaics by applying offsets in intensity
         to give the most consistent result amongst the overlapping regions.
         The mosaic is not trimmed to the dimensions of a single frame, thus
         the noise will be greater in the few pixels in the peripheral areas
         having received less exposure time.  The mosaic is not normalised by
         its exposure time (that being the exposure time of a single frame).

         \sstitem
         For each cycle of twelve frames, the recipe creates mosaics
         for each beam and waveplate angle.  Each mosaic has its bad pixels
         filled and after the first cycle is then added into its own master
         mosaic of improving signal to noise.  The exposure time is also
         summed and stored in each master mosaic's EXP\_TIME (UFTI) or
         DEXPTIME (IRCAM) header.  Likewise the end airmass header, AMEND,
         is updated to match that of the last-observed frame contributing to
         the mosaic.

         \sstitem
         The polarised intensity is corrected for the statistical bias
         of the noise by subtracting the variance of $Q$ or $U$.

         \sstitem
         An offset of 6.3 degrees clockwise is applied to the rotation
         angle for the orientation of the analyser with respect to north.

         \sstitem
         The polarisation data for each pixel are also stored in
         catalogues.  See \htmlref{``Output Data''}{pj_data}.

         \sstitem
         The intensity image may be displayed with vectors overlaid.
         Steps are taken to reduce the number of noisy or insignificant
         pixels, as well as clutter.  First, the polarisation catalogue data
         are averaged in 3-by-3-pixel bins.  Second, a binned pixel is
         rejected if its polarisation is greater than 50\% or is not positive,
         or its polarisation signal to noise less than 3, or its polarisation
         error is greater 5\%.  The bin size and thresholds can readily be
         changed by supplying arguments to the \_CALC\_STOKES\_ primitive.

         \sstitem
         At the end of each cycle, the grand mosaics are registered, and
         new polarisation maps and catalogues constructed.

         \sstitem
         Intermediate frames are deleted except for the flat-fielded ({\tt\_ff}
         suffix) frames and the mosaics ({\tt\_mos} or
         {\tt\_mos\_c$<$cycle\_number$>$} suffix).
      }
   }
   \label{pj_data}
   \sstdiytopic{
      Output Data
   }{
      \sstitemlist{

         \sstitem
         The integrated mosaic in {\tt$<$m$>$$<$date$>$\_$<$group\_number$>$\_$<$beam$>$$<$angle$>$\_mos},
         where {\tt$<$m$>$} is {\tt{gf}} for UFTI and {\tt{gi}} for IRCAM.  Before 2000 August these
         were {\tt{g}} and {\tt{rg}} respectively.  Token {\tt$<$beam$>$} is {\tt{e}} or {\tt{o}};
         and $<$angle$>$ is {\tt{0}}, {\tt{22}}, {\tt{45}}, or {\tt{67}}.

         \sstitem
         A mosaic for each cycle of jittered frames per beam and angle in \\
         {\tt$<$m$>$$<$date$>$\_$<$group\_number$>$\_$<$beam$>$$<$angle$>$\_mos\_c$<$cycle\_number$>$}, where
         {\tt$<$cycle\_number$>$} counts from 0.

         \sstitem
         The individual flat-fielded frames in {\tt$<$i$>$$<$date$>$\_$<$obs\_number$>$\_ff},
         where {\tt$<$i$>$} is {\tt{f}} for UFTI and {\tt{i}} for IRCAM.  Before 2000 August
         IRCAM frames had prefix {\tt{ro}}.

         \sstitem
         Polarisation frames {\tt$<$m$>$$<$date$>$\_$<$group\_number$>$\_$<$suffix$>$}, each with a
         different suffix for the each parameter.  The suffices are:\\
         \begin{tabular}{cl}
             I &  intensity \\
             P &  percentage polarisation \\
             PI &  polarisation intensity \\
             Q  &  Stokes $Q$ \\
             TH & polarisation angle \\
             U  &  Stokes $U$ \\
         \end{tabular}

         \sstitem
         A FITS binary-table catalogue of the binned and culled
         polarisation data, called {\tt$<$m$>$$<$date$>$\_$<$group\_number$>$\_I.FIT}.  For
         each point it tabulates the $x$-$y$ co-ordinates, the total intensity,
         the Stokes parameters, the percentage polarisation, the polarisation
         angle and intensity.  There are additional columns giving the
         standard deviation on each of the tabulated values (excluding the
         co-ordinates).  Likewise \\
         {\tt$<$m$>$$<$date$>$\_$<$group\_number$>$\_all.FIT} and \\
         {\tt$<$m$>$$<$date$>$\_$<$group\_number$>$\_bin.FIT} store the full and
         binned catalogues respectively.
      }
   }
   \sstparameters{
      \sstsubsection{
         NUMBER = INTEGER
      }{
         The number of frames in the jitter pattern, per waveplate angle.
         If this is not set, the number of offsets, as given by FITS
         header NOFFSETS, minus one is used.  If neither is available, 3
         is the default.  An error state arises if the number of jittered
         frames is fewer than 3.  For observations prior to the
         availability of full ORAC, header NOFFSETS will be absent.  {\tt[]}
      }
   }
   \sstdiytopic{
      Related Recipes
   }{
      \htmlref{POL\_ANGLE\_JITTER}{POL\_ANGLE\_JITTER},
      \htmlref{POL\_EXTENDED}{POL\_EXTENDED},
      \htmlref{SKY\_FLAT\_POL}{SKY\_FLAT\_POL},
      \htmlref{SKY\_FLAT\_POL\_ANGLE}{SKY\_FLAT\_POL\_ANGLE}.
   }
   \sstimplementationstatus{
      \sstitemlist{

         \sstitem
         The processing engines are from the Starlink packages: \xref{\CCDPACK}{sun139}{}
         \xref{\KAPPA}{sun95}{}, \xref{\POLPACK}{sun223}{}, and \xref{\CURSA}{sun190}{}.

         \sstitem
         Uses the Starlink NDF format.

         \sstitem
         History is recorded within the data files.

         \sstitem
         The title of the data is propagated through intermediate files
         to the mosaics.  The polarisation maps have new titles as follows
         using the suffices described in {\tt{Output Data}}.  I: {\tt{Intensity}};
         P: {\tt{Polarisation}}; PI: {\tt{Polarised Intensity}}; Q: {\tt{Stokes Q}};
         TH: {\tt{Polarisation Angle}}; U: {\tt{Stokes U}}.

         \sstitem
         The origins of the generated polarisation maps are set to [1,1].
         The WCS current frame is unchanged.

         \sstitem
         The units are set for the frames with suffices (see \htmlref{``Output Data''}{pj_data})
         {\tt{P}} to {\tt{\%}}, and {\tt{TH}} to {\tt{degrees}}.
      }
   }
   \sstdiytopic{
      Deprecated Variants
   }{
      POL\_JITTER3.
   }
}

%\newpage
\sstroutine{
   QUADRANT\_JITTER
}{
   Reduces a ``Quadrant Jitter'' observation, including object masking
}{
   \sstdescription{
      This script reduces a ``quadrant jitter'' photometry observation
      with UFTI or IRCAM data.  It takes an imaging observation comprising
      one or more series of four object frames where the target is
      approximately centred in each quadrant; and a dark frame to make
      a calibrated, untrimmed mosaic automatically.

      It performs bad-pixel masking, null debiassing, dark subtraction,
      flat-field creation and division, feature detection and matching
      between object frames, and resampling.   See the ``Notes'' for
      further information.

      This recipe is suitable for faint objects or objects within
      a comparatively bright core embedded in faint extended emission,
      e.g. a quasar; or extended objects less than 45 arcseconds across
      with UFTI and 10 arcseconds with IRCAM.  The object need not be
      isolated, as the recipe masks objects within the other quadrants,
      and hence does not introduce significant artifacts into the flat
      field.  For isolated objects use \htmlref{QUADRANT\_JITTER\_NO\_MASK}{QUADRANT\_JITTER\_NO\_MASK}; or where
      speed is critical, use \htmlref{QUADRANT\_JITTER\_BASIC}{QUADRANT\_JITTER\_BASIC} instead.
   }
   \sstnotes{
      \sstitemlist{

         \sstitem
         A World Co-ordinate System (WCS) using the AIPS convention is
         created in the headers should no WCS already exist.

         \sstitem
         For IRCAM, old headers are reordered and structured with
         headings before groups of related keywords.  The comments have
         units added or appear in a standard format.  Four deprecated
         deprecated are removed.  FITS-violating headers are corrected.
         Spurious instrument names are changed to IRCAM3.

         \sstitem
         The bad-pixel mask applied is {\tt\$ORAC\_DATA\_CAL/bpm}.

         \sstitem
         Each dark-subtracted frame has thresholds applied beyond which
         pixels are flagged as bad.  The lower limit is 5 standard
         deviations below the mode, but constrained to the range $-$100 to 1.
         The upper limit is 1000 above the saturation limit for the detector
         in the mode used.

         \sstitem
         The flat field is created iteratively.  First the quadrant
         containing the object is masked in each object frame.  Second an
         approximate flat field is created by combining the normalised
         and masked object frames using the clipped median at each pixel.
         This flat field is applied to the object frames.  Sources within
         the flat-fielded frames are detected, and masked in the
         dark-subtracted frames.  The second stage is repeated but applied
         to the masked frames to create the final flat field.

         \sstitem
         Registration is performed using common point sources in the
         overlap regions.  If the recipe cannot identify sufficient common
         objects, it matches the centroid of the central source.  If this
         fails, the script resorts to using the telescope offsets
         transformed to pixels.

         \sstitem
         The resampling applies non-integer shifts of origin using
         bilinear interpolation.  There is no rotation to align the
         Cartesian axes with the cardinal directions.

         \sstitem
         The recipe makes the mosaics by applying offsets in intensity
         to give the most consistent result amongst the overlapping regions.
         The mosaic is not trimmed to the dimensions of a single frame.  Thus
         the noise will be greater in the peripheral areas having received
         less exposure time.  The full signal will be in the central ninth
         containing the main object.  The mosaic is not normalised by its
         exposure time (that being the exposure time of a single frame).

         \sstitem
         For each cycle of four, the recipe creates a mosaic, which has
         its bad pixels filled and is then added into a master mosaic of
         improving signal to noise.  The exposure time is also summed and
         stored in the mosaic's EXP\_TIME (UFTI) or DEXPTIME (IRCAM) header.
         Likewise the end airmass header, AMEND, is updated to match that of
         the last-observed frame contributing to the mosaic.

         \sstitem
         Intermediate frames are deleted except for the flat-fielded ({\tt\_ff}
         suffix) frames.
      }
   }
   \sstdiytopic{
      Output Data
   }{
      \sstitemlist{

         \sstitem
         The integrated mosaic in {\tt$<$m$>$$<$date$>$\_$<$group\_number$>$\_mos}, where {\tt$<$m$>$}
         is {\tt{gf}} for UFTI and {\tt{gi}} for IRCAM.  Before 2000 August these
         were {\tt{g}} and {\tt{rg}} respectively.

         \sstitem
         A mosaic for each cycle of four in \\
         {\tt$<$m$>$$<$date$>$\_$<$group\_number$>$\_mos$<$cycle\_number$>$}, where {\tt$<$cycle\_number$>$}\\
         counts from 0.

         \sstitem
         The individual flat-fielded frames in {\tt$<$i$>$$<$date$>$\_$<$obs\_number$>$\_ff},
         where {\tt$<$i$>$} is {\tt{f}} for UFTI and {\tt{i}} for IRCAM.  Before 2000 August
         IRCAM frames had prefix {\tt{ro}}.

         \sstitem
         The created flat fields in {\tt{flat\_$<$filter$>$\_$<$group\_number$>$}} for the
         first or only cycle, and {\tt{{\tt{flat\_$<$filter$>$\_$<$group\_number$>$}}\_c$<$cycle\_number$>$}}
         for subsequent cycles.
      }
   }
   \sstdiytopic{
      Related Recipes
   }{
      \htmlref{QUADRANT\_JITTER\_BASIC}{QUADRANT\_JITTER\_BASIC},
      \htmlref{QUADRANT\_JITTER\_NO\_MASK}{QUADRANT\_JITTER\_NO\_MASK},
      \htmlref{EXTENDED\_3x3}{EXTENDED\_3x3},\\
      \htmlref{EXTENDED\_5x5}{EXTENDED\_5x5}.
   }
   \sstimplementationstatus{
      \sstitemlist{

         \sstitem
         The processing engines are from the Starlink packages: \xref{\CCDPACK}{sun139}{}
         \xref{\KAPPA}{sun95}{}, and \xref{\PISA}{sun109}{}.

         \sstitem
         Uses the Starlink NDF format.

         \sstitem
         History is recorded within the data files.

         \sstitem
         The title of the data is propagated through intermediate files
         to the mosaic.

         \sstitem
         Error propagation is not used.
      }
   }
}

%\newpage
\sstroutine{
   QUADRANT\_JITTER\_BASIC
}{
   Reduces a ``Quadrant Jitter'' observation, using just the basic
   operations for speed
}{
   \sstdescription{
      This script reduces a ``quadrant jitter'' photometry observation
      with UFTI or IRCAM data.  It takes an imaging observation comprising
      one or more series of four object frames where the target is
      approximately centred in each quadrant; and a dark frame to make
      a calibrated, untrimmed mosaic automatically.

      It performs bad-pixel masking, null debiassing, dark subtraction,
      flat-field creation and division, and registration using telescope
      offsets.  See the ``Notes'' for further information.

      This recipe aims to keep pace with the pipeline's incoming
      data and many options which improve the final mosaic are omitted.
      This recipe is suitable for faint objects or objects within
      a comparatively bright core embedded in faint extended emission,
      e.g. a quasar; or extended objects less than 45 arcseconds across
      with UFTI and 10 arcseconds with IRCAM.  If the object is not
      isolated, there will be artifacts introduced into the flat field.
      These arise from the contribution of sources outside the quadrant
      containing the primary object.  This variant of
      \htmlref{QUADRANT\_JITTER}{QUADRANT\_JITTER} is
      best for isolated objects or where speed is critical.  Use
      QUADRANT\_JITTER itself if object masking is required instead.
   }
   \sstnotes{
      \sstitemlist{

         \sstitem
         A World Co-ordinate System (WCS) using the AIPS convention is
         created in the headers should no WCS already exist.

         \sstitem
         For IRCAM, old headers are reordered and structured with
         headings before groups of related keywords.  The comments have
         units added or appear in a standard format.  Four deprecated
         deprecated are removed.  FITS-violating headers are corrected.
         Spurious instrument names are changed to IRCAM3.

         \sstitem
         The bad-pixel mask applied is {\tt\$ORAC\_DATA\_CAL/bpm}.

         \sstitem
         Each dark-subtracted frame has thresholds applied beyond which
         pixels are flagged as bad.  The lower limit is 5 standard
         deviations below the mode, but constrained to the range $-$100 to 1.
         The upper limit is 1000 above the saturation limit for the detector
         in the mode used.

         \sstitem
         The flat field is created in two steps.  The quadrant
         containing the object is masked in each object frame.  Then the
         recipe combines the normalised and quadrant-masked object frames
         using the median at each pixel.

         \sstitem
         Registration is performed using the telescope offsets
         transformed to pixels.

         \sstitem
         There is no resampling, merely integer shifts of origin.

         \sstitem
         The recipe makes the mosaics by applying offsets in intensity
         to give the most consistent result amongst the overlapping regions.
         The mosaic is not trimmed to the dimensions of a single frame.  Thus
         the noise will be greater in the peripheral areas having received
         less exposure time.  The full signal will be in the central ninth
         containing the main object.  The mosaic is not normalised by its
         exposure time (that being the exposure time of a single frame).

         \sstitem
         For each cycle of four, the recipe creates a mosaic, which is
         added into a master mosaic of improving signal to noise.  The
         exposure time is also summed and stored in the mosaic's EXP\_TIME
         (UFTI) or DEXPTIME (IRCAM) header.  Likewise the end airmass header,
         AMEND, is updated to match that of the last-observed frame
         contributing to the mosaic.

         \sstitem
         Intermediate frames are deleted except for the flat-fielded ({\tt\_ff}
         suffix) frames.
      }
   }
   \sstdiytopic{
      Output Data
   }{
      \sstitemlist{

         \sstitem
         The integrated mosaic in {\tt$<$m$>$$<$date$>$\_$<$group\_number$>$\_mos}, where {\tt$<$m$>$}
         is {\tt{gf}} for UFTI and {\tt{gi}} for IRCAM.  Before 2000 August these
         were {\tt{g}} and {\tt{rg}} respectively.

         \sstitem
         A mosaic for each cycle of four in \\
         {\tt$<$m$>$$<$date$>$\_$<$group\_number$>$\_mos$<$cycle\_number$>$}, where {\tt$<$cycle\_number$>$}\\
         counts from 0.

         \sstitem
         The individual flat-fielded frames in {\tt$<$i$>$$<$date$>$\_$<$obs\_number$>$\_ff},
         where {\tt$<$i$>$} is {\tt{f}} for UFTI and {\tt{i}} for IRCAM.  Before 2000 August
         IRCAM frames had prefix {\tt{ro}}.

         \sstitem
         The created flat fields in {\tt{flat\_$<$filter$>$\_$<$group\_number$>$}} for the
         first or only cycle, and {\tt{{\tt{flat\_$<$filter$>$\_$<$group\_number$>$}}\_c$<$cycle\_number$>$}}
         for subsequent cycles.
      }
   }
   \sstdiytopic{
      Related Recipes
   }{
      \htmlref{QUADRANT\_JITTER}{QUADRANT\_JITTER},
      \htmlref{QUADRANT\_JITTER\_NO\_MASK}{QUADRANT\_JITTER\_NO\_MASK},
      \htmlref{EXTENDED\_3x3\_BASIC}{EXTENDED\_3x3\_BASIC},\\
      \htmlref{EXTENDED\_5x5\_BASIC}{EXTENDED\_5x5\_BASIC}.
   }
   \sstimplementationstatus{
      \sstitemlist{

         \sstitem
         The processing engines are from the Starlink packages: \xref{\CCDPACK}{sun139}{}
         \xref{\KAPPA}{sun95}{}, and \xref{\PISA}{sun109}{}.

         \sstitem
         Uses the Starlink NDF format.

         \sstitem
         History is recorded within the data files.

         \sstitem
         The title of the data is propagated through intermediate files
         to the mosaic.

         \sstitem
         Error propagation is not used.
      }
   }
}

%\newpage
\sstroutine{
   QUADRANT\_JITTER\_NO\_MASK
}{
   Reduces a ``Quadrant Jitter'' observation without object masking
}{
   \sstdescription{
      This script reduces a ``quadrant jitter'' photometry observation
      with UFTI or IRCAM data.  It takes an imaging observation comprising
      one or more series of four object frames where the target is
      approximately centred in each quadrant; and a dark frame to make
      a calibrated, untrimmed mosaic automatically.

      It performs bad-pixel masking, null debiassing, dark subtraction,
      flat-field creation and division, feature detection and matching
      between object frames, and resampling.   See the ``Notes'' for
      further information.

      This recipe is suitable for faint objects or objects within
      a comparatively bright core embedded in faint extended emission,
      e.g. a quasar; or extended objects less than 45 arcseconds across
      with UFTI and 10 arcseconds with IRCAM.  If there are other objects
      of comparable brightness to the principal target in other quadrants,
      they will introduce artifacts into the flat field.  Use
      \htmlref{QUADRANT\_JITTER}{QUADRANT\_JITTER} for those.
   }
   \sstnotes{
      \sstitemlist{

         \sstitem
         A World Co-ordinate System (WCS) using the AIPS convention is
         created in the headers should no WCS already exist.

         \sstitem
         For IRCAM, old headers are reordered and structured with
         headings before groups of related keywords.  The comments have
         units added or appear in a standard format.  Four deprecated
         deprecated are removed.  FITS-violating headers are corrected.
         Spurious instrument names are changed to IRCAM3.

         \sstitem
         The bad-pixel mask applied is {\tt\$ORAC\_DATA\_CAL/bpm}.

         \sstitem
         Each dark-subtracted frame has thresholds applied beyond which
         pixels are flagged as bad.  The lower limit is 5 standard
         deviations below the mode, but constrained to the range $-$100 to 1.
         The upper limit is 1000 above the saturation limit for the detector
         in the mode used.

         \sstitem
         The flat field is created in two steps.  The quadrant
         containing the object is masked in each object frame.  Then the
         recipe combines the normalised and quadrant-masked object frames
         using the median at each pixel.

         \sstitem
         Registration is performed using common point sources in the
         overlap regions.  If the recipe cannot identify sufficient common
         objects, it matches the centroid of the central source.  If this
         fails, the script resorts to using the telescope offsets
         transformed to pixels.

         \sstitem
         The resampling applies non-integer shifts of origin using
         bilinear interpolation.  There is no rotation to align the
         Cartesian axes with the cardinal directions.

         \sstitem
         The recipe makes the mosaics by applying offsets in intensity
         to give the most consistent result amongst the overlapping regions.
         The mosaic is not trimmed to the dimensions of a single frame.  Thus
         the noise will be greater in the peripheral areas having received
         less exposure time.  The full signal will be in the central ninth
         containing the main object.  The mosaic is not normalised by its
         exposure time (that being the exposure time of a single frame).

         \sstitem
         For each cycle of four, the recipe creates a mosaic, which has
         its bad pixels filled and is then added into a master mosaic of
         improving signal to noise.  The exposure time is also summed and
         stored in the mosaic's EXP\_TIME (UFTI) or DEXPTIME (IRCAM) header.
         Likewise the end airmass header, AMEND, is updated to match that of
         the last-observed frame contributing to the mosaic.

         \sstitem
         Intermediate frames are deleted except for the flat-fielded ({\tt\_ff}
         suffix) frames.
      }
   }
   \sstdiytopic{
      Output Data
   }{
      \sstitemlist{

         \sstitem
         The integrated mosaic in {\tt$<$m$>$$<$date$>$\_$<$group\_number$>$\_mos}, where {\tt$<$m$>$}
         is {\tt{gf}} for UFTI and {\tt{gi}} for IRCAM.  Before 2000 August these
         were {\tt{g}} and {\tt{rg}} respectively.

         \sstitem
         A mosaic for each cycle of four in \\
         {\tt$<$m$>$$<$date$>$\_$<$group\_number$>$\_mos$<$cycle\_number$>$}, where {\tt$<$cycle\_number$>$}\\
         counts from 0.

         \sstitem
         The individual flat-fielded frames in {\tt$<$i$>$$<$date$>$\_$<$obs\_number$>$\_ff},
         where {\tt$<$i$>$} is {\tt{f}} for UFTI and {\tt{i}} for IRCAM.  Before 2000 August
         IRCAM frames had prefix {\tt{ro}}.

         \sstitem
         The created flat fields in {\tt{flat\_$<$filter$>$\_$<$group\_number$>$}} for the
         first or only cycle, and {\tt{{\tt{flat\_$<$filter$>$\_$<$group\_number$>$}}\_c$<$cycle\_number$>$}}
         for subsequent cycles.
      }
   }
   \sstdiytopic{
      Related Recipes
   }{
      \htmlref{QUADRANT\_JITTER}{QUADRANT\_JITTER},
      \htmlref{QUADRANT\_JITTER\_BASIC}{QUADRANT\_JITTER\_BASIC},
      \htmlref{EXTENDED\_3x3\_BASIC}{EXTENDED\_3x3\_BASIC},\\
      \htmlref{EXTENDED\_5x5\_BASIC}{EXTENDED\_5x5\_BASIC}.
   }
   \sstimplementationstatus{
      \sstitemlist{

         \sstitem
         The processing engines are from the Starlink packages: \xref{\CCDPACK}{sun139}{}
         and \xref{\KAPPA}{sun95}{}.

         \sstitem
         Uses the Starlink NDF format.

         \sstitem
         History is recorded within the data files.

         \sstitem
         The title of the data is propagated through intermediate files
         to the mosaic.

         \sstitem
         Error propagation is not used.
      }
   }
}

\newpage
\sstroutine{
   QUADRANT\_JITTER\_TELE
}{
   Reduces a ``Quadrant Jitter'' observation, using object masking,
   and telescope offsets for registration
}{
   \sstdescription{
      This script reduces a ``quadrant jitter'' photometry observation
      with UFTI or IRCAM data.  It takes an imaging observation comprising
      one or more series of four object frames where the target is
      approximately centred in each quadrant; and a dark frame to make
      a calibrated, untrimmed mosaic automatically.

      It performs bad-pixel masking, null debiassing, dark subtraction,
      flat-field creation and division, registration using telescope
      offsets, and resampling.  See the ``Notes'' for further information.

      This recipe is intended for extended moving sources (comets)
      tracked by the telescope.  The source extent should not exceed 45
      arcseconds for UFTI or 10 arcseconds for IRCAM, in moderately
      crowded fields.  Sources may include those with a comparatively
      bright core embedded in faint extended emission.  The object need
      not be isolated, as the recipe masks objects within the other
      quadrants, and hence does not introduce significant artifacts into
      the flat field.  This recipe should not be used for frames where
      the telescope has not guided on the moving object.  In that case
      reduction should be performed by 
      \htmlref{MOVING\_QUADRANT\_JITTER}{MOVING\_QUADRANT\_JITTER}, which
      corrects the registration for the motion of the source using an
      ephemeris.
   }
   \sstnotes{
      \sstitemlist{

         \sstitem
         A World Co-ordinate System (WCS) using the AIPS convention is
         created in the headers should no WCS already exist.

         \sstitem
         For IRCAM, old headers are reordered and structured with
         headings before groups of related keywords.  The comments have
         units added or appear in a standard format.  Four deprecated
         deprecated are removed.  FITS-violating headers are corrected.
         Spurious instrument names are changed to IRCAM3.

         \sstitem
         The bad pixel mask applied is {\tt\$ORAC\_DATA\_CAL/bpm}.

         \sstitem
         Each dark-subtracted frame has thresholds applied beyond which
         pixels are flagged as bad.  The lower limit is 5 standard
         deviations below the mode, but constrained to the range -100 to 1.
         The upper limit is 1000 above the saturation limit for the detector
         in the mode used.

         \sstitem
         The flat field is created iteratively.  First the quadrant
         containing the object is masked in each object frame.  Second an
         approximate flat field is created by combining the normalised
         and masked object frames using the clipped median at each pixel.
         This flat field is applied to the object frames.  Sources within
         the flat-fielded frames are detected, and masked in the
         dark-subtracted frames.  The second stage is repeated but applied
         to the masked frames to create the final flat field.

         \sstitem
         Registration is performed using the telescope offsets
         transformed to pixels.

         \sstitem
         The resampling applies non-integer shifts of origin using
         bilinear interpolation.  There is no rotation to align the
         Cartesian axes with the cardinal directions.

         \sstitem
         The recipe makes the mosaics by applying offsets in intensity
         to give the most consistent result amongst the overlapping regions.
         The mosaic is not trimmed to the dimensions of a single frame.  Thus
         the noise will be greater in the peripheral areas having received
         less exposure time.  The full signal will be in the central ninth
         containing the main object.  The mosaic is not normalised by its
         exposure time (that being the exposure time of a single frame).

         \sstitem
         For each cycle of four, the recipe creates a mosaic, which has
         its bad pixels filled and is then added into a master mosaic of
         improving signal to noise.  The exposure time is also summed and
         stored in the mosaic's EXP\_TIME (UFTI) or DEXPTIME (IRCAM) header.
         Likewise the end airmass header, AMEND, is updated to match that of
         the last-observed frame contributing to the mosaic.

         \sstitem
         Intermediate frames are deleted except for the flat-fielded ({\tt\_ff}
         suffix) frames.
      }
   }
   \sstdiytopic{
      Output Data
   }{
      \sstitemlist{

         \sstitem
         The integrated mosaic in {\tt$<$m$>$$<$date$>$\_$<$group\_number$>$\_mos}, where {\tt$<$m$>$}
         is {\tt{gf}} for UFTI and {\tt{gi}} for IRCAM.  Before 2000 August these
         were {\tt{g}} and {\tt{rg}} respectively.

         \sstitem
         A mosaic for each cycle of four in \\
         {\tt$<$m$>$$<$date$>$\_$<$group\_number$>$\_mos$<$cycle\_number$>$}, where {\tt$<$cycle\_number$>$}\\
         counts from 0.

         \sstitem
         The individual flat-fielded frames in {\tt$<$i$>$$<$date$>$\_$<$obs\_number$>$\_ff},
         where {\tt$<$i$>$} is {\tt{f}} for UFTI and {\tt{i}} for IRCAM.  Before 2000 August
         IRCAM frames had prefix {\tt{ro}}.

         \sstitem
         The created flat fields in {\tt{flat\_$<$filter$>$\_$<$group\_number$>$}} for the
         first or only cycle, and {\tt{{\tt{flat\_$<$filter$>$\_$<$group\_number$>$}}\_c$<$cycle\_number$>$}}
         for subsequent cycles.

      }
   }
   \sstdiytopic{
      Related Recipes
   }{
      \htmlref{JITTER\_SELF\_FLAT\_TELE}{JITTER\_SELF\_FLAT\_TELE},
      \htmlref{MOVING\_JITTER\_SELF\_FLAT}{MOVING\_JITTER\_SELF\_FLAT},\\
      \htmlref{MOVING\_QUADRANT\_JITTER}{MOVING\_QUADRANT\_JITTER},
      \htmlref{QUADRANT\_JITTER}{QUADRANT\_JITTER}.
   }
   \sstimplementationstatus{
      \sstitemlist{

         \sstitem
         The processing engines are from the Starlink packages: \xref{\CCDPACK}{sun139}{}
         \xref{\KAPPA}{sun95}{}, and \xref{\PISA}{sun109}{}.

         \sstitem
         Uses the Starlink NDF format.

         \sstitem
         History is recorded within the data files.

         \sstitem
         The title of the data is propagated through intermediate files
         to the mosaic.

         \sstitem
         Error propagation is not used.
      }
   }
}

%\newpage
\sstroutine{
   REDUCE\_DARK
}{
   File an observation as the current dark
}{
   \sstdescription{
      This recipe reduces a dark-frame observation with UFTI or IRCAM
      data.  It files the dark in the dark index file.  Reduction
      comprises only bad-pixel masking.
   }
   \sstnotes{
      \sstitemlist{

         \sstitem
         The bad pixel mask applied is {\tt\$ORAC\_DATA\_CAL/bpm}.

         \sstitem
         Each dark-subtracted frame has thresholds applied beyond which
         pixels are flagged as bad.  The lower limit is 5 standard
         deviations below the mode, but constrained to the range $-$100 to 1.
         The upper limit is 1000 above the saturation limit for the detector
         in the mode used.

         \sstitem
         Intermediate frames are deleted.

         \sstitem
         Sub-arrays are supported.
      }
   }
   \sstdiytopic{
      Output Data
   }{
      \sstitemlist{

         \sstitem
         The dark is called {\tt{dark\_$<$frame\_number$>$}}.

         \sstitem
         The dark is filed in {\tt\$ORAC\_DATA\_OUT/index.dark}.
      }
   }
   \sstimplementationstatus{
      \sstitemlist{

         \sstitem
         The processing engines are from the Starlink package \xref{\KAPPA}{sun95}{}.

         \sstitem
         Uses the Starlink NDF format.

         \sstitem
         History is recorded within the data files.

         \sstitem
         The title of the data is propagated through the intermediate file
         to the dark.

         \sstitem
         Error propagation is not used.
      }
   }
}

%\newpage
\sstroutine{
   SKY\_AND\_JITTER
}{
   Reduces a ``combined jitter'' photometry observation
}{
   \sstdescription{
      This script reduces a ``combined jitter'' photometry observation
      with UFTI or IRCAM data.  It takes an imaging observation
      comprising one or more sets of frames, each set containing a sky
      frame, followed by jittered object frames; and a pre-determined
      flat-field frame to make a calibrated, trimmed mosaic automatically.

      This recipe performs bad-pixel masking, null debiassing, sky
      subtraction, flat-field division, feature detection and matching
      between object frames, and resampling.  See the ``Notes'' for details.

      This recipe is suitable for moderately faint point sources.
   }
   \sstnotes{
      \sstitemlist{

         \sstitem
         A World Co-ordinate System (WCS) using the AIPS convention is
         created in the headers should no WCS already exist.

         \sstitem
         For IRCAM, old headers are reordered and structured with
         headings before groups of related keywords.  The comments have
         units added or appear in a standard format.  Four deprecated
         deprecated are removed.  FITS-violating headers are corrected.
         Spurious instrument names are changed to IRCAM3.

         \sstitem
         You may use \htmlref{SKY\_FLAT}{SKY\_FLAT} or \htmlref{SKY\_FLAT\_MASKED}{SKY\_FLAT\_MASKED} to make the flat field.

         \sstitem
         The bad-pixel mask applied is {\tt\$ORAC\_DATA\_CAL/bpm}.

         \sstitem
         Each dark-subtracted frame has thresholds applied beyond which
         pixels are flagged as bad.  The lower limit is 5 standard
         deviations below the mode, but constrained to the range $-$100 to 1.
         The upper limit is 1000 above the saturation limit for the detector
         in the mode used.

         \sstitem
         The most-recent sky frame is used for the sky subtraction.

         \sstitem
         Where automatic registration is not possible, the recipe matches
         the centroid of central source, and should that fail, it resorts
         to using the telescope offsets transformed to pixels.

         \sstitem
         The resampling applies non-integer shifts of origin using
         bilinear interpolation.  There is no rotation to align the
         Cartesian axes with the cardinal directions.

         \sstitem
         The recipe makes the mosaic by applying offsets in intensity to
         give the most consistent result amongst the overlapping regions.
         The mosaic is trimmed to the dimensions of an input frame.  The
         mosaic is not normalised by its exposure time (that being the
         exposure time of a single frame).

         \sstitem
         For each cycle of jittered frames, the recipe creates a mosaic,
         which is then added into a master mosaic of improving signal to
         noise.  The exposure time is also summed and stored in the mosaic's
         EXP\_TIME (UFTI) or DEXPTIME (IRCAM) header.  Likewise the end
         airmass header, AMEND, is updated to match that of the last-observed
         frame contributing to the mosaic.

         \sstitem
         Intermediate frames are deleted except for the flat-fielded ({\tt\_ff}
         suffix) frames.

         \sstitem
         Sub-arrays are supported.
      }
   }
   \sstdiytopic{
      Output Data
   }{
      \sstitemlist{

         \sstitem
         The resultant mosaic in {\tt$<$m$>$$<$date$>$\_$<$group\_number$>$\_mos}, where {\tt$<$m$>$}
         is {\tt{gf}} for UFTI and {\tt{gi}} for IRCAM.  Before 2000 August these
         were {\tt{g}} and {\tt{rg}} respectively.

         \sstitem
         The individual flat-fielded frames in {\tt$<$i$>$$<$date$>$\_$<$obs\_number$>$\_ff},
         where {\tt$<$i$>$} is {\tt{f}} for UFTI and {\tt{i}} for IRCAM.  Before 2000 August
         IRCAM frames had prefix {\tt{ro}}.
      }
   }
   \sstparameters{
      \sstsubsection{
         NUMBER = INTEGER
      }{
         The number of frames in the jitter, excluding the sky frame.
         If absent, the number of offsets, as given by header NOFFSETS,
         minus two is used.  If neither is available, 5 is used.  An
         error state arises if the number of jittered frames is fewer
         than 3.  {\tt[]}
      }
   }
   \sstdiytopic{
      Related Recipes
   }{
      \htmlref{JITTER\_SELF\_FLAT}{JITTER\_SELF\_FLAT},
      \htmlref{NOD\_SELF\_FLAT\_NO\_MASK}{NOD\_SELF\_FLAT\_NO\_MASK},
      \htmlref{SKY\_AND\_JITTER\_APHOT}{SKY\_AND\_JITTER\_APHOT},\\
      \htmlref{SKY\_FLAT}{SKY\_FLAT},
      \htmlref{SKY\_FLAT\_MASKED}{SKY\_FLAT\_MASKED}.
   }
   \sstimplementationstatus{
      \sstitemlist{

         \sstitem
         The processing engines are from the Starlink packages: \xref{\CCDPACK}{sun139}{}
         and \xref{\KAPPA}{sun95}{}.

         \sstitem
         Uses the Starlink NDF format.

         \sstitem
         History is recorded within the data files.

         \sstitem
         The title of the data is propagated through intermediate files
         to the mosaic.

         \sstitem
         Error propagation is not used.
      }
   }
   \sstdiytopic{
      Deprecated Variants
   }{
      SKY\_AND\_JITTER5.
   }
}

%\newpage
\sstroutine{
   SKY\_AND\_JITTER\_APHOT
}{
   Reduces a ``combined jitter'' photometry observation, and
   performs aperture photometry
}{
   \sstdescription{
      This script reduces a ``combined jitter'' photometry observation
      with UFTI or IRCAM data.  It takes an imaging observation
      comprising one or more sets of frames, each set containing a sky
      frame, followed by jittered object frames; and a pre-determined
      flat-field frame to make a calibrated, trimmed mosaic automatically.

      This recipe performs bad-pixel masking, null debiassing, sky
      subtraction, flat-field division, feature detection and matching
      between object frames, and resampling.  See the ``Notes'' for details.

      Photometry of the point source using a fixed 5-arcsecond aperture
      is calculated for each jitter frame and the mosaic.  The results
      appear in {\tt\$ORAC\_DATA\_OUT/aphot\_results.txt} in the form of a Starlink
      small text list.  The analysis of each star is appended to this file.

      This recipe is suitable for moderately faint point sources.
   }
   \sstnotes{
      \sstitemlist{

         \sstitem
         A World Co-ordinate System (WCS) using the AIPS convention is
         created in the headers should no WCS already exist.

         \sstitem
         For IRCAM, old headers are reordered and structured with
         headings before groups of related keywords.  The comments have
         units added or appear in a standard format.  Four deprecated
         deprecated are removed.  FITS-violating headers are corrected.
         Spurious instrument names are changed to IRCAM3.

         \sstitem
         You may use \htmlref{SKY\_FLAT}{SKY\_FLAT} or \htmlref{SKY\_FLAT\_MASKED}{SKY\_FLAT\_MASKED} to make the flat field.

         \sstitem
         The bad-pixel mask applied is {\tt\$ORAC\_DATA\_CAL/bpm}.

         \sstitem
         Each dark-subtracted frame has thresholds applied beyond which
         pixels are flagged as bad.  The lower limit is 5 standard
         deviations below the mode, but constrained to the range $-$100 to 1.
         The upper limit is 1000 above the saturation limit for the detector
         in the mode used.

         \sstitem
         The most-recent sky frame is used for the sky subtraction.

         \sstitem
         Where automatic registration is not possible, the recipe matches
         the centroid of central source, and should that fail, it resorts
         to using the telescope offsets transformed to pixels.

         \sstitem
         The resampling applies non-integer shifts of origin using
         bilinear interpolation.  There is no rotation to align the
         Cartesian axes with the cardinal directions.

         \sstitem
         The recipe makes the mosaic by applying offsets in intensity to
         give the most consistent result amongst the overlapping regions.
         The mosaic is trimmed to the dimensions of an input frame.  The
         mosaic is not normalised by its exposure time (that being the
         exposure time of a single frame).

         \sstitem
         For each cycle of jittered frames, the recipe creates a mosaic,
         which is then added into a master mosaic of improving signal to
         noise.  The exposure time is also summed and stored in the mosaic's
         EXP\_TIME (UFTI) or DEXPTIME (IRCAM) header.  Likewise the end
         airmass header, AMEND, is updated to match that of the last-observed
         frame contributing to the mosaic.

         \sstitem
         The photometry tabulation includes the file name, source
         name, time, filter, airmass, the catalogue magnitude and
         estimates of the zero-point with and without the application
         of a mean extinction.  There are headings at the top of each
         column.

         \sstitem
         The photometry uses the mode calculated from
         \mbox{3 $\lsk$ median $-$ 2 $\lsk$ mean} and Chauvenet's
         rejection criterion to estimate the sky level in an annulus
         about the source.  The inner annulus diameter is 1.3 times
         that of the aperture (6.5 arcsec); the outer annulus is 2.5
         times (12.5 arcsec) for UFTI and twice the aperture (10 arcsec)
         for IRCAM. The errors are internal, based on the sky noise.

         \sstitem
         Intermediate frames are deleted except for the flat-fielded ({\tt\_ff}
         suffix) frames.

         \sstitem
         Sub-arrays are supported.
      }
   }
   \sstdiytopic{
      Output Data
   }{
      \sstitemlist{

         \sstitem
         The resultant mosaic in {\tt$<$m$>$$<$date$>$\_$<$group\_number$>$\_mos}, where {\tt$<$m$>$}
         is {\tt{gf}} for UFTI and {\tt{gi}} for IRCAM.  Before 2000 August these
         were {\tt{g}} and {\tt{rg}} respectively.

         \sstitem
         The individual flat-fielded frames in {\tt$<$i$>$$<$date$>$\_$<$obs\_number$>$\_ff},
         where {\tt$<$i$>$} is {\tt{f}} for UFTI and {\tt{i}} for IRCAM.  Before 2000 August
         IRCAM frames had prefix {\tt{ro}}.
      }
   }
   \sstparameters{
      \sstsubsection{
         NUMBER = INTEGER
      }{
         The number of frames in the jitter, excluding the sky frame.
         If absent, the number of offsets, as given by header NOFFSETS,
         minus two is used.  If neither is available, 5 is used.  An
         error state arises if the number of jittered frames is fewer
         than 3.  {\tt[]}
      }
   }
   \sstdiytopic{
      Related Recipes
   }{
      \htmlref{JITTER\_SELF\_FLAT\_APHOT}{JITTER\_SELF\_FLAT\_APHOT},
      \htmlref{NOD\_SELF\_FLAT\_NO\_MASK\_APHOT}{NOD\_SELF\_FLAT\_NO\_MASK\_APHOT},\\
      \htmlref{SKY\_AND\_JITTER}{SKY\_AND\_JITTER},
      \htmlref{SKY\_FLAT}{SKY\_FLAT}, \htmlref{SKY\_FLAT\_MASKED}{SKY\_FLAT\_MASKED}.
   }
   \sstimplementationstatus{
      \sstitemlist{

         \sstitem
         The processing engines are from the Starlink packages: \xref{\CCDPACK}{sun139}{}
         and \xref{\KAPPA}{sun95}{}.

         \sstitem
         Uses the Starlink NDF format.

         \sstitem
         History is recorded within the data files.

         \sstitem
         The title of the data is propagated through intermediate files
         to the mosaic.

         \sstitem
         Error propagation is not used.
      }
   }
   \sstdiytopic{
      Deprecated Variants
   }{
      SKY\_AND\_JITTER5\_APHOT.
   }
}

%\newpage
\sstroutine{
   SKY\_FLAT
}{
   Creates and files a flat field derived from jittered frames
}{
   \sstdescription{
      This recipe makes a sky flat for UFTI or IRCAM from a series of sky
      or object frames which are combined using one of a selection of
      statistics.  It expects one dark frame followed by jittered sky frames.

      It performs a null debiassing, bad-pixel masking, and dark subtraction
      before combining normalised frames pixel by pixel using the median.
      Details of the flat are filed in the index of flats for future
      selection and use of the flat.  See the ``Notes'' for further details.

      For best results the field observed should contain few stars and no
      bright ones.  In contaminated sky regions, recipe \htmlref{SKY\_FLAT\_MASKED}{SKY\_FLAT\_MASKED} will
      greatly reduce artifacts appearing in the resultant flat.
   }
   \sstnotes{
      \sstitemlist{

         \sstitem
         A World Co-ordinate System (WCS) using the AIPS convention is
         created in the headers should no WCS already exist.

         \sstitem
         For IRCAM, old headers are reordered and structured with
         headings before groups of related keywords.  The comments have
         units added or appear in a standard format.  Four deprecated
         deprecated are removed.  FITS-violating headers are corrected.
         Spurious instrument names are changed to IRCAM3.

         \sstitem
         The bad-pixel mask applied is {\tt\$ORAC\_DATA\_CAL/bpm}.

         \sstitem
         Each dark-subtracted frame has thresholds applied beyond which
         pixels are flagged as bad.  The lower limit is 5 standard
         deviations below the mode, but constrained to the range $-$100 to 1.
         The upper limit is 1000 above the saturation limit for the detector
         in the mode used.

         \sstitem
         Intermediate frames are deleted.

         \sstitem
         Sub-arrays are supported.
      }
   }
   \sstdiytopic{
      Output Data
   }{
      \sstitemlist{

         \sstitem
         The created flat fields in {\tt{flat\_$<$filter$>$\_$<$group\_number$>$}} for the
         first or only cycle, and {\tt{{\tt{flat\_$<$filter$>$\_$<$group\_number$>$}}\_c$<$cycle\_number$>$}}
         for subsequent recipe cycles.  Token {\tt$<$filter$>$} is the filter name,
         {\tt$<$group\_number$>$} is the frame number of the group, and {\tt$<$cycle\_number$>$}
         is the number of the cycle, counting from one.

         \sstitem
         The flats are filed in {\tt\$ORAC\_DATA\_OUT/index.flat}.
      }
   }
   \sstparameters{
      \sstsubsection{
         NUMBER = INTEGER
      }{
         The number of frames in the jitter.  If absent, the number of
         offsets, as given by header NOFFSETS, minus one is used.  If
         neither is available, 5 is used.  An error state arises if
         the number of jittered frames is fewer than 3.  {\tt[]}
      }
   }
   \sstdiytopic{
      Related Recipes
   }{
      \htmlref{SKY\_FLAT\_FP}{SKY\_FLAT\_FP},
      \htmlref{SKY\_FLAT\_MASKED}{SKY\_FLAT\_MASKED},
      \htmlref{SKY\_FLAT\_POL}{SKY\_FLAT\_POL}.
   }
   \sstimplementationstatus{
      \sstitemlist{

         \sstitem
         The processing engines are from the Starlink packages \xref{\CCDPACK}{sun139}{}
         and \xref{\KAPPA}{sun95}{}.

         \sstitem
         Uses the Starlink NDF format.

         \sstitem
         History is recorded within the data files.

         \sstitem
         The title of the data is propagated through the intermediate file
         to the flat.

         \sstitem
         Error propagation is not used.
      }
   }
}

%\newpage
\sstroutine{
   SKY\_FLAT\_FP
}{
   Creates and files a flat field derived from multiples of four frames using
   object masking to reduce artifacts
}{
   \sstdescription{
      This recipe make a sky flat for UFTI from a series of four
      (or multiples of four) sky or object frames combined using one of a
      selection of statistics.  It is intended to be used to make a flat for
      Fabry-Perot data.

      It performs a null debiassing, bad-pixel masking, and dark
      subtraction before combining the sky frames pixel by pixel to
      make the flat.  See the ``Notes'' for further details.  The parameters
      of the flat are filed in the index of flats for future selection
      and use of the flat.

      For best results the field observed should contain few stars and no
      bright ones.
   }
   \sstnotes{
      \sstitemlist{

         \sstitem
         A World Co-ordinate System (WCS) using the AIPS convention is
         created in the headers should no WCS already exist.

         \sstitem
         The bad-pixel mask applied is {\tt\$ORAC\_DATA\_CAL/bpm}.

         \sstitem
         Each dark-subtracted frame has thresholds applied beyond which
         pixels are flagged as bad.  The lower limit is 5 standard
         deviations below the mode, but constrained to the range $-$100 to 1.
         The upper limit is 1000 above the saturation limit for the detector
         in the mode used.

         \sstitem
         The flat field is created iteratively.  First an approximate
         flat-field is created by combining normalised sky frames using
         the median at each pixel.  This flat field is applied to the sky
         frames.  Sources within the flat-fielded frames are detected, and
         masked in the dark-subtracted frames.  The first stage is repeated
         but applied to the masked frames to create the final flat field.

         \sstitem
         Intermediate frames are deleted.

         \sstitem
         Sub-arrays are supported.
      }
   }
   \sstdiytopic{
      Output Data
   }{
      \sstitemlist{

         \sstitem
         The created flat field in {\tt{flat\_$<$filter$>$\_$<$group\_number$>$}} for the
         first or only cycle, and {\tt{{\tt{flat\_$<$filter$>$\_$<$group\_number$>$}}\_c$<$cycle\_number$>$}}
         for subsequent recipe cycles.  Token {\tt$<$filter$>$} is the filter name,
         {\tt$<$group\_number$>$} is the frame number of the group, and {\tt$<$cycle\_number$>$}
         is the number of the cycle, counting from one.

         \sstitem
         The flats are filed in {\tt\$ORAC\_DATA\_OUT/index.flat}.
      }
   }
   \sstdiytopic{
      Related Recipes
   }{
      \htmlref{SKY\_FLAT}{SKY\_FLAT},
      \htmlref{SKY\_FLAT\_MASKED}{SKY\_FLAT\_MASKED}.
   }
   \sstimplementationstatus{
      \sstitemlist{

         \sstitem
         The processing engines are from the Starlink packages \xref{\CCDPACK}{sun139}{},
         \xref{\KAPPA}{sun95}{}, and \xref{\PISA}{sun109}{}.

         \sstitem
         Uses the Starlink NDF format.

         \sstitem
         History is recorded within the data files.

         \sstitem
         The title of the data is propagated through the intermediate file
         to the flat.

         \sstitem
         Error propagation is not used.
      }
   }
}

%\newpage
\sstroutine{
   SKY\_FLAT\_MASKED
}{
   Creates and files a flat field derived from jittered frames using
   object masking to reduce artifacts
}{
   \sstdescription{
      This recipe makes a sky flat for UFTI or IRCAM from a series of sky
      or object frames which are combined using one of a selection of
      statistics.  It expects one dark frame followed by jittered sky frames.

      It performs a null debiassing, bad-pixel masking, and dark subtraction
      before combining the sky frames pixel by pixel to make the flat.  See
      the ``Notes'' for further details.  The parameters of the flat are filed
      in the index of flats for future selection and use of the flat.

      For best results the field observed should contain few stars and no
      bright ones.  In sparse fields, recipe \htmlref{SKY\_FLAT}{SKY\_FLAT} is a faster alternative.
   }
   \sstnotes{
      \sstitemlist{

         \sstitem
         A World Co-ordinate System (WCS) using the AIPS convention is
         created in the headers should no WCS already exist.

         \sstitem
         For IRCAM, old headers are reordered and structured with
         headings before groups of related keywords.  The comments have
         units added or appear in a standard format.  Four deprecated
         deprecated are removed.  FITS-violating headers are corrected.
         Spurious instrument names are changed to IRCAM3.

         \sstitem
         The bad pixel mask applied is {\tt\$ORAC\_DATA\_CAL/bpm}.

         \sstitem
         Each dark-subtracted frame has thresholds applied beyond which
         pixels are flagged as bad.  The lower limit is 5 standard
         deviations below the mode, but constrained to the range $-$100 to 1.
         The upper limit is 1000 above the saturation limit for the detector
         in the mode used.

         \sstitem
         The flat field is created iteratively.  First an approximate
         flat-field is created by combining normalised sky frames using
         the median at each pixel.  This flat field is applied to the sky
         frames.  Sources within the flat-fielded frames are detected, and
         masked in the dark-subtracted frames.  The first stage is repeated
         but applied to the masked frames to create the final flat field.

         \sstitem
         Intermediate frames are deleted.

         \sstitem
         Sub-arrays are supported.
      }
   }
   \sstdiytopic{
      Output Data
   }{
      \sstitemlist{

         \sstitem
         The created flat fields in {\tt{flat\_$<$filter$>$\_$<$group\_number$>$}} for the
         first or only cycle, and {\tt{{\tt{flat\_$<$filter$>$\_$<$group\_number$>$}}\_c$<$cycle\_number$>$}}
         for subsequent recipe cycles.  Token {\tt$<$filter$>$} is the filter name,
         {\tt$<$group\_number$>$} is the frame number of the group, and {\tt$<$cycle\_number$>$}
         is the number of the cycle, counting from one.

         \sstitem
         The flats are filed in {\tt\$ORAC\_DATA\_OUT/index.flat}.
      }
   }
   \sstparameters{
      \sstsubsection{
         NUMBER = INTEGER
      }{
         The number of frames in the jitter.  If absent, the number of
         offsets, as given by header NOFFSETS, minus one is used.  If
         neither is available, 5 is used.  An error state arises if
         the number of jittered frames is fewer than 3.  {\tt[]}
      }
   }
   \sstdiytopic{
      Related Recipes
   }{
      \htmlref{SKY\_FLAT}{SKY\_FLAT}.
   }
   \sstimplementationstatus{
      \sstitemlist{

         \sstitem
         The processing engines are from the Starlink packages \xref{\CCDPACK}{sun139}{},
         \xref{\KAPPA}{sun95}{}, and \xref{\PISA}{sun109}{}.

         \sstitem
         Uses the Starlink NDF format.

         \sstitem
         History is recorded within the data files.

         \sstitem
         The title of the data is propagated through the intermediate file
         to the flat.

         \sstitem
         Error propagation is not used.
      }
   }
}

%\newpage
\sstroutine{
   SKY\_FLAT\_POL
}{
   Creates and files a flat field derived from eight frames using
   object masking to reduce artifacts
}{
   \sstdescription{
      This recipe make a sky flat for UFTI or IRCAM from a series of
      eight sky or object frames combined using one of a selection of
      statistics.  It is intended to be used to make a flat for
      polarimetry data.  The data should comprise two spatial positions
      at the waveplate angles 0, 45, 22.5, and 67.5 degrees.

      It performs a null debiassing, bad-pixel masking, and dark
      subtraction before combining the sky frames pixel by pixel to
      to make the flat.  See the ``Notes'' for further details.  The
      parameters of the flat are filed in the index of flats for
      future selection and use of the flat.

      For best results the field observed should contain few stars and
      no bright ones.
   }
   \sstnotes{
      \sstitemlist{

         \sstitem
         A World Co-ordinate System (WCS) using the AIPS convention is
         created in the headers should no WCS already exist.

         \sstitem
         For IRCAM, old headers are reordered and structured with
         headings before groups of related keywords.  The comments have
         units added or appear in a standard format.  Four deprecated
         deprecated are removed.  FITS-violating headers are corrected.
         Spurious instrument names are changed to IRCAM3.

         \sstitem
         The bad-pixel mask applied is {\tt\$ORAC\_DATA\_CAL/bpm}.

         \sstitem
         Each dark-subtracted frame has thresholds applied beyond which
         pixels are flagged as bad.  The lower limit is 5 standard
         deviations below the mode, but constrained to the range $-$100 to 1.
         The upper limit is 1000 above the saturation limit for the detector
         in the mode used.

         \sstitem
         The flat field is created iteratively.  First an approximate
         flat-field is created by combining normalised sky frames using
         the median at each pixel.  This flat field is applied to the sky
         frames.  Sources within the flat-fielded frames are detected, and
         masked in the dark-subtracted frames.  The first stage is repeated
         but applied to the masked frames to create the final flat field.

         \sstitem
         Intermediate frames are deleted.

         \sstitem
         Sub-arrays are supported.
      }
   }
   \sstdiytopic{
      Output Data
   }{
      \sstitemlist{

         \sstitem
         There are flats for each waveplate angle, made by copying the
         original flat frame.  This is to permit both flats made at each
         angle (\htmlref{SKY\_FLAT\_POL\_ANGLE}{SKY\_FLAT\_POL\_ANGLE}), or with the angles combined as here.
         Each is called
{\tt{flat\_$<$filter$>$\_pol$<$waveplate\_angle$>$\_}\-{\tt$<$group\_number$>$}},
         The {\tt$<$waveplate\_angle$>$} is the integer part of the angle, e.g. 22,
         67; where {\tt$<$filter$>$} is the filter name (excluding the {\tt{+pol}}),
         and {\tt$<$group\_number$>$} is the frame number of the group.  For each
         subsequent cycle of the recipe, the recipe makes new flats
         which have a {\tt\_c$<$cycle\_number$>$} suffix, where {\tt$<$cycle\_number$>$} is
         the number of the cycle, counting from one.

         \sstitem
         The flats are filed in {\tt\$ORAC\_DATA\_OUT/index.flat}.
      }
   }
   \sstdiytopic{
      Related Recipes
   }{
      \htmlref{SKY\_FLAT\_FP}{SKY\_FLAT\_FP},
      \htmlref{SKY\_FLAT\_MASKED}{SKY\_FLAT\_MASKED},
      \htmlref{SKY\_FLAT\_POL\_ANGLE}{SKY\_FLAT\_POL\_ANGLE}.
   }
   \sstimplementationstatus{
      \sstitemlist{

         \sstitem
         The processing engines are from the Starlink packages \xref{\CCDPACK}{sun139}{},
         \xref{\KAPPA}{sun95}{}, and \xref{\PISA}{sun109}{}.

         \sstitem
         Uses the Starlink NDF format.

         \sstitem
         History is recorded within the data files.

         \sstitem
         The title of the data is propagated through the intermediate file
         to the flat.

         \sstitem
         Error propagation is not used.
      }
   }
}

%\newpage
\sstroutine{
   SKY\_FLAT\_POL\_ANGLE
}{
   Creates and files flat fields derived from jittered frames
   at each waveplate angle, using object masking to reduce artifacts
}{
   \sstdescription{
      This recipe make a sky flat for UFTI or IRCAM from a series of
      sky or object frames combined using one of a selection of
      statistics.  It is intended to be used to make flats at each
      waveplate angle for polarimetry data.  The data should
      comprise at least three spatial positions for each waveplate
      angle 0, 45, 22.5, and 67.5 degrees in turn.

      It performs a null debiassing, bad-pixel masking, and dark
      subtraction before combining the sky frames pixel by pixel to
      to make the flat.  See the ``Notes'' for further details.  The
      parameters of the flat are filed in the index of flats for
      future selection and use of the flat.

      For best results the field observed should contain few stars and
      no bright ones.
   }
   \sstnotes{
      \sstitemlist{

         \sstitem
         A World Co-ordinate System (WCS) using the AIPS convention is
         created in the headers should no WCS already exist.

         \sstitem
         For IRCAM, old headers are reordered and structured with
         headings before groups of related keywords.  The comments have
         units added or appear in a standard format.  Four deprecated
         deprecated are removed.  FITS-violating headers are corrected.
         Spurious instrument names are changed to IRCAM3.

         \sstitem
         The bad-pixel mask applied is {\tt\$ORAC\_DATA\_CAL/bpm}.

         \sstitem
         Each dark-subtracted frame has thresholds applied beyond which
         pixels are flagged as bad.  The lower limit is 5 standard
         deviations below the mode, but constrained to the range $-$100 to 1.
         The upper limit is 1000 above the saturation limit for the detector
         in the mode used.

         \sstitem
         The flat field is created iteratively.  First an approximate
         flat-field is created by combining normalised sky frames using
         the median at each pixel.  This flat field is applied to the sky
         frames.  Sources within the flat-fielded frames are detected, and
         masked in the dark-subtracted frames.  The first stage is repeated
         but applied to the masked frames to create the final flat field.

         \sstitem
         Intermediate frames are deleted.

         \sstitem
         Sub-arrays are supported.
      }
   }
   \sstdiytopic{
      Output Data
   }{
      \sstitemlist{

         \sstitem
         The flats are called {\tt{flat\_$<$filter$>$\_pol$<$waveplate\_angle$>$\_$<$group\_number$>$}},
         The {\tt$<$waveplate\_angle$>$} is the integer part of the angle, e.g. 22, 67;
         {\tt$<$filter$>$} is the filter name (excluding any {\tt{+pol}}); and {\tt$<$group\_number$>$}
         is the frame number of the group.  For each subsequent cycle of the
         recipe, the recipe makes new flats which have a {\tt\_c$<$cycle\_number$>$}
         suffix, where {\tt$<$cycle\_number$>$} is the number of the cycle, counting
         from one.

         \sstitem
         The flats are filed in {\tt\$ORAC\_DATA\_OUT/index.flat}.
      }
   }
   \sstdiytopic{
      Related Recipes
   }{
      \htmlref{SKY\_FLAT\_FP}{SKY\_FLAT\_FP},
      \htmlref{SKY\_FLAT\_MASKED}{SKY\_FLAT\_MASKED},
      \htmlref{SKY\_FLAT\_POL}{SKY\_FLAT\_POL}.
   }
   \sstimplementationstatus{
      \sstitemlist{

         \sstitem
         The processing engines are from the Starlink packages \xref{\CCDPACK}{sun139}{},
         \xref{\KAPPA}{sun95}{}, and \xref{\PISA}{sun109}{}.

         \sstitem
         Uses the Starlink NDF format.

         \sstitem
         History is recorded within the data files.

         \sstitem
         The title of the data is propagated through the intermediate file
         to the flat.

         \sstitem
         Error propagation is not used.
      }
   }
}

\newpage
\section{\xlabel{se_changes2p1}Release Notes---V2.1-0\label{se_changes2p1}}
\markboth{{\stardocname}~ --- Release Notes---V2.1-0}
{{\stardocname}~--- Release Notes---V2.1-0}

The main changes are the addition of spatially jittered Fabry-Perot
recipes, and recipes for compact comets.

\subsection{New recipes}
\begin{description}
   \item [\htmlref{FP\_JITTER}{FP\_JITTER}]
      Reduction of a Fabry-Perot observation, comprising eight frames,
      on and off the source, and on and off the spectral line both to
      the blue and to the red.  This is repeated for a series of spatial
      positions of the source.
   \item [\htmlref{FP\_JITTER\_NO\_SKY}{FP\_JITTER\_NO\_SKY}]
      Reduction of a Fabry-Perot observation, comprising four frames,
      all on the source, and on and off the spectral line both to
      the blue and to the red.  This is repeated for a series of spatial
      positions of the source.
   \item [\htmlref{MOVING\_QUADRANT\_JITTER}{MOVING\_QUADRANT\_JITTER}]
      As \htmlref{QUADRANT\_JITTER}{QUADRANT\_JITTER} but registration
      is adjusted to track the motion of a comet using ephemeris
      data.  The comet should $<$45 arcsec diameter for UFTI, or $<$10
      arcsec for IRCAM.  Compared with 
      \htmlref{MOVING\_JITTER\_SELF\_FLAT}{MOVING\_JITTER\_SELF\_FLAT}, 
      this recipe avoids cometary artifacts appearing in the flat field.
   \item [\htmlref{QUADRANT\_JITTER\_TELE}{QUADRANT\_JITTER\_TELE}]
      As QUADRANT\_JITTER, but registers using the telescope offsets.
      This is used for observing compact comets (limiting angular
      sizes as above), when the telescope has tracked the
      nucleus.
\end{description}

\subsection{Modified recipes}

\begin{itemize}
    \item \htmlref{ARRAY\_TESTS}{ARRAY\_TESTS}
       For UFTI, the ADU conversion is obtained from the GAIN header,
       rather than being hardwired at 7.0.
    \item \htmlref{CHOP\_SKY\_JITTER}{CHOP\_SKY\_JITTER}
       This now functions correctly for multiple cycles of the recipe.
    \item \htmlref{FP}{FP}
       Documentation improvements especially in the description.
       Primitive \_FP\_STEER\_ has a new steering parameter, NPAIRS,
       and parameter NUMBER has changed to its normal meaning.
    \item \htmlref{NIGHT\_LOG}{NIGHT\_LOG}
       This can start from observation numbers other than 1.  A bug
       has been fixed where the dimensions appeared as zero for UFTI.
       It arose because certain headers no longer existed after 2000
       August.
    \item \htmlref{SKY\_FLAT\_FP}{SKY\_FLAT\_FP}
       This is no longer limited to eight frames.
    \item \htmlref{SKY\_AND\_JITTER}{SKY\_AND\_JITTER}
       A bug has been fixed where some intermediate files were not being
       removed for this recipe and its variant.
\end{itemize}

\subsection{Global changes}

The main changes from a user perspective were as follows.

\begin{itemize}
   \item The aperture for \_APHOT recipes is now 5 arcseconds.
   \item A new photometry catalogue {\tt fs2001.dat}, supplied in
      original form by Sandy Leggett, is used by the \_APHOT recipes.
      The new catalogue contains $IZLM$ magnitudes for the first time,
      and the $JHK$ data have been refined to account for recent
      observations.  The 2000 edition is accessed for $JHK$ photometry
      if the 2001 catalogue is unavailable. 
   \item Another new UFTI bad-pixel mask.  Old masks are available on
      request.
   \item A bug that affected some rare mixed-method registrations has
      been fixed.  The $y$ offset had the wrong sign.
   \item Allow for a special case in the mosaicking registration, when
      the nearest-neighbour method is used to align the various images,
      {\em{and}} an automatic multiple-object registration was found.
      It was possible not to find a common object identifier in all
      the fields when the target is a low-surface brightness galaxy.
      While this is still possible, it is far less likely than before.
   \item The IRCAM saturation level was refined upwards to 20000, or
      33000 if header SPD\_GAIN is {\tt{"Deepwell"}}.  SPD\_GAIN
      is created by the recipe when it is absent, based upon the value
      of the detector bias.
   \item The message concerning an AST SKY Frame creation,
      which could be confused with a data frame of blank sky, has been
      clarified.  The same script allows for a missing CROTA2 header
      in old data when inserting world co-ordinate system headers.
   \item Various minor improvements and corrections to the
      documentation, such as correcting the former prefix for UFTI
      mosaics in the reference section, and excluding references to
      IRCAM in the FP recipes.
\end{itemize}


\section{\xlabel{se_changes2p0}Release Notes---V2.0-1\label{se_changes2p0}}
\markboth{{\stardocname}~ --- Release Notes---V2.0-0}
{{\stardocname}~ --- Release Notes---V2.0-0}

The major changes are the move to generic recipes, and the introduction
of many new recipes, especially for polarimetry and Fabry-Perot data.

\subsection{New recipes}
\begin{description}
   \item [\htmlref{CHOP\_SKY\_JITTER}{CHOP\_SKY\_JITTER}]
      Reduction of alternating sky-target jitters for extended objects
      of size comparable to the detector's field of view.
   \item [\htmlref{CHOP\_SKY\_JITTER\_BASIC}{CHOP\_SKY\_JITTER\_BASIC}]
      A basic (faster) version of CHOP\_SKY\_JITTER.
   \item [\htmlref{FP}{FP}]
      Reduction of a Fabry-Perot observation, without jittering.
   \item [\htmlref{JITTER\_SELF\_FLAT\_NCOLOUR}{JITTER\_SELF\_FLAT\_NCOLOUR}]
      Reduction of multi-colour standard jitters.  This will become
      the new JITTER\_SELF\_FLAT once the recipes are colour generic.
   \item [\htmlref{POL\_ANGLE\_JITTER}{POL\_ANGLE\_JITTER}]
      Reduces an imaging polarimetry observation, in which the
      waveplate angle iterates at each jitter position.
   \item [\htmlref{POL\_EXTENDED}{POL\_EXTENDED}]
      Reduces an imaging polarimetry observation of an extended
      source.
   \item [\htmlref{POL\_JITTER}{POL\_JITTER}]
      Reduces an imaging polarimetry observation, in which the
      spatial is position jittered before moving the waveplate
      angle.
   \item [\htmlref{SKY\_FLAT\_FP}{SKY\_FLAT\_FP}]
      Creates and files a Fabry-Perot flat field derived from eight
      frames, using object masking.
   \item [\htmlref{SKY\_FLAT\_POL}{SKY\_FLAT\_POL}]
      Creates a polarimetry flat field derived from eight frames
      (two at each waveplate angle), using object masking.  It copies
      the flat for each waveplate angle and files them.
   \item [\htmlref{SKY\_FLAT\_POL\_ANGLE}{SKY\_FLAT\_POL\_ANGLE}]
      Creates and files polarimetry flat fields derived from jittered
      frames at each waveplate angle, using object masking.
\end{description}

\subsection{Modified recipes}
Many former recipes had numerous variants for different jitter sizes.
These have largely been superseded by generic equivalents.  Only the
EXTENDED recipes await conversion.  The families of recipes changed
are listed below.

\begin{itemize}
    \item \htmlref{JITTER\_SELF\_FLAT}{JITTER\_SELF\_FLAT} 
          (Six recipes in the family)  There were three-,
          five-, seven-, and nine-point versions, but most were only
          available for one or two recipe variants.
    \item \htmlref{MOVING\_JITTER\_SELF\_FLAT}{MOVING\_JITTER\_SELF\_FLAT} 
          (Two recipes)  These were limited to nine-point jitters.
    \item \htmlref{NOD\_SELF\_FLAT\_NO\_MASK}{NOD\_SELF\_FLAT\_NO\_MASK}
          (Four recipes) There were limited to fixed sizes of four- and
          eight-point jitters.
    \item \htmlref{SKY\_AND\_JITTER}{SKY\_AND\_JITTER} (Two recipes)
          Only five-point jitters were available.
\end{itemize}

In addition certain recipes had a fixed jitter size, but no longer.
These families are as follows.

\begin{itemize}
    \item \htmlref{BRIGHT\_POINT\_SOURCE}{BRIGHT\_POINT\_SOURCE} (Two recipes)
       These were formerly restricted to five-point jitters.
\end{itemize}

\subsection{Global changes}

The main changes from a user perspective were as follows.

\begin{itemize}
   \item Recipes and primitives are instrument generic.
   \item \htmlref{Improved registration}{automatic_registration} allowing
      mixed solutions.  There is a new offset type---beam
      separa\-tions---for combining polarimetry mosaics.
   \item \htmlref{Thresholding}{bad_pixels} of dark-subtracted data
      to prevent bizarre values affecting the pipeline processing.
      In addition \htmlref{JITTER\_SELF\_FLAT}{JITTER\_SELF\_FLAT} and
      \htmlref{SKY\_FLAT}{SKY\_FLAT} recipes and their variants which
      use object masking to make a flat, now have the deviant pixels of
      the initial flat-field frame reflagged as bad.
   \item Editing of the FITS headers to create a world co-ordinate
      system which \GAIA\ and \KAPPA\ recognise.  Also pre-ORAC IRCAM
      data have their headers structured and comments edited to bring
      them closer towards the \htmladdnormallink{UKIRT FITS
      standard}{http://www.jach.hawaii.edu/JACpublic/UKIRT/software/orac/docs/orac016-fith.fm.A4.ps},
      and to make the headers easier to read and comprehend.
   \item The names of flats have changed.  The filter name is included
      for easy identification.  Certain characters have special
      meaning to HDS, therefore {\tt []\{\}} are removed and a decimal
      point becomes a {\tt p} in the flat's name.  The {\tt{\_cycle}}
      suffix is shortened to {\tt{\_c}}, and all {\tt{\_subgrp}}
      strings removed from the group number.  Flats are are not
      combined over multiple cycles over recipes.  SKY\_FLAT and
      SKY\_FLAT\_MASKED are limited to jitters between three to five
      points for compatibility with other ORAC tools.  It is still
      possible to make private variants of these recipes in which the
      number of jitter positions is set by the NUMBER steering
      parameter.
   \item Mosaics are combined using the mean at each pixel.  This was
      formerly the median.  The change was made to correct the
      photometry.  Poor registration from telescope offsets due to
      sparse fields leads to multiple peaks in the mosaic's grid, and
      given the steep point-spread function's profile, the median
      preferentially selects pixels not at a peak.  This resulted in a
      typical underestimate of the flux of standard stars by 1--3\%.
      The \_MAKE\_MOSAIC\_ primitive now has an argument to select
      various estimators should you prefer not to use the mean.
   \item For the aperture photometry the default sky annulus radii
      have been increased.  This reduces the error estimating the sky
      level, both statistically and from the extend low-level pedestal
      in the point-spread function.  The area is increased 2.9$\times$
      for IRCAM and 5.6$\times$ for UFTI. 
   \item To counteract the spike artifact in the histogram of sky values
      of flat-fielded frames where the pre-flattened frame itself
      contributed to the flat (`self-flat' recipes), the mode is
      now calculated using multiple standard-deviation clipping for
      \htmlref{JITTER\_SELF\_FLAT\_APHOT}{JITTER\_SELF\_FLAT\_APHOT} and
      \htmlref{NOD\_SELF\_FLAT\_NO\_MASK\_APHOT}{NOD\_SELF\_FLAT\_NO\_MASK\_APHOT}.
      The previous estimation was from iterative application of
      Chauvenet's criterion and using the \mbox{3 $\lsk$ median $-$ 2 $\lsk$ mean}
      formula, and lead to a underestimate of the sky level.  For
      typical standard stars this systematic error led to a
      brightening of 1--2\%.  The latter method is still used for
      photometric recipes which do not self flat, such as
      \htmlref{BRIGHT\_POINT\_SOURCE\_APHOT}{BRIGHT\_POINT\_SOURCE\_APHOT}.
      If you create and apply a `superflat', the artifact is much
      reduced and therefore, the former estimator is appropriate.
   \item More and better processing-status messages.  For example, all
      floating-point numbers are now reported with a sensible number of
      decimal places, and the names of calibration frames are reported.
   \item Addition of FWHM to the aperture photometry results and small
      text list.  The file name column is 3 characters wider to
      accommodate the positive and negative suffices of recipe
      \htmlref{NOD\_SELF\_FLAT\_NO\_MASK\_APHOT}{NOD\_SELF\_FLAT\_NO\_MASK\_APHOT}.
   \item The saturation level in the aperture photometry was a constant.
      Now it is set to the appropriate value by instrument and mode.
   \item It is possible to use versions of \KAPPA\  other than the
     latest.  The changed argument lists in various tasks are adjusted
     for the \KAPPA\  version.
   \item More intermediate files, mostly the text files, are removed.
   \item Pipeline activates two more application engines: \POLPACK\ and 
      {\footnotesize CATSELECT}.
   \item A new UFTI bad-pixel mask.
   \item Added waveplate angle to flat rules file ({\tt\$ORAC\_DATA\_CAL/rules.flat}).
        The angle defaults to zero if it does not have a value in the FITS
        headers.
   \item Bug fixes and documentation improvements, especially links in
      the Perl POD.
\end{itemize}

% ? End of main text
\end{document}

%%% Local Variables: 
%%% mode: latex
%%% TeX-master: t
%%% End: 
