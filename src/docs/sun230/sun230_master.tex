\documentclass[twoside,11pt]{article}

% +
%  Name:
%     sun230.tex

%  Purpose:
%     SUN documentation for ORAC-DR overview (SUN/230)

%  Authors:
%     Frossie Economou (JACH)
%     Tim Jenness (JACH)

%  Copyright:
%     Copyright (C) 1997-2003 Particle Physics and Astronomy
%     Research Council. All Rights Reserved.

%  Notes:
%     The final SUN is generated automatically from POD. Once the
%     latex is generated the following changes are required:
%     - Replace ORAC-DR with \oracdr\
%     - Remove any junk from the start and end of each pod section.
%       Usually involves removing everything up to and including the
%       DESCRIPTION section declaration, leaving the description in place
%       as a section introduction. Remove COPYRIGHT, AUTHOR and REVISION
%       sections.
%     - Remove % from end of pod2latex generated \item commands (and
%       \index commands) since they are not needed
%       and they ruin the star2html conversion.
%     - Usually replace 'this document' with 'this section'
%     - Add xlabels to all sections/subsections
%     - References to CGS4DR should be replaced with \cgsdr\
%     - References to GAIA should be replaced with \gaia\
%     - References to GWM should be replaced with \gwm\
%     - References to Kappa should be replaced with \Kappa\
%     - References to KAPVIEW should be replaced with \kapview\
%     - References to EXTRACTOR should be replaced with \extractor\
%     - Replace P4 with \textsc{p4}
%     - Replace (C) with \copyright
%     - Replace SUN/231 and SUN/232 references with real xrefs

%  History:
%     $Log$
%     Revision 1.10  2004/05/27 20:10:09  bradc
%     updates for ORAC-DR v4.1-0
%
%     Revision 1.9  2003/06/13 03:58:01  timj
%     Update for V4.0 release.
%
%     Revision 1.8  2002/09/16 04:29:42  timj
%     - fix up header, abstract and release notes for V3.1-3
%
%     Revision 1.7  2001/11/19 04:11:16  timj
%     Add release notes
%
%     Revision 1.6  2001/11/17 01:28:43  timj
%     Prep version 3.0 starlink release
%
%     Revision 1.5  2001/04/11 03:52:26  timj
%     Add stardoccopyright
%
%     Revision 1.4  2001/04/10 21:52:19  timj
%     - update doc number
%     - Add Alasdair Allan
%     - Add CGS4
%
%     Revision 1.3  2000/02/09 20:19:41  timj
%     Correct stardocnumber to .1 rather than .0
%
%     Revision 1.2  2000/02/08 04:41:10  timj
%     Fixes to documentation required by Starlink for release of V1.0-0
%
%     Revision 1.1.1.1  2000/02/03 09:44:59  timj
%     first starlink docs
%

%  Revision:
%     $Id$

% -

% ? Specify used packages
\usepackage{graphicx}        %  Use this one for final production.
\usepackage{times}
% \usepackage[draft]{graphicx} %  Use this one for drafting.
% ? End of specify used packages

\pagestyle{myheadings}

% -----------------------------------------------------------------------------
% ? Document identification
% Fixed part
\newcommand{\stardoccategory}  {Starlink User Note}
\newcommand{\stardocinitials}  {SUN}
\newcommand{\stardocsource}    {sun\stardocnumber}

% Variable part - replace [xxx] as appropriate.
\newcommand{\stardocnumber}    {230.6}
\newcommand{\stardocauthors}   {Frossie Economou, Tim Jenness,\\
Malcolm Currie, Andy Adamson, Alasdair Allan, Brad Cavanagh\\
Joint Astronomy Centre, Hilo, Hawaii}
\newcommand{\stardoccopyright} {Copyright \copyright\ 1997-2004 Particle Physics and Astronomy Research Council}
\newcommand{\stardocdate}      {June 2004}
\newcommand{\stardoctitle}     {ORAC-DR: Overview and General Introduction}
\newcommand{\stardocversion}   {4.1-0}
\newcommand{\stardocmanual}    {}


\newcommand{\stardocabstract} {\textsc{orac-dr} is a general purpose automatic
  data reduction pipeline environment. It currently supports data reduction
  for the United Kingdom Infrared Telescope (UKIRT) instruments UFTI, IRCAM,
  UIST  and CGS4, for the James Clerk Maxwell Telescope (JCMT) instrument
  SCUBA, for the William Herschel Telescope (WHT) instrument INGRID, for the
  European Southern Observatory (ESO) instrument ISAAC and
  for the Anglo-Australian Telescope (AAT) instrument IRIS-2. This document
  describes the general pipeline environment. For specific information on how
  to reduce the data for a particular instrument, please consult the
  appropriate \textsc{orac-dr} instrument guide.}

% ? End of document identification
% -----------------------------------------------------------------------------

% +
%  Name:
%     sun.tex
%
%  Purpose:
%     Template for Starlink User Note (SUN) documents.
%     Refer to SUN/199
%
%  Authors:
%     AJC: A.J.Chipperfield (Starlink, RAL)
%     BLY: M.J.Bly (Starlink, RAL)
%     PWD: Peter W. Draper (Starlink, Durham University)
%
%  History:
%     17-JAN-1996 (AJC):
%        Original with hypertext macros, based on MDL plain originals.
%     16-JUN-1997 (BLY):
%        Adapted for LaTeX2e.
%        Added picture commands.
%     13-AUG-1998 (PWD):
%        Converted for use with LaTeX2HTML version 98.2 and
%        Star2HTML version 1.3.
%     {Add further history here}
%
% -

\newcommand{\stardocname}{\stardocinitials /\stardocnumber}
\markboth{\stardocname}{\stardocname}
\setlength{\textwidth}{160mm}
\setlength{\textheight}{230mm}
\setlength{\topmargin}{-2mm}
\setlength{\oddsidemargin}{0mm}
\setlength{\evensidemargin}{0mm}
\setlength{\parindent}{0mm}
\setlength{\parskip}{\medskipamount}
\setlength{\unitlength}{1mm}

% -----------------------------------------------------------------------------
%  Hypertext definitions.
%  ======================
%  These are used by the LaTeX2HTML translator in conjunction with star2html.

%  Comment.sty: version 2.0, 19 June 1992
%  Selectively in/exclude pieces of text.
%
%  Author
%    Victor Eijkhout                                      <eijkhout@cs.utk.edu>
%    Department of Computer Science
%    University Tennessee at Knoxville
%    104 Ayres Hall
%    Knoxville, TN 37996
%    USA

%  Do not remove the %begin{latexonly} and %end{latexonly} lines (used by
%  LaTeX2HTML to signify text it shouldn't process).
%begin{latexonly}
\makeatletter
\def\makeinnocent#1{\catcode`#1=12 }
\def\csarg#1#2{\expandafter#1\csname#2\endcsname}

\def\ThrowAwayComment#1{\begingroup
    \def\CurrentComment{#1}%
    \let\do\makeinnocent \dospecials
    \makeinnocent\^^L% and whatever other special cases
    \endlinechar`\^^M \catcode`\^^M=12 \xComment}
{\catcode`\^^M=12 \endlinechar=-1 %
 \gdef\xComment#1^^M{\def\test{#1}
      \csarg\ifx{PlainEnd\CurrentComment Test}\test
          \let\html@next\endgroup
      \else \csarg\ifx{LaLaEnd\CurrentComment Test}\test
            \edef\html@next{\endgroup\noexpand\end{\CurrentComment}}
      \else \let\html@next\xComment
      \fi \fi \html@next}
}
\makeatother

\def\includecomment
 #1{\expandafter\def\csname#1\endcsname{}%
    \expandafter\def\csname end#1\endcsname{}}
\def\excludecomment
 #1{\expandafter\def\csname#1\endcsname{\ThrowAwayComment{#1}}%
    {\escapechar=-1\relax
     \csarg\xdef{PlainEnd#1Test}{\string\\end#1}%
     \csarg\xdef{LaLaEnd#1Test}{\string\\end\string\{#1\string\}}%
    }}

%  Define environments that ignore their contents.
\excludecomment{comment}
\excludecomment{rawhtml}
\excludecomment{htmlonly}

%  Hypertext commands etc. This is a condensed version of the html.sty
%  file supplied with LaTeX2HTML by: Nikos Drakos <nikos@cbl.leeds.ac.uk> &
%  Jelle van Zeijl <jvzeijl@isou17.estec.esa.nl>. The LaTeX2HTML documentation
%  should be consulted about all commands (and the environments defined above)
%  except \xref and \xlabel which are Starlink specific.

\newcommand{\htmladdnormallinkfoot}[2]{#1\footnote{#2}}
\newcommand{\htmladdnormallink}[2]{#1}
\newcommand{\htmladdimg}[1]{}
\newcommand{\hyperref}[4]{#2\ref{#4}#3}
\newcommand{\htmlref}[2]{#1}
\newcommand{\htmlimage}[1]{}
\newcommand{\htmladdtonavigation}[1]{}

\newenvironment{latexonly}{}{}
\newcommand{\latex}[1]{#1}
\newcommand{\html}[1]{}
\newcommand{\latexhtml}[2]{#1}
\newcommand{\HTMLcode}[2][]{}

%  Starlink cross-references and labels.
\newcommand{\xref}[3]{#1}
\newcommand{\xlabel}[1]{}

%  LaTeX2HTML symbol.
\newcommand{\latextohtml}{\LaTeX2\texttt{HTML}}

%  Define command to re-centre underscore for Latex and leave as normal
%  for HTML (severe problems with \_ in tabbing environments and \_\_
%  generally otherwise).
\renewcommand{\_}{\texttt{\symbol{95}}}

% -----------------------------------------------------------------------------
%  Debugging.
%  =========
%  Remove % on the following to debug links in the HTML version using Latex.

% \newcommand{\hotlink}[2]{\fbox{\begin{tabular}[t]{@{}c@{}}#1\\\hline{\footnotesize #2}\end{tabular}}}
% \renewcommand{\htmladdnormallinkfoot}[2]{\hotlink{#1}{#2}}
% \renewcommand{\htmladdnormallink}[2]{\hotlink{#1}{#2}}
% \renewcommand{\hyperref}[4]{\hotlink{#1}{\S\ref{#4}}}
% \renewcommand{\htmlref}[2]{\hotlink{#1}{\S\ref{#2}}}
% \renewcommand{\xref}[3]{\hotlink{#1}{#2 -- #3}}
%end{latexonly}
% -----------------------------------------------------------------------------
% ? Document specific \newcommand or \newenvironment commands.

\def\C++{{\rm C\kern-.05em\raise.3ex\hbox{\footnotesize ++}}}
\newcommand{\underscore}{\_}
\newcommand{\oracdr}{\textsc{orac-dr}}

% For HTML redefine hfil since latex2html does not understand it
\html{\renewcommand{\hfil}{ }}

\newcommand{\recipe}[1]{{\small\textsf{#1}}}
\newcommand{\primitive}[1]{{\small\texttt{#1}}}

\newcommand{\Kappa}{\xref{{\textsc{Kappa}}}{sun95}{}}
\newcommand{\kapview}{\textsc{kapview}}
\newcommand{\gaia}{\xref{{\textsc{Gaia}}}{sun214}{}}
\newcommand{\cgsdr}{\xref{{\textsc{cgs4dr}}}{sun27}{}}
\newcommand{\gwm}{\xref{\textsc{gwm}}{sun219}{}}
\newcommand{\extractor}{\xref{\textsc{extractor}}{sun226}{}}

% ? End of document specific commands
% -----------------------------------------------------------------------------
%  Title Page.
%  ===========
\renewcommand{\thepage}{\roman{page}}
\begin{document}
\thispagestyle{empty}

%  Latex document header.
%  ======================
\begin{latexonly}
   CCLRC / \textsc{Rutherford Appleton Laboratory} \hfill \textbf{\stardocname}\\
   {\large Science and Technology Facilities Council}\\
   {\large Starlink Project\\}
   {\large \stardoccategory\ \stardocnumber}
   \begin{flushright}
   \stardocauthors\\
   \stardocdate
   \end{flushright}
   \vspace{-4mm}
   \rule{\textwidth}{0.5mm}
   \vspace{5mm}
   \begin{center}
   {\Huge\textbf{\stardoctitle \\ [2.5ex]}}
   {\LARGE\textbf{\stardocversion \\ [4ex]}}
   {\Huge\textbf{\stardocmanual}}
   \end{center}
   \vspace{5mm}

% ? Add picture here if required for the LaTeX version.
%   e.g. \includegraphics[scale=0.3]{filename.ps}
\begin{center}
\includegraphics[width=1.0in]{sun230_logo.eps}
\end{center}
% ? End of picture

% ? Heading for abstract if used.
   \vspace{10mm}
   \begin{center}
      {\Large\textbf{Abstract}}
   \end{center}
% ? End of heading for abstract.
\end{latexonly}

%  HTML documentation header.
%  ==========================
\begin{htmlonly}
   \xlabel{}
   \begin{rawhtml} <H1> \end{rawhtml}
      \stardoctitle\\
      \stardocversion\\
      \stardocmanual
   \begin{rawhtml} </H1> <HR> \end{rawhtml}

% ? Add picture here if required for the hypertext version.
%   e.g. \includegraphics[scale=0.7]{filename.ps}
\includegraphics[width=1.0in]{sun230_logo.eps}
% ? End of picture

   \begin{rawhtml} <P> <I> \end{rawhtml}
   \stardoccategory\ \stardocnumber \\
   \stardocauthors \\
   \stardocdate
   \begin{rawhtml} </I> </P> <H3> \end{rawhtml}
      \htmladdnormallink{CCLRC / Rutherford Appleton Laboratory}
                        {http://www.cclrc.ac.uk} \\
      \htmladdnormallink{Science and Technology Facilities Council}
                        {http://www.stfc.ac.uk} \\
   \begin{rawhtml} </H3> <H2> \end{rawhtml}
      \htmladdnormallink{Starlink Project}{http://www.starlink.ac.uk/}
   \begin{rawhtml} </H2> \end{rawhtml}
   \htmladdnormallink{\htmladdimg{source.gif} Retrieve hardcopy}
      {http://www.starlink.ac.uk/cgi-bin/hcserver?\stardocsource}\\

%  HTML document table of contents.
%  ================================
%  Add table of contents header and a navigation button to return to this
%  point in the document (this should always go before the abstract \section).
  \label{stardoccontents}
  \begin{rawhtml}
    <HR>
    <H2>Contents</H2>
  \end{rawhtml}
  \htmladdtonavigation{\htmlref{\htmladdimg{contents_motif.gif}}
        {stardoccontents}}

% ? New section for abstract if used.
  \section{\xlabel{abstract}Abstract}
% ? End of new section for abstract
\end{htmlonly}

% -----------------------------------------------------------------------------
% ? Document Abstract. (if used)
%  ==================
\stardocabstract
% ? End of document abstract

% -----------------------------------------------------------------------------
% ? LateX Copyright Statement
%  =========================
\begin{latexonly}
\newpage
\vspace*{\fill}
\stardoccopyright
\end{latexonly}
% ? End of Latex copyright statement

% -----------------------------------------------------------------------------
% ? Latex document Table of Contents (if used).
%  ===========================================
  \newpage
  \begin{latexonly}
    \setlength{\parskip}{0mm}
    \tableofcontents
    \setlength{\parskip}{\medskipamount}
    \markboth{\stardocname}{\stardocname}
  \end{latexonly}
% ? End of Latex document table of contents
% -----------------------------------------------------------------------------
\cleardoublepage
\renewcommand{\thepage}{\arabic{page}}
\setcounter{page}{1}

% ? Main text

%% ORACDRDOC_HOWTO:Introduction

%% ORACDRDOC_HOWTO:Jargon

%% ORACDRDOC_HOWTO:SettingUp

%% ORACDRDOC_HOWTO:Components

%% ORACDRDOC_HOWTO:Xoracdr

%% ORACDRDOC_BIN:oracdr

%% ORACDRDOC_HOWTO:Credits

\section{Release Notes}

\begin{description}

\item[V4.0]

\begin{itemize}

\item Support for UIST in all observation modes.

\item Support for INGRID in all observation modes.

\item Support for ISAAC in imaging mode, and preliminary support for
spectroscopy mode.

\item New document, \xref{SUN/246}{sun246}{}, describing integral field
  spectroscopy reduction and recipes.

\item New \texttt{ORAC\_KEEP} environment variable to retain intermediate
  frames.

\item Spectroscopy:

\begin{itemize}

\item Widened optimal extraction windows for better profile fitting.

\item Flux calibration for I-band.

\end{itemize}

\item Imaging:

\begin{itemize}

\item Modification of EXTRACTOR object-detection parameters to obtain a
    flatter, more accurate flat-fielded mosaic.

\item Offset patterns need not be centered at centre of the array.

\item Four new recipes including NOD\_SKY\_FLAT\_THERMAL recipe for reduction
    of thermal data using sky observations for flat-fielding.

\item REDUCE\_DARK supports variance creation and propagation by default.

\item Expanded \xref{SUN/232}{sun232}{} with more description of the
  primitives, and information for programmers wishing to adapt the recipes.

\end{itemize}

\item SCUBA:

\begin{itemize}

\item CSO Tau fits up-to-date to January 2003 (when the tau meter broke).

\item Flux Conversion Factors verified up to March 2003.

\item More robust error handling for poor data.

\end{itemize}

\end{itemize}

\item[V3.1]

\begin{itemize}

\item Support for AAT IRIS2 data

\item Spectroscopy:

\begin{itemize}

\item Extracts "sky-arcs" to enable wavelength calibration of Michelle data.

\item Now handles offset patterns that don't originate at (0,0).

\item Peak-up routines for Michelle.

\item Single beam polarimetry now much more robust.

\item Masking of off-slit areas of image improved.

\item Better bad pixel detection in flat fields.

\end{itemize}

\item Imaging:

\begin{itemize}

\item Addition of NOD\_CHOP\_FAINT (faint mid-IR) and
    NOD\_CHOP\_SCAN (scan pattern mid-IR) recipes.

\item Addition of ADDWCS (adds WCS to headers) recipe.

\end{itemize}

\end{itemize}

\item[V3.0]

\begin{itemize}

\item Support for Michelle data.

\item Support for multi-mode instruments, such as Michelle and UIST.

\item Three Michelle recipes for nodded and chopped data (vanilla,
  photometry and moving target).  New NQ standards file.

\item Easy to switch on variance creation and propagation; calculates
  correct data variance for UKIRT IR imagers.

\item  Faster object masking using EXTRACTOR instead of PISA.

\item Better registration of sparse fields using astrometry.

\item Tidier output for easier reading.  Added content to messages.

\item Comments in calibration rules files.

\item SCUBA: improvements in calibration of SCUBA data (both for flux
    conversion factors and extinction correction using
     \texttt{--calib tausys=csofit}).

\item Use of internal headers and directory reorganisation, permitting
     generic-named recipes and primitives, and optimising code use;
     and easier to add new instruments.

\end{itemize}

\item[V2.1]

\begin{itemize}

\item New GUI (xoracdr) to simplify use of the pipeline.

\item Enhanced CGS4 and imaging recipes.

\end{itemize}

\item[V2.0]

\begin{itemize}

\item Support for CGS4

\item SCUBA: Jiggle map calibration

\item IR Imaging: Polarimetry support plus recipe generalization.

\end{itemize}

\item[V1.0]

\begin{itemize}

\item First release to Starlink. Includes SCUBA and imaging (UFTI FITS
and IRCAM) recipes.

\end{itemize}

\end{description}


\appendix


%% ORACDRDOC_HOWTO:DataLoops

%% ORACDRDOC_HOWTO:DisplaySystem

%% ORACDRDOC_HOWTO:Calibrating

%% ORACDRDOC_HOWTO:ShellVariables

%% ORACDRDOC_BIN:oracdisp

%% ORACDRDOC_BIN:oracdr_nuke



% ? End of main text
\end{document}
