\newpage
\sstroutine{
   \_ADD\_AUTO\_ASTROMETRY\_
}{
   Performs automated astrometric corrections
}{
   \sstdescription{
      This primitive automatically corrects astrometry for a given
      observation. It does so by downloading a catalogue from a source
      (typically 2MASS), detects objects in the observation, correlates
      between the two catalogues, and calculates an astrometric
      solution.
   }
   \sstnotes{
      \sstitemlist{

         \sstitem
         This primitive is suitable for infrared imaging instruments.

         \sstitem
         Processing only occurs when the steering header ADD\_ASTROMETRY
         is true.

         \sstitem
         Processing occurs on the current Group object.

         \sstitem
         Astrometric correction will probably fail if fewer than five
         objects are detected in the frame. Should this occur, the original
         WCS will be retained.

         \sstitem
         Should an astrometric solution be found, the WCS in the current
         Group will be overwritten with the solution.
      }
   }
   \sstdiytopic{
      Required Perl Modules
   }{
      Starlink::Autoastrom.
   }
}
\newpage
\sstroutine{
   \_ADD\_PIPELINE\_VERSION\_
}{
   Adds the current pipeline version to the file
}{
   \sstdescription{
      This primitive adds the current pipeline version to the file
   }
   \sstnotes{
      \sstitemlist{

         \sstitem
            This primitive is suitable for LCOGT optical imaging
            instruments.

         \sstitem
            This primitive needs to have SVN version properties set to
            update properly
      }
   }
   \sstdiytopic{
      Tasks
   }{
      KAPPA
         FITSMOD
   }
}
\newpage
\sstroutine{
   \_AVERAGE\_BIASES\_
}{
   Averages the current frame into the current bias
}{
   \sstdescription{
      This primitive averages or copies the current frame into the
      current bias frame. Copying occurs if the bias is new. Both steps
      reported.
   }
   \sstnotes{
      \sstitemlist{

         \sstitem
         Only applies to bias frames as specified by user header
         ORAC\_OBSERVATION\_TYPE.

         \sstitem
         The bias frame is a copy of the supplied frame, if it is the
         first (and probably only) contributing frae. Otherwise the new
         frame is averaged into the bias using (bias(n-1)$+$frame)/n, where n
         is the number of contributing frames.

         \sstitem
         The number of contributing frames comes from a hash stored in
         user header BIAS\_FRAMES with keys equal to the bias name given by
         primitive \_GET\_BIAS\_NAME\_.
      }
   }
   \sstdiytopic{
      Output Data
   }{
      Frame bias\_$<$group\_number$>$.
   }
   \sstdiytopic{
      Tasks
   }{
      KAPPA: MATHS, NDFCOPY.
   }
}
\newpage
\sstroutine{
   \_CALCULATE\_IMAGE\_STATISTICS\_
}{
   Calculate various image quality statistics based on an input
   catalogue
}{
   \sstdescription{
      This primitive calculates image quality statistics based on an
      input catalogue. This catalogue is typically the output from
      either Starlink::Extractor or Starlink::Autoastrom, so that object
      morphology information is available for calculations.

      This primitive calculates the mean axial ratio of bright sources,
      the average object diameter converted to K-band equivalent, the
      sky brightness, the limiting magnitude, and the instrumental zero
      point.
   }
   \sstnotes{
      \sstitemlist{

         \sstitem
         Files results with the Calibration system. FWHM is obtained
         with the fwhm() method, axial ratio is obtained with the
         axial\_ratio() method, sky brightness is obtained with the
         sky\_brightness() method, limiting magnitude is obtained with the
         limiting\_mag() method, and the zero point is obtained with the
         zeropoint() method.
      }
   }
   \sstdiytopic{
      Output Data
   }{
      None.
   }
   \sstdiytopic{
      Required Perl Modules
   }{
      Astro::Catalog.
   }
}
\newpage
\sstroutine{
   \_CALCULATE\_SEEING\_STATS\_
}{
   Extract objects and determine the average FWHM, ellipticity, and
   position angle across a field
}{
   \sstdescription{
      For the group file, this primitive finds good sources in the field
      and calculates the average FWHM, ellipticity, and position angle.
      It then displays these averages along with errors.
   }
   \sstnotes{
      \sstitemlist{

         \sstitem
         Currently uses SEXTRACTOR for source extraction.

         \sstitem
         The SEXTRACTOR configuration file is found in
         \$ORAC\_DATA\_CAL/extractor\_seeing\_stats.sex.

         \sstitem
         This primitive runs only when the CALCULATE\_SEEING\_STATS
         internal header is true.

         \sstitem
         This primitive only operates on the current Group file.
      }
   }
}
\newpage
\sstroutine{
   \_CALCULATE\_ZEROPOINT\_
}{
   Calculate various image quality statistics based on an input
   catalogue
}{
   \sstdescription{
      This primitive calculates image quality statistics based on an
      input catalogue. This catalogue is typically the output from
      either Starlink::Extractor or Starlink::Autoastrom, so that object
      morphology information is available for calculations.

      This primitive calculates the mean axial ratio of bright sources,
      the average object diameter converted to K-band equivalent, the
      sky brightness, the limiting magnitude, and the instrumental zero
      point.
   }
   \sstnotes{
      \sstitemlist{

         \sstitem
         Files results with the Calibration system. FWHM is obtained
         with the fwhm() method, axial ratio is obtained with the
         axial\_ratio() method, sky brightness is obtained with the
         sky\_brightness() method, limiting magnitude is obtained with the
         limiting\_mag() method, and the zero point is obtained with the
         zeropoint() method.
      }
   }
   \sstdiytopic{
      Output Data
   }{
      None.
   }
   \sstdiytopic{
      Required Perl Modules
   }{
      Astro::Catalog.
   }
}
\newpage
\sstroutine{
   \_CHECK\_PROPID\_
}{
   Checks whether the proposal id in the frame header is known to the
      science archive
}{
   \sstdescription{
      The primitive checks whether the proposal id in the frame header
      is known to the IPAC Science Archive or the LCOGT Proposal DB,
      depending on the setting of \$test\_proposaldb.
   }
   \sstnotes{
      \sstitemlist{

         \sstitem
            This primitive is suitable for imaging cameras.

         \sstitem
            Processing only occurs for object frames.

         \sstitem
            Email sending requires Mail::Sendmail to be installed

         \sstitem
            The files to be checked are \$ORAC\_CAL\_ROOT/lcogt\_propid.txt or
            \$ORAC\_CAL\_ROOT/proposal\_list depening on whether the IPAC
            Science Archive or the LCOGT Proposal DB is being checked.
      }
   }
   \sstdiytopic{
      Required Perl Modules
   }{
      Mail::Sendmail
   }
}
\newpage
\sstroutine{
   \_CLIPPED\_STATS\_MEDIAN\_
}{
   Finds the clipped mean, median and standard deviation of a frame
}{
   \sstdescription{
      Use progressive sigma-clipping to find a representative mean,
      median and standard deviation of a frame. The default clipping
      thresholds give a reasonable approximation to the mode.
   }
   \sstnotes{
      \sstitemlist{

         \sstitem
         This primitive is suitable for all instruments.
      }
   }
   \sstdiytopic{
      Tasks
   }{
      KAPPA: STATS.
   }
}
\newpage
\sstroutine{
   CONVERT\_TO\_FITS
}{
   Converts current observation to FITS
}{
   \sstdescription{
      Converts current observation to FITS
   }
}
\newpage
\sstroutine{
   \_CREATE\_GRAPHIC\_FROM\_FILE\_
}{
   Create a PNG, GIF, or JPG graphic from a given file
}{
   \sstdescription{
      This primitive creates a PNG, GIF, or JPG graphic from the
      supplied file. It currently only supports 1-D and 2-D files.
   }
   \sstnotes{
      \sstitemlist{

         \sstitem
         This primitive is suitable only for 1-D and 2-D input files.

         \sstitem
         The graphic file will have the same root filename as the input
         file, with the appropriate file extension.
      }
   }
}
\newpage
\sstroutine{
   \_CREATE\_IMAGE\_CATALOGUE\_\_
}{
   source extraction and photometry on all sources
}{
   \sstdescription{
      For the group file, find all the sources and calculate the flux of
      each detected source. Write the results to a catalogue file.
   }
   \sstnotes{
      \sstitemlist{

         \sstitem
         Currently uses SEXTRACTOR for source extraction and for
         photometry.
      }
   }
   \sstdiytopic{
      Required Perl Modules
   }{
      Starlink::Extractor, Astro::WaveBand, Astro::Catalog.
   }
}
\newpage
\sstroutine{
   \_CREATE\_RAW\_FRAME\_
}{
   Creates a raw frame in ORAC\_DATA\_OUT
}{
   \sstdescription{
      Null primitive for LCOSBIG.
   }
   \sstnotes{
      \sstitemlist{

         \sstitem
         This primitive is suitable for LCOSBIG.
      }
   }
}
\newpage
\newpage
\sstroutine{
   \_DATA\_QC\_TEST\_
}{
   Evaluates the image data quality control flag and sets the
      relevant bitmask keywords in the data catalogue product
}{
   \sstdescription{
      Evaluates the image data quality control flag and sets the
      relevant bitmask keywords in the data catalogue product.
   }
   \sstnotes{
      \sstitemlist{

         \sstitem
            This primitive is suitable for imaging cameras.

         \sstitem
            Processing only occurs for object frames.
      }
   }
   \sstdiytopic{
      Output Data
   }{
      The results of the comparison are returned in
      \$\_DATA\_QC\_TEST\_\{QCPARAM\} and will contain either {\tt '}T{\tt '} if the
      comparison passed, {\tt '}F{\tt '} if the comparison failed or {\tt '}U{\tt '} if the
      comparison couldn{\tt '}t be made.
   }
   \sstdiytopic{
      Output Files
   }{
      None.
   }
}
\newpage
\sstroutine{
   \_DELETE\_RAW\_FRAME\_
}{
   Remove the raw frame file
}{
   \sstdescription{
      Generic primitive to remove the actual raw frame file.
   }
   \sstnotes{
      In rare cases you do not care about the Frame data product (maybe
      because the group product is the only product of interest) and do
      not want it to remain on disk. This primitive will erase the
      current frame file.
   }
}
\newpage
\sstroutine{
   \_DERIVED\_PRODUCTS\_STEER\_
}{
   Steers processing for DERIVED\_PRODUCTS recipe
}{
   \sstdescription{
      This primitive control processing for DERIVED\_PRODUCTS recipe
      through steering headers listed below.
   }
   \sstnotes{
      \sstitemlist{

         \sstitem
         This primitive is suitable for LCOGT imaging CCD cameras.

         \sstitem
         Processing only occurs for object frames.
      }
   }
   \sstdiytopic{
      Steering Headers
   }{
      ADD\_ASTROMETRY = LOGICAL
         Whether or not automated astrometry is to occur. This equates
         to argument DOASTROM.
      CREATE\_CATALOGUE = LOGICAL
         Whether or not catalogue creation is to occur. This equates to
         argument DOIMGCAT.
      CALCULATE\_SEEING\_STATS = LOGICAL
         Whether or not seeing statistics should be calculated. This
         equates to argument DOSEEING.
      QUICK\_LOOK = LOGICAL
         Whether or not to perform quicklook processing. This equates to
         argument DOQUICKLOOK.
   }
}
\newpage
\sstroutine{
   \_DERIVED\_PRODUCTS\_
}{
   Create derived products from the processed frames
}{
   \sstdescription{
      This primitive creates the derived products from the processed
      frames. The following items are performed:

      \sstitemlist{

         \sstitem
         headers are updated with the details of the master calibration
         frames and correction steps used,

         \sstitem
         the saturation and trimmed section are updated with the values
         used in the pipeline processing,

         \sstitem
         A clipped mean, median and sigma are calculated and added to
         the header,

         \sstitem
         the QC modules is run to produce the four QC flags in the
         headers,

         \sstitem
         incorrect values of the TAGID and/or PROPID are flagged and
         corrected and the PROPID is checked against the lists of known
         proposals,

         \sstitem
         access rights and state of the data are set,

         \sstitem
         a 512x512 PNG thumbnail of the processed image is created,

         \sstitem
         the frame is converted to FITS format

         \sstitem
         the pytimecorrect.py code is called to compute per-star
         barycentric time correction and airmass and updates the header of
         the BCD image and source catalog,

         \sstitem
         if a final destination is defined by the enviroment variable
         FINAL\_DATA\_OUT, the derived products are moved there.
      }
   }
   \sstnotes{
      \sstitemlist{

         \sstitem
            This primitive is suitable for imaging cameras.

         \sstitem
            Processing only occurs for object frames.

         \sstitem
            Various incorrect/invalid values of the PROPID and TAGID are
            corrected and overwritten.

         \sstitem
            Data is normally set to private with a publication date of 1
            year from the frames{\tt '} date. Data with a PROPID beginning with
            {\tt "}ENG{\tt "} or Engineering is set to private indefinitely. EPO with a
            PROPID or TAGID beginning with LCOEPO$|$FTPEPO$|$HAWEPO is set to
            public immediate.
      }
   }
   \sstdiytopic{
      External Tasks
   }{
      The following external tasks are used:
      KAPPA
         FITSMOD
      ORAC-DR PRIMITIVES
         \_ADD\_PIPELINE\_VERSION\_, \_CHECK\_PROPID\_, \_CLIPPED\_STATS\_MEDIAN\_,
         \_CONVERT\_TO\_FITS\_, \_CREATE\_GRAPHIC\_FROM\_FILE\_,
         \_GET\_SATURATION\_LEVEL\_, \_SET\_FILE\_FITS\_ITEM\_, \_SET\_QC\_FLAGS\_
   }
   \sstdiytopic{
      Output Files
   }{
      Depending on whether the Frame uhdr entry QUICK\_LOOK is set to 1
      or not, the 00\_bp\_ff suffix of the processed SDF frame is replaced
      with a 10.fits or 90.fits when converting to the output FITS
      frame. The processed SDF frame is also converted to a PNG image
      with a 10.png or 90.png suffix, depending on whether the Frame
      uhdr entry QUICK\_LOOK.
   }
}
\newpage
\sstroutine{
   \_DIVIDE\_BY\_FLAT\_
}{
   Flat-fields a frame
}{
   \sstdescription{
      This primitive divides the current frame by the most-recent and
      matching flat-field frame from \$Cal-$>$flat method.
   }
   \sstnotes{
      \sstitemlist{

         \sstitem
         This primitive is suitable for UFTI, IRCAM, and Michelle in
         imaging mode.

         \sstitem
         Processing only occurs for object and sky frames.

         \sstitem
         The flat-fielded image can be displayed.

         \sstitem
         The frame title is propagated.
      }
   }
   \sstdiytopic{
      Output Data
   }{
      \sstitemlist{

         \sstitem
         Flat-fielded frame inheriting the frame{\tt '}s name but with the \_ff
         suffix.
      }
   }
   \sstdiytopic{
      Tasks
   }{
      CCDPACK: FLATCOR.
   }
}
\newpage
\sstroutine{
   \_FLAT\_QC\_
}{
   To perform Quality Control on flat fields produced by this
   pipeline
}{
   \sstdescription{
      This primitive looks for the separate master flat fields produces
      in each filter at both the start and the end of a single night of
      data. It performs the following tests:

      \sstitemlist{

         \sstitem
         If morning and evening twilight master flats were produced:

         \sstitem
            Compare the am/pm masters and check the RMS of the residuals is
            within bounds (implementation of J. Eastman{\tt '}s IDL algorithm).
      }
   }
   \sstnotes{
      This is an ORAC-DR/Perl implementation of J. Eastman{\tt '}s IDL
      algorithm.
   }
   \sstdiytopic{
      External Tasks
   }{
      KAPPA:
         STATS, SUB
      ORAC-DR PRIMITIVES
         \_DELETE\_A\_FRAME\_
   }
   \sstdiytopic{
      Output Data
   }{
      None.
   }
}
\newpage
\sstroutine{
   \_GET\_FILTER\_PARAMETERS\_
}{
   Returns LCOGT optical filter characteristics
}{
   \sstdescription{
      This primitive using a switch structure to return two
      characteristics or relating to the LCOGT optical imaging filters
      through arguments. Default values are returned if the filter is
      not recognised.
   }
   \sstnotes{
      \sstitemlist{

         \sstitem
         This primitive is suitable for LCOGT optical photometry.

         \sstitem
         The filter name comes from the user header ORAC\_FILTER.

         \sstitem
         The recognised filters have names ending with U, B, V, R, I,
         up, gp, rp, ip, zp, zs and Y.

         \sstitem
         The mean extinction coefficients are: U: 0.53, B: 0.27, V:
         0.14, R: 0.10, I: 0.05, u{\tt '}: 0.56, g{\tt '}: 0.20, r{\tt '}:0.11, i{\tt '}: 0.05, z{\tt '}:
         0.04 and Y: 0.03. Zero extinction applies to any other filter.
         Effective wavelengths are from Bessell (1998); extinction is from
         AJP{\tt '}s loc\_m1\_assm.c code as of 2012/05/08 for ELP; M1 inst.
         zeropoints are from AJP{\tt '}s loc\_m1\_assm.c code as of 2012/05/08 for
         1m0 kb72.
      }
   }
}
\newpage
\sstroutine{
   \_GET\_GAIN\_
}{
   Finds the LCO instrument gain in electrons per ADU for the current
   Frame
}{
   \sstdescription{
      This primitive obtains the gain in electrons per ADU for the
      current frame. It first attempts to find a value from the
      ORAC\_GAIN header. If this is null or less than 3.0, the primitive
      uses a time-dependent default value, and it reports the use of the
      default.

      The gain is returned through an argument.
   }
   \sstnotes{
      \sstitemlist{

         \sstitem
         This primitive is suitable for LCO optical instruments.
      }
   }
}
\newpage
\sstroutine{
   \_GET\_READNOISE\_
}{
   Finds the LCO instrument readnoise in electrons for the current
   Frame
}{
   \sstdescription{
      This primitive obtains the readnoise in electrons for the current
      frame. It first attempts to find a value for the calibration
      system. Where there is no value, it tries a header for the value,
      and if that{\tt '}s not defined, the primitive assigns a default.

      The readnoise is returned through an argument.
   }
   \sstnotes{
      \sstitemlist{

         \sstitem
         This primitive is suitable for LCO optical instruments.

         \sstitem
         The read noise comes from the readnoise calibration, or failing
         that the header RDNOISE.
      }
   }
}
\newpage
\sstroutine{
   \_GET\_SATURATION\_LEVEL\_
}{
   Finds the LCOSBIG saturation level in ADU the current Frame
}{
   \sstdescription{
      This primitive obtains the LCOSBIG saturation level in ADU for the
      current frame.

      The saturation level is returned through an argument.
   }
   \sstnotes{
      \sstitemlist{

         \sstitem
         This primitive is only suitable for LCOSBIG.

         \sstitem
         The values are estimated from fullwell and gain on the
         instrument{\tt '}s Web page, viz.
         http://www.sbig.com/large\_format/6303E\_specs.htm.
      }
   }
}
\newpage
\newpage
\newpage
\sstroutine{
   \_INSTRUMENT\_HELLO\_
}{
   Performs the LCOSBIG-specific imaging setup
}{
   \sstdescription{
      This primitive is performs the instrument specific setup for
      imaging. It{\tt '}s needed for the generic \_IMAGING\_HELLO\_. In this case
      it merely reports that the set-up operations are complete.
   }
   \sstnotes{
      \sstitemlist{

         \sstitem
         This primitive is suitable for LCOSBIG.
      }
   }
}
\newpage
\sstroutine{
   \_LCOGT\_STANDARD\_MAGNITUDE\_
}{
   Obtains the catalogue magnitude of a LCOGT standard
}{
   \sstdescription{
      This primitive reads the faint-standard catalogue or its
      predecessor. A case- and space-insensitive comparison of the
      supplied object name with the entries in the table provides a
      catalogue magnitude in U, B, V, R, or I for a standard star.
   }
   \sstnotes{
      \sstitemlist{

         \sstitem
         This primitive is suitable for LCOGT optical imagers.

         \sstitem
         Processing only occurs when it is time to perform photometry,
         i.e. when the steering header DO\_APHOT is true.

         \sstitem
         An error occurs when the filter is not one of U, B, V, R, or I.

         \sstitem
         The standard-star catalogue used is
         \$ORAC\_DATA\_CAL/landolt\_ubvri.dat. An error results when the
         catalogue cannot be opened.
      }
   }
}
\newpage
\sstroutine{
   \_MAKE\_BIAS\_FROM\_GROUP\_
}{
   Makes a masterbias from the current group of frames
}{
   \sstdescription{
      This primitive makes a master bias from the current group. The
      primitive files the resultant bias in its calibration index.
   }
   \sstnotes{
      \sstitemlist{

         \sstitem
         This primitive is suitable for optical imaging instruments.

         \sstitem
         Processing only occurs for bias frames, and when the steering
         header MAKE\_BIAS is true.

         \sstitem
         The bias is displayed.
      }
   }
   \sstdiytopic{
      Output Data
   }{
      The masterbias. It is called {\tt "}bias\_$<$instrument$>$\_$<$UT
      night$>$\_bin$<$Xbin$>$x$<$Ybin$>${\tt "} where $<$instrument$>$ is the LCOGT
      instrument identifier and $<$Xbin$>$, $<$Ybin$>$ are the binning factors
      in X and Y.
   }
   \sstdiytopic{
      Tasks
   }{
      CCDPACK: MAKEBIAS; KAPPA: FITSMOD, NDFCOPY.
   }
}
\newpage
\sstroutine{
   \_MAKE\_DARK\_FROM\_GROUP\_
}{
   Makes a masterdark from the current group of frames
}{
   \sstdescription{
      This primitive makes a master dark from the current group. The
      primitive files the resultant dark in its calibration index.
   }
   \sstnotes{
      \sstitemlist{

         \sstitem
         This primitive is suitable for optical imaging instruments.

         \sstitem
         Processing only occurs for dark frames, and when the steering
         header MAKE\_DARK is true.

         \sstitem
         The dark is displayed.
      }
   }
   \sstdiytopic{
      Output Data
   }{
      The masterdark. It is called {\tt "}dark\_$<$instrument$>$\_$<$UT
      night$>$\_bin$<$Xbin$>$x$<$Ybin$>${\tt "} where $<$instrument$>$ is the LCOGT
      instrument identifier and $<$Xbin$>$, $<$Ybin$>$ are the binning factors
      in X and Y.
   }
   \sstdiytopic{
      Tasks
   }{
      CCDPACK: MAKEDARK; KAPPA: FITSMOD, NDFCOPY.
   }
}
\newpage
\sstroutine{
   \_MAKE\_FLAT\_FROM\_GROUP\_
}{
   Makes flats from the current group of frames
}{
   \sstdescription{
      This primitive makes self flats from the current group, one for
      each distinct observation filter. For eacg flat it uses a median
      to combine the frames pixel by pixel, and then divides the
      resultant image by its mean form the flat field. The primitive
      files the resultant flat in its calibration index.
   }
   \sstnotes{
      \sstitemlist{

         \sstitem
         This primitive is suitable for infrared imaging instruments.

         \sstitem
         Processing only occurs for object, sky, or calibration-lamp
         frames, and when the steering header MAKE\_FLAT is true.

         \sstitem
         The list of filters present in the group is listed in an array
         stored by reference in the group user header FILTER\_LIST. If this
         is undefined, only a single flat is made for filter stored in the
         current Frame{\tt '}s user header ORAC\_FILTER.

         \sstitem
         There is special behaviour for a combined polarimetry flat (see
         {\tt "}OUTPUT DATA{\tt "}). The string {\tt "}pol{\tt "} in the filter name is used to
         indicate polarimetry data.

         \sstitem
         The flat is displayed.
      }
   }
   \sstdiytopic{
      Output Data
   }{
      The flat field. It is called is {\tt "}flat\_$<$filter$>$\_$<$groupnumber$>${\tt "} for
      the first cycle, and {\tt "}flat\_$<$filter$>$\_$<$groupnumber$>$\_c$<$cyclenumber$>${\tt "}
      for subsequent cycles, where $<$groupnumber$>$ is the frame number of
      the group, $<$filter$>$ is the filter name, and $<$cyclenumber$>$ is the
      cycle number derived from steering header CYCLE\_NUMBER.
      An exception is for polarimetric data, where the name becomes
      flat\_$<$filter$>$\_pol$<$waveplate\_angle$>$\_$<$groupnumber$>$. The
      $<$waveplate\_angle$>$ is the integer part of the angle, e.g. 22, 67,
      from internal header ORAC\_WAVEPLATE\_ANGLE. Subsequent cycles for
      polarimetry also have the {\tt "}\_c$<$cyclenumber$>${\tt "} suffix, but the cycle
      comes from steering header POL\_CYCLE\_NUMBER. When steering header
      WAVEPLATE\_FLAT is false (0), copies of the flat are made, one for
      each angle, using the above nomenclature. Each has its waveplate
      angle set to its nominal angle. This allows a single
      ORAC\_WAVEPLATE\_ANGLE rule entry irrespective of whether all
      waveplate angles were combined to make a flat or not.
   }
   \sstdiytopic{
      Tasks
   }{
      CCDPACK: MAKEFLAT; KAPPA: FITSMOD, NDFCOPY.
   }
}
\newpage
\sstroutine{
   \_NIGHT\_LOG\_
}{
   Produces a text file log of a night{\tt '}s imaging observations
}{
   \sstdescription{
      This recipe takes a night{\tt '}s imaging observations, and creates a
      text file containing a headed tabulation of parameters for each
      frame.

      The parameters are: observation number, group number, object name,
      observation type, UT start time, exposure time, number of coadds,
      read mode and speed, filter, start airmass, frame dimensions in
      pixels, base equatorial co-ordinates, and data-reduction recipe
      name.
   }
   \sstnotes{
      \sstitemlist{

         \sstitem
         The $<$date$>$ comes from the internal header keyword ORAC\_UTDATE.

         \sstitem
         The logfile created by this primitive does not follow the
         standard ORAC-DR naming convention (log.xxxx) since it can be used
         to write log files to directories other than \$ORAC\_DATA\_OUT and
         unique file names are required.

         \sstitem
         Fudges missing or old headers.

         \sstitem
         Uses user header ORAC\_INSTRUMENT to specify the file name.

         \sstitem
         Specification provided by Sandy Leggett.
      }
   }
   \sstdiytopic{
      Output Data
   }{
      \sstitemlist{

         \sstitem
         The text log file \$ORAC\_DATA\_IN/$<$date$<$.nightlog, where $<$date$>$
         is the UT date, unless the OUT argument is set, whereupon the log
         is in \$ORAC\_DATA\_OUT. This enables a separate on-the-fly log. For
         the multi-mode instruments UIST, Michelle, IRIS2, and ISAAC the
         file is \$ORAC\_DATA\_IN/$<$date$>$\_im.nightlog.
         For LCOGT data, we make use of the DAY-OBS header keyword to use
         as the observation date (and hence log root) to prevent writing
         two logs after we go over the UTC midnight boundary.
         The on-the-fly log in \$ORAC\_DATA\_OUT is always appended to, being
         created only if it doesn{\tt '}t exist. Thus multiple entries for the
         same observation may exist in the on-the-fly log if the pipeline
         is rerun.
         The {\tt "}clean{\tt "} log file in F\$$<$ORAC\_DATA\_IN$>$ is re-started if the
         observation number equals 1 and is appended to otherwise, being
         created as necessary.
      }
   }
}
\newpage
\sstroutine{
   \_OFFLINE\_REDUCTION\_HELLO\_
}{
   Sets up data-reduction tasks and data for OFFLINE\_REDUCTION
   recipes
}{
   \sstdescription{
      Sets up CCDPACK-related global parameters for OFFLINE\_REDUCTION
      recipes. The settings are as follows.

      \sstitemlist{

         \sstitem
         The readout bounds in the internal headers ORAC\_X\_LOWER\_BOUND,
         ORAC\_Y\_LOWER\_BOUND, ORAC\_X\_UPPER\_BOUND, ORAC\_Y\_UPPER\_BOUND define
         the pixel limits for processing, i.e. there are no bias strips and
         interpolation direction.

         \sstitem
         Error processing is disabled so the readout noise and
         analogue-to-digital conversions are not specified.

         \sstitem
         There is no deferred charge.

         \sstitem
         Position list processing tasks expect to find the names of
         lists stored within NDFs.

         \sstitem
         Logging is to the terminal.

         \sstitem
         The data type of NDF arrays is preserved.

         \sstitem
         Does not detect saturated pixels.

         \sstitem
         Parameters are neither saved from or to a `restoration{\tt '} file.

      }
      The script also performs the following tasks.

      \sstitemlist{

         \sstitem
         Calls the steering primitive to set steering headers.

         \sstitem
      }
   }
   \sstnotes{
      \sstitemlist{

         \sstitem
         This primitive is suitable for imaging instruments.
      }
   }
   \sstdiytopic{
      Tasks
   }{
      CCDPACK: CCDSETUP.
   }
}
\newpage
\sstroutine{
   \_QUICK\_LOOK\_HELLO\_
}{
   Sets up data-reduction tasks and data for QUICK\_LOOK recipes
}{
   \sstdescription{
      Sets up CCDPACK-related global parameters for QUICK\_LOOK recipes.
      The settings are as follows.

      \sstitemlist{

         \sstitem
         The readout bounds in the internal headers ORAC\_X\_LOWER\_BOUND,
         ORAC\_Y\_LOWER\_BOUND, ORAC\_X\_UPPER\_BOUND, ORAC\_Y\_UPPER\_BOUND define
         the pixel limits for processing, i.e. there are no bias strips and
         interpolation direction.

         \sstitem
         Error processing is disabled so the readout noise and
         analogue-to-digital conversions are not specified.

         \sstitem
         There is no deferred charge.

         \sstitem
         Position list processing tasks expect to find the names of
         lists stored within NDFs.

         \sstitem
         Logging is to the terminal.

         \sstitem
         The data type of NDF arrays is preserved.

         \sstitem
         Does not detect saturated pixels.

         \sstitem
         Parameters are neither saved from or to a `restoration{\tt '} file.

      }
      The script also performs the following tasks.

      \sstitemlist{

         \sstitem
         Calls the steering primitive to set steering headers.

         \sstitem
      }
   }
   \sstnotes{
      \sstitemlist{

         \sstitem
         This primitive is suitable for imaging instruments.
      }
   }
   \sstdiytopic{
      Tasks
   }{
      CCDPACK: CCDSETUP.
   }
}
\newpage
\sstroutine{
   \_REDUCE\_BIAS\_HELLO\_
}{
   Sets up data-reduction tasks and data for REDUCE\_BIAS recipes
}{
   \sstdescription{
      This primitive sets up CCDPACK-related global parameters for the
      REDUCE\_BIAS recipe and performs preliminary data reduction.

      The CCDPACK settings are as follows.

      \sstitemlist{

         \sstitem
         The readout bounds in the internal headers ORAC\_X\_LOWER\_BOUND,
         ORAC\_Y\_LOWER\_BOUND, ORAC\_X\_UPPER\_BOUND, ORAC\_Y\_UPPER\_BOUND define
         the pixel limits for processing, i.e. there are no bias strips and
         interpolation direction.

         \sstitem
         Error processing is disabled so the readout noise and
         analogue-to-digital conversions are not specified.

         \sstitem
         There is no deferred charge.

         \sstitem
         Position list processing tasks expect to find the names of
         lists stored within NDFs.

         \sstitem
         Logging is to the terminal.

         \sstitem
         The data type of NDF arrays is preserved.

         \sstitem
         Does not detect saturated pixels.

         \sstitem
         Parameters are neither saved from or to a `restoration{\tt '} file.

      }
      The script also performs the following tasks.

      \sstitemlist{

         \sstitem
         Calls the steering primitive to set steering headers.

         \sstitem
         Calls \_MASK\_BAD\_PIXELS\_ primitive to mask bad pixels.
      }
   }
   \sstnotes{
      \sstitemlist{

         \sstitem
         This primitive is suitable for imaging instruments.
      }
   }
   \sstdiytopic{
      Steering Headers
   }{
      USE\_VARIANCE = LOGICAL
         Whether or not variance processing is to occur. This equates to
         argument USEVAR.
   }
}
\newpage
\sstroutine{
   \_REDUCE\_BIAS\_STEER\_
}{
   Steers processing for REDUCE\_BIAS recipe
}{
   \sstdescription{
      This primitive control processing for REDUCE\_BIAS recipe through
      steering headers listed below.
   }
   \sstnotes{
      \sstitemlist{

         \sstitem
         This primitive is suitable for imaging optical cameras.

         \sstitem
         Processing only occurs for bias frames.
      }
   }
   \sstdiytopic{
      Steering Headers
   }{
      BIAS\_FRAMES = HASH
         The name of the bias for a given exposure time, and the
         corresponding number of frames used to create it.
      USE\_VARIANCE = LOGICAL
         Whether or not variance processing is to occur. This equates to
         argument USEVAR.
   }
}
\newpage
\sstroutine{
   \_BIAS\_GROUP\_
}{
   Reduce a group of bias frames for array tests to determine a group
   bias
}{
   \sstdescription{
      Reduces a group bias frames for array tests to determine a group
      bias. This primitive determines the mean and variance for the bias
      frames, converts the variance into a population variance estimate,
      and stores these in the data and variance arrays of a file called
      bias\_gNNN, where NNN is the group number. The bias\_gNNN file is
      also filed with the Cal system.

      This primitive is used with the UIST array test sequence. It must
      be called after \_ARRAY\_TESTS\_STEER\_ so that internal headers can
      be used.
   }
}
\newpage
\sstroutine{
   \_REDUCE\_BIAS\_TIDY\_
}{
   Removes unwanted intermediate files for the REDUCE\_BIAS recipe
}{
   \sstdescription{
      Removes all intermediate frames. Files are only removed when they
      are no longer needed, as guided by the steering header MAKE\_BIAS.
   }
}
\newpage
\sstroutine{
   \_REDUCE\_DARK\_STEER\_
}{
   Steers processing for REDUCE\_DARK recipe
}{
   \sstdescription{
      This primitive control processing for REDUCE\_DARK recipe through
      steering headers listed below.
   }
   \sstnotes{
      \sstitemlist{

         \sstitem
         This primitive is suitable for imaging optical cameras.

         \sstitem
         Processing only occurs for dark frames.
      }
   }
   \sstdiytopic{
      Steering Headers
   }{
      USE\_VARIANCE = LOGICAL
         Whether or not variance processing is to occur. This equates to
         argument USEVAR.
   }
}
\newpage
\sstroutine{
   \_REDUCE\_DARK\_TIDY\_
}{
   Removes unwanted intermediate files for the REDUCE\_DARK recipe
}{
   \sstdescription{
      Removes all intermediate frames. Files are only removed when they
      are no longer needed, as guided by the steering header MAKE\_DARK.
   }
}
\newpage
\sstroutine{
   \_REMOVE\_BIAS\_
}{
   Subtracts a bias frame
}{
   \sstdescription{
      This primitive subtracts a zero bias from the current frame;
      unless the data have variance information and were taken using a
      non-ND mode (i.e. where the bias has not already been subtracted),
      whereupon a bias frame, if available, is subtracted.

      For most instruments there is no bias to subtract so it is
      something of a placeholder primitive. Its main purpose is to set
      up CCDPACK for subsequent processing. For instance, CCDPACK will
      complain if debiassing is not performed before say flat-fielding.
      The primitive reports a successful bias subtraction and the frames
      concerned.
   }
   \sstnotes{
      \sstitemlist{

         \sstitem
         This primitive is suitable for UFTI, IRCAM, INGRID, and
         Michelle in imaging mode.

         \sstitem
         Processing occurs for all frames, and sub-frames therein.

         \sstitem
         Where a bias frame is used, it is the most-recent and matching
         given by \$Cal-$>$bias method.

         \sstitem
         The observing mode (read type) comes from user header
         ORAC\_DETECTOR\_READ\_TYPE.

         \sstitem
         The bias-subtracted image can be displayed.

         \sstitem
         The frame title is propagated.
      }
   }
   \sstdiytopic{
      Output Data
   }{
      \sstitemlist{

         \sstitem
         Bias-subtracted frame inheriting the frame{\tt '}s name but with the
         \_db suffix.
      }
   }
   \sstdiytopic{
      Tasks
   }{
      CCDPACK: DEBIAS.
   }
}
\newpage
\sstroutine{
   \_SET\_ORIGIN\_
}{
   Sets the origin of an observation
}{
   \sstdescription{
      This primitive sets the origin of an observation, including all
      integrations, using the ORAC\_X\_LOWER\_BOUND and ORAC\_Y\_LOWER\_BOUND
      user headers in the frame. It is needed to correct the raw data
      from a sub-array for which the origin is still at the default. If
      either header is undefined, the primitive creates a default origin
      (1,1). The origin is not set if ORAC\_X\_LOWER\_BOUND does not exist.
   }
   \sstnotes{
      \sstitemlist{

         \sstitem
         This primitive is suitable for UFTI, IRCAM, and Michelle and
         UIST in imaging mode.
      }
   }
   \sstdiytopic{
      Tasks
   }{
      KAPPA: SETORIGIN.
   }
}
\newpage
\newpage
\sstroutine{
   \_SKY\_FLAT\_STEER\_
}{
   Steers processing for SKY\_FLAT recipes
}{
   \sstdescription{
      This primitive control processing for SKY\_FLAT recipes through
      steering headers listed below.
   }
   \sstnotes{
      \sstitemlist{

         \sstitem
         This primitive is suitable for imaging infrared cameras.

         \sstitem
         Processing only occurs for object, sky, or calibration lamp
         frames.

         \sstitem
         The data are deemed to be polarimetry if the frame internal
         header ORAC\_FILTER contains the string {\tt "}pol{\tt "}.

         \sstitem
         A list of the distinct filters within the group is stored in an
         array stored by reference in the group user header FILTER\_LIST.
      }
   }
   \sstdiytopic{
      Steering Headers
   }{
      CYCLE\_NUMBER = INTEGER
         Number of the cycle, a cycle being a set of frames to complete
         a pass through the recipe. The first cycle is 0.
      JITTER\_NUMBER = INTEGER
         The number of frames in the jitter.
      MAKE\_FLAT = LOGICAL
         Whether or not to make the flat. The flat is made once all the
         jittered target frames in a cycle are available.
      MASK\_OBJECTS = LOGICAL
         Whether or not to mask the objects. Masking occurs when all the
         jittered frames in a cycle are available.
      POL\_CYCLE\_NUMBER = INTEGER
         Number of the polarimetry cycle, a cycle being a set of frames
         to complete a pass through the recipe for all waveplate angles.
         The first cycle is 0.
      USE\_VARIANCE = LOGICAL
         Whether or not variance processing is to occur. This equates to
         argument USEVAR.
      WAVEPLATE\_FLAT = LOGICAL
         See the argument of the same name. This header merely
         propagates the value of the argument.
   }
}
\newpage
\sstroutine{
   \_STANDARD\_MAGNITUDE\_
}{
   Obtains the catalogue magnitude of a standard
}{
   \sstdescription{
      This primitive reads the faint-standard catalogue or its
      predecessor. A case- and space-insensitive comparison of the
      supplied object name with the entries in the table provides a
      catalogue magnitude in U, B, V, R, or I for a standard star.
   }
   \sstnotes{
      \sstitemlist{

         \sstitem
         This primitive is suitable for LCOGT optical imagers.

         \sstitem
         Processing only occurs when it is time to perform photometry,
         i.e. when the steering header DO\_APHOT is true.

         \sstitem
         An error occurs when the filter is not one of U, B, V, R, or I.

         \sstitem
         Invokes \_LCOGT\_STANDARD\_MAGNITUDE\_ to obtain the magnitude and
         catalogue name.
      }
   }
}
\newpage
\sstroutine{
   \_SUBTRACT\_DARK\_NO\_THRESH\_
}{
   Subtracts a dark frame
}{
   \sstdescription{
      This primitive subtracts from the current frame the most-recent
      and matching dark frame given by \$Cal-$>$dark method. It reports a
      successful dark subtraction and the frames concerned.
   }
   \sstnotes{
      \sstitemlist{

         \sstitem
         This primitive is suitable for UFTI, IRCAM, and Michelle in
         imaging mode.

         \sstitem
         Processing only occurs for object and sky frames.

         \sstitem
         The dark-subtracted image can be displayed.

         \sstitem
         The subtraction assumes the same exposure time for the dark and
         object frame. That validation should be done by the \$Cal-$>$dark
         method.

         \sstitem
         The frame title is propagated.
      }
   }
   \sstdiytopic{
      Output Data
   }{
      \sstitemlist{

         \sstitem
         Dark-subtracted frame inheriting the frame{\tt '}s name but with the
         \_dk suffix.
      }
   }
   \sstdiytopic{
      Tasks
   }{
      CCDPACK: CALCOR.
   }
}
\newpage
\sstroutine{
   \_SUBTRACT\_DARK\_
}{
   Subtracts a dark frame
}{
   \sstdescription{
      This primitive subtracts from the current frame the most-recent
      and matching dark frame given by \$Cal-$>$dark method. It reports a
      successful dark subtraction and the frames concerned.

      Since transient `hot{\tt '} and `cold{\tt '} pixels can be present despite the
      application of a bad-pixel mask, the primitive also thresholds the
      dark-subtracted frame, setting values beyond the limits to be bad
      (i.e. undefined), to remove these non-physical values. Such values
      can lead to problems later in the pipeline. In a sense this
      processing step augments the bad-pixel mask.
   }
   \sstnotes{
      \sstitemlist{

         \sstitem
         This primitive is suitable for infrared imaging.

         \sstitem
         Processing only occurs for object, sky, and flat frames.

         \sstitem
         The dark-subtracted image can be displayed.

         \sstitem
         The subtraction assumes the same exposure time for the dark and
         object frame. That validation should be done by the \$Cal-$>$dark
         method.

         \sstitem
         The lower threshold limit is the clipped mean (mode) minus five
         standard deviations, subject to the constraint that the limit lies
         between -100 and 1. The upper limit is 1000 above the nominal
         saturation level for the instrument and its mode.

         \sstitem
         The primitive issues a warning if the dark-subtracted frame{\tt '}s
         mode is negative, allowing for the error of the mode. It aborts
         with an error message if the modal dark-subtracted signal is more
         than one standard deviation negative.

         \sstitem
         The frame title is propagated.
      }
   }
   \sstdiytopic{
      Output Data
   }{
      \sstitemlist{

         \sstitem
         Dark-subtracted frame inheriting the frame{\tt '}s name but with the
         \_dk suffix.

         \sstitem
         An array with bad-value substitution beyond thresholds and
         inheriting the frame{\tt '}s name but with the \_th suffix.
      }
   }
   \sstdiytopic{
      Tasks
   }{
      CCDPACK: CALCOR; KAPPA: THRESH.
   }
}
\newpage
\sstroutine{
   \_TURN\_ON\_HISTORY\_
}{
   Switches on history recording
}{
   \sstdescription{
      This primitive enables NDF history recording for each integration
      in an observation.
   }
   \sstnotes{
      \sstitemlist{

         \sstitem
         This primitive is suitable for UFTI, IRCAM, and Michelle and
         UIST in imaging mode.

         \sstitem
         If the ORAC\_HISTORY\_OFF environment variable is set, then
         history will be disabled.
      }
   }
   \sstdiytopic{
      Tasks
   }{
      KAPPA: HISSET.
   }
}
